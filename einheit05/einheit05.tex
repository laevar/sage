\documentclass[notes=hide,hyperref={dvipdfmx,pdfpagelabels=false}]{beamer}
\mode<article>
{
  \usepackage{fullpage}
  \usepackage{pgf}
  \usepackage[xetex]{hyperref}
  \setjobnamebeamerversion{beamer}
}

\mode<presentation>
{
  %\usetheme{Frankfurt}
 %\usetheme{My}
  \usetheme{Madrid}
  % or ...
%\usecolortheme{seagull}
  %\setbeamercovered{transparent}
  %\setbeamercovered{dynamic}
  % or whatever (possibly just delete it)
}
\usenavigationsymbolstemplate{}
\usefonttheme{structurebold}
\usepackage{multimedia}
%\usepackage{tikz}
\usepackage{fontspec,xunicode,xltxtra}

%\usepackage{polyglossia}
%\setdefaultlanguage[spelling=new, latesthyphen=true]{german}
%\setsansfont{DejaVu Sans}
%\setsansfont{Verdana}
%\setsansfont{Arial}
%\setromanfont{Linux Libertine O}
%\setsansfont{Linux Biolinum O}

\setbeamertemplate{footline}
{
\leavevmode
%\hbox{\begin{beamercolorbox}[wd=.5\paperwidth,ht=2.5ex,dp=1.125ex,
%leftskip=.3cm plus1fill,rightskip=.3cm]{author in head/foot}%
%    \usebeamerfont{author in head/foot}\insertshortauthor
%  \end{beamercolorbox}%
%  \begin{beamercolorbox}[wd=.5\paperwidth,ht=2.5ex,dp=1.125ex,leftskip=.3cm,
%rightskip=.3cm plus1fil]{title in head/foot}%
%    \usebeamerfont{title in head/foot}\insertshorttitle\hfill

\hfill\insertframenumber  \hspace{3pt}

%\inserttotalframenumber
%\hspace*{2ex}
%  \end{beamercolorbox}}%
  \vskip3pt%
}

\usepackage[ngerman]{babel}
\selectlanguage{ngerman}

%
% math/symbols
%
\usepackage{amssymb}
\usepackage{amsthm}
% \usepackage{latexsym}
\usepackage{amsmath}
%\usepackage{amsxtra} %Weitere Extrasymbole
%\usepackage{empheq} %Gleichungen hervorheben
%\usepackage{bm}
 %\bm{A} Boldface im Mathemodus

\usepackage{cellspace}
\setlength{\cellspacetoplimit}{2pt}
\setlength{\cellspacebottomlimit}{2pt}

%%%%%%%%%%%%%%%%%% Fuer Frames [fragile]-Option verwenden!
%Programm-Listing
%%%%%%%%%%%%%%%%%%
%Listingsumgebung fuer verbatim
%Grauhinterlegeter Text
%Automatischer Zeilenumbruch ist aktiviert
\usepackage{listings}
\definecolor{lgray}{gray}{0.80}
%\lstset{backgroundcolor=\color{lgray}, frame=single, basicstyle=\ttfamily, breaklines=true}
\lstnewenvironment{sage}{\lstset{backgroundcolor=\color{lgray},language=Python, emphstyle=\color{red}, frame=single, basicstyle=\ttfamily, breaklines=true,mathescape =true,escapechar=§}}{}


\usepackage{mydef}
%\usepackage{cmap} % you can search in the pdf for umlauts and ligatures
\usepackage{colonequals} %corrects the definition-symbols \colonequals (besides others)
\title{Einführung in Sage}
%
%\subtitle{Disputation} % (optional)

\author{Jochen Schulz}
% - Use the \inst{?} command only if the authors have different
%   affiliation.

\institute{Georg-August Universit\"at G\"ottingen \pgfimage[height=0.5cm]{../figures/unilogo3}}
% - Use the \inst command only if there are several affiliations.
% - Keep it simple, no one is interested in your street address.

\date{\today}

\subject{Sage}
% This is only inserted into the PDF information catalog. Can be left
% out. 

% If you have a file called "university-logo-filename.xxx", where xxx
% is a graphic format that can be processed by latex or pdflatex,
% resp., then you can add a logo as follows:

%\logo{\pgfimage[height=0.5cm]{figures/unilogo3}}


% Delete this, if you do not want the table of contents to pop up at
% the beginning of each subsection:

\AtBeginSection[]
{
  \begin{frame}<beamer>
    \frametitle{Aufbau}
    \tableofcontents[currentsection,currentsubsection]
  \end{frame}
}

\AtBeginSubsection[]
{
  \begin{frame}<beamer>
    \frametitle{Aufbau}
    \tableofcontents[currentsection,currentsubsection]
  \end{frame}
}



%%%%%%%%%%%%%%%%%%%
%Neue Definitionen
%%%%%%%%%%%%%%%%%%%

%Newcommands
\newcommand{\Fun}[1]{\mathcal{#1}}      %Mathcal fuer Funktoren
\newcommand{\field}[1]{\mathbb{#1}}     %Grundkoerper ?? in mathds

\newcommand{\A}{\field{A}}              %Affines A
\newcommand{\C}{\field{C}}              %Complexes C
\newcommand{\Fp}{\field{F}_{\!p}}       %Endlicher Koerper mit p Elementen
\newcommand{\Fq}{\field{F}_{\!q}}       %Endlicher Koerper mit q Elementen
\newcommand{\Ga}{\field{G}_{a}}         %Add Gruppenschema
\newcommand{\K}{\field{K}}              %Generischer Koerper 
\newcommand{\N}{\field{N}}              %Nat Zahlen
\newcommand{\Pj}{\field{P}}             %Projektives P
\newcommand{\R}{\field{R}} 		%Reelle Zahlen
\newcommand{\Q}{\field{Q}}              %Rationale Zahlen  
\newcommand{\Qt}{\field{H}}             %Quaternionen 
\newcommand{\V}{\field{V}}              %Vektorbuendel V
\newcommand{\Z}{\field{Z}}              %Ganze Zahlen

\newcommand{\fdg}{\;|\;}                 %fuer die gilt

%Operatoren
\DeclareMathOperator{\Abb}{Abb}
%\usepackage{sagetex}

\begin{document}
\lstset{basicstyle={\lstbasicfont\footnotesize}}


\subtitle{Einheit 5}
\maketitle

\begin{frame}{Aufbau}
\tableofcontents
\end{frame}

% \begin{frame}[fragile]{Übersicht}
% \begin{itemize}
% \item Datencontainer in MuPAD
% \item Lineare Abbildungen und Matrizen
% \item Eigenwerte und Eigenvektoren
% \item Lineare Gleichungssysteme
% \end{itemize}
% \end{frame}


%===============================================
\section{Datencontainer in Sage}
%===============================================

\begin{frame}{Datencontainer in Sage}
 \begin{itemize}
\item Matrizen (bereits besprochen)
\item Mengen (bereits besprochen)
\item Folgen/Tuple
\item Listen
\item Wörterbücher
\item Felder
\end{itemize}
\end{frame}

\begin{frame}[fragile]{Folgen}
\begin{itemize}
\item  Folgen sind ein grundlegender Typ. Zum Beispiel werden Listen
  und Mengen aus Folgen aufgebaut.
\item Folgen sind Sage-Ausdrücke vom Typ \verb+tuple+.
\item Eine Folge ist eine Aneinanderreihung beliebiger Sage-Objekte,
  welche durch Kommata getrennt sind.
\item Wir kennen Folgen bereits durch die Eingabe mehrerer Befehle in
  einer Zeile.
 \end{itemize}
\end{frame}

\begin{frame}[fragile]{Konstruktion von Folgen I}
\begin{itemize}
\item Einfache Definition
\begin{sage}
>> t=var('a,b,c,d,e');Folge1 = a,b,c; Folge2 = c,d,e; Folge1 ; Folge2
\end{sage}
\begin{sage}
(a, b, c)
(c, d, e)
\end{sage}
\item Verbinden von Folgen
\begin{sage}
>> Folge3 = Folge1+Folge2; Folge3
\end{sage}
\begin{sage}
(a, b, c, c, d, e)
\end{sage}
\end{itemize}
\end{frame}

\begin{frame}[fragile]{Konstruktion von Folgen II}
\begin{itemize}
\item Konstruktion durch \verb+tuple()+:  Der Aufruf 
{\color{red}tuple(range(a,b)+}, für
  $a,b \in \mathbb{Z}$
  erzeugt die Folge $a,a+1,\dots,b-1$:
\begin{sage}
>> tuple(range(-1,7))
\end{sage}
\begin{sage}
  (-1, 0, 1, 2, 3, 4, 5, 6)
\end{sage}
\item Der Aufruf {\color{red}\verb+tuple(Objekt(i) for i in range(m,n))+} erzeugt die Folge

\centering{Objekt(m), Objekt(m+1), $\ldots$ ,Objekt(n-1)}
\begin{sage}
>> tuple(i^2 for i in range(2,8))
\end{sage}
\begin{sage}
(4, 9, 16, 25, 36, 49)
\end{sage}
\begin{sage}
>> tuple(x^i for i in range(4,8))
\end{sage}
\begin{sage}
(x^4, x^5, x^6, x^7)
\end{sage}
\end{itemize}
\end{frame}

\begin{frame}[fragile]{Konstruktion von Folgen III}
\begin{itemize}
\item Erzeugen von $n$ identischen Objekten
\begin{sage}
>> tuple(sin(x) for i in range(0,5))
\end{sage}
\begin{sage}
  (sin(x),sin(x),sin(x),sin(x),sin(x))
\end{sage}
\item Erzeugen von $n$ funktionalen Objekten
\begin{sage}
>> x = tuple(sin(i) for i in range(1,5)); x
\end{sage}
\begin{sage}
(sin(1), sin(2), sin(3), sin(4))
\end{sage}  
\end{itemize}
\end{frame}

\begin{frame}[fragile]{Konstruktion und Zugriff}
\begin{itemize}
\item Erzeugen einer leeren Folge
\begin{sage}
>> folge = (); folge2 = folge,2,3; folge2
\end{sage}
\begin{sage}
 ((), 2, 3)
\end{sage}
\item Zugriff auf Elemente
\begin{sage}
>> x = 1,2,3,4; x[2] 
\end{sage}
\begin{sage}
  3
\end{sage}
\begin{sage}
>> x[2]; x[0:2]
\end{sage}
\begin{sage}
3
(1, 2)
\end{sage}
\end{itemize}
\end{frame}

% \begin{frame}[fragile]{Konstruktion und Zugriff}
% \begin{itemize}
% \item Löschen von Einträgen -- Folge verkürzt sich:
% \begin{sage}
% >> delete x[2]: x
% \end{sage}
% \begin{sage}
%   1, 3, 4
% \end{sage}
% \begin{sage}
% >> delete x[2]: x
% \end{sage}
% \begin{sage}
%   1, 4
% \end{sage}
% \begin{sage}
% >> delete x[2]: x
% \end{sage}
% \begin{sage}
%   1
% \end{sage}
% \begin{sage}
% >> delete x[2]: x
% \end{sage}
% \begin{sage}
%   Error: Invalid index [delete]
% \end{sage}
% \end{itemize}
% \end{frame}

\begin{frame}[fragile]{Listen}
\begin{itemize}
\item Eine Liste ist eine geordnete Folge beliebiger Sage Objekte.
\item In Sage ist eine Liste in eckigen Klammern eingeschlossen.
\item Listen haben den Datentyp \verb+list+.
\item Matrizen werden als geschachtelte Liste definiert.
\item Sie baut auf den eben beschriebenen Folgen auf.
\end{itemize}
\end{frame}

\begin{frame}[fragile]{Konstruktion von Listen}
\begin{itemize}
\item Konstruktion 'per Hand'
\begin{sage}
>> Liste = [1,[1,2], Set([1,2,3]),x]; Liste
\end{sage}
\begin{sage}
[1, [1, 2], {1, 2, 3}, (1, 2, 3, 4)]
\end{sage}
\item Listen können leer sein:
\begin{sage}
>> Liste = []; Liste
\end{sage}
\begin{sage}
  []
\end{sage}

\item Erzeugen von Listen mit funktionalen Objekten
\begin{sage}
>> Liste = [2^i for i in range(1,9)]; Liste
\end{sage}
\begin{sage}
[2, 4, 8, 16, 32, 64, 128, 256]
\end{sage}
\end{itemize}
\end{frame}

\begin{frame}[fragile]{map}
Mit Hilfe der Funktion {\color{blue} \verb+map(<f>,<Liste>)+} kann eine Funktion \verb+f+
auf alle Elemente
der Liste \verb+<Liste>+ angewendet werden. Erwartet eine Funktion
mehrere Argumente, muss diese Funktion erst gekapselt werden.
\begin{sage}
>> map(sin,[x,1,0,pi,0.3])
\end{sage}
\begin{sage}
[sin(x), sin(1), 0, 0, 0.295520206661340]
\end{sage}
\begin{sage}
>> map(is_prime,[2,3,4,5,6,7])
\end{sage}
\begin{sage}
[True, True, False, True, False, True]
\end{sage}
\end{frame}

\begin{frame}[fragile]{Zugriff auf Listen}
Der Zugriff funktioniert genau wie bei  Folgen.
Beispiele:
\begin{sage}
>> Liste = [(x,i) for x in range(1,4) for i in range(0,x)];  Liste
\end{sage}
\begin{sage}
[(1, 0), (2, 0), (2, 1), (3, 0), (3, 1), (3, 2)]
\end{sage}
\begin{sage}
>> Liste[3], Liste[5]
\end{sage}
\begin{sage}
((3, 0), (3, 2))
\end{sage}
\begin{sage}
>> Liste[5] = 42; Liste
\end{sage}
\begin{sage}
  [(1, 0), (2, 0), (2, 1), (3, 0), (3, 1), 42]
\end{sage}
\end{frame}

\begin{frame}[fragile]{Weitere Befehle für Listen I}
\begin{itemize}
\item Entfernen eines Elements aus der Liste
\begin{sage}
>> Liste = [a,b,c]; del Liste[1]; Liste 
\end{sage}
\begin{sage}
  [a, c]
\end{sage}
\item Anhängen von Elementen mittels \verb+append+
\begin{sage}
>> Liste.append([3,4,5]); print Liste
\end{sage}
\begin{sage}
[a, c, [3, 4, 5]]
\end{sage}
\end{itemize}
\end{frame}

\begin{frame}[fragile]{Weitere Befehle für Listen II}
\begin{itemize}
\item Zusammenfügen von Listen mit dem $+$- und $*$-Operator. 
\begin{sage}
>> Liste2 = Liste+[3,4,5]; Liste2
\end{sage}
\begin{sage}
[a, c, 3, 4, 5]
\end{sage}
\begin{sage}
>> Liste*2
\end{sage}
\begin{sage}
 [a, c, [3, 4, 5], a, c, [3, 4, 5]]
\end{sage}
\item Sortieren von Listen mit {\color{blue} \verb+sort+}
\begin{sage}
>> Liste = [4,23,1,3]; Liste.sort(); Liste
\end{sage}
\begin{sage}
[1, 3, 4, 23]
\end{sage}
\end{itemize}
\end{frame}

\begin{frame}[fragile]{contains}
Mit {\color{blue} \verb+<Objekt> in <Liste>+} kann geprüft werden, ob ein
            Objekt \verb+<Objekt>+ in der Liste \verb+<Liste>+ enthalten ist.
\begin{sage}
>> Liste = [x+1,a,x+1,sin(b)]
>> x+1 in Liste
\end{sage}
\begin{sage}
  True
\end{sage}
Mit {\color{blue} \verb+<Liste>.index(<Objekt>)+} erhält man die Position des Objektes in der Liste. Ist es nicht vorhanden, bekommt man eine enstprechende Meldung.
\begin{sage}
>> Liste.index(sin(b))
\end{sage}
\begin{sage}
  3
\end{sage}
\end{frame}

\begin{frame}[fragile]{filter ?? gabs schon..}
\begin{itemize}
\item Mittels {\color{blue} \verb+select(Liste, Funktion, Parameter)+} können Objekte mit bestimmten Eigenschaften aus einer
Liste ausgewählt werden. 
\item Dabei fungiert eine Funktion \verb+Funktion+
als Auswahlkriterium. Sie muss Boolsche Werte zurückliefern. 
\item Die
\verb+Parameter+ sind Eingabewerte für die Funktion \verb+Funktion+ 
\item Der Befehl \verb+select+ funktioniert auch für Mengen, Tabellen oder Ausdrücke.
\end{itemize}
\end{frame}

\begin{frame}[fragile]{Beispiele für select ??}
\begin{sage}
>> filter(bool,[1==1, 1==2, 3==3, 4==5, 7==7])
\end{sage}
\begin{sage}
[True, True, True]
\end{sage}
\begin{sage}
>> select([a+2,x,y,z,sin(a)],has,a)
\end{sage}
\begin{sage}
  [a + 2, sin(a)]
\end{sage}
\begin{sage}
>> select(11*x*y*(1-x)^2*5,has,x)
\end{sage}
\begin{sage}
           2
  x (x - 1)
\end{sage}
\end{frame}

% \begin{frame}[fragile]{split}
% Durch \verb+split+ wird eine Liste bezüglich eines Kriteriums in drei
% Teillisten $A,B,C$ aufgeteilt.
% $A$ steht für \verb+TRUE+, $B$ für
% \verb+FALSE+ und $C$ für \verb+UNKNOWN+.
% Durch den Befehl
% \begin{center}
% {\color{blue} \verb+split(Liste, f, par)+}
% \end{center}
% wird die \verb+Liste+ bzglüglich einer Boolschen Funktion  \verb+f+ mit
% Parameter \verb+par+ aufgeteilt.
% \begin{sage}
% >> [A,B,C]:=split([sin(x),x+y,y,11],has,x):
% >> A,B,C
% \end{sage}
% \begin{sage}
%  [sin(x), x + y], [y, 11], []
% \end{sage}
% \end{frame}

\begin{frame}[fragile]{zip}
Mittels {\color{blue} \verb+zip(<Liste1>, <Liste2>)+} werden die Elemente zweier Listen
                        paarweise zu einer neuen Liste verknüpft.
\begin{sage}
>> Liste1 = [a,b,c]; Liste2 = [e,f,g]
>> zip(Liste1,Liste2)
\end{sage}
\begin{sage}
[(a, e), (b, f), (c, g)]
\end{sage}
% \begin{sage}
% >> zip(Liste1,Liste2,_power)
% \end{sage}
% \begin{sage}
%     e   f   g
%   [a , b , c ]
% \end{sage}
\end{frame}

\begin{frame}[fragile]{Wörterbücher (Dictonaries)}
\begin{itemize}
\item Eine Tabelle besteht aus einer Ansammlung von \glq Assoziationen\grq\  der 
Form \verb+<Index>:<Wert>+. Indizes und Werte können dabei beliebige Sage
Objekte sein. 
\item Tabellen haben den Typ \verb+dict+.
\item  Tabellen sind Datenstrukturen, die für das Speichern großer
Datenmengen gut geeignet sind.
\item In Sage ist der indizierte Zugriff auf einzelne Elemente sehr
schnell, da intern nicht die gesamte Datenstruktur durchsucht wird.
\end{itemize}
\end{frame}

\begin{frame}[fragile]{Konstruktion von Dictonaries I}
\begin{itemize}
\item Konstruktion: ({\color{blue} \verb+{<Index>:<Wert>, ... }+})
\begin{sage}
>> T = {a: b, c: d}; T
\end{sage}
\begin{sage}
{c: d, a: b}
\end{sage}
\end{itemize}
\end{frame}

\begin{frame}[fragile]{Konstruktion von Dictonaries II}
\begin{itemize}
\item Einträge können durch Zuweisungen der Form \verb+<Dict>[<Index>] = <Wert>+ erzeugt oder verändert werden. 
\begin{sage}
>> T[f(x)] = sin(x); T[1,2] = 5
>> T[1,2,3] = [a,b,c]; T[a] = d;
>> T
\end{sage}
\begin{sage}
{(1, 2): 5, f: sin(x), c: d, a: d, (1, 2, 3): [a, b, c]}
\end{sage}
\item Erzeugen einer leeren Tabelle
\begin{sage}
>> T1 = {}; T1
\end{sage}
\begin{sage}
  {}
\end{sage}
\end{itemize}
\end{frame}

\begin{frame}[fragile]{Konstruktion und Zugriff}
\begin{itemize}
\item Zugriff auf ein Dictonary
\begin{sage}
>> T[a],T[1,2],T[c]
\end{sage}
\begin{sage}
(d, 5, d)
\end{sage}
\begin{sage}
 float(T[f](4))
\end{sage}
\begin{sage}
 -0.7568024953079282
\end{sage}
Wird ein \verb+Index+ nicht gefunden, gibt es eine Fehlermeldung
\item Löschen von Einträgen
\begin{sage}
>> del T[a]
\end{sage}
\end{itemize}
\end{frame}

\begin{frame}[fragile]{Befehle wie bei Listen}
\begin{center}
\begin{tabular}{|l|p{7cm}|}
\hline
\verb+contains+ & Es wird überprüft, ob ein \verb+Index+ in einer
Tabelle vorkommt. \\
\verb+select+ & Filtert eine Tabelle nach Kriterien. Prüft sowohl
\verb+Werte+ als auch \verb+Indizes+.\\
\verb+split+ & Sortiert die Tabelle in Teiltabellen bzgl. eines
Kriteriums.  Prüft sowohl
\verb+Werte+ als auch \verb+Indizes+.\\
\verb+map+ & Wendet eine Funktion auf die Werte an.\\
\hline
\end{tabular}
\end{center}
\end{frame}

\begin{frame}[fragile]{Beispiele}

\verb+contains+ prüft ob \verb+a+ oder \verb+b+ als Index in einer Tabelle auftaucht.
\begin{sage}
>> Z[a]:=b: Z[c]:=d: Z[x] := b
>> contains(Z,a), contains(Z,b)
\end{sage}
\begin{sage}
  TRUE, FALSE
\end{sage}
\verb+a+ taucht als Index auf, \verb+b+ nur als Wert.

\verb+select+ prüft sowohl den Index als auch den Wert auf \verb+a+ oder \verb+b+ ab und liefert eine Tabelle die nur noch den entsprechenden Index=Wert Eintrag enthält zurück.
\begin{sage}
>> select(Z,has,a), select(Z,has,b)
\end{sage}
\begin{sage}
  table(    table(
    a = b,    a = b,
  )           x = b
            )
\end{sage}
\end{frame}

\begin{frame}[fragile]{Felder}
\begin{itemize}
\item Felder sind spezielle Tabellen. Die Indizes müssen
           cartesische Produkte von ganzen (!) Zahlen sein (Multiindizes).
\item Matrizen lassen sich als zweidimensionale Felder interpretieren, d.h. die Indizes sind Paare von ganzen Zahlen.
\item Es können Felder beliebiger Dimension erzeugt werden. Felder
sind geeignet
für Datenmengen fixierter Größe.
\item Felder haben den Datentyp \verb+DOM_ARRAY+.
\item Die Matrixklasse hat den Vorteil, dass man dort Matrizen
addieren und multiplizieren kann.
\end{itemize}
\end{frame}

\begin{frame}[fragile]{Konstruktion und Zugriff I}
\begin{sage}
>> A:=array(0..1,1..3)
\end{sage}
\begin{sage}
  +-                           -+
  |  ?[0, 1], ?[0, 2], ?[0, 3]  |
  |                             |
  |  ?[1, 1], ?[1, 2], ?[1, 3]  |
  +-                           -+
\end{sage}
Das \verb+?+ bedeutet, dass die entsprechenden Werte nicht belegt sind.
\end{frame}


\begin{frame}[fragile]{Konstruktion und Zugriff II}
\begin{sage}
>> A[0,1]:=1: A[1,3]:=HALLO: A
\end{sage}
\begin{sage}
  +-                           -+
  |     1,    ?[0, 2], ?[0, 3]  |
  |                             |
  |  ?[1, 1], ?[1, 2],  HALLO   |
  +-                           -+
\end{sage}
\begin{sage}
>> A[1,3]
\end{sage}
\begin{sage}
  HALLO
\end{sage}
\end{frame}


%===============================================
\section{Lineare Abbildungen}
%===============================================

\begin{frame}{Lineare Abbildungen}
Seien $K$-Vektorräume $V$ und $W$ gegeben. Eine Abbildung
\[ F: \ V \rightarrow \ W \]
heißt {\color{red} linear}, falls für $v,w\in V$ und $\lambda \in K$ gilt:
\begin{itemize}
\item [(L1)] $F(v+w)=F(v)+F(w)$
\item [(L2)] $F(\lambda \cdot v)=\lambda \cdot F(v)$
\end{itemize} 
Ist $F$ bijektiv, so heißt $F$ {\color{red} Isomorphismus}. 

Gilt $V=W$, so
spricht man von einem {\color{red} Endomorphismus}. Im Falle von $V=W$ und
Bijektivität spricht man von einem {\color{red} Automorphismus}. 
\end{frame}

\begin{frame}{Bemerkungen}
\begin{itemize}
\item Sei $(v_i)_{i\in I}$ eine Basis in $V$ und $(w_i)_{i\in I}$
seien Vektoren in $W$. Dann gibt es genau eine lineare Abbildung $F:V
\rightarrow W$ mit $F(v_i)=w_i$ für alle $i \in I$.
\item Das {\color{red} Bild} von $F$  ist $\mathop{Im}(F) = F(V):=\{ F(v), v \in V \}$.
\item Der {\color{red} Kern} von F ist $\mathop{Ker}(F):=\{v \ \in V \ | \ F(v)=0 \}$
\item Kern und Bild sind Untervektorräume.
\item Dimensionsformel:
\[\dim V = \dim F(V) + \dim Ker(F)\]
\item Die Menge der linearen Abbildungen von $V$ nach $W$ wird mit
{\color{red} $\operatorname{Hom}_K(V,W)$} bezeichnet. Sie ist ein Vektorraum durch punktweise Addition und Skalarmultiplikation.
\end{itemize}
\end{frame}

\begin{frame}{Lineare Abbildungen und Matrizen}
\begin{itemize}
\item Jeder Matrix $A \in K^{m \times n}$ läßt sich durch 
\[
 L_A: K^n \rightarrow K^m,\; 
\begin{pmatrix}
 x_1\\
\vdots\\
x_n
\end{pmatrix}
%\rMapsto
A
\begin{pmatrix}
 x_1\\
\vdots\\
x_n
\end{pmatrix}
\]
eine lineare Abbildung zuordnen.
\item Es gilt $\operatorname{dim}(L_A(K^m))=\operatorname{Rang}(A)$.
\end{itemize}
\end{frame}

\begin{frame}{Koordinatenvektor}
Sei $V$ ein $K$-Vektorraum mit Basis $\mathcal{V}=(v_1, \dots
,v_n)$.
\begin{itemize}
\item Die lineare Abbildung 
$\Phi_\mathcal{V}:K^n \ \rightarrow \ V$ mit
\[\Phi_\mathcal{V}(x_1,\dots ,x_n)=x_1v_1+ \dots +x_nv_n\]
ist ein Isomorphismus. Man nennt $\Phi_\mathcal{V}$ ein
{\color{red} Koordinatensystem} in $V$ und $x=(x_1,\dots ,x_n)=\Phi_\mathcal{V}^{-1}(v)$
den {\color{red} Koordinatenvektor} zu $v \in V$.
\item Ist $\mathcal{Z}$ eine weitere Basis in $V$, so erhält man die
Basiswechselabbildung von $\mathcal{V}$ nach $\mathcal{Z}$ durch $T:=
 \Phi_\mathcal{Z}^{-1} \circ \Phi_\mathcal{V}$.
 \end{itemize}
\end{frame}

\begin{frame}[fragile]{Isomorphismus}
Seien $K$-Vektorräume $V$ und $W$ mit Basen $\mathcal{V}=(v_1, \dots
,v_n)$ und $\mathcal{W}=(w_1, \dots ,w_m)$ gegeben. 

 Für eine Matrix $A\in K^{m
 \times n}$ wird durch
\begin{eqnarray*}{ccc}
F(v_1) & := & a_{11}w_1 + \dots +a_{m1} w_m\\
\vdots &    & \vdots     \\
F(v_n) & := & a_{1n} w_1 + \dots + a_{mn}w_m
\end{eqnarray*}
eine lineare Abbildung $F$ definiert. Dies ergibt einen Isomorphismus
\[
L^\mathcal{V}_\mathcal{W}: K^{m \times n} \ \rightarrow \
\mathrm{Hom}_K(V,W), \quad A \ \mapsto \ F.
\] 
\end{frame}

\begin{frame}{Kanonisches Beispiel}
Seien $K^n$ und $K^m$ mit den kanonischen Basen $\mathcal{K}_n$ und
$\mathcal{K}_m$ versehen.
\begin{itemize}
\item Die Abbildungen $\Phi_{\mathcal{K}_n}$ und $\Phi_{\mathcal{K}_m}$
sind Identitäten.
\item  Die Abbildung $L^{\mathcal{K}_n}_{\mathcal{K}_m}$ ist gegeben
durch 
\[ L^{\mathcal{K}_n}_{\mathcal{K}_m} (A)(x)=Ax,\; x \in K^n. \]
\item Die Spaltenvektoren von $A$ sind die Bilder der Einheitsvektoren
unter der Abbildung  $L^{\mathcal{K}_n}_{\mathcal{K}_m}(A)$.
\end{itemize}
\end{frame}

\begin{frame}[fragile]{Kommutierendes Diagramm}
  Seien $K$-Vektorräume $V$ und $W$ mit Basen $\mathcal{V}=(v_1, \dots
,v_n)$ und $\mathcal{W}=(w_1, \dots ,w_m)$ und eine lineare Abbildung
$F$ gegeben. Dann kommutiert das folgende Diagramm:
%\begin{diagram}
% V & \rTo^F & W\\
% \uTo^{\Phi_V} && \uTo^_{\Phi_W}\\
%K^n & \rTo_{(L_\mathcal{W}^\mathcal{V})^{-1}(F)} & K^m
%\end{diagram}
\end{frame}

\begin{frame}[fragile]{Drehung und Spiegelung I}
Drehung um den Winkel $\alpha$ -- Drehmatrix $G$:
\[ G(\alpha):= \left ( \begin{array}{cc}
\cos(\alpha) & -\sin(\alpha) \\
\sin(\alpha) & \cos(\alpha)
\end{array} \right)
\]
\begin{sage}
>> A:=matrix([[cos(a),-sin(a)],
[sin(a),cos(a)]]):
>>B:=matrix([[cos(b),-sin(b)],
[sin(b),cos(b)]]):
>>simplify(A*B)
\end{sage}
\end{frame}

\begin{frame}[fragile]{Drehung und Spiegelung II}
Spiegelung bezüglich der Ebene 
\[H(a):=\{ x \in \mathbb{R}^3 | x^T a=0  \}, \|a\|=1\]
 durch \[S(a):=I - 2 a a^T.\] 
\begin{sage}
>> a:=matrix(3,1,[1,2,3]):
>> a:=a/norm(a,2):
>> I_n:=matrix(3,3,[1,1,1],Diagonal)
>> S:=I_n-2*a*linalg::transpose(a)
>> norm(S*S-I_n)
\end{sage}
\begin{sage}
   0
\end{sage}
\end{frame}

\section{Eigenwerte und Eigenvektoren}

\begin{frame}{Eigenwerte und Eigenvektoren}
Sei $A\in K^{n\times n}$. Ein Element $\lambda \in K$ heißt {\color{red} Eigenwert} von
$A$, wenn ein $x \in K^n\smallsetminus \{0 \}$ existiert, 
\[ {\color{red}A x = \lambda x} \] 
gilt. Der Vektor $x \in K^n$ heißt {\color{red} Eigenvektor} zum Eigenwert $\lambda$.
\begin{itemize}
\item Die Eigenwerte sind die Nullstellen des {\color{red} charakteristischen
Polynoms} 
\[p(t):=\det(A-t \ I_n).\] 
\item Es gibt höchstens $n$ Eigenwerte.
\end{itemize}
\end{frame}

\begin{frame}{Bemerkungen}
\begin{itemize}
\item Eigenvektoren zu paarweise verschiedenen Eigenwerten  sind linear
unabhängig.
\item Gibt es eine Basis aus Eigenvektoren, so ist $A$
{\color{red} diagonalisierbar}, d.h. man kann die Abbildung $L_A$ bei
geeigneter Basiswahl durch eine Diagonalmatrix repräsentieren.
\item Jeder Endomorphismus eines komplexen Vektorraums läßt sich durch
eine Matrix in \alert{Jordanscher Normalform} darstellen.
\end{itemize}
\end{frame}

\begin{frame}[fragile]{Eigenwerte in MuPAD}
\begin{itemize}
\item Bestimmung von Eigenwerten
\begin{sage}
>> A:=matrix([[cos(al), sin(al)],
[sin(al),-cos(al)]]):
>> simplify(linalg::eigenvalues(A))
\end{sage}
\begin{sage}
  {-1, 1}
\end{sage}
\item Bestimmung von Eigenvektoren
\begin{sage}
>> simplify(linalg::eigenvectors(A))
\end{sage} 
\end{itemize}
\end{frame}

\begin{frame}[fragile]{Eigenwerte in MuPAD}
\begin{itemize}
\item Bestimmung des charakteristischen Polynoms
\begin{sage}
>> I_n:=matrix(2,2,[1,1],Diagonal):
>> p:=linalg::det(A-lambda*I_n)
\end{sage}
\begin{sage}
        2          2          2
  lambda  - sin(al)  - cos(al)
\end{sage}
\begin{sage}
>> simplify(solve(p=0,lambda))
\end{sage}
\begin{sage}
  {-1, 1}
\end{sage}
\end{itemize}
\end{frame}

\begin{frame}{Lineare Gleichungssysteme (LGS)}
Sei $A\in K^{m \times n}$ und $b \in K^m$. Gesucht ist die Menge der
Lösungen (Lösungsraum): 
\[ \{ x \in K^n \;|\; A x = b\} \]
\begin{itemize}
\item Ist $b=0$, so spricht man von einem {\color{red} homogenen
System}. Ansonsten spricht man von einem {\color{red} inhomogenen System}.
\item Der Lösungsraum $W$ des homogenen Systems bildet einen
Untervektorraum des $K^n$. Die Dimension ist 
\[ \dim (W)=n - \mathop{rang}(A). \] 
\end{itemize}
\end{frame}

\begin{frame}{Struktur des Lösungsraums}
\begin{itemize}
\item Die Lösungen des inhomogenen Systems ($b \neq 0$) bilden einen affinen
Unterraum des $K^n$. $X \subset K^n$  heißt {\color{red} affiner Unterraum}, wenn ein
Unterraum $W$ von $K^n$ und ein $v \in K^n$ existiert, so dass 
\[X=v+W\]
gilt. Der Unterraum $W$ ist durch $X$ eindeutig bestimmt, $v$ kann jeder Vektor aus $X$ sein.
\item Ist $W$ der Lösungsraum des homogenen Systems und $v \in K^n$
eine beliebige Lösung von $Ax=b$, dann ist der Lösungsraum $X$ von $Ax=b$
gegeben durch  $X=v+W$.
\item Zwei Lösungen des inhomogenen Systems unterscheiden sich durch
eine Lösung des homogenen Systems.
\end{itemize}
\end{frame}

\begin{frame}{Lösbarkeit}
\begin{itemize}
\item Das inhomogene System ist genau dann für alle $b$
lösbar, wenn $\mathop{rang}(A)=m$ gilt.
\item Das homogene bzw. das inhomogene System besitzt höchstens eine
Lösung, genau dann wenn $\mathop{rang}(A)=n$ gilt.
\item Der Lösungsraum des inhomogenen Systems ist genau dann nicht
leer, wenn $rang(A)=rang(A, b)$ gilt.
\item Praktisch kann ein LGS mit dem {\color{red} Gausschen
Eliminationsverfahren} gelöst werden.
\end{itemize}
\end{frame}

\begin{frame}[fragile]{LGS in MuPAD}
Berechnung der Lösungen von $Ax=b$ mit \verb+linalg::matlinsolve+. Der
erste Rückgabewert ist eine spezielle Lösung. Dann wird ein Basis des
Lösungsraums des homogenen Systems zurückgegeben.
\begin{sage}
>> A:=matrix([[1,2,3],[4,5,6],[7,8,9]]):
>> b1:=matrix(3,1,[0,0,0]):
>> b2:=matrix(3,1,[1,0,0]):
>> b3:=A*b2:
>> linalg::matlinsolve(A,b1)
>> linalg::matlinsolve(A,b2)
>> linalg::matlinsolve(A,b3)
>> linalg::gaussElim(A)
\end{sage}
\end{frame}
\end{document}
