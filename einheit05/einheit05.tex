\documentclass[notes=hide,hyperref={dvipdfmx,pdfpagelabels=false}]{beamer}
\title{Einführung in Sage - Einheit 5}
\subtitle{Datencontainer, Lineare Abbildungen, Eigenwert und Eigenvektoren}
\usepackage{tikz}
\mode<article>
{
  \usepackage{fullpage}
  \usepackage{pgf}
  \usepackage[xetex]{hyperref}
  \setjobnamebeamerversion{beamer}
}

\mode<presentation>
{
  %\usetheme{Frankfurt}
 %\usetheme{My}
  \usetheme{Madrid}
  % or ...
%\usecolortheme{seagull}
  %\setbeamercovered{transparent}
  %\setbeamercovered{dynamic}
  % or whatever (possibly just delete it)
}
\usenavigationsymbolstemplate{}
\usefonttheme{structurebold}
\usepackage{multimedia}
%\usepackage{tikz}
\usepackage{fontspec,xunicode,xltxtra}

%\usepackage{polyglossia}
%\setdefaultlanguage[spelling=new, latesthyphen=true]{german}
%\setsansfont{DejaVu Sans}
%\setsansfont{Verdana}
%\setsansfont{Arial}
%\setromanfont{Linux Libertine O}
%\setsansfont{Linux Biolinum O}

\setbeamertemplate{footline}
{
\leavevmode
%\hbox{\begin{beamercolorbox}[wd=.5\paperwidth,ht=2.5ex,dp=1.125ex,
%leftskip=.3cm plus1fill,rightskip=.3cm]{author in head/foot}%
%    \usebeamerfont{author in head/foot}\insertshortauthor
%  \end{beamercolorbox}%
%  \begin{beamercolorbox}[wd=.5\paperwidth,ht=2.5ex,dp=1.125ex,leftskip=.3cm,
%rightskip=.3cm plus1fil]{title in head/foot}%
%    \usebeamerfont{title in head/foot}\insertshorttitle\hfill

\hfill\insertframenumber  \hspace{3pt}

%\inserttotalframenumber
%\hspace*{2ex}
%  \end{beamercolorbox}}%
  \vskip3pt%
}

\usepackage[ngerman]{babel}
\selectlanguage{ngerman}

%
% math/symbols
%
\usepackage{amssymb}
\usepackage{amsthm}
% \usepackage{latexsym}
\usepackage{amsmath}
%\usepackage{amsxtra} %Weitere Extrasymbole
%\usepackage{empheq} %Gleichungen hervorheben
%\usepackage{bm}
 %\bm{A} Boldface im Mathemodus

\usepackage{cellspace}
\setlength{\cellspacetoplimit}{2pt}
\setlength{\cellspacebottomlimit}{2pt}

%%%%%%%%%%%%%%%%%% Fuer Frames [fragile]-Option verwenden!
%Programm-Listing
%%%%%%%%%%%%%%%%%%
%Listingsumgebung fuer verbatim
%Grauhinterlegeter Text
%Automatischer Zeilenumbruch ist aktiviert
\usepackage{listings}
\definecolor{lgray}{gray}{0.80}
%\lstset{backgroundcolor=\color{lgray}, frame=single, basicstyle=\ttfamily, breaklines=true}
\lstnewenvironment{sage}{\lstset{backgroundcolor=\color{lgray},language=Python, emphstyle=\color{red}, frame=single, basicstyle=\ttfamily, breaklines=true,mathescape =true,escapechar=§}}{}


\usepackage{mydef}
%\usepackage{cmap} % you can search in the pdf for umlauts and ligatures
\usepackage{colonequals} %corrects the definition-symbols \colonequals (besides others)
\title{Einführung in Sage}
%
%\subtitle{Disputation} % (optional)

\author{Jochen Schulz}
% - Use the \inst{?} command only if the authors have different
%   affiliation.

\institute{Georg-August Universit\"at G\"ottingen \pgfimage[height=0.5cm]{../figures/unilogo3}}
% - Use the \inst command only if there are several affiliations.
% - Keep it simple, no one is interested in your street address.

\date{\today}

\subject{Sage}
% This is only inserted into the PDF information catalog. Can be left
% out. 

% If you have a file called "university-logo-filename.xxx", where xxx
% is a graphic format that can be processed by latex or pdflatex,
% resp., then you can add a logo as follows:

%\logo{\pgfimage[height=0.5cm]{figures/unilogo3}}


% Delete this, if you do not want the table of contents to pop up at
% the beginning of each subsection:

\AtBeginSection[]
{
  \begin{frame}<beamer>
    \frametitle{Aufbau}
    \tableofcontents[currentsection,currentsubsection]
  \end{frame}
}

\AtBeginSubsection[]
{
  \begin{frame}<beamer>
    \frametitle{Aufbau}
    \tableofcontents[currentsection,currentsubsection]
  \end{frame}
}



%%%%%%%%%%%%%%%%%%%
%Neue Definitionen
%%%%%%%%%%%%%%%%%%%

%Newcommands
\newcommand{\Fun}[1]{\mathcal{#1}}      %Mathcal fuer Funktoren
\newcommand{\field}[1]{\mathbb{#1}}     %Grundkoerper ?? in mathds

\newcommand{\A}{\field{A}}              %Affines A
\newcommand{\C}{\field{C}}              %Complexes C
\newcommand{\Fp}{\field{F}_{\!p}}       %Endlicher Koerper mit p Elementen
\newcommand{\Fq}{\field{F}_{\!q}}       %Endlicher Koerper mit q Elementen
\newcommand{\Ga}{\field{G}_{a}}         %Add Gruppenschema
\newcommand{\K}{\field{K}}              %Generischer Koerper 
\newcommand{\N}{\field{N}}              %Nat Zahlen
\newcommand{\Pj}{\field{P}}             %Projektives P
\newcommand{\R}{\field{R}} 		%Reelle Zahlen
\newcommand{\Q}{\field{Q}}              %Rationale Zahlen  
\newcommand{\Qt}{\field{H}}             %Quaternionen 
\newcommand{\V}{\field{V}}              %Vektorbuendel V
\newcommand{\Z}{\field{Z}}              %Ganze Zahlen

\newcommand{\fdg}{\;|\;}                 %fuer die gilt

%Operatoren
\DeclareMathOperator{\Abb}{Abb}
%\usepackage{sagetex}

\begin{document}
\lstset{basicstyle={\lstbasicfont\footnotesize}}


\maketitle

\begin{frame}{Aufbau}
\tableofcontents
\end{frame}













% \begin{frame}[fragile]{Felder}
% \begin{itemize}
% \item Felder sind spezielle Tabellen. Die Indizes müssen
%            cartesische Produkte von ganzen (!) Zahlen sein (Multiindizes).
% \item Matrizen lassen sich als zweidimensionale Felder interpretieren, d.h. die Indizes sind Paare von ganzen Zahlen.
% \item Es können Felder beliebiger Dimension erzeugt werden. Felder
% sind geeignet
% für Datenmengen fixierter Größe.
% \item Felder haben den Datentyp \isage{DOM_ARRAY}.
% \item Die Matrixklasse hat den Vorteil, dass man dort Matrizen
% addieren und multiplizieren kann.
% \end{itemize}
% \end{frame}
% 
% \begin{frame}[fragile]{Konstruktion und Zugriff I}
% \begin{sagein}
% A:=array(0..1,1..3)
% \end{sagein}
% \begin{sageout}
%   +-                           -+
%   |  ?[0, 1], ?[0, 2], ?[0, 3]  |
%   |                             |
%   |  ?[1, 1], ?[1, 2], ?[1, 3]  |
%   +-                           -+
% \end{sageout}
% Das \isage{?} bedeutet, dass die entsprechenden Werte nicht belegt sind.
% \end{frame}


% \begin{frame}[fragile]{Konstruktion und Zugriff II}
% \begin{sagein}
% A[0,1]:=1: A[1,3]:=HALLO: A
% \end{sagein}
% \begin{sageout}
%   +-                           -+
%   |     1,    ?[0, 2], ?[0, 3]  |
%   |                             |
%   |  ?[1, 1], ?[1, 2],  HALLO   |
%   +-                           -+
% \end{sageout}
% \begin{sagein}
% A[1,3]
% \end{sagein}
% \begin{sageout}
%   HALLO
% \end{sageout}
% \end{frame}


%===============================================
\section{Lineare Abbildungen}
%===============================================

\begin{frame}{Lineare Abbildungen}
Seien $K$-Vektorräume $V$ und $W$ gegeben. Eine Abbildung
\[ F: \ V \rightarrow \ W \]
heißt {\color{red} linear}, falls für $v,w\in V$ und $\lambda \in K$ gilt:
\begin{itemize}
\item [(L1)] $F(v+w)=F(v)+F(w)$
\item [(L2)] $F(\lambda \cdot v)=\lambda \cdot F(v)$
\end{itemize} 
\begin{itemize}
 \item {\color{red} Isomorphismus}: $F$ bijektiv. 
\item {\color{red} Endomorphismus}: $V=W$.
\item {\color{red} Automorphismus}: $V=W$ und $F$ bijektiv.
\end{itemize}

\end{frame}

\begin{frame}{Bemerkungen}
\begin{itemize}
\item Sei $(v_i)_{i\in I}$ eine Basis in $V$ und $(w_i)_{i\in I}$
seien Vektoren in $W$. Dann gibt es genau eine lineare Abbildung $F:V
\rightarrow W$ mit $F(v_i)=w_i$ für alle $i \in I$.
\item {\color{red} Bild} von $F$:   $\mathop{Im}(F) = F(V):=\{ F(v), v \in V \}$.
\item {\color{red} Kern} von F: $\mathop{Ker}(F):=\{v \ \in V \ | \ F(v)=0 \}$
\item Kern und Bild sind Untervektorräume.
\item Dimensionsformel:
\[\dim V = \dim F(V) + \dim Ker(F)\]
\item {\color{red} $\operatorname{Hom}_K(V,W)$}: Die Menge der linearen Abbildungen von $V$ nach $W$. 
Sie ist ein Vektorraum durch punktweise Addition und Skalarmultiplikation.
\end{itemize}
\end{frame}

\begin{frame}{Lineare Abbildungen und Matrizen}
\begin{itemize}
\item Jeder Matrix $A \in K^{m \times n}$ läßt sich durch 
\[
 L_A: K^n \rightarrow K^m,\; 
\begin{pmatrix}
 x_1\\
\vdots\\
x_n
\end{pmatrix}
\longmapsto
A
\begin{pmatrix}
 x_1\\
\vdots\\
x_n
\end{pmatrix}
\]
eine lineare Abbildung zuordnen.
\item Es gilt $\operatorname{dim}(L_A(K^m))=\operatorname{Rang}(A)$.
\end{itemize}
\end{frame}

\begin{frame}{Koordinatenvektor}
Sei $V$ ein $K$-Vektorraum mit Basis $\mathcal{V}=(v_1, \dots
,v_n)$.
\begin{itemize}
\item Die lineare Abbildung 
$\Phi_\mathcal{V}:K^n \ \rightarrow \ V$ mit
\[\Phi_\mathcal{V}(x_1,\dots ,x_n)=x_1v_1+ \dots +x_nv_n\]
ist ein Isomorphismus. Man nennt $\Phi_\mathcal{V}$ ein
{\color{red} Koordinatensystem} in $V$ und $x=(x_1,\dots ,x_n)=\Phi_\mathcal{V}^{-1}(v)$
den {\color{red} Koordinatenvektor} zu $v \in V$.
\item Basiswechselabbildung von $\mathcal{V}$ nach Basis $\mathcal{Z}$:
\[T:= \Phi_\mathcal{Z}^{-1} \circ \Phi_\mathcal{V}\].
 \end{itemize}
\end{frame}

\begin{frame}[fragile]{Isomorphismus}
Seien $K$-Vektorräume $V$ und $W$ mit Basen $\mathcal{V}=(v_1, \dots
,v_n)$ und $\mathcal{W}=(w_1, \dots ,w_m)$ gegeben. 

 Für eine Matrix $A\in K^{m
 \times n}$ wird durch
\begin{eqnarray*}
F(v_1) & := & a_{11}w_1 + \dots +a_{m1} w_m\\
\vdots &    & \vdots     \\
F(v_n) & := & a_{1n} w_1 + \dots + a_{mn}w_m
\end{eqnarray*}
eine lineare Abbildung $F$ definiert. Dies ergibt einen Isomorphismus
\[
L^\mathcal{V}_\mathcal{W}: K^{m \times n} \ \rightarrow \
\mathrm{Hom}_K(V,W), \quad A \ \mapsto \ F.
\] 
\end{frame}

\begin{frame}{Kanonisches Beispiel}
Seien $K^n$ und $K^m$ mit den kanonischen Basen $\mathcal{K}_n$ und
$\mathcal{K}_m$ versehen.
\begin{itemize}
\item Die Abbildungen $\Phi_{\mathcal{K}_n}$ und $\Phi_{\mathcal{K}_m}$
sind Identitäten.
\item  Die Abbildung $L^{\mathcal{K}_n}_{\mathcal{K}_m}$ ist gegeben
durch 
\[ L^{\mathcal{K}_n}_{\mathcal{K}_m} (A)(x)=Ax,\; x \in K^n. \]
\item Die Spaltenvektoren von $A$ sind die Bilder der Einheitsvektoren
unter der Abbildung  $L^{\mathcal{K}_n}_{\mathcal{K}_m}(A)$.
\end{itemize}
\end{frame}

\begin{frame}[fragile]{Kommutierendes Diagramm}
  Seien $K$-Vektorräume $V$ und $W$ mit Basen $\mathcal{V}=(v_1, \dots
,v_n)$ und $\mathcal{W}=(w_1, \dots ,w_m)$ und eine lineare Abbildung
$F$ gegeben. Dann gilt das folgende kommutierende Diagramm:\\
\begin{center}
  \begin{tikzpicture}
  \draw (0,2) node[] (a) {$V$}
   (5,2) node[] (b) {$W$}
   (0,0)  node[] (c) {$K^n$}
   (5,0)  node[] (d) {$K^m$};
  \draw[-latex,thick] (a) -- (b) node[midway, above] {$F$};
 \draw[-latex,thick] (d) -- (b) node[midway,right] {$\Phi_W$};
  \draw[-latex,thick] (c) -- (d) node[midway,below] {$(L_\mathcal{W}^\mathcal{V})^{-1}(F)$};
  \draw[-latex,thick] (c) -- (a) node[midway,left] {$\Phi_V$};
  \end{tikzpicture}
\end{center}
\end{frame}

\begin{frame}[fragile]{Drehung}
Drehung um den Winkel $\alpha$ - Drehmatrix $G$:
\[ G(\alpha):= \left ( \begin{array}{cc}
\cos(\alpha) & -\sin(\alpha) \\
\sin(\alpha) & \cos(\alpha)
\end{array} \right)
\]
\begin{sagein}
var('a,b');A = matrix([[cos(a),-sin(a)],[sin(a),cos(a)]])
A(a=pi/2)*vector([1,1])
\end{sagein}
\begin{sageout}
 (-1, 1)
\end{sageout}

\end{frame}

\begin{frame}[fragile]{Spiegelung}
Spiegelung bezüglich der Ebene 
\[H(a):=\{ x \in \mathbb{R}^3 | x^T a=0  \}, \|a\|=1\]
 durch \[S(a):=I - 2 a a^T.\] 
\begin{sagein}
a = matrix(3,1,[1,2,3])
a = a/norm(a)
I_n = identity_matrix(3)
S = I_n - 2*a*a.transpose()
norm(S*S-I_n)
\end{sagein}
\begin{sageout}
   0.0
\end{sageout}
\end{frame}

\section{Eigenwerte und Eigenvektoren}

\begin{frame}{Eigenwerte und Eigenvektoren}
Sei $A\in K^{n\times n}$. Ein Element $\lambda \in K$ heißt {\color{red} Eigenwert} von
$A$, wenn ein $x \in K^n\smallsetminus \{0 \}$ existiert, 
\[ {\color{red}A x = \lambda x} \] 
gilt. Der Vektor $x \in K^n$ heißt {\color{red} Eigenvektor} zum Eigenwert $\lambda$.
\begin{itemize}
\item Die Eigenwerte sind die Nullstellen des {\color{red} charakteristischen
Polynoms} 
\[p(t):=\det(A-t \ I_n).\] 
\item Es gibt höchstens $n$ Eigenwerte.
\end{itemize}
\end{frame}

\begin{frame}{Bemerkungen}
\begin{itemize}
\item Eigenvektoren zu paarweise verschiedenen Eigenwerten  sind linear
unabhängig.
\item Gibt es eine Basis aus Eigenvektoren, so ist $A$
{\color{red} diagonalisierbar}, d.h. man kann die Abbildung $L_A$ bei
geeigneter Basiswahl durch eine Diagonalmatrix repräsentieren.
\item Jeder Endomorphismus eines komplexen Vektorraums läßt sich durch
eine Matrix in \alert{Jordanscher Normalform} darstellen.
\end{itemize}
\end{frame}

\begin{frame}[fragile]{Eigenwerte in Sage}
\begin{itemize}
\item Bestimmung von Eigenwerten
\begin{sagein}
_=var('al');A = matrix([[cos(al), sin(al)],[sin(al),-cos(al)]])
[ m.full_simplify() for m in  A.eigenvalues()]
\end{sagein}
\begin{sageout}
  [-1, 1]
\end{sageout}
\item Bestimmung von Eigenvektoren
\begin{sagein}
A.eigenvectors_right()
\end{sagein} 
\end{itemize}
\end{frame}

\begin{frame}[fragile]{Eigenwerte in Sage}
\begin{itemize}
\item Bestimmung des charakteristischen Polynoms
\begin{sagein}
E = identity_matrix(2)
p = (A-x*E).det(); p
\end{sagein}
\begin{sageout}
(x - cos(al))*(x + cos(al)) - sin(al)^2
\end{sageout}
Alternative (Vorsicht: gleich bis auf Vorzeichen!):
\begin{sagein}
A.charpoly() 
\end{sagein}

\begin{sagein}
[m.full_simplify() for m in solve(p==0,x)]
\end{sagein}
\begin{sageout}
  [x == -1, x == 1]
\end{sageout}
\end{itemize}
\end{frame}

\begin{frame}{Lineare Gleichungssysteme (LGS)}
Sei $A\in K^{m \times n}$ und $b \in K^m$. Gesucht ist die Menge der
Lösungen (Lösungsraum): 
\[ \{ x \in K^n \;|\; A x = b\} \]
\begin{itemize}
\item Ist $b=0$, so spricht man von einem {\color{red} homogenen
System}. Ansonsten spricht man von einem {\color{red} inhomogenen System}.
\item Der Lösungsraum $W$ des homogenen Systems bildet einen
Untervektorraum des $K^n$. Die Dimension ist 
\[ \dim (W)=n - \rang(A). \] 
\end{itemize}
\end{frame}

\begin{frame}{Struktur des Lösungsraums}
\begin{itemize}
\item {\color{red} affiner Unterraum}  $X \subset K^n$: wenn ein
Unterraum $W$ von $K^n$ und ein $v \in K^n$ existiert, so dass 
\[X=v+W\]
%Der Unterraum $W$ ist durch $X$ eindeutig bestimmt, $v$ kann jeder Vektor aus $X$ sein.
\item Die Lösungen des inhomogenen Systems ($b \neq 0$) bilden einen affinen
Unterraum des $K^n$.   

\item Ist $W$ der Lösungsraum des homogenen Systems und $v \in K^n$
eine beliebige Lösung von $Ax=b$, dann ist der Lösungsraum $X$ von $Ax=b$
gegeben durch  $X=v+W$.
\item Zwei Lösungen des inhomogenen Systems unterscheiden sich durch
eine Lösung des homogenen Systems.
\end{itemize}
\end{frame}

\begin{frame}{Lösbarkeit}
\begin{itemize}
\item Das inhomogene System ist genau dann für alle $b$
lösbar, wenn $\rang(A)=m$ gilt.
\item Das homogene bzw. das inhomogene System besitzt höchstens eine
Lösung, genau dann wenn $\rang(A)=n$ gilt.
\item Der Lösungsraum des inhomogenen Systems ist genau dann nicht
leer, wenn $\rang(A)=\rang(A, b)$ gilt. %A,b : erweiterte Koeffizientenmatrix
\item Praktisch kann ein LGS mit dem {\color{red} Gausschen
Eliminationsverfahren} gelöst werden.
\end{itemize}
\end{frame}

\begin{frame}[fragile]{LGS in Sage}
Berechnung der Lösungen von $Ax=b$: 
\begin{sagein}
A = matrix([[1,2,3],[4,5,6],[7,8,9]])
b1 = vector([0,0,0])
b2 = vector([1,0,0])
b3 = A*b2
print A\b1
print A\b3
\end{sagein}
\begin{sageout}
(0, 0, 0)
(1, 0, 0) 
\end{sageout}

%linalg::gaussElim(A)
\end{frame}
\end{document}
