\documentclass[a4paper,12pt,DIV15]{scrartcl}
\usepackage[xetex,bookmarks=true,pdfstartview=FitH,bookmarksopen=true,
    colorlinks,citecolor=Blue,linkcolor=DarkBlue,urlcolor=Green,
    pagebackref=true,plainpages=false,pdfpagelabels=true,unicode=true,
    breaklinks=true,naturalnames=false,setpagesize=true,a4paper=true,hyperindex]{hyperref}
\usepackage[svgnames,hyperref]{xcolor} %color definition
\usepackage{tikz}

%\usepackage{fontspec,xunicode}
% %\usepackage{polyglossia}
%\setdefaultlanguage[spelling=new, latesthyphen=true]{german}
%\setsansfont{DejaVu Sans}
%\setsansfont{Verdana}
%\setsansfont{Arial}
%\setromanfont[Mapping=tex-text]{Linux Libertine}
%\setsansfont[Mapping=tex-text]{Myriad Pro}
%\setmonofont[Mapping=tex-text]{Courier New}

%\setsansfont{Linux Biolinum}

\usepackage[ngerman]{babel}
\selectlanguage{ngerman}

%
% math/symbols
%
\usepackage{amssymb}
\usepackage{amsthm}
% \usepackage{latexsym}
\usepackage{amsmath}
%\usepackage{amsxtra} %Weitere Extrasymbole
%\usepackage{empheq} %Gleichungen hervorheben
%\usepackage{bm}
 %\bm{A} Boldface im Mathemodus
\usepackage{fontspec,xunicode,xltxtra}

\usepackage{multimedia}
%\usepackage{tikz}

\usepackage{cellspace}
\setlength{\cellspacetoplimit}{2pt}
\setlength{\cellspacebottomlimit}{2pt}

%%%%%%%%%%%%%%%%%% Fuer Frames [fragile]-Option verwenden!
%Programm-Listing
%%%%%%%%%%%%%%%%%%
%Listingsumgebung fuer verbatim
%Grauhinterlegeter Text
%Automatischer Zeilenumbruch ist aktiviert
%\usepackage{listings}
\usepackage[framed]{mcode}
%\usepackage{mcode}
% This command allows you to typeset syntax highlighted Matlab
% code ``inline''.
% mcode fuer matlab

\definecolor{lgray}{gray}{0.80}
\definecolor{gray}{gray}{0.3}
\definecolor{darkgreen}{rgb}{0,0.4,0}
\definecolor{darkblue}{rgb}{0,0,0.8}
\definecolor{key}{rgb}{0,0.5,0} 
\definecolor{NU0}{RGB}{68,85,136} % #458
\definecolor{KW3}{RGB}{85,68,136}
\definecolor{KW4}{RGB}{153,0,0}
\definecolor{dred}{RGB}{221,17,68} % #d14
\definecolor{BG}{RGB}{240,240,240}
%\lstset{backgroundcolor=\color{lgray}, frame=single, basicstyle=\ttfamily, breaklines=true}
%\lstnewenvironment{sage}{\lstset{,language=python, keywordstyle=color{blue},    commentstyle=color{green}, emphstyle=\color{red}, %frame=single, stringstyle=\color{red}, basicstyle=\ttfamily, ,mathescape =true,escapechar=§}}{}

\lstdefinelanguage{fooHaskell} {%
  basicstyle=\footnotesize\ttfamily,%
  commentstyle=\slshape\color{gray},%
  keywordstyle=\bfseries,%\color{KW4},
  breaklines=true,
  sensitive=true,
  xleftmargin=1pc,
  emph={[1]
    FilePath,IOError,abs,acos,acosh,all,and,any,appendFile,approxRational,asTypeOf,asin,
    asinh,atan,atan2,atanh,basicIORun,break,catch,ceiling,chr,compare,concat,concatMap,
    const,cos,cosh,curry,cycle,decodeFloat,denominator,digitToInt,div,divMod,drop,
    dropWhile,either,elem,encodeFloat,enumFrom,enumFromThen,enumFromThenTo,enumFromTo,
    error,even,exp,exponent,fail,filter,flip,floatDigits,floatRadix,floatRange,floor,
    fmap,foldl,foldl1,foldr,foldr1,fromDouble,fromEnum,fromInt,fromInteger,fromIntegral,
    fromRational,fst,gcd,getChar,getContents,getLine,head,id,inRange,index,init,intToDigit,
    interact,ioError,isAlpha,isAlphaNum,isAscii,isControl,isDenormalized,isDigit,isHexDigit,
    isIEEE,isInfinite,isLower,isNaN,isNegativeZero,isOctDigit,isPrint,isSpace,isUpper,iterate,
    last,lcm,length,lex,lexDigits,lexLitChar,lines,log,logBase,lookup,map,mapM,mapM_,max,
    maxBound,maximum,maybe,min,minBound,minimum,mod,negate,not,notElem,null,numerator,odd,
    or,ord,otherwise,pi,pred,primExitWith,print,product,properFraction,putChar,putStr,putStrLn,quot,
    quotRem,range,rangeSize,read,readDec,readFile,readFloat,readHex,readIO,readInt,readList,readLitChar,
    readLn,readOct,readParen,readSigned,reads,readsPrec,realToFrac,recip,rem,repeat,replicate,return,
    reverse,round,scaleFloat,scanl,scanl1,scanr,scanr1,seq,sequence,sequence_,show,showChar,showInt,
    showList,showLitChar,showParen,showSigned,showString,shows,showsPrec,significand,signum,sin,
    sinh,snd,span,splitAt,sqrt,subtract,succ,sum,tail,take,takeWhile,tan,tanh,threadToIOResult,toEnum,
    toInt,toInteger,toLower,toRational,toUpper,truncate,uncurry,undefined,unlines,until,unwords,unzip,
    unzip3,userError,words,writeFile,zip,zip3,zipWith,zipWith3,listArray,doParse
  },%
  emphstyle={[1]\color{NU0}},%
  emph={[2]
    Bool,Char,Double,Either,Float,IO,Integer,Int,Maybe,Ordering,Rational,Ratio,ReadS,Show,ShowS,String,
    Word8,InPacket
  },%
  emphstyle={[2]\bfseries\color{KW4}},%
  emph={[3]
    case,class,data,deriving,do,else,if,import,in,infixl,infixr,instance,let,
    module,of,primitive,then,type,where
  },
  emphstyle={[3]\color{darkblue}},
  emph={[4]
    quot,rem,div,mod,elem,notElem,seq
  },
  emphstyle={[4]\color{NU0}\bfseries},
  emph={[5]
    EQ,False,GT,Just,LT,Left,Nothing,Right,True,Show,Eq,Ord,Num
  },
  emphstyle={[5]\color{KW4}\bfseries},
  morestring=[b]",%
  morestring=[b]',%
  stringstyle=\color{darkgreen},%
  showstringspaces=false
}
\lstnewenvironment{hs}
{\lstset{language=fooHaskell,backgroundcolor=\color{BG}}}
{\smallskip}
\newcommand{\ihs}[1]{\lstset{language=fooHaskell,basicstyle=\color[gray]{0.6}}\lstinline|#1|}


\lstdefinelanguage{fooMatlab} {%
backgroundcolor=\color[gray]{0.9},
breaklines=true,
basicstyle=\ttfamily\small,
%otherkeywords={ =},
%keywordstyle=\color{blue},
%stringstyle=\color{darkgreen},
showstringspaces=false,
%emph={for, while, if, elif, else, not, and, or, printf, break, continue, return, end, function},
%emphstyle=\color{blue},
%emph={[2]True, False, None, self, NaN, NULL},
%emphstyle=[2]\color{key},
%emph={[3]from, import, as},
%emphstyle=[3]\color{blue},
%upquote=true,
%morecomment=[s]{"""}{"""},
%commentstyle=\color{gray}\slshape,
%framexleftmargin=1mm, framextopmargin=1mm, 
%title=\tiny matlab,
frame=single,
%mathescape =true,
%escapechar=§
}
\newcommand{\imatlab}[1]{\lstset{language=fooMatlab,basicstyle=\color[gray]{0.6}}\lstinline|#1|}
\lstnewenvironment{matlab}[1][]{\lstset{language=fooMatlab,xleftmargin=0.2cm,frame=none,backgroundcolor=\color{white},basicstyle=\color{darkblue}\ttfamily\small,#1}}{} 
\lstnewenvironment{matlabin}[1][]{\lstset{language=fooMatlab,#1}}{} 
\newcommand{\matinput}[1]{\lstset{language=fooMatlab}\lstinputlisting{#1}}

\lstdefinelanguage{fooPython} {%
language=python,
backgroundcolor=\color[gray]{0.7},
breaklines=true,
basicstyle=\ttfamily\small,
%otherkeywords={ =},
keywordstyle=\color{blue},
stringstyle=\color{darkgreen},
morestring=[b]",%
morestring=[b]',%
showstringspaces=false,
emph={class, pass, in, for, while, if, is, elif, else, not, and, or,
def, print, exec, break, continue, return, import, from, lambda, null},
emphstyle=\color{blue},
emph={[2]True, False, None, self},
emphstyle=[2]\color{key},
emph={[3]from, import, as},
emphstyle=[3]\color{blue},
upquote=true,
morecomment=[s]{"""}{"""},
comment=[l]{\#},
commentstyle=\color{gray},
%framexleftmargin=1mm, framextopmargin=1mm, 
%title=\tiny python,
%caption=python,
frame=single
%frameround=tttt,
%mathescape =true,
%escapechar=§
}

\newcommand{\pyinput}[1]{\lstset{language=fooPython}\lstinputlisting{#1}}
\newcommand{\isage}[1]{{\lstset{language=fooPython,basicstyle=\color[gray]{0.3}}\lstinline|#1|}}

\lstnewenvironment{pyout}[1][]{\lstset{language=fooPython,xleftmargin=0.2cm,frame=none,backgroundcolor=\color{white},basicstyle=\color{darkblue}\ttfamily\small,#1}}{}
\lstnewenvironment{pyin}[1][]{\lstset{language=fooPython,#1}}{}
\lstnewenvironment{sageout}[1][]{\lstset{language=fooPython,xleftmargin=0.2cm,frame=none,backgroundcolor=\color{white},basicstyle=\color{darkblue}\ttfamily\small,#1}}{}
\lstnewenvironment{sagein}[1][]{\lstset{language=fooPython,#1}}{}

%\usepackage{caption}
%\DeclareCaptionFont{white}{ \color{white} }
%\DeclareCaptionFormat{listing}{
%  \colorbox[cmyk]{0.43, 0.35, 0.35,0.01 }{
%      \parbox{\textwidth}{\hspace{15pt}#1#2#3}
%        }
%        }
%        \captionsetup[lstlisting]{ format=listing, labelfont=white, textfont=white, singlelinecheck=false, margin=0pt, font={bf,footnotesize} }


\usepackage{mydef}
%\usepackage{cmap} % you can search in the pdf for umlauts and ligatures
\usepackage{colonequals} %corrects the definition-symbols \colonequals (besides others)

\usepackage{ifthen}

%%%%%%%%%%%%%%%%%%%
%Neue Definitionen
%%%%%%%%%%%%%%%%%%%

%Newcommands
\newcommand{\Fun}[1]{\mathcal{#1}}      %Mathcal fuer Funktoren
\newcommand{\field}[1]{\mathbb{#1}}     %Grundkoerper ?? in mathds

\newcommand{\A}{\field{A}}              %Affines A
\newcommand{\Fp}{\field{F}_{\!p}}       %Endlicher Koerper mit p Elementen
\newcommand{\Fq}{\field{F}_{\!q}}       %Endlicher Koerper mit q Elementen
\newcommand{\Ga}{\field{G}_{a}}         %Add Gruppenschema
\newcommand{\K}{\field{K}}              %Generischer Koerper 
\newcommand{\N}{\field{N}}              %Nat Zahlen
\newcommand{\Pj}{\field{P}}             %Projektives P
\newcommand{\R}{\field{R}} 		%Reelle Zahlen
\newcommand{\Q}{\field{Q}}              %Rationale Zahlen  
\newcommand{\Qt}{\field{H}}             %Quaternionen 
\newcommand{\V}{\field{V}}              %Vektorbuendel V
\newcommand{\Z}{\field{Z}}              %Ganze Zahlen
\DeclareMathOperator{\Real}{Re}

\newcommand{\fdg}{\;|\;}                 %fuer die gilt

%Operatoren
\DeclareMathOperator{\Abb}{Abb}
%\usepackage{sagetex}


%
% Aufgaben
%
\parindent0cm % Abs�tze nicht einr�cken 
% Definieren einer neuen Farbe
\definecolor{light-gray}{gray}{.9}

\newcounter{zaehler}     % neuen Z�hler einf�hren
\newenvironment{aufgn}[2][0]
%---- Header
{\begin{samepage}%
%\colorbox{light-gray}{%                         % Box in gray
% \makebox[\textwidth]{%                           % Box in linewidth
%\textbf{Aufgabe \arabic{zaehler} } }\hspace{-\textwidth}\makebox[\textwidth]{\hfill #1 Punkte} }\\[0.05cm]       % Header
\dotfill\\
{\large\textbf{Aufgabe \arabic{zaehler} \ifthenelse{ \equal{#2}{} }{}{: \emph{ #2 } }}\ifthenelse{-1=#1}{(testierbar)}{}\ifthenelse{0=#1 \or -1=#1}{}{\hfill #1 Punkte} }\\[0.4cm]
%{\large\textbf{Exercise \arabic{zaehler}  #2 }\ifthenelse{-1=#1}{(testierbar)}{}\ifthenelse{0=#1 \or -1=#1}{}{\hfill #1 Punkte} }\\[0.4cm]
\begin{minipage}{\textwidth}%
}%
%-----  foot
{\end{minipage}\nopagebreak%\begin{minipage}{1cm} \end{minipage}
%\\ 
%\begin{minipage}{0.1cm} \end{minipage} 
%\hrulefill \begin{minipage}{1cm} \end{minipage}\\[1cm]  
\stepcounter{zaehler}                           % increase counter
\end{samepage}%
\\%
\bigskip%
}


\newenvironment{aufg}[1][0]
%---- Header
{\begin{samepage}%
\refstepcounter{zaehler}% increase counter
%\colorbox{light-gray}{%                         % Box in gray
% \makebox[\textwidth]{%                           % Box in linewidth
%\textbf{Aufgabe \arabic{zaehler} } }\hspace{-\textwidth}\makebox[\textwidth]{\hfill #1 Punkte} }\\[0.05cm]       % Header
\dotfill\\
{\large\textbf{Aufgabe \arabic{zaehler} }\ifthenelse{-1=#1}{(testierbar)}{}\ifthenelse{0=#1 \or -1=#1}{}{\hfill #1 Punkte} }\\[0.4cm]
\begin{minipage}{\textwidth}%
}%
%-----  foot
{\end{minipage}\nopagebreak%\begin{minipage}{1cm} \end{minipage}
%\\ 
%\begin{minipage}{0.1cm} \end{minipage} 
%\hrulefill \begin{minipage}{1cm} \end{minipage}\\[1cm]  
\end{samepage}%
\\%
\bigskip%
}

\begin{document}
\begin{center}
    \textbf{\LARGE Einführung in Sage}\\
    {\large Einheit 05}\\
    {\large Zusammenfassung: lineare Abbildungen}
\end{center}

\section{Lineare Abbildungen}
\begin{defn}[Lineare Abbildung]
Seien $K$-Vektorräume $V$ und $W$ gegeben. Eine Abbildung
\[ F: \ V \rightarrow \ W \]
heißt {\color{red} linear}, falls für $v,w\in V$ und $\lambda \in K$ gilt:
\begin{itemize}
\item [(L1)] $F(v+w)=F(v)+F(w)$
\item [(L2)] $F(\lambda \cdot v)=\lambda \cdot F(v)$
\end{itemize} 
\begin{itemize}
 \item {\color{red} Isomorphismus}: $F$ bijektiv. 
\item {\color{red} Endomorphismus}: $V=W$.
\item {\color{red} Automorphismus}: $V=W$ und $F$ bijektiv.
\end{itemize}
\end{defn}

\paragraph{Bemerkungen:}
\begin{itemize}
\item Sei $(v_i)_{i\in I}$ eine Basis in $V$ und $(w_i)_{i\in I}$
seien Vektoren in $W$. Dann gibt es genau eine lineare Abbildung $F:V
\rightarrow W$ mit $F(v_i)=w_i$ für alle $i \in I$.
\item {\color{red} Bild} von $F$:   $\mathop{Im}(F) = F(V):=\{ F(v), v \in V \}$.
\item {\color{red} Kern} von F: $\mathop{Ker}(F):=\{v \ \in V \ | \ F(v)=0 \}$
\item Kern und Bild sind Untervektorräume.
\item Dimensionsformel:
\[\dim V = \dim F(V) + \dim Ker(F)\]
\item {\color{red} $\operatorname{Hom}_K(V,W)$}: Die Menge der linearen Abbildungen von $V$ nach $W$. 
Sie ist ein Vektorraum durch punktweise Addition und Skalarmultiplikation.
\end{itemize}

\paragraph{Lineare Abbildungen und Matrizen}
\begin{itemize}
\item Jeder Matrix $A \in K^{m \times n}$ läßt sich durch 
\[
 L_A: K^n \rightarrow K^m,\; 
\begin{pmatrix}
 x_1\\
\vdots\\
x_n
\end{pmatrix}
\longmapsto
A
\begin{pmatrix}
 x_1\\
\vdots\\
x_n
\end{pmatrix}
\]
eine lineare Abbildung zuordnen.
\item Es gilt $\operatorname{dim}(L_A(K^m))=\operatorname{Rang}(A)$.
\end{itemize}

\begin{defn}[Koordinatenvektor]
Sei $V$ ein $K$-Vektorraum mit Basis $\mathcal{V}=(v_1, \dots
,v_n)$.
\begin{itemize}
\item Die lineare Abbildung 
$\Phi_\mathcal{V}:K^n \ \rightarrow \ V$ mit
\[\Phi_\mathcal{V}(x_1,\dots ,x_n)=x_1v_1+ \dots +x_nv_n\]
ist ein Isomorphismus. Man nennt $\Phi_\mathcal{V}$ ein
{\color{red} Koordinatensystem} in $V$ und $x=(x_1,\dots ,x_n)=\Phi_\mathcal{V}^{-1}(v)$
den {\color{red} Koordinatenvektor} zu $v \in V$.
\item Basiswechselabbildung von $\mathcal{V}$ nach Basis $\mathcal{Z}$:
\[T:= \Phi_\mathcal{Z}^{-1} \circ \Phi_\mathcal{V}\].
 \end{itemize}
\end{defn}

\begin{defn}[Isomorphismus]
Seien $K$-Vektorräume $V$ und $W$ mit Basen $\mathcal{V}=(v_1, \dots
,v_n)$ und $\mathcal{W}=(w_1, \dots ,w_m)$ gegeben. 

 Für eine Matrix $A\in K^{m
 \times n}$ wird durch
\begin{eqnarray*}
F(v_1) & := & a_{11}w_1 + \dots +a_{m1} w_m\\
\vdots &    & \vdots     \\
F(v_n) & := & a_{1n} w_1 + \dots + a_{mn}w_m
\end{eqnarray*}
eine lineare Abbildung $F$ definiert. Dies ergibt einen Isomorphismus
\[
L^\mathcal{V}_\mathcal{W}: K^{m \times n} \ \rightarrow \
\mathrm{Hom}_K(V,W), \quad A \ \mapsto \ F.
\] 
\end{defn}

\paragraph{Kanonisches Beispiel:}
Seien $K^n$ und $K^m$ mit den kanonischen Basen $\mathcal{K}_n$ und
$\mathcal{K}_m$ versehen.
\begin{itemize}
\item Die Abbildungen $\Phi_{\mathcal{K}_n}$ und $\Phi_{\mathcal{K}_m}$
sind Identitäten.
\item  Die Abbildung $L^{\mathcal{K}_n}_{\mathcal{K}_m}$ ist gegeben
durch 
\[ L^{\mathcal{K}_n}_{\mathcal{K}_m} (A)(x)=Ax,\; x \in K^n. \]
\item Die Spaltenvektoren von $A$ sind die Bilder der Einheitsvektoren
unter der Abbildung  $L^{\mathcal{K}_n}_{\mathcal{K}_m}(A)$.
\end{itemize}

\begin{defn}[Kommutierendes Diagramm]
  Seien $K$-Vektorräume $V$ und $W$ mit Basen $\mathcal{V}=(v_1, \dots
,v_n)$ und $\mathcal{W}=(w_1, \dots ,w_m)$ und eine lineare Abbildung
$F$ gegeben. Dann gilt das folgende kommutierende Diagramm:\\
\begin{center}
  \begin{tikzpicture}
  \draw (0,2) node[] (a) {$V$}
   (5,2) node[] (b) {$W$}
   (0,0)  node[] (c) {$K^n$}
   (5,0)  node[] (d) {$K^m$};
  \draw[-latex,thick] (a) -- (b) node[midway, above] {$F$};
 \draw[-latex,thick] (d) -- (b) node[midway,right] {$\Phi_W$};
  \draw[-latex,thick] (c) -- (d) node[midway,below] {$(L_\mathcal{W}^\mathcal{V})^{-1}(F)$};
  \draw[-latex,thick] (c) -- (a) node[midway,left] {$\Phi_V$};
  \end{tikzpicture}
\end{center}
\end{defn}

\section{Eigenwerte und Eigenvektoren}
\begin{defn}[Eigenwert und Eigenvektor]
Sei $A\in K^{n\times n}$. Ein Element $\lambda \in K$ heißt {\color{red} Eigenwert} von
$A$, wenn ein $x \in K^n\smallsetminus \{0 \}$ existiert, 
\[ {\color{red}A x = \lambda x} \] 
gilt. Der Vektor $x \in K^n$ heißt {\color{red} Eigenvektor} zum Eigenwert $\lambda$.
\begin{itemize}
\item Die Eigenwerte sind die Nullstellen des {\color{red} charakteristischen
Polynoms} 
\[p(t):=\det(A-t \ I_n).\] 
\item Es gibt höchstens $n$ Eigenwerte.
\end{itemize}
\end{defn}

\paragraph{Bemerkungen}
\begin{itemize}
\item Eigenvektoren zu paarweise verschiedenen Eigenwerten  sind linear
unabhängig.
\item Gibt es eine Basis aus Eigenvektoren, so ist $A$
{\color{red} diagonalisierbar}, d.h. man kann die Abbildung $L_A$ bei
geeigneter Basiswahl durch eine Diagonalmatrix repräsentieren.
\item Jeder Endomorphismus eines komplexen Vektorraums läßt sich durch
eine Matrix in \textsl{Jordanscher Normalform} darstellen.
\end{itemize}


\begin{defn}[Lineare Gleichungssysteme (LGS)]
Sei $A\in K^{m \times n}$ und $b \in K^m$. Gesucht ist die Menge der
Lösungen (Lösungsraum): 
\[ \{ x \in K^n \;|\; A x = b\} \]
\begin{itemize}
\item Ist $b=0$, so spricht man von einem {\color{red} homogenen
System}. Ansonsten spricht man von einem {\color{red} inhomogenen System}.
\item Der Lösungsraum $W$ des homogenen Systems bildet einen
Untervektorraum des $K^n$. Die Dimension ist 
\[ \dim (W)=n - \rang(A). \] 
\end{itemize}
\end{defn}

\paragraph{Struktur des Lösungsraums:}
\begin{itemize}
\item {\color{red} affiner Unterraum}  $X \subset K^n$: wenn ein
Unterraum $W$ von $K^n$ und ein $v \in K^n$ existiert, so dass 
\[X=v+W\]
%Der Unterraum $W$ ist durch $X$ eindeutig bestimmt, $v$ kann jeder Vektor aus $X$ sein.
\item Die Lösungen des inhomogenen Systems ($b \neq 0$) bilden einen affinen
Unterraum des $K^n$.   

\item Ist $W$ der Lösungsraum des homogenen Systems und $v \in K^n$
eine beliebige Lösung von $Ax=b$, dann ist der Lösungsraum $X$ von $Ax=b$
gegeben durch  $X=v+W$.
\item Zwei Lösungen des inhomogenen Systems unterscheiden sich durch
eine Lösung des homogenen Systems.
\end{itemize}

\begin{thm}[Lösbarkeit]
\begin{itemize}
\item Das inhomogene System ist genau dann für alle $b$
lösbar, wenn $\rang(A)=m$ gilt.
\item Das homogene bzw. das inhomogene System besitzt höchstens eine
Lösung, genau dann wenn $\rang(A)=n$ gilt.
\item Der Lösungsraum des inhomogenen Systems ist genau dann nicht
leer, wenn $\rang(A)=\rang(A, b)$ gilt. %A,b : erweiterte Koeffizientenmatrix
\item Praktisch kann ein LGS mit dem {\color{red} Gausschen
Eliminationsverfahren} gelöst werden.
\end{itemize}
\end{thm}
\end{document}
