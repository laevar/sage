\documentclass[a4paper,10pt,DIV15]{scrartcl}
\usepackage[psamsfonts]{amssymb}
\usepackage{amsmath}
\usepackage[svgnames]{xcolor} %color definitions

\usepackage{fontspec,xunicode,xltxtra}
%\usepackage{fontspec,xunicode}
%\usepackage{polyglossia}
%\setdefaultlanguage[spelling=new, latesthyphen=true]{german}
%\setsansfont{DejaVu Sans}
%\setsansfont{Verdana}
%\setsansfont{Arial}
%\setromanfont[Mapping=tex-text]{Linux Libertine}
%\setsansfont[Mapping=tex-text]{Myriad Pro}
%\setmonofont[Mapping=tex-text]{Courier New}

%\setsansfont{Linux Biolinum}

\usepackage[ngerman]{babel}
\selectlanguage{ngerman}

%
% math/symbols
%
\usepackage{amssymb}
\usepackage{amsthm}
% \usepackage{latexsym}
\usepackage{amsmath}
%\usepackage{amsxtra} %Weitere Extrasymbole
%\usepackage{empheq} %Gleichungen hervorheben
%\usepackage{bm}
 %\bm{A} Boldface im Mathemodus

\usepackage{multimedia}
%\usepackage{tikz}

\usepackage{cellspace}
\setlength{\cellspacetoplimit}{2pt}
\setlength{\cellspacebottomlimit}{2pt}

%%%%%%%%%%%%%%%%%% Fuer Frames [fragile]-Option verwenden!
%Programm-Listing
%%%%%%%%%%%%%%%%%%
%Listingsumgebung fuer verbatim
%Grauhinterlegeter Text
%Automatischer Zeilenumbruch ist aktiviert
\usepackage{listings}
% This command allows you to typeset syntax highlighted Matlab
% code ``inline''.
\newcommand{\isage}[1]{\lstinline|#1|}

\definecolor{lgray}{gray}{0.80}
\definecolor{gray}{gray}{0.3}
\definecolor{darkgreen}{rgb}{0,0.4,0}
\definecolor{darkblue}{rgb}{0,0,0.8}
\definecolor{key}{rgb}{0,0.5,0} 
%\lstset{backgroundcolor=\color{lgray}, frame=single, basicstyle=\ttfamily, breaklines=true}
\lstnewenvironment{sage}[1][]{\lstset{xleftmargin=0.2cm,frame=none,backgroundcolor=\color{white},basicstyle=\color{darkblue}\ttfamily\small,#1}}{} 
\lstnewenvironment{sagein}[1][]{\lstset{#1}}{} 
%\lstnewenvironment{sage}{\lstset{,language=python, keywordstyle=color{blue},    commentstyle=color{green}, emphstyle=\color{red}, %frame=single, stringstyle=\color{red}, basicstyle=\ttfamily, ,mathescape =true,escapechar=§}}{}

\lstset{
language=python,
backgroundcolor=\color{lgray},
breaklines=true,
basicstyle=\ttfamily\small,
%otherkeywords={ =},
keywordstyle=\color{blue},
stringstyle=\color{darkgreen},
showstringspaces=false,
emph={class, pass, in, for, while, if, is, elif, else, not, and, or,
def, print, exec, break, continue, return},
emphstyle=\color{blue},
emph={[2]True, False, None, self},
emphstyle=[2]\color{key},
emph={[3]from, import, as},
emphstyle=[3]\color{blue},
upquote=true,
morecomment=[s]{"""}{"""},
commentstyle=\color{gray}\slshape,
%framexleftmargin=1mm, framextopmargin=1mm, 
frame=single,
mathescape =true,
escapechar=§
}


\usepackage{mydef}
%\usepackage{cmap} % you can search in the pdf for umlauts and ligatures
\usepackage{colonequals} %corrects the definition-symbols \colonequals (besides others)
\usepackage{ifthen}
%%%%%%%%%%%%%%%%%%%
%Neue Definitionen
%%%%%%%%%%%%%%%%%%%

%Newcommands
\newcommand{\Fun}[1]{\mathcal{#1}}      %Mathcal fuer Funktoren
\newcommand{\field}[1]{\mathbb{#1}}     %Grundkoerper ?? in mathds

\newcommand{\A}{\field{A}}              %Affines A
\newcommand{\C}{\field{C}}              %Complexes C
\newcommand{\Fp}{\field{F}_{\!p}}       %Endlicher Koerper mit p Elementen
\newcommand{\Fq}{\field{F}_{\!q}}       %Endlicher Koerper mit q Elementen
\newcommand{\Ga}{\field{G}_{a}}         %Add Gruppenschema
\newcommand{\K}{\field{K}}              %Generischer Koerper 
\newcommand{\N}{\field{N}}              %Nat Zahlen
\newcommand{\Pj}{\field{P}}             %Projektives P
\newcommand{\R}{\field{R}} 		%Reelle Zahlen
\newcommand{\Q}{\field{Q}}              %Rationale Zahlen  
\newcommand{\Qt}{\field{H}}             %Quaternionen 
\newcommand{\V}{\field{V}}              %Vektorbuendel V
\newcommand{\Z}{\field{Z}}              %Ganze Zahlen
\DeclareMathOperator{\Real}{Re}

\newcommand{\fdg}{\;|\;}                 %fuer die gilt

%Operatoren
\DeclareMathOperator{\Abb}{Abb}
%\usepackage{sagetex}

%
% Aufgaben
%
\parindent0cm % Abs�tze nicht einr�cken 
% Definieren einer neuen Farbe
\definecolor{light-gray}{gray}{.9}

\newcounter{zaehler}     % neuen Z�hler einf�hren
\stepcounter{zaehler}    % Z�hler einen hochz�hlen

\newenvironment{aufg}[1][0]
%---- Header
{\begin{samepage}%
%\colorbox{light-gray}{%                         % Box in gray
% \makebox[\textwidth]{%                           % Box in linewidth
%\textbf{Aufgabe \arabic{zaehler} } }\hspace{-\textwidth}\makebox[\textwidth]{\hfill #1 Punkte} }\\[0.05cm]       % Header
\dotfill\\
{\large\textbf{Aufgabe \arabic{zaehler} }\ifthenelse{0=#1}{}{\hfill #1 Punkte} }\\[0.4cm]
\begin{minipage}{\textwidth}
}
%-----  foot
{\end{minipage} \nopagebreak %\begin{minipage}{1cm} \end{minipage}
%\\ 
%\begin{minipage}{0.1cm} \end{minipage} 
%\hrulefill \begin{minipage}{1cm} \end{minipage}\\[1cm]  
\stepcounter{zaehler}                           % increase counter
\end{samepage}%
\\%
\bigskip%
}


%\usepackage{tikz}
%\usetikzlibrary{shadows}
%\usetikzlibrary{fit}
%\usetikzlibrary{shapes}
%\usetikzlibrary{backgrounds}

\parindent0cm % Abs�tze nicht einr�cken 

% Definieren einer neuen Farbe
\definecolor{light-gray}{gray}{.9}

\newcounter{zaehler}     % neuen Z�hler einf�hren
\stepcounter{zaehler}    % Z�hler einen hochz�hlen

\newenvironment{aufg}%
%---- Header
{\begin{samepage}
\colorbox{light-gray}{                         % Box in gray
 \makebox[\textwidth]{                           % Box in linewidth
\textbf{Aufgabe} \arabic{zaehler} :}}\\[0.1cm]       % Header
%\begin{minipage}{0.5cm} \end{minipage}    % Insert 0.5cm
\begin{minipage}{\textwidth}}
%-----  foot
{\end{minipage} \nopagebreak %\begin{minipage}{1cm} \end{minipage}
\\[0.1cm] 
%\begin{minipage}{0.1cm} \end{minipage} 
%\hrulefill \begin{minipage}{1cm} \end{minipage}\\[1cm]  
\stepcounter{zaehler}                           % increase counter
 \end{samepage}%
}

%-------------------------------------------------------------------------------
\begin{document}
%-------------------------------------------------------------------------------

%--------------------------------------------------- Header
\begin{center}
\textbf{\LARGE Einf\"uhrung in Sage }\\
\end{center}
\begin{minipage}{6cm}
Dr. J. Schulz\\
C. Rügge
\end{minipage}\hfill
\begin{minipage}{2.5cm}
\begin{flushright}
\textbf{Einheit 5}\\
WS 2009/2010
\end{flushright}
\end{minipage}\\[1cm]

\begin{aufg}
  Gegeben seien die Listen \(L1=[x,x^3,x^5,x^7,x^9]\) und
  \(L2=[x,x^2,x^6,x^{24},x^{120}]\). 
  \begin{enumerate}
  \item
    Erstellen Sie diese beiden Listen mit Hilfe des \verb+[... for ...]+ Konstruktes.
  \item Bilden sie $L3=L1+L2$
  \item
    Bilden Sie von jedem Eintrag von $L3$ die Ableitung und  die Stammfunktion.
  \item
Bestimmen Sie von jedem Element von $L3$ den Funktionswert an der
Stelle $x=3$. 
Entfernen Sie den größten Eintrag aus der resultierenden Liste.
  \end{enumerate}
\end{aufg}

\begin{aufg}
  Gegeben sei:
\begin{sage}
L=[16,81,125,512,729,4096,19683,78125,262144,390625,505,22343243,512]
\end{sage}
\begin{enumerate}
\item
Bestimmen Sie aus L alle Elemente, die durch $3$ teilbar sind und
entfernen Sie diese
aus der Liste.
\item
Bestimmen Sie aus den restlichen Elementen alle Elemente, die durch $2$ teilbar
sind, und entfernen diese 
aus der Liste.
\item
Bestimmen Sie zuletzt alle Elemente, die durch $5$ teilbar sind.

\end{enumerate}

\end{aufg}

\begin{aufg}
  \begin{enumerate}
  \item
    Bestimmen Sie das charakteristische Polynom, die Eigenwerte und die
    Eigenvektoren der Matrizen:\[ 
    A = \left(\begin{array}{ccc}19&-2&4\\4&10&-2\\4&-8&25\end{array}\right), 
    B = \left(\begin{array}{ccc}1&-3&3\\3&-5&3\\6&-6&4\end{array}\right), 
    C = \left(\begin{array}{ccc}-3&1&-1\\-7&5&-1\\-6&6&-2\end{array}\right). 
    \]
    \"Uberpr\"ufen Sie diese Matrizen ebenfalls auf Diagonalisierbarkeit.
  \item
    Bestimmen Sie die Eigenwerte der $(2\times 2)$-Drehmatrix aus der
    Vorlesung. Überlegen Sie sich, für welche Winkel reelle
    Eigenwerte existieren. Überlegen Sie sich eine geometrische
    Begründung. 
  \end{enumerate}
\end{aufg}

\begin{aufg}
  Gegeben seien Basen $V=( v_1, v_2, v_3)$, $W=(
  w_1, w_2, w_3)$   des
  $\mathbb{R}^3$ mit Vektoren 
  \[
   v_1 = \left(\begin{array}{c}2\\0\\2\end{array}\right),
   v_2 = \left(\begin{array}{c}0\\2\\0\end{array}\right), 
   v_3 = \left(\begin{array}{c}1\\2\\3\end{array}\right) \mbox{ und }
   w_1 = \left(\begin{array}{c}1\\1\\1\end{array}\right),
   w_2 = \left(\begin{array}{c}0\\0\\1\end{array}\right), 
   w_3 = \left(\begin{array}{c}1\\0\\1\end{array}\right).\]
 Berechnen Sie die Basiswechselmatrix bzgl. eines Basiswechsels von
  $V$ nach $W$!
\end{aufg}

\begin{aufg}
    Gegeben seien die Matrizen \(
    A = \left(\begin{array}{ccc}0&2&-2\\3&x&4\\-1&y&1\end{array}\right),
    B = \left(\begin{array}{ccc}1&8&x\\3&2&y\\6&4&4\end{array}\right) 
    \). 
    Bestimmen Sie x und y so, dass $AB$ invertierbar ist. \"Uberpr\"ufen Sie
    das Ergebnis durch Einsetzen.
\end{aufg}

\begin{aufg}
Untersuchen Sie für $m=0,1,\dots ,41$ jeweils, wieviele  der Zahlen
$n^2+n+m^2$ mit $n=1,2,\dots,100$ Primzahlen sind.
\end{aufg}

\begin{aufg}
Berechnen Sie mit Hilfe einer \verb+for+-Schleife die ersten 10 Glieder der Folge
\[ x_{n+1}=x_n - \frac{x_n^2-2}{2 x_n}, n \in \mathbb{N}   \]
mit Startwert $x_0=1$. Erraten Sie den Grenzwert der Folge!
Ersetzen Sie den Startwert $x_0$ durch den Gleitkommawert $x_0=1.0$..
\end{aufg}


\begin{aufg}
Schreiben Sie eine Prozedur \verb+fak+, die die Fakultät 
\[ a!= \prod_{i=1}^a i\]
von einer natürlichen Zahl $a$ berechnet. 
\end{aufg}
\end{document}

