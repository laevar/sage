\documentclass[notes=hide,hyperref={dvipdfmx,pdfpagelabels=false}]{beamer}
\title{Einführung in Sage - Einheit 4}
\subtitle{Matrizen, Vektorräume, Funktionen}
\mode<article>
{
  \usepackage{fullpage}
  \usepackage{pgf}
  \usepackage[xetex]{hyperref}
  \setjobnamebeamerversion{beamer}
}

\mode<presentation>
{
  %\usetheme{Frankfurt}
 %\usetheme{My}
  \usetheme{Madrid}
  % or ...
%\usecolortheme{seagull}
  %\setbeamercovered{transparent}
  %\setbeamercovered{dynamic}
  % or whatever (possibly just delete it)
}
\usenavigationsymbolstemplate{}
\usefonttheme{structurebold}
\usepackage{multimedia}
%\usepackage{tikz}
\usepackage{fontspec,xunicode,xltxtra}

%\usepackage{polyglossia}
%\setdefaultlanguage[spelling=new, latesthyphen=true]{german}
%\setsansfont{DejaVu Sans}
%\setsansfont{Verdana}
%\setsansfont{Arial}
%\setromanfont{Linux Libertine O}
%\setsansfont{Linux Biolinum O}

\setbeamertemplate{footline}
{
\leavevmode
%\hbox{\begin{beamercolorbox}[wd=.5\paperwidth,ht=2.5ex,dp=1.125ex,
%leftskip=.3cm plus1fill,rightskip=.3cm]{author in head/foot}%
%    \usebeamerfont{author in head/foot}\insertshortauthor
%  \end{beamercolorbox}%
%  \begin{beamercolorbox}[wd=.5\paperwidth,ht=2.5ex,dp=1.125ex,leftskip=.3cm,
%rightskip=.3cm plus1fil]{title in head/foot}%
%    \usebeamerfont{title in head/foot}\insertshorttitle\hfill

\hfill\insertframenumber  \hspace{3pt}

%\inserttotalframenumber
%\hspace*{2ex}
%  \end{beamercolorbox}}%
  \vskip3pt%
}

\usepackage[ngerman]{babel}
\selectlanguage{ngerman}

%
% math/symbols
%
\usepackage{amssymb}
\usepackage{amsthm}
% \usepackage{latexsym}
\usepackage{amsmath}
%\usepackage{amsxtra} %Weitere Extrasymbole
%\usepackage{empheq} %Gleichungen hervorheben
%\usepackage{bm}
 %\bm{A} Boldface im Mathemodus

\usepackage{cellspace}
\setlength{\cellspacetoplimit}{2pt}
\setlength{\cellspacebottomlimit}{2pt}

%%%%%%%%%%%%%%%%%% Fuer Frames [fragile]-Option verwenden!
%Programm-Listing
%%%%%%%%%%%%%%%%%%
%Listingsumgebung fuer verbatim
%Grauhinterlegeter Text
%Automatischer Zeilenumbruch ist aktiviert
\usepackage{listings}
\definecolor{lgray}{gray}{0.80}
%\lstset{backgroundcolor=\color{lgray}, frame=single, basicstyle=\ttfamily, breaklines=true}
\lstnewenvironment{sage}{\lstset{backgroundcolor=\color{lgray},language=Python, emphstyle=\color{red}, frame=single, basicstyle=\ttfamily, breaklines=true,mathescape =true,escapechar=§}}{}


\usepackage{mydef}
%\usepackage{cmap} % you can search in the pdf for umlauts and ligatures
\usepackage{colonequals} %corrects the definition-symbols \colonequals (besides others)
\title{Einführung in Sage}
%
%\subtitle{Disputation} % (optional)

\author{Jochen Schulz}
% - Use the \inst{?} command only if the authors have different
%   affiliation.

\institute{Georg-August Universit\"at G\"ottingen \pgfimage[height=0.5cm]{../figures/unilogo3}}
% - Use the \inst command only if there are several affiliations.
% - Keep it simple, no one is interested in your street address.

\date{\today}

\subject{Sage}
% This is only inserted into the PDF information catalog. Can be left
% out. 

% If you have a file called "university-logo-filename.xxx", where xxx
% is a graphic format that can be processed by latex or pdflatex,
% resp., then you can add a logo as follows:

%\logo{\pgfimage[height=0.5cm]{figures/unilogo3}}


% Delete this, if you do not want the table of contents to pop up at
% the beginning of each subsection:

\AtBeginSection[]
{
  \begin{frame}<beamer>
    \frametitle{Aufbau}
    \tableofcontents[currentsection,currentsubsection]
  \end{frame}
}

\AtBeginSubsection[]
{
  \begin{frame}<beamer>
    \frametitle{Aufbau}
    \tableofcontents[currentsection,currentsubsection]
  \end{frame}
}



%%%%%%%%%%%%%%%%%%%
%Neue Definitionen
%%%%%%%%%%%%%%%%%%%

%Newcommands
\newcommand{\Fun}[1]{\mathcal{#1}}      %Mathcal fuer Funktoren
\newcommand{\field}[1]{\mathbb{#1}}     %Grundkoerper ?? in mathds

\newcommand{\A}{\field{A}}              %Affines A
\newcommand{\C}{\field{C}}              %Complexes C
\newcommand{\Fp}{\field{F}_{\!p}}       %Endlicher Koerper mit p Elementen
\newcommand{\Fq}{\field{F}_{\!q}}       %Endlicher Koerper mit q Elementen
\newcommand{\Ga}{\field{G}_{a}}         %Add Gruppenschema
\newcommand{\K}{\field{K}}              %Generischer Koerper 
\newcommand{\N}{\field{N}}              %Nat Zahlen
\newcommand{\Pj}{\field{P}}             %Projektives P
\newcommand{\R}{\field{R}} 		%Reelle Zahlen
\newcommand{\Q}{\field{Q}}              %Rationale Zahlen  
\newcommand{\Qt}{\field{H}}             %Quaternionen 
\newcommand{\V}{\field{V}}              %Vektorbuendel V
\newcommand{\Z}{\field{Z}}              %Ganze Zahlen

\newcommand{\fdg}{\;|\;}                 %fuer die gilt

%Operatoren
\DeclareMathOperator{\Abb}{Abb}
%\usepackage{sagetex}

\begin{document}
\lstset{basicstyle={\lstbasicfont\footnotesize}}


\begin{document}
\maketitle

\begin{frame}{Aufbau}
\tableofcontents
\end{frame}


%===================================================
\section{Vektoren}
%==================================================
\subsection{Matrizen}

\begin{frame}{Matrizen}

{\color{red} $m \times n$ Matrix} $A=(a_{ij}) \in K^{m\times n}$ über einen Körper $K$ 

\[ A = \left( \begin{array}{cccc}
a_{11} & a_{12} & \cdots & a_{1n} \\
a_{21} & a_{22} & \cdots & a_{2n} \\
\vdots & \vdots & \ddots & \vdots \\
a_{m1} & a_{m2} & \cdots & a_{mn} 
\end{array} \right) \] 
$a_{ij} \in K$, Zeilenindex $i \in [1,m]$, Spaltenindex $j \in [1,n]$

\end{frame}

\begin{frame}{Definitionen}
\begin{itemize}
\item {\color{red} Transponiert} von $A=(a_{ij})$: $A^T:=(a_{ji})$ .
\item {\color{red} Symmetrisch}: wenn $A=A^T$ gilt.
\item \alert{Adjungiert} von $A=(a_{ij})\in \mathbb{C}^{m\times n}$: $A^* :=
(\overline{a_{ji}}) \in \mathbb{C}^{n \times m}$.
\item \alert{Einheitsmatrix}: $I:=I_n:=(\delta_{ij}) \in K^{n \times
n}$
\item {\color{red} Addition}: Seien $A=(a_{ij}),B=(b_{ij}) \in {K}^{n \times m}$, dann
\[ C=(c_{ij}):=A+B \in {K}^{n \times m} \]
mit $c_{ij}=a_{ij}+b_{ij}$.  
\end{itemize}
\end{frame} 

\begin{frame}{Definitionen}
\begin{itemize}
\item {\color{red} Multiplikation}: Seien $A=(a_{ij}) \in {K}^{m \times n}$ und $B=(b_{ij})
\in {K}^{n \times p}$, dann 
\[ C=(c_{ij}):=A \cdot B \in{K}^{m \times p} \]
mit $c_{ij}=\sum_{k=1}^n a_{ik} b_{kj}$. 
\item {\color{red} orthogonal}: $A \cdot A^T=A^T \cdot A=I_n$ für $A\in K^{n \times n}$ 
\item {\color{red} unitär}: $A \cdot A^*=A^* \cdot A=I_n$ für $A\in \mathbb{C}^{n \times n}$.
\item {\color{red} invertierbar}: $A\in K^{n \times n}$ heißt , wenn eine
Matrix $A^{-1}\in K^{n \times n}$ existiert mit  $A \cdot
A^{-1}=A^{-1} \cdot A=I_n$.
\end{itemize}
\end{frame} 

\begin{frame}{Definitionen und Bemerkungen}
\begin{itemize}
\item Die Multiplikation ist assoziativ aber in der Regel \alert{nicht kommutativ}. 
\item Die Matrizen aus $K^{m \times n}$ bilden einen Vektorraum über
$K$ (mit komponentenweiser Skalarmultiplikation).
\item {\color{red} allgemeine
lineare Gruppe} $\operatorname{GL}(K,n) = \operatorname{GL}_n(K) = \operatorname{GL}(n,K)$: 
Die Menge der invertierbaren Matrizen aus $K^{n \times n}$ bilden bezüglich der Multiplikation eine Gruppe.
\item {\color{red} orthogonale Gruppe}: $O(n)$: Die Menge der orthogonalen Matrizen in $GL(\mathbb{R},n)$ bilden
 eine Untergruppe von  $GL(\mathbb{R},n)$. 
\item  {\color{red} unitäre Gruppe} $U(n)$: Die entsprechende Untergruppe der unitären Matrizen in $GL(\mathbb{C},n)$.
\end{itemize}
\end{frame}








\subsection{Vektorräume}
\begin{frame}{Vektorraum}
Ein Tripel $(V,+,\cdot)$, bestehend aus einer nichtleeren Menge $V$
und Verknüpfungen
\[ +:V \times V \ \rightarrow \ V, \qquad \cdot:K\times V \ \rightarrow \ V\]
heißt {\color{red} Vektorraum} über einem Körper
$K$, wenn gilt:
\begin{enumerate}
\item $(V,+)$ ist eine abelsche Gruppe.
\item Für alle $v,w \in V$ und alle $\lambda, \mu \in K$ gilt:
\begin{enumerate}
 \item $(\lambda + \mu) \cdot v  =(\lambda \cdot v) + ( \mu \cdot v)$.
\item $\lambda \cdot (v + w )  = ( \lambda \cdot v) + ( \lambda \cdot w)$.
\item $(\lambda \mu) \cdot v = \lambda \cdot (\mu \cdot v)$.
\item $1 \cdot v = v$.
\end{enumerate}
\end{enumerate}
\end{frame}

\begin{frame}{Begriffe}
\begin{itemize}
\item {\color{red} Vektoren}: Die Elemente eines Vektorraums.
\item {\color{red} Skalarmultiplikation}: Die Abbildung $\cdot : K\times V \ \rightarrow \ V$. Die Elemente des Körpers $K$ nennt man
  {\color{red} Skalare}.
\item {\color{red} Untervektorraum} oder {\color{red} Unterraum} von $V$: Ist $U \subset V$ eine Teilmenge des Vektorraums $V$ und es gelten
  alle Vektorraumaxiome.
\item \alert{Vorsicht!} man muß zwischen der $0$ des Körpers und der $0$ des Vektorraums (Nullvektor) unterscheiden. \\
Es gilt $0 \cdot v  = 0$ für alle $v \in V$. 
\end{itemize}
\end{frame}

\begin{frame}{Beispiele für Vektorräume}
\begin{itemize}
\item $K^n := \{(x_1,\ldots,x_n) \;|\; x_1, \ldots, x_n \in K\}$, $n \in \mathbb{N}$
\item Sei $M$ eine beliebige Menge. Die Menge der Abbildungen von $M$
  in $K$, $\text{Abb}(M,K)$, mit den punktweise definiertenVerknüpfungen
\begin{eqnarray*}
(f+g)(x) & :=& f(x)+g(x), \forall\; x \in M\\
(\alpha \cdot f)(x) & :=& \alpha \cdot f(x), \forall\; x \in M  
\end{eqnarray*}
für $\alpha \in K$, $f,g:M \mapsto K$.
\item Die Menge der Polynome bis zum Grad $n$.
\item Die Menge aller Polynome.
\item $\mathbb{R}$ als $\mathbb{Q}$-Vektorraum.
\item $\mathbb{C}$ als $\mathbb{R}$-Vektorraum.
\end{itemize}
\end{frame} 



\begin{frame}{Lineare Abhängigkeit}
Sei $V$ ein $K$-Vektorraum und $(v_1,\dots ,v_r)$ eine Familie von
Elementen aus $V$.
\begin{itemize}
\item {\color{red} Linearkombination} $v \in V$ von $(v_1,\dots ,v_r)$: 
  falls $\exists \lambda_1, \dots, \lambda_r \in K$  mit
  $ v= \lambda_1 v_1 + \dots + \lambda_r v_r$. 
\item {\color{red} Lineare Hülle} $\mathop{span}\{v_1, \dots, v_n\}$: Die Menge aller Linearkombinationen. Die Lineare
  Hülle ist ein Unterraum von $V$.
\item {\color{red} linear unabhängig}: 
  Sind $\lambda_1, \dots , \lambda_r \in K$ und ist $\lambda_1 v_1 +
  \dots + \lambda_r v_r=0$ so folgt $\lambda_1= \dots =
  \lambda_r=0$. Andernfalls {\color{red} linear abhängig}. 
\begin{itemize}
\item Ist $M \subseteq V$ eine unendliche Menge, dann ist $M$ linear unabhängig falls alle endlichen Teilmengen von $M$ linear unabhängig sind.
\end{itemize}
\end{itemize}
\end{frame}

\begin{frame}{Weitere Notationen und Bemerkungen}
Sei $V$ ein $K$-Vektorraum und $(v_1,\dots ,v_r)$ eine Familie von
Elementen aus $V$
\begin{itemize}
\item $(v_1,\dots ,v_r)$ sind genau dann linear unabhängig, wenn sich
jeder Vektor $v \in span\{v_1, \dots ,v_r\}$ eindeutig linear kombinieren
läßt. 
\item Vektoren sind linear unabhängig wenn die Determinante der korrelierenden Matrix ungleich 0 ist.
\item Gilt $V=span\{v_1,\dots ,v_r \}$, so ist $(v_1, \dots ,v_r)$ ein
{\color{red} Erzeugendensystem}. Sind $(v_1, \dots ,v_r)$ zusätzlich linear
unabhängig, so ist $(v_1, \dots ,v_r)$ eine {\color{red} Basis}.
\item Aus jedem Erzeugendensystem kann man eine Basis auswählen. 
\end{itemize}
\end{frame}


\begin{frame}{Beispiele für Basen}
\begin{itemize}
\item Seien $(e_i)_{i=1,\ldots,n} \in \mathbb{R}^n$ die Einheitsvektoren. $(e_1, \dots
,e_n)$ ist eine Basis des $\mathbb{R}^n$.
\item Die Monombasis $(1,x,x^2,\dots, x^n)$ ist eine Basis des
Vektorraums der Polynome $n$-ten Grades.
\item $(1,i)$ ist eine Basis von $\mathbb{C}$ als
$\mathbb{R}$-Vektorraum. 
\item $\mathbb{R}$ als $\mathbb{Q}$-Vektorraum hat keine endliche
Basis. 
\end{itemize}
\end{frame}

\begin{frame}{Basis und Dimension}
\begin{itemize}
\item {\color{red} Dimension} des Vektorraums $V$: die  Anzahl der Basiselemente einer Basis $(v_1,\dots, v_n)$.
\item Jeder Vektorraum besitzt eine Basis.
\item Seien $W,Z$ Unterräume von $V$. Dann ist {\color{red} $W+Z:=span(W \cup Z)$}
die {\color{red} Summe} von $W$ und $Z$. Es gilt:
\[ \dim(W+Z)=\dim(W) + \dim (Z) - \dim( W \cap Z) \]
\end{itemize}
\end{frame}






\begin{frame}{Normen auf Vektorräumen}
Sei $V$ ein Vektorraum über $K=\mathbb{R}$ oder $K=\mathbb{C}$.\\
Eine {\color{red} Norm} auf $V$ ist eine Abbildung
\[ \| \cdot \|: V \ \rightarrow \mathbb{R}, v \mapsto \| v \|, \]
so dass für  alle $\alpha \in K$, $u,v \in V$ gilt 
\[ \begin{array} {rcl}
\| v \| & \geq & 0 \\
\| v \| & = & 0 \mbox{ impliziert } v=0\\
\| \alpha v \| & = & | \alpha | \| v \|\\
\| u + v \| & \leq & \| u \| + \| v \| \mbox{ (Dreiecksungleichung)}.\\
\end{array} \]
$(V,\| \cdot \|)$
heißt {\color{red} normierter Raum}.
\end{frame}

\begin{frame}[fragile]{Skalarprodukt}
Eine skalarwertige binäre Abbildung
\[ ( \cdot, \cdot ): V \times V: \ \rightarrow \ K\]
auf einem Vektorraum $V$ über $K=\mathbb{R}$ oder $K=\mathbb{C}$ heißt
{\color{red} Skalarprodukt}, wenn für alle $x,y,z \in V$, $\alpha, \beta \in
K$ gilt 
\begin{eqnarray*}
(x,x) & \geq & 0\\
(x,x) & = & 0 \mbox{ impliziert } x=0.\\
(x,y) & = & \overline{(y,x)}\\
(\alpha x+\beta y,z) & = & \alpha (x,z)+ \beta (y,z)
\end{eqnarray*}
\end{frame}

\begin{frame}{Bemerkungen}
\begin{itemize}
\item Ein VR $V$ mit Skalarprodukt heißt {\color{red} Prä-Hilbert-Raum}. Ist
$K=\mathbb{R}$ so heißt der Raum auch {\color{red} euklidisch}.
\item Durch $\|v\|:=\sqrt{(v,v)}$, $v \in V$ läßt sich eine Norm
definieren. Es gilt die {\color{red} Cauchy-Schwarzsche Ungleichung}
\[ |(u,v)| \leq \| u \| \|v\|. \]
\item Im euklidischen Raum ist der Winkel $\alpha$ zwischen zwei Vektoren
$u,v \in V\smallsetminus \{ 0 \}$ definiert durch
\[ \cos(\alpha) = \frac{(u,v)}{\|u\| \|v \|}. \]
   \end{itemize}
\end{frame}


\begin{frame}{Bemerkungen}
\begin{itemize}
\item {\color{red} Orthogonal}: wenn $(u,v)=0$ gilt.
\item {\color{red} Orthogonalbasis}: Eine Basis aus paarweise orthogonalen Vektoren. 
\item {\color{red} Orthonormalbasis}: Eine Orthogonalbasis, bei der alle Vektoren die Norm $1$
haben.
\item Jeder endlichdimensionale Prä-Hilbert-Raum hat eine
Orthonormalbasis. 
\item {\color{red} Orthogonalraum}:
\[
{U}^\perp := \{ v \in V \ | \ (v,u)=0 \mbox{ für alle } u \in U \}
\]
wenn $U$ ein Unterraum von $V$ ist.
\item Es gilt: $\dim U + \dim U^\perp = \dim V$, insb. $U \cap U^\perp = 0$.
\end{itemize}
\end{frame}


\begin{frame}{Sage}
\begin{center}
\url{https://sage.math.uni-goettingen.de/home/pub/16/}
\end{center}
\end{frame}

\section{Funktionen}

\begin{frame}{Sage}
\begin{center}
\url{https://sage.math.uni-goettingen.de/home/pub/17/}
\end{center}
\end{frame}


\end{document}
