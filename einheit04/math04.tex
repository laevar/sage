\documentclass[a4paper,12pt,DIV15]{scrartcl}
\usepackage[xetex,bookmarks=true,pdfstartview=FitH,bookmarksopen=true,
    colorlinks,citecolor=Blue,linkcolor=DarkBlue,urlcolor=Green,
    pagebackref=true,plainpages=false,pdfpagelabels=true,unicode=true,
    breaklinks=true,naturalnames=false,setpagesize=true,a4paper=true,hyperindex]{hyperref}
\usepackage[svgnames,hyperref]{xcolor} %color definition

%\usepackage{fontspec,xunicode}
% %\usepackage{polyglossia}
%\setdefaultlanguage[spelling=new, latesthyphen=true]{german}
%\setsansfont{DejaVu Sans}
%\setsansfont{Verdana}
%\setsansfont{Arial}
%\setromanfont[Mapping=tex-text]{Linux Libertine}
%\setsansfont[Mapping=tex-text]{Myriad Pro}
%\setmonofont[Mapping=tex-text]{Courier New}

%\setsansfont{Linux Biolinum}

\usepackage[ngerman]{babel}
\selectlanguage{ngerman}

%
% math/symbols
%
\usepackage{amssymb}
\usepackage{amsthm}
% \usepackage{latexsym}
\usepackage{amsmath}
%\usepackage{amsxtra} %Weitere Extrasymbole
%\usepackage{empheq} %Gleichungen hervorheben
%\usepackage{bm}
 %\bm{A} Boldface im Mathemodus
\usepackage{fontspec,xunicode,xltxtra}

\usepackage{multimedia}
%\usepackage{tikz}

\usepackage{cellspace}
\setlength{\cellspacetoplimit}{2pt}
\setlength{\cellspacebottomlimit}{2pt}

%%%%%%%%%%%%%%%%%% Fuer Frames [fragile]-Option verwenden!
%Programm-Listing
%%%%%%%%%%%%%%%%%%
%Listingsumgebung fuer verbatim
%Grauhinterlegeter Text
%Automatischer Zeilenumbruch ist aktiviert
%\usepackage{listings}
\usepackage[framed]{mcode}
%\usepackage{mcode}
% This command allows you to typeset syntax highlighted Matlab
% code ``inline''.
% mcode fuer matlab

\definecolor{lgray}{gray}{0.80}
\definecolor{gray}{gray}{0.3}
\definecolor{darkgreen}{rgb}{0,0.4,0}
\definecolor{darkblue}{rgb}{0,0,0.8}
\definecolor{key}{rgb}{0,0.5,0} 
\definecolor{NU0}{RGB}{68,85,136} % #458
\definecolor{KW3}{RGB}{85,68,136}
\definecolor{KW4}{RGB}{153,0,0}
\definecolor{dred}{RGB}{221,17,68} % #d14
\definecolor{BG}{RGB}{240,240,240}
%\lstset{backgroundcolor=\color{lgray}, frame=single, basicstyle=\ttfamily, breaklines=true}
%\lstnewenvironment{sage}{\lstset{,language=python, keywordstyle=color{blue},    commentstyle=color{green}, emphstyle=\color{red}, %frame=single, stringstyle=\color{red}, basicstyle=\ttfamily, ,mathescape =true,escapechar=§}}{}

\lstdefinelanguage{fooHaskell} {%
  basicstyle=\footnotesize\ttfamily,%
  commentstyle=\slshape\color{gray},%
  keywordstyle=\bfseries,%\color{KW4},
  breaklines=true,
  sensitive=true,
  xleftmargin=1pc,
  emph={[1]
    FilePath,IOError,abs,acos,acosh,all,and,any,appendFile,approxRational,asTypeOf,asin,
    asinh,atan,atan2,atanh,basicIORun,break,catch,ceiling,chr,compare,concat,concatMap,
    const,cos,cosh,curry,cycle,decodeFloat,denominator,digitToInt,div,divMod,drop,
    dropWhile,either,elem,encodeFloat,enumFrom,enumFromThen,enumFromThenTo,enumFromTo,
    error,even,exp,exponent,fail,filter,flip,floatDigits,floatRadix,floatRange,floor,
    fmap,foldl,foldl1,foldr,foldr1,fromDouble,fromEnum,fromInt,fromInteger,fromIntegral,
    fromRational,fst,gcd,getChar,getContents,getLine,head,id,inRange,index,init,intToDigit,
    interact,ioError,isAlpha,isAlphaNum,isAscii,isControl,isDenormalized,isDigit,isHexDigit,
    isIEEE,isInfinite,isLower,isNaN,isNegativeZero,isOctDigit,isPrint,isSpace,isUpper,iterate,
    last,lcm,length,lex,lexDigits,lexLitChar,lines,log,logBase,lookup,map,mapM,mapM_,max,
    maxBound,maximum,maybe,min,minBound,minimum,mod,negate,not,notElem,null,numerator,odd,
    or,ord,otherwise,pi,pred,primExitWith,print,product,properFraction,putChar,putStr,putStrLn,quot,
    quotRem,range,rangeSize,read,readDec,readFile,readFloat,readHex,readIO,readInt,readList,readLitChar,
    readLn,readOct,readParen,readSigned,reads,readsPrec,realToFrac,recip,rem,repeat,replicate,return,
    reverse,round,scaleFloat,scanl,scanl1,scanr,scanr1,seq,sequence,sequence_,show,showChar,showInt,
    showList,showLitChar,showParen,showSigned,showString,shows,showsPrec,significand,signum,sin,
    sinh,snd,span,splitAt,sqrt,subtract,succ,sum,tail,take,takeWhile,tan,tanh,threadToIOResult,toEnum,
    toInt,toInteger,toLower,toRational,toUpper,truncate,uncurry,undefined,unlines,until,unwords,unzip,
    unzip3,userError,words,writeFile,zip,zip3,zipWith,zipWith3,listArray,doParse
  },%
  emphstyle={[1]\color{NU0}},%
  emph={[2]
    Bool,Char,Double,Either,Float,IO,Integer,Int,Maybe,Ordering,Rational,Ratio,ReadS,Show,ShowS,String,
    Word8,InPacket
  },%
  emphstyle={[2]\bfseries\color{KW4}},%
  emph={[3]
    case,class,data,deriving,do,else,if,import,in,infixl,infixr,instance,let,
    module,of,primitive,then,type,where
  },
  emphstyle={[3]\color{darkblue}},
  emph={[4]
    quot,rem,div,mod,elem,notElem,seq
  },
  emphstyle={[4]\color{NU0}\bfseries},
  emph={[5]
    EQ,False,GT,Just,LT,Left,Nothing,Right,True,Show,Eq,Ord,Num
  },
  emphstyle={[5]\color{KW4}\bfseries},
  morestring=[b]",%
  morestring=[b]',%
  stringstyle=\color{darkgreen},%
  showstringspaces=false
}
\lstnewenvironment{hs}
{\lstset{language=fooHaskell,backgroundcolor=\color{BG}}}
{\smallskip}
\newcommand{\ihs}[1]{\lstset{language=fooHaskell,basicstyle=\color[gray]{0.6}}\lstinline|#1|}


\lstdefinelanguage{fooMatlab} {%
backgroundcolor=\color[gray]{0.9},
breaklines=true,
basicstyle=\ttfamily\small,
%otherkeywords={ =},
%keywordstyle=\color{blue},
%stringstyle=\color{darkgreen},
showstringspaces=false,
%emph={for, while, if, elif, else, not, and, or, printf, break, continue, return, end, function},
%emphstyle=\color{blue},
%emph={[2]True, False, None, self, NaN, NULL},
%emphstyle=[2]\color{key},
%emph={[3]from, import, as},
%emphstyle=[3]\color{blue},
%upquote=true,
%morecomment=[s]{"""}{"""},
%commentstyle=\color{gray}\slshape,
%framexleftmargin=1mm, framextopmargin=1mm, 
%title=\tiny matlab,
frame=single,
%mathescape =true,
%escapechar=§
}
\newcommand{\imatlab}[1]{\lstset{language=fooMatlab,basicstyle=\color[gray]{0.6}}\lstinline|#1|}
\lstnewenvironment{matlab}[1][]{\lstset{language=fooMatlab,xleftmargin=0.2cm,frame=none,backgroundcolor=\color{white},basicstyle=\color{darkblue}\ttfamily\small,#1}}{} 
\lstnewenvironment{matlabin}[1][]{\lstset{language=fooMatlab,#1}}{} 
\newcommand{\matinput}[1]{\lstset{language=fooMatlab}\lstinputlisting{#1}}

\lstdefinelanguage{fooPython} {%
language=python,
backgroundcolor=\color[gray]{0.7},
breaklines=true,
basicstyle=\ttfamily\small,
%otherkeywords={ =},
keywordstyle=\color{blue},
stringstyle=\color{darkgreen},
morestring=[b]",%
morestring=[b]',%
showstringspaces=false,
emph={class, pass, in, for, while, if, is, elif, else, not, and, or,
def, print, exec, break, continue, return, import, from, lambda, null},
emphstyle=\color{blue},
emph={[2]True, False, None, self},
emphstyle=[2]\color{key},
emph={[3]from, import, as},
emphstyle=[3]\color{blue},
upquote=true,
morecomment=[s]{"""}{"""},
comment=[l]{\#},
commentstyle=\color{gray},
%framexleftmargin=1mm, framextopmargin=1mm, 
%title=\tiny python,
%caption=python,
frame=single
%frameround=tttt,
%mathescape =true,
%escapechar=§
}

\newcommand{\pyinput}[1]{\lstset{language=fooPython}\lstinputlisting{#1}}
\newcommand{\isage}[1]{{\lstset{language=fooPython,basicstyle=\color[gray]{0.3}}\lstinline|#1|}}

\lstnewenvironment{pyout}[1][]{\lstset{language=fooPython,xleftmargin=0.2cm,frame=none,backgroundcolor=\color{white},basicstyle=\color{darkblue}\ttfamily\small,#1}}{}
\lstnewenvironment{pyin}[1][]{\lstset{language=fooPython,#1}}{}
\lstnewenvironment{sageout}[1][]{\lstset{language=fooPython,xleftmargin=0.2cm,frame=none,backgroundcolor=\color{white},basicstyle=\color{darkblue}\ttfamily\small,#1}}{}
\lstnewenvironment{sagein}[1][]{\lstset{language=fooPython,#1}}{}

%\usepackage{caption}
%\DeclareCaptionFont{white}{ \color{white} }
%\DeclareCaptionFormat{listing}{
%  \colorbox[cmyk]{0.43, 0.35, 0.35,0.01 }{
%      \parbox{\textwidth}{\hspace{15pt}#1#2#3}
%        }
%        }
%        \captionsetup[lstlisting]{ format=listing, labelfont=white, textfont=white, singlelinecheck=false, margin=0pt, font={bf,footnotesize} }


\usepackage{mydef}
%\usepackage{cmap} % you can search in the pdf for umlauts and ligatures
\usepackage{colonequals} %corrects the definition-symbols \colonequals (besides others)

\usepackage{ifthen}

%%%%%%%%%%%%%%%%%%%
%Neue Definitionen
%%%%%%%%%%%%%%%%%%%

%Newcommands
\newcommand{\Fun}[1]{\mathcal{#1}}      %Mathcal fuer Funktoren
\newcommand{\field}[1]{\mathbb{#1}}     %Grundkoerper ?? in mathds

\newcommand{\A}{\field{A}}              %Affines A
\newcommand{\Fp}{\field{F}_{\!p}}       %Endlicher Koerper mit p Elementen
\newcommand{\Fq}{\field{F}_{\!q}}       %Endlicher Koerper mit q Elementen
\newcommand{\Ga}{\field{G}_{a}}         %Add Gruppenschema
\newcommand{\K}{\field{K}}              %Generischer Koerper 
\newcommand{\N}{\field{N}}              %Nat Zahlen
\newcommand{\Pj}{\field{P}}             %Projektives P
\newcommand{\R}{\field{R}} 		%Reelle Zahlen
\newcommand{\Q}{\field{Q}}              %Rationale Zahlen  
\newcommand{\Qt}{\field{H}}             %Quaternionen 
\newcommand{\V}{\field{V}}              %Vektorbuendel V
\newcommand{\Z}{\field{Z}}              %Ganze Zahlen
\DeclareMathOperator{\Real}{Re}

\newcommand{\fdg}{\;|\;}                 %fuer die gilt

%Operatoren
\DeclareMathOperator{\Abb}{Abb}
%\usepackage{sagetex}


%
% Aufgaben
%
\parindent0cm % Abs�tze nicht einr�cken 
% Definieren einer neuen Farbe
\definecolor{light-gray}{gray}{.9}

\newcounter{zaehler}     % neuen Z�hler einf�hren
\newenvironment{aufgn}[2][0]
%---- Header
{\begin{samepage}%
%\colorbox{light-gray}{%                         % Box in gray
% \makebox[\textwidth]{%                           % Box in linewidth
%\textbf{Aufgabe \arabic{zaehler} } }\hspace{-\textwidth}\makebox[\textwidth]{\hfill #1 Punkte} }\\[0.05cm]       % Header
\dotfill\\
{\large\textbf{Aufgabe \arabic{zaehler} \ifthenelse{ \equal{#2}{} }{}{: \emph{ #2 } }}\ifthenelse{-1=#1}{(testierbar)}{}\ifthenelse{0=#1 \or -1=#1}{}{\hfill #1 Punkte} }\\[0.4cm]
%{\large\textbf{Exercise \arabic{zaehler}  #2 }\ifthenelse{-1=#1}{(testierbar)}{}\ifthenelse{0=#1 \or -1=#1}{}{\hfill #1 Punkte} }\\[0.4cm]
\begin{minipage}{\textwidth}%
}%
%-----  foot
{\end{minipage}\nopagebreak%\begin{minipage}{1cm} \end{minipage}
%\\ 
%\begin{minipage}{0.1cm} \end{minipage} 
%\hrulefill \begin{minipage}{1cm} \end{minipage}\\[1cm]  
\stepcounter{zaehler}                           % increase counter
\end{samepage}%
\\%
\bigskip%
}


\newenvironment{aufg}[1][0]
%---- Header
{\begin{samepage}%
\refstepcounter{zaehler}% increase counter
%\colorbox{light-gray}{%                         % Box in gray
% \makebox[\textwidth]{%                           % Box in linewidth
%\textbf{Aufgabe \arabic{zaehler} } }\hspace{-\textwidth}\makebox[\textwidth]{\hfill #1 Punkte} }\\[0.05cm]       % Header
\dotfill\\
{\large\textbf{Aufgabe \arabic{zaehler} }\ifthenelse{-1=#1}{(testierbar)}{}\ifthenelse{0=#1 \or -1=#1}{}{\hfill #1 Punkte} }\\[0.4cm]
\begin{minipage}{\textwidth}%
}%
%-----  foot
{\end{minipage}\nopagebreak%\begin{minipage}{1cm} \end{minipage}
%\\ 
%\begin{minipage}{0.1cm} \end{minipage} 
%\hrulefill \begin{minipage}{1cm} \end{minipage}\\[1cm]  
\end{samepage}%
\\%
\bigskip%
}

\begin{document}
\begin{center}
    \textbf{\LARGE Einführung in Sage}\\
    {\large Einheit 04}\\
    {\large Zusammenfassung: Vektoren, Matrizen und Vektorräume}
\end{center}

\section{Matrizen}

\begin{defn}[Matrix]
    \label{def:matrix}
{\color{red} $m \times n$ Matrix} $A=(a_{ij}) \in K^{m\times n}$ über einen Körper $K$ 

\[ A = \left( \begin{array}{cccc}
a_{11} & a_{12} & \cdots & a_{1n} \\
a_{21} & a_{22} & \cdots & a_{2n} \\
\vdots & \vdots & \ddots & \vdots \\
a_{m1} & a_{m2} & \cdots & a_{mn} 
\end{array} \right) \] 
$a_{ij} \in K$, Zeilenindex $i \in [1,m]$, Spaltenindex $j \in [1,n]$
\end{defn}

\begin{defn}[Eigenschaften von Matrizen] 
    Sei A eine Matrix aus Definition \ref{def:matrix}.
\begin{itemize}
\item {\color{red} Transponiert} von $A=(a_{ij})$: $A^T:=(a_{ji})$ .
\item {\color{red} Symmetrisch}: wenn $A=A^T$ gilt.
\item {\color{red} Adjungiert} von $A=(a_{ij})\in \mathbb{C}^{m\times n}$: $A^* :=
(\overline{a_{ji}}) \in \mathbb{C}^{n \times m}$.
\item {\color{red} Einheitsmatrix}: $I:=I_n:=(\delta_{ij}) \in K^{n \times
n}$
\item {\color{red} Addition}: Seien $A=(a_{ij}),B=(b_{ij}) \in {K}^{n \times m}$, dann
\[ C=(c_{ij}):=A+B \in {K}^{n \times m} \]
mit $c_{ij}=a_{ij}+b_{ij}$.  
\item {\color{red} Multiplikation}: Seien $A=(a_{ij}) \in {K}^{m \times n}$ und $B=(b_{ij})
\in {K}^{n \times p}$, dann 
\[ C=(c_{ij}):=A \cdot B \in{K}^{m \times p} \]
mit $c_{ij}=\sum_{k=1}^n a_{ik} b_{kj}$. 
\item {\color{red} orthogonal}: $A \cdot A^T=A^T \cdot A=I_n$ für $A\in K^{n \times n}$ 
\item {\color{red} unitär}: $A \cdot A^*=A^* \cdot A=I_n$ für $A\in \mathbb{C}^{n \times n}$.
\item {\color{red} invertierbar}: $A\in K^{n \times n}$ heißt , wenn eine
Matrix $A^{-1}\in K^{n \times n}$ existiert mit  $A \cdot
A^{-1}=A^{-1} \cdot A=I_n$.
\end{itemize}
\end{defn} 

\paragraph{Bemerkungen:}
\begin{itemize}
\item Die Multiplikation ist assoziativ aber in der Regel \textsl{nicht kommutativ}. 
\item Die Matrizen aus $K^{m \times n}$ bilden einen Vektorraum über
$K$ (mit komponentenweiser Skalarmultiplikation).
\item {\color{red} allgemeine
lineare Gruppe} $\operatorname{GL}(K,n) = \operatorname{GL}_n(K) = \operatorname{GL}(n,K)$: 
Die Menge der invertierbaren Matrizen aus $K^{n \times n}$ bilden bezüglich der Multiplikation eine Gruppe.
\item {\color{red} orthogonale Gruppe}: $O(n)$: Die Menge der orthogonalen Matrizen in $GL(\mathbb{R},n)$ bilden
 eine Untergruppe von  $GL(\mathbb{R},n)$. 
\item  {\color{red} unitäre Gruppe} $U(n)$: Die entsprechende Untergruppe der unitären Matrizen in $GL(\mathbb{C},n)$.
\end{itemize}


\subsection{Vektorräume}
\begin{defn}[Vektorraum]
Ein Tripel $(V,+,\cdot)$, bestehend aus einer nichtleeren Menge $V$
und Verknüpfungen
\[ +:V \times V \ \rightarrow \ V, \qquad \cdot:K\times V \ \rightarrow \ V\]
heißt {\color{red} Vektorraum} über einem Körper
$K$, wenn gilt:
\begin{enumerate}
\item $(V,+)$ ist eine abelsche Gruppe.
\item Für alle $v,w \in V$ und alle $\lambda, \mu \in K$ gilt:
\begin{enumerate}
 \item $(\lambda + \mu) \cdot v  =(\lambda \cdot v) + ( \mu \cdot v)$.
\item $\lambda \cdot (v + w )  = ( \lambda \cdot v) + ( \lambda \cdot w)$.
\item $(\lambda \mu) \cdot v = \lambda \cdot (\mu \cdot v)$.
\item $1 \cdot v = v$.
\end{enumerate}
\end{enumerate}
\end{defn}

\paragraph{Eigenschaften:}
\begin{itemize}
\item {\color{red} Vektoren}: Die Elemente eines Vektorraums.
\item {\color{red} Skalarmultiplikation}: Die Abbildung $\cdot : K\times V \ \rightarrow \ V$. Die Elemente des Körpers $K$ nennt man
  {\color{red} Skalare}.
\item {\color{red} Untervektorraum} oder {\color{red} Unterraum} von $V$: Ist $U \subset V$ eine Teilmenge des Vektorraums $V$ und es gelten
  alle Vektorraumaxiome.
\item \textsl{Vorsicht!} man muß zwischen der $0$ des Körpers und der $0$ des Vektorraums (Nullvektor) unterscheiden. \\
Es gilt $0 \cdot v  = 0$ für alle $v \in V$. 
\end{itemize}

\paragraph{Beispiele für Vektorräume:}
\begin{itemize}
\item $K^n := \{(x_1,\ldots,x_n) \;|\; x_1, \ldots, x_n \in K\}$, $n \in \mathbb{N}$
\item Sei $M$ eine beliebige Menge. Die Menge der Abbildungen von $M$
  in $K$, $\text{Abb}(M,K)$, mit den punktweise definiertenVerknüpfungen
\begin{eqnarray*}
(f+g)(x) & :=& f(x)+g(x), \forall\; x \in M\\
(\alpha \cdot f)(x) & :=& \alpha \cdot f(x), \forall\; x \in M  
\end{eqnarray*}
für $\alpha \in K$, $f,g:M \mapsto K$.
\item Die Menge der Polynome bis zum Grad $n$.
\item Die Menge aller Polynome.
\item $\mathbb{R}$ als $\mathbb{Q}$-Vektorraum.
\item $\mathbb{C}$ als $\mathbb{R}$-Vektorraum.
\end{itemize}



\begin{defn}[Lineare Abhängigkeit]
Sei $V$ ein $K$-Vektorraum und $(v_1,\dots ,v_r)$ eine Familie von
Elementen aus $V$.
\begin{itemize}
\item {\color{red} Linearkombination} $v \in V$ von $(v_1,\dots ,v_r)$: 
  falls $\exists \lambda_1, \dots, \lambda_r \in K$  mit
  $ v= \lambda_1 v_1 + \dots + \lambda_r v_r$. 
\item {\color{red} Lineare Hülle} $\mathop{span}\{v_1, \dots, v_n\}$: Die Menge aller Linearkombinationen. Die Lineare
  Hülle ist ein Unterraum von $V$.
\item {\color{red} linear unabhängig}: 
  Sind $\lambda_1, \dots , \lambda_r \in K$ und ist $\lambda_1 v_1 +
  \dots + \lambda_r v_r=0$ so folgt $\lambda_1= \dots =
  \lambda_r=0$. Andernfalls {\color{red} linear abhängig}. 
\begin{itemize}
\item Ist $M \subseteq V$ eine unendliche Menge, dann ist $M$ linear unabhängig falls alle endlichen Teilmengen von $M$ linear unabhängig sind.
\end{itemize}
\end{itemize}
\end{defn}

\begin{bemerk}
Sei $V$ ein $K$-Vektorraum und $(v_1,\dots ,v_r)$ eine Familie von
Elementen aus $V$
\begin{itemize}
\item $(v_1,\dots ,v_r)$ sind genau dann linear unabhängig, wenn sich
jeder Vektor $v \in span\{v_1, \dots ,v_r\}$ eindeutig linear kombinieren
läßt. 
\item Vektoren sind linear unabhängig wenn die Determinante der korrelierenden Matrix ungleich 0 ist.
\item Gilt $V=span\{v_1,\dots ,v_r \}$, so ist $(v_1, \dots ,v_r)$ ein
{\color{red} Erzeugendensystem}. Sind $(v_1, \dots ,v_r)$ zusätzlich linear
unabhängig, so ist $(v_1, \dots ,v_r)$ eine {\color{red} Basis}.
\item Aus jedem Erzeugendensystem kann man eine Basis auswählen. 
\end{itemize}
\end{bemerk}


\paragraph{Beispiele für Basen:}
\begin{itemize}
\item Seien $(e_i)_{i=1,\ldots,n} \in \mathbb{R}^n$ die Einheitsvektoren. $(e_1, \dots
,e_n)$ ist eine Basis des $\mathbb{R}^n$.
\item Die Monombasis $(1,x,x^2,\dots, x^n)$ ist eine Basis des
Vektorraums der Polynome $n$-ten Grades.
\item $(1,i)$ ist eine Basis von $\mathbb{C}$ als
$\mathbb{R}$-Vektorraum. 
\item $\mathbb{R}$ als $\mathbb{Q}$-Vektorraum hat keine endliche
Basis. 
\end{itemize}

\begin{defn}[Basis und Dimension] Jeder Vektorraum besitzt eine Basis.
\begin{itemize}
\item {\color{red} Dimension} des Vektorraums $V$: die  Anzahl der Basiselemente einer Basis $(v_1,\dots, v_n)$.
\item Seien $W,Z$ Unterräume von $V$. Dann ist {\color{red} $W+Z:=span(W \cup Z)$}
die {\color{red} Summe} von $W$ und $Z$. Es gilt:
\[ \dim(W+Z)=\dim(W) + \dim (Z) - \dim( W \cap Z) \]
\end{itemize}
\end{defn}



\begin{defn}[Normen auf Vektorräumen]
Sei $V$ ein Vektorraum über $K=\mathbb{R}$ oder $K=\mathbb{C}$.\\
Eine {\color{red} Norm} auf $V$ ist eine Abbildung
\[ \| \cdot \|: V \ \rightarrow \mathbb{R}, v \mapsto \| v \|, \]
so dass für  alle $\alpha \in K$, $u,v \in V$ gilt 
\[ \begin{array} {rcl}
\| v \| & \geq & 0 \\
\| v \| & = & 0 \mbox{ impliziert } v=0\\
\| \alpha v \| & = & | \alpha | \| v \|\\
\| u + v \| & \leq & \| u \| + \| v \| \mbox{ (Dreiecksungleichung)}.\\
\end{array} \]
$(V,\| \cdot \|)$
heißt {\color{red} normierter Raum}.
\end{defn}

\begin{defn}[Skalarprodukt]
Eine skalarwertige binäre Abbildung
\[ ( \cdot, \cdot ): V \times V: \ \rightarrow \ K\]
auf einem Vektorraum $V$ über $K=\mathbb{R}$ oder $K=\mathbb{C}$ heißt
{\color{red} Skalarprodukt}, wenn für alle $x,y,z \in V$, $\alpha, \beta \in
K$ gilt 
\begin{eqnarray*}
(x,x) & \geq & 0\\
(x,x) & = & 0 \mbox{ impliziert } x=0.\\
(x,y) & = & \overline{(y,x)}\\
(\alpha x+\beta y,z) & = & \alpha (x,z)+ \beta (y,z)
\end{eqnarray*}
\end{defn}

\paragraph{Bemerkungen:}

\begin{itemize}
\item Ein VR $V$ mit Skalarprodukt heißt {\color{red} Prä-Hilbert-Raum}. Ist
$K=\mathbb{R}$ so heißt der Raum auch {\color{red} euklidisch}.
\item Durch $\|v\|:=\sqrt{(v,v)}$, $v \in V$ läßt sich eine Norm
definieren. Es gilt die {\color{red} Cauchy-Schwarzsche Ungleichung}
\[ |(u,v)| \leq \| u \| \|v\|. \]
\item Im euklidischen Raum ist der Winkel $\alpha$ zwischen zwei Vektoren
$u,v \in V\smallsetminus \{ 0 \}$ definiert durch
\[ \cos(\alpha) = \frac{(u,v)}{\|u\| \|v \|}. \]
   \end{itemize}


\begin{itemize}
\item {\color{red} Orthogonal}: wenn $(u,v)=0$ gilt.
\item {\color{red} Orthogonalbasis}: Eine Basis aus paarweise orthogonalen Vektoren. 
\item {\color{red} Orthonormalbasis}: Eine Orthogonalbasis, bei der alle Vektoren die Norm $1$
haben.
\item Jeder endlichdimensionale Prä-Hilbert-Raum hat eine
Orthonormalbasis. 
\item {\color{red} Orthogonalraum}:
\[
{U}^\perp := \{ v \in V \ | \ (v,u)=0 \mbox{ für alle } u \in U \}
\]
wenn $U$ ein Unterraum von $V$ ist.
\item Es gilt: $\dim U + \dim U^\perp = \dim V$, insb. $U \cap U^\perp = 0$.
\end{itemize}
\end{document}
