\documentclass[a4paper,9pt,DIV15,twocolumn]{scrartcl}
\usepackage[xetex,bookmarks=true,pdfstartview=FitH,bookmarksopen=true,
    colorlinks,citecolor=Blue,linkcolor=DarkBlue,urlcolor=Green,
    pagebackref=true,plainpages=false,pdfpagelabels=true,unicode=true,
    breaklinks=true,naturalnames=false,setpagesize=true,a4paper=true,hyperindex]{hyperref}
\usepackage[svgnames,hyperref]{xcolor} %color definition
\usepackage{fontspec,xunicode,xltxtra}
%\usepackage{fontspec,xunicode}
%\usepackage{polyglossia}
%\setdefaultlanguage[spelling=new, latesthyphen=true]{german}
%\setsansfont{DejaVu Sans}
%\setsansfont{Verdana}
%\setsansfont{Arial}
%\setromanfont[Mapping=tex-text]{Linux Libertine}
%\setsansfont[Mapping=tex-text]{Myriad Pro}
%\setmonofont[Mapping=tex-text]{Courier New}

%\setsansfont{Linux Biolinum}

\usepackage[ngerman]{babel}
\selectlanguage{ngerman}

%
% math/symbols
%
\usepackage{amssymb}
\usepackage{amsthm}
% \usepackage{latexsym}
\usepackage{amsmath}
%\usepackage{amsxtra} %Weitere Extrasymbole
%\usepackage{empheq} %Gleichungen hervorheben
%\usepackage{bm}
 %\bm{A} Boldface im Mathemodus

\usepackage{multimedia}
%\usepackage{tikz}

\usepackage{cellspace}
\setlength{\cellspacetoplimit}{2pt}
\setlength{\cellspacebottomlimit}{2pt}

%%%%%%%%%%%%%%%%%% Fuer Frames [fragile]-Option verwenden!
%Programm-Listing
%%%%%%%%%%%%%%%%%%
%Listingsumgebung fuer verbatim
%Grauhinterlegeter Text
%Automatischer Zeilenumbruch ist aktiviert
\usepackage{listings}
% This command allows you to typeset syntax highlighted Matlab
% code ``inline''.
\newcommand{\isage}[1]{\lstinline|#1|}

\definecolor{lgray}{gray}{0.80}
\definecolor{gray}{gray}{0.3}
\definecolor{darkgreen}{rgb}{0,0.4,0}
\definecolor{darkblue}{rgb}{0,0,0.8}
\definecolor{key}{rgb}{0,0.5,0} 
%\lstset{backgroundcolor=\color{lgray}, frame=single, basicstyle=\ttfamily, breaklines=true}
\lstnewenvironment{sage}[1][]{\lstset{xleftmargin=0.2cm,frame=none,backgroundcolor=\color{white},basicstyle=\color{darkblue}\ttfamily\small,#1}}{} 
\lstnewenvironment{sagein}[1][]{\lstset{#1}}{} 
%\lstnewenvironment{sage}{\lstset{,language=python, keywordstyle=color{blue},    commentstyle=color{green}, emphstyle=\color{red}, %frame=single, stringstyle=\color{red}, basicstyle=\ttfamily, ,mathescape =true,escapechar=§}}{}

\lstset{
language=python,
backgroundcolor=\color{lgray},
breaklines=true,
basicstyle=\ttfamily\small,
%otherkeywords={ =},
keywordstyle=\color{blue},
stringstyle=\color{darkgreen},
showstringspaces=false,
emph={class, pass, in, for, while, if, is, elif, else, not, and, or,
def, print, exec, break, continue, return},
emphstyle=\color{blue},
emph={[2]True, False, None, self},
emphstyle=[2]\color{key},
emph={[3]from, import, as},
emphstyle=[3]\color{blue},
upquote=true,
morecomment=[s]{"""}{"""},
commentstyle=\color{gray}\slshape,
%framexleftmargin=1mm, framextopmargin=1mm, 
frame=single,
mathescape =true,
escapechar=§
}


\usepackage{mydef}
%\usepackage{cmap} % you can search in the pdf for umlauts and ligatures
\usepackage{colonequals} %corrects the definition-symbols \colonequals (besides others)
\usepackage{ifthen}
%%%%%%%%%%%%%%%%%%%
%Neue Definitionen
%%%%%%%%%%%%%%%%%%%

%Newcommands
\newcommand{\Fun}[1]{\mathcal{#1}}      %Mathcal fuer Funktoren
\newcommand{\field}[1]{\mathbb{#1}}     %Grundkoerper ?? in mathds

\newcommand{\A}{\field{A}}              %Affines A
\newcommand{\C}{\field{C}}              %Complexes C
\newcommand{\Fp}{\field{F}_{\!p}}       %Endlicher Koerper mit p Elementen
\newcommand{\Fq}{\field{F}_{\!q}}       %Endlicher Koerper mit q Elementen
\newcommand{\Ga}{\field{G}_{a}}         %Add Gruppenschema
\newcommand{\K}{\field{K}}              %Generischer Koerper 
\newcommand{\N}{\field{N}}              %Nat Zahlen
\newcommand{\Pj}{\field{P}}             %Projektives P
\newcommand{\R}{\field{R}} 		%Reelle Zahlen
\newcommand{\Q}{\field{Q}}              %Rationale Zahlen  
\newcommand{\Qt}{\field{H}}             %Quaternionen 
\newcommand{\V}{\field{V}}              %Vektorbuendel V
\newcommand{\Z}{\field{Z}}              %Ganze Zahlen
\DeclareMathOperator{\Real}{Re}

\newcommand{\fdg}{\;|\;}                 %fuer die gilt

%Operatoren
\DeclareMathOperator{\Abb}{Abb}
%\usepackage{sagetex}

%
% Aufgaben
%
\parindent0cm % Abs�tze nicht einr�cken 
% Definieren einer neuen Farbe
\definecolor{light-gray}{gray}{.9}

\newcounter{zaehler}     % neuen Z�hler einf�hren
\stepcounter{zaehler}    % Z�hler einen hochz�hlen

\newenvironment{aufg}[1][0]
%---- Header
{\begin{samepage}%
%\colorbox{light-gray}{%                         % Box in gray
% \makebox[\textwidth]{%                           % Box in linewidth
%\textbf{Aufgabe \arabic{zaehler} } }\hspace{-\textwidth}\makebox[\textwidth]{\hfill #1 Punkte} }\\[0.05cm]       % Header
\dotfill\\
{\large\textbf{Aufgabe \arabic{zaehler} }\ifthenelse{0=#1}{}{\hfill #1 Punkte} }\\[0.4cm]
\begin{minipage}{\textwidth}
}
%-----  foot
{\end{minipage} \nopagebreak %\begin{minipage}{1cm} \end{minipage}
%\\ 
%\begin{minipage}{0.1cm} \end{minipage} 
%\hrulefill \begin{minipage}{1cm} \end{minipage}\\[1cm]  
\stepcounter{zaehler}                           % increase counter
\end{samepage}%
\\%
\bigskip%
}


\begin{document}
%--------------------------------------------------- Header
\begin{center}
    \textbf{\LARGE Einführung in Sage}\\
    {\large Zusammenfassung Einheit 04}
\end{center}
\textsl{Hinweis:} Textbausteine mit <name> weisen darauf hin, das anstatt des Ausdrucks eine passende Variable eingefügt werden muss.\\
Die {\color{Green}grün} markierten Wörter Sind web-links zu der jeweiligen Dokumentation.

\medskip
\textbf{Matrix- } \href{https://sage.math.uni-goettingen.de/doc/static/reference/sage/matrix/constructor.html#sage.matrix.constructor.Matrix}{\textbf{matrix()}}
\begin{itemize}
\item Deklaration $n\times m$
\begin{sagein}
matrix([<Koerper>,<n>,<m>,[a11,a12,...],[a21,a22,..],..])
\end{sagein}
\begin{itemize}
\item Erzeuge Diagonalmatrix aus einem Vektor-\\ \href{https://sage.math.uni-goettingen.de/doc/static/reference/sage/matrix/constructor.html#sage.matrix.constructor.diagonal_matrix}{diagonal\_matrix()}
\begin{sagein}
diagonal_matrix(<vector>)
\end{sagein}
\end{itemize}
\item Zeilen- und Spaltenzahl
\begin{itemize}
 \item Spaltenzahl- \href{https://sage.math.uni-goettingen.de/doc/static/reference/sage/matrix/matrix0.html?highlight=matrix.rank#sage.matrix.matrix0.Matrix.ncols}{ncols()}
\begin{sagein}
<matrix>.ncols()
\end{sagein}
\item Zeilenzahl- \href{https://sage.math.uni-goettingen.de/doc/static/reference/sage/matrix/matrix0.html?highlight=matrix.rank#sage.matrix.matrix0.Matrix.nrows}{nrows()}
\begin{sagein}
<matrix>.nrows()
\end{sagein}
\end{itemize}
\item Informationen über überliegendes Objekt
\begin{sagein}
<matrix>.parent()
\end{sagein}
\item Zugriff
\begin{itemize}
 \item Eintrag in Zeile $i$ und Spalte $j$
\begin{sagein}
<matrix>[i,j] , <matrix>[i1:i2,j1:j2]
\end{sagein}
\item Manipulation Eintrag Zeile $i$, Spalte $j$
\begin{sagein}
<matrix>[i,j]=<Wert>
\end{sagein}
\item Extrahieren von Zeilen/Spalten
\begin{sagein}
<matrix>.row(0)
<matrix>.column(4)
\end{sagein}
\end{itemize}
\item Addition/Multiplikation
\begin{sagein}
A+B; A*B
\end{sagein}
\item Rang- \href{https://sage.math.uni-goettingen.de/doc/static/reference/sage/matrix/matrix0.html?highlight=matrix.rank#sage.matrix.matrix0.Matrix.rank}{rank()}
\begin{sagein}
<matrix>.rank()        
\end{sagein}
\item Symmetrie - \href{https://sage.math.uni-goettingen.de/doc/static/reference/sage/matrix/matrix0.html?highlight=matrix.rank#sage.matrix.matrix0.Matrix.is_symmetric}{is\_symmetric()}
\begin{sagein}
<matrix>.is_symmetric() 
\end{sagein}
\item Invertierbarkeit- \href{https://sage.math.uni-goettingen.de/doc/static/reference/sage/matrix/matrix0.html?highlight=matrix.rank#sage.matrix.matrix0.Matrix.is_invertible}{is\_invertible()}
    \begin{sagein}
<matrix>.is_invertible()
    \end{sagein}
\item Determinante- \href{https://sage.math.uni-goettingen.de/doc/static/reference/sage/matrix/matrix2.html?highlight=matrix.det#sage.matrix.matrix2.Matrix.det}{det()}
\begin{sagein}
<matrix>.det()
\end{sagein}
\item Rekursive Maps
\begin{sagein}
map_threaded(<function>,<matrix oder verschachtelte Listen>)
\end{sagein}
\end{itemize}

\textbf{Vektorraum- } \href{https://sage.math.uni-goettingen.de/doc/static/reference/sage/categories/vector_spaces.html#vector-spaces}{\textbf{vector spaces}}

\begin{itemize}
 \item Deklaration Vektor- \href{https://sage.math.uni-goettingen.de/doc/static/reference/sage/modules/free_module_element.html#sage.modules.free_module_element.vector}{vector()}
\begin{sagein}
 vector([v1,v2,..]) 
\end{sagein}

\end{itemize}

\textbf{Funktionen}

\begin{itemize}
 \item Deklaration- \href{http://docs.python.org/reference/compound_stmts.html#function-definitions}{def}
\begin{sagein}
def <Name><(a,b,..)>:
    """ <Hilfetext> """
    <Code-Block>
    return <ret>
\end{sagein}
\end{itemize}

\end{document}

