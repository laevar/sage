\documentclass[notes=hide,hyperref={dvipdfmx,pdfpagelabels=false}]{beamer}
\mode<article>
{
  \usepackage{fullpage}
  \usepackage{pgf}
  \usepackage[xetex]{hyperref}
  \setjobnamebeamerversion{beamer}
}

\mode<presentation>
{
  %\usetheme{Frankfurt}
 %\usetheme{My}
  \usetheme{Madrid}
  % or ...
%\usecolortheme{seagull}
  %\setbeamercovered{transparent}
  %\setbeamercovered{dynamic}
  % or whatever (possibly just delete it)
}
\usenavigationsymbolstemplate{}
\usefonttheme{structurebold}
\usepackage{multimedia}
%\usepackage{tikz}
\usepackage{fontspec,xunicode,xltxtra}

%\usepackage{polyglossia}
%\setdefaultlanguage[spelling=new, latesthyphen=true]{german}
%\setsansfont{DejaVu Sans}
%\setsansfont{Verdana}
%\setsansfont{Arial}
%\setromanfont{Linux Libertine O}
%\setsansfont{Linux Biolinum O}

\setbeamertemplate{footline}
{
\leavevmode
%\hbox{\begin{beamercolorbox}[wd=.5\paperwidth,ht=2.5ex,dp=1.125ex,
%leftskip=.3cm plus1fill,rightskip=.3cm]{author in head/foot}%
%    \usebeamerfont{author in head/foot}\insertshortauthor
%  \end{beamercolorbox}%
%  \begin{beamercolorbox}[wd=.5\paperwidth,ht=2.5ex,dp=1.125ex,leftskip=.3cm,
%rightskip=.3cm plus1fil]{title in head/foot}%
%    \usebeamerfont{title in head/foot}\insertshorttitle\hfill

\hfill\insertframenumber  \hspace{3pt}

%\inserttotalframenumber
%\hspace*{2ex}
%  \end{beamercolorbox}}%
  \vskip3pt%
}

\usepackage[ngerman]{babel}
\selectlanguage{ngerman}

%
% math/symbols
%
\usepackage{amssymb}
\usepackage{amsthm}
% \usepackage{latexsym}
\usepackage{amsmath}
%\usepackage{amsxtra} %Weitere Extrasymbole
%\usepackage{empheq} %Gleichungen hervorheben
%\usepackage{bm}
 %\bm{A} Boldface im Mathemodus

\usepackage{cellspace}
\setlength{\cellspacetoplimit}{2pt}
\setlength{\cellspacebottomlimit}{2pt}

%%%%%%%%%%%%%%%%%% Fuer Frames [fragile]-Option verwenden!
%Programm-Listing
%%%%%%%%%%%%%%%%%%
%Listingsumgebung fuer verbatim
%Grauhinterlegeter Text
%Automatischer Zeilenumbruch ist aktiviert
\usepackage{listings}
\definecolor{lgray}{gray}{0.80}
%\lstset{backgroundcolor=\color{lgray}, frame=single, basicstyle=\ttfamily, breaklines=true}
\lstnewenvironment{sage}{\lstset{backgroundcolor=\color{lgray},language=Python, emphstyle=\color{red}, frame=single, basicstyle=\ttfamily, breaklines=true,mathescape =true,escapechar=§}}{}


\usepackage{mydef}
%\usepackage{cmap} % you can search in the pdf for umlauts and ligatures
\usepackage{colonequals} %corrects the definition-symbols \colonequals (besides others)
\title{Einführung in Sage}
%
%\subtitle{Disputation} % (optional)

\author{Jochen Schulz}
% - Use the \inst{?} command only if the authors have different
%   affiliation.

\institute{Georg-August Universit\"at G\"ottingen \pgfimage[height=0.5cm]{../figures/unilogo3}}
% - Use the \inst command only if there are several affiliations.
% - Keep it simple, no one is interested in your street address.

\date{\today}

\subject{Sage}
% This is only inserted into the PDF information catalog. Can be left
% out. 

% If you have a file called "university-logo-filename.xxx", where xxx
% is a graphic format that can be processed by latex or pdflatex,
% resp., then you can add a logo as follows:

%\logo{\pgfimage[height=0.5cm]{figures/unilogo3}}


% Delete this, if you do not want the table of contents to pop up at
% the beginning of each subsection:

\AtBeginSection[]
{
  \begin{frame}<beamer>
    \frametitle{Aufbau}
    \tableofcontents[currentsection,currentsubsection]
  \end{frame}
}

\AtBeginSubsection[]
{
  \begin{frame}<beamer>
    \frametitle{Aufbau}
    \tableofcontents[currentsection,currentsubsection]
  \end{frame}
}



%%%%%%%%%%%%%%%%%%%
%Neue Definitionen
%%%%%%%%%%%%%%%%%%%

%Newcommands
\newcommand{\Fun}[1]{\mathcal{#1}}      %Mathcal fuer Funktoren
\newcommand{\field}[1]{\mathbb{#1}}     %Grundkoerper ?? in mathds

\newcommand{\A}{\field{A}}              %Affines A
\newcommand{\C}{\field{C}}              %Complexes C
\newcommand{\Fp}{\field{F}_{\!p}}       %Endlicher Koerper mit p Elementen
\newcommand{\Fq}{\field{F}_{\!q}}       %Endlicher Koerper mit q Elementen
\newcommand{\Ga}{\field{G}_{a}}         %Add Gruppenschema
\newcommand{\K}{\field{K}}              %Generischer Koerper 
\newcommand{\N}{\field{N}}              %Nat Zahlen
\newcommand{\Pj}{\field{P}}             %Projektives P
\newcommand{\R}{\field{R}} 		%Reelle Zahlen
\newcommand{\Q}{\field{Q}}              %Rationale Zahlen  
\newcommand{\Qt}{\field{H}}             %Quaternionen 
\newcommand{\V}{\field{V}}              %Vektorbuendel V
\newcommand{\Z}{\field{Z}}              %Ganze Zahlen

\newcommand{\fdg}{\;|\;}                 %fuer die gilt

%Operatoren
\DeclareMathOperator{\Abb}{Abb}
%\usepackage{sagetex}

\begin{document}
\lstset{basicstyle={\lstbasicfont\footnotesize}}


\subtitle{Einheit 6}
\maketitle

\begin{frame}{Aufbau}
\tableofcontents
\end{frame}

% \begin{frame}{Übersicht}
% \begin{itemize}
% \item Funktionen
% \item Funktionen in MuPAD
% \item Grenzwerte und Stetigkeit
% \item Funktionenfolgen
% \item Grafik
% \end{itemize}
% \end{frame}

%===================================================
\section{Funktionen}
%==================================================

\begin{frame}{Funktionen}
Man spricht von einer (reellen) {\color{red} Funktion}, wenn ein {\color{red}
  Definitionsbereich} $D \subset \mathbb{R}$, $D \neq \emptyset$
  gegeben ist und eine Vorschrift, die jedem $x \in D$ in eindeutiger
  Weise eine reelle Zahl $f(x)$ zuordnet. Man schreibt 
\[f: \ D \ \rightarrow \ \mathbb{R}.\]

Die Menge $f(D)$ ist die Menge aller rellen Zahlen, die als Werte der
Funktion vorkommen. Die Menge $f(D)$ wird als {\color{red} Wertebereich} bezeichnet. Der
{\color{red} Graph} einer Funktion ist die Menge aller Punkte 
\[ \{ (x,f(x)) \in \mathbb{R}^2 \;|\; x \in D\}. \]
\end{frame}

\begin{frame}[fragile]{Verknüpfungen}
Seien $f$ und $g$ Funktionen mit einem gemeinsamen Definitionsbereich. Dann
definiert man:
\begin{itemize}
\item Summe: $(f+g)(x):=f(x)+g(x)$
\item Differenz: $(f-g)(x):=f(x)-g(x)$
\item Produkt: $(f\cdot g)(x):=f(x) \cdot g(x)$
\item Quotient: $(\frac{f}{g})(x):=\frac{f(x)}{g(x)}$, falls $g(x) \neq
0$ für alle $x \in D$ 
\end{itemize}

Sind $f:D_f \rightarrow \mathbb{R}$ und $g:D_g \rightarrow \mathbb{R}$
mit $f(D_f) \subset D_g$ so ist die Komposition definiert durch:
\[(g \circ f) (x):=g(f(x)).\] 
\end{frame}

\begin{frame}{Mehrere Veränderliche}
Ist $D \subseteq \mathbb{R}^n$ und $f : D \Rightarrow \mathbb{R}$ dann spricht man von
einer reellen Funktion in \alert{mehreren Veränderlichen}. Das Studium dieser Funktionen ist einer der Hauptinhalte der Diff2-Vorlesung.

\bigskip

Weiterhin können Funktionen auch Wertebereiche außerhalb der reellen Zahlen haben.
Z.B. 
\[f : D \Rightarrow \mathbb{R}^m.\]
 Im physikalischen Umfeld spricht man für $m=1$ dann von \alert{skalarwertigen Funktionen} und für $m>1$ von \alert{vektorwertigen Funktionen} oder \alert{Vektorfeldern}.
 
\end{frame}


\begin{frame}[fragile]{Abbildungen in Sage I}
In Sage wird eine Abbildung $f$ durch einen Ausdruck der Form {\color{red}\verb+f(x,y,...)+} gebildet, wobei $x$,$y$ die sind.
\begin{sage}
>> f(x,y) = x^2+y^2; f
\end{sage}
\begin{sage}
(x, y) |--> x^2 + y^2
\end{sage}
Die so definierte Funktion $f$ kann wie jede beliebige andere Funktion
aufgerufen werden. Funktionen haben den Datentyp \verb+expression+.
\begin{sage}
>> _=var('a,b');f(a,b+1)
\end{sage}
\begin{sage}
(b + 1)^2 + a^2
\end{sage}
\begin{sage}
>> type(f)
\end{sage}
\begin{sage}
<type 'sage.symbolic.expression.Expression'>
\end{sage}
\end{frame}

\begin{frame}[fragile]{Abbildungen in Sage II}
Wie gewohnt können Abbildungen addiert, subtrahiert, multipliziert und
dividiert werden:
\begin{sage}
f(x) = 1/(1+x); g(x) = sin(x^2)
h = f+g; k = f*g; l = f/g
h(a),k(a),l(a)
\end{sage}
\begin{sage}
(1/(a + 1) + sin(a^2), sin(a^2)/(a + 1), 1/((a + 1)*sin(a^2)))
\end{sage}
\end{frame}


\begin{frame}[fragile]{Kompositionen in Sage I}
Eine Komposition $f\circ g$ wird in Sage durch Ineinanderschachteln gelöst.
\begin{sage}
f_g(x) = f(g); g_f(x) = g(f)
f_g(x), g_f(x)
\end{sage}
\begin{sage}
(1/(sin(x^2) + 1), sin((x + 1)^(-2)))
\end{sage}
Mehrfaches Hintereinanderschalten $f(f(\cdots f(\cdot)))=f \circ \dots
\circ f(\cdot)$ wird in Sage ebenso druchgeführt.
\begin{sage}
g4(x) = g(g(g(g))); g4
\end{sage}
\begin{sage}
x |--> sin(sin(sin(sin(x^2)^2)^2)^2)
\end{sage}
\end{frame}

\begin{frame}[fragile]{Kompositionen in Sage II}
Diese Konstruktionen funktionieren auch mit Systemfunktionen:
\begin{sage}
>> abs(real(-2+3*I))
\end{sage}
\begin{sage}
  2
\end{sage}
Kompliziertere Funktionen können besser durch selbst definierte Funktionen erklärt
werden. Dies sind im Wesentlichen kleine Programme, die mit
\verb+def <func>():+ beginnen (vgl. letzte Vorlesung).
\end{frame}

\begin{frame}[fragile]{Ausdrücke und Funktionen I}
Man kann in Sage wählen, ob man eine Funktion $f$ als Funktion oder
als Ausdruck darstellt. Die Funktionsauswertung ist i.A. allerdings unterschiedlich:
\begin{sage}
>> Funktion(x) = 2*x*cos(x); Funktion(1)
\end{sage}
\begin{sage}
  2*cos(1)
\end{sage}
\begin{sage}
Ausdruck = 2*x*cos(x); Ausdruck(x=1)
\end{sage}
\begin{sage}
  2*cos(1)
\end{sage}

\end{frame}

% \begin{frame}[fragile]{Ausdrücke und Funktionen II}
% Man kann durch Einsetzen aus einer Funktion einen Ausdruck machen:
% \begin{sage}
% >> bool(Funktion=Ausdruck), bool(Funktion(x)=Ausdruck)
% \end{sage}
% \begin{sage}
%   FALSE, TRUE
% \end{sage}
% Umgekehrt kann man aber auch aus einem Ausdruck eine Funktion
% machen. Dazu gibt es in der Bibliothek \verb+fp+ den Befehl
% \verb+unapply+. 
% \begin{sage}
% >> Ausdruck:=2*x*cos(x):
% >> (h:=fp::unapply(Ausdruck)), h(2), Ausdruck(2)
% \end{sage}
% \begin{sage}
%   x -> 2*x*cos(x), 4*cos(2), 2*x(2)*cos(x)(2)
% \end{sage}
% \end{frame}

\begin{frame}[fragile]{Ausdrücke und Funktionen II}
Auch mehrere Veränderliche  sind möglich:
\begin{sage}
_=var('y');Funktion2(x) = x+sin(y); Funktion2
\end{sage}
\begin{sage}
 x |--> x + sin(y)
\end{sage}
\begin{sage}
>> Funktion3(x,y) = x+sin(y); Funktion3
\end{sage}
\begin{sage}
  (x, y) |--> x + sin(y)
\end{sage}
\end{frame}

% \begin{frame}[fragile]{Warnung im Umgang mit Funktionen}
% 
% {\color{red} Funktionen werden bei der Zuweisung nicht vollständig ausgewertet.}
% \begin{sage}
% >> n:=1: f.n := x -> x^2+n
% \end{sage}
% \begin{sage}
%  x -> x^2 + n
% \end{sage}
% \begin{sage}
% >> f1(4); n:=0: f1(4)
%    17  16
% \end{sage}
% \begin{sage}
% >> (f.n := x -> x^2+n) $\text{dollar}$ n=4..5
% \end{sage}
% \begin{sage}
%   x -> x^2 + n, x -> x^2 + n
% \end{sage}
% Ausweg: Verwendung des Operators {\color{red}\verb+-->+}
% \begin{sage}
% >> (f.n := x --> x^2+n) $\text{dollar}$ n=4..5 
% \end{sage}
% \begin{sage}
%   x -> x^2 + 4, x -> x^2 + 5
% \end{sage} 
% \end{frame}

%----------------------------------------
\section{Grenzwerte und Stetigkeit}
%----------------------------------------

\begin{frame}{Grenzwerte von Funktionen}
Sei $f$ eine Funktion mit Definitionsbereich $D$ und $a\in D$.
$f$ strebt für $x \rightarrow a$ gegen den {\color{red} Grenzwert} $b \in
\mathbb{R}$, wenn es zu jedem $\varepsilon >0$ ein $\delta >0$ gibt, so
dass für alle $x \in D\smallsetminus\{a \}$ mit $|x-a|<\delta$ gilt 
\[ |f(x)-b| < \varepsilon .\]
Der Grenzwert $b$ ist eindeutig bestimmt und man schreibt
\[ \lim_{x \rightarrow a} f(x) =b \mbox{ oder } f(x) \rightarrow b
\mbox{ für } x \rightarrow a. \]
Die Aussage überträgt sich sinngemäß auf $a=\pm \infty$.
\end{frame}

\begin{frame}{Bemerkungen}
\begin{itemize}
\item Folgenkriterium: Es gilt $ \lim_{x \rightarrow a} f(x) =b$
genau dann, wenn für \alert{jede} Folge $a_n \in D$ mit $a_n \neq a$ und $a_n \rightarrow a$
gilt $\lim_{n \rightarrow \infty} f(a_n)=b$.
\item Es gelten die üblichen Rechenregeln:
\begin{eqnarray*}
\lim_{x \rightarrow a}(f(x)+g(x)) &=&\lim_{x \rightarrow a} f(x) +
\lim_{x \rightarrow a} g(x) \\
\lim_{x \rightarrow a}(f(x) \cdot g(x)) &=& \lim_{x \rightarrow a}
f(x) \cdot \lim_{x \rightarrow a} g(x)
\end{eqnarray*}
wenn $\lim_{x \rightarrow a} f(x)$ \alert{und} $\lim_{x \rightarrow a}g(x)$
\alert{existieren}. 
\item Gilt $\lim_{x \rightarrow a} f(x)=b$, $\lim_{x \rightarrow b}
g(x)=c$ bei entsprechenden Definitionsgebieten für $f$ und $g$, so
folgt $\lim_{x \rightarrow a} g(f(x)) =c$.
\end{itemize}
\end{frame}

\begin{frame}[fragile]{Sage}
Grenzwerte werden in Sage mit dem Befehl \verb+limit+ gebildet. 
Die Syntax des Befehls lautet
\begin{sage}
>> expr.limit(x = a, dir=None, taylor=False)
>> limit(expr, x = a, dir=None, taylor=False)
\end{sage}
Hierdurch wird der Grenzwert eines Ausdrucks mit Unbekannten $x$ an
der Stelle $a$ bestimmt. $a$ kann auch $\pm \infty$ sein (in
Sage \verb+infinity+ oder \verb+oo+).

Ruft man \verb+limit+ ohne \verb+option+ auf, so wird der beidseitige
Limes berechnet. Falls \verb+dir='minus'+ ist, wird der linksseitige
Limes berechnet; für \verb+dir='plus'+ der rechtsseitige.
\end{frame}

\begin{frame}[fragile]{Beispiele in Sage I}
\begin{itemize}
\item Bestimme den Grenzwert $\lim_{x \rightarrow 0}
\frac{\sin(x)}{x}$
\begin{sage}
>> limit(sin(x)/x,x=0)
\end{sage}
\begin{sage}
  1
\end{sage}
\item Bestimme den Grenzwert $\lim_{x \rightarrow \infty}
\frac{\log(x)}{x}$
\begin{sage}
>> limit(log(x)/x,x=infinity)
\end{sage}
\begin{sage}
  0
\end{sage}
\item Bestimme den Grenzwert $\lim_{x \rightarrow \infty} \sqrt[x]{x}$
\begin{sage}
>> limit(x^(1/x),x=infinity)
\end{sage}
\begin{sage}
  1
\end{sage}
\end{itemize}
\end{frame}

\begin{frame}[fragile]{Beispiele in Sage II}
\begin{itemize}
\item Bestimme den Grenzwert $\lim_{x \rightarrow 0}
\sin(1/x)$
\begin{sage}
>> limit(sin(1/x),x=0)
\end{sage}
\begin{sage}
  ind
\end{sage}
Der Grenzwert existiert nicht. Sage gibt in diesem Fall \verb+ind+ (indefinite) zurück. 
\item Bestimme den Grenzwert $\lim_{x \rightarrow 0} |x|' $
\begin{sage}
>> limit(diff(abs(x),x),x=0),
limit(diff(abs(x),x),x=0,dir='minus'),
limit(diff(abs(x),x),x=0,dir='plus')
\end{sage}
\begin{sage}
  (und, -1, 1)
\end{sage}
\end{itemize}
\end{frame}

\begin{frame}{Stetigkeit}
Eine Funktion $f:D \ \rightarrow  \ \mathbb{R}$ heißt {\color{red} stetig an
der Stelle $x_0 \in D$}, wenn es zu jedem $\varepsilon>0$ ein $\delta>0$
gibt, so dass für alle $x \in D$ mit $|x - x_0| < \delta$ gilt
\[ |f(x)-f(x_0) | < \varepsilon .\]
Man sagt, dass $f$ {\color{red} stetig} ist, wenn $f$ an jeder Stelle $x_0
\in D$ stetig ist. \\
Sind $f$ und $g$ an $x_0$ stetig, so auch $f+g$, $f-g$, $f \cdot g$
und $\frac{f}{g}$ (falls $g(x_0) \neq 0$). 
\end{frame}

\begin{frame}{Wichtige Sätze I}
\begin{itemize}
\item Sei $f$ auf einem offenen Intervall $I$ definiert. $f$ ist an
$x_0 \in I$ genau dann stetig, wenn gilt
\[ \lim_{x \rightarrow x_0} f(x) = f(x_0). \]
\item Für $f:I \rightarrow \mathbb{R}$ und $g:J \rightarrow
\mathbb{R}$ gelte $f(I) \subset J$ und es seien $f$ an $x_0 \in I$ und
$g$ an $y_0=f(x_0)$ stetig. Dann ist $g \circ f$ an $x_0$ stetig.
\item Eine Funktion $f: D \rightarrow \mathbb{R}$ ist {\color{red}
linksstetig} bzw. {\color{red} rechtsstetig}, wenn $f|_{D\cap (-\infty,x_0)}$
bzw  $f|_{D\cap (x_0,\infty)}$ an $x_0$ stetig ist. Eine Funktion $f$
ist dann an $x_0$ stetig, genau dann wenn $f$ links- und rechtsstetig
an $x_0$ ist.
\end{itemize}
\end{frame}

\begin{frame}{Wichtige Sätze II}
\begin{itemize}
\item Eine stetige Funktion auf einem abgeschlossenen Intervall $I=[a,b]$
besitzt ein Maximum und ein Minimum.
\item Eine stetige Funktion $f$ auf einem abgeschlossenen  Intervall
$[a,b]$ nimmt in $I$ jeden Wert zwischen $f(a)$ und $f(b)$ an.
\item Potenzreihen $f(x)=\sum_{n=0}^\infty a_n (x-x_0)^n$ sind stetig
innerhalb ihres Konvergenzintervalls.
\end{itemize}
\end{frame}


\begin{frame}{Gleichmäßige Stetigkeit}
$f: D \rightarrow \mathbb{R}$ heißt {\color{red} gleichmäßig stetig auf $D$},
wenn es zu jedem $\varepsilon >0$ ein $\delta>0$ gibt, so dass für alle
Paare $x,x_0 \in D$ mit $|x - x_0|< \delta$ gilt
\[ | f(x)-f(x_0)| < \varepsilon. \]
\begin{itemize}
\item Die Exponentialfunktion ist auf jedem kompakten Intervall
gleichmäßig stetig (aber nicht auf ganz $\mathbb{R}$). 
\item $\log:(0,1) \rightarrow \mathbb{R}$ ist stetig aber nicht
gleichmäßig stetig.
\end{itemize}
\end{frame}

\begin{frame}[fragile]{Stetigkeit in MuPAD}
Für die Diskussion der Stetigkeit einer Funktion $f$ an einer Stelle
$x_0$ sei auf den Abschnitt zu Grenzwerten verwiesen. \\
(Komplexe) Unstetigkeitsstellen oder Definitionslücken können mittels
\verb+discont+ aufgespürt werden:
\begin{sage}
>> discont(sin(1/x)*x,x)
\end{sage}
\begin{sage}
  {0}
\end{sage}
\begin{sage}
>> discont(exp(x),x)
\end{sage}
\begin{sage}
  {}
\end{sage}
\begin{sage}
>> discont(tan(x),x)
\end{sage}
\begin{sage}
  { 1/2*PI + X1*PI |  X1 in Z_ }
\end{sage}
\end{frame}

\begin{frame}[fragile]{Stetigkeit in MuPAD}
\begin{sage}
>> discont(1/sin(x),x=-1..10)
\end{sage}
\begin{sage}
  {PI, 0, 2 PI, 3 PI}
\end{sage}
Vorsicht! Es werden komplexe Unstetigkeitsstellen gesucht!
\begin{sage}
>> discont(ln(x),x)
\end{sage}
\begin{sage}
  (-infinity, 0]
\end{sage}
\begin{sage}
>> discont(ln(x),x,Dom::Real)
\end{sage}
\begin{sage}
  {0}
\end{sage}
\begin{sage}
>> ln(-2)
\end{sage}
\begin{sage}
  I PI + ln(2)
\end{sage}
\end{frame}

%----------------------------------------
\section{Funktionenfolgen}
%----------------------------------------

\begin{frame}{Funktionenfolgen}
Seien $f_n: D \ \rightarrow \ \mathbb{R}$, $n \in
\mathbb{N}$  rellwertige Funktionen auf  $D \subset \mathbb{R}$.
\begin{itemize}
\item $(f_n)_n$ heißt {\color{red} Funktionenfolge.}
\item Ist für jedes $x\in D$ die Folge $(f_n(x))_n$ konvergent, so wird durch 
\[ f(x):= \lim_{n \rightarrow \infty} f_n(x), \quad x \in D \]
die {\color{red} Grenzfunktion} $f:D \ \rightarrow \ \mathbb{R}$ definiert.
\item Man sagt $f_n$ strebe {\color{red} punktweise} auf $D$ gegen $f$.  
\item Durch $\sum_{i=1}^\infty f_i$ definierte {\color{red} Funktionenreihen}
sind spezielle Funktionenfolgen.
\end{itemize}  
\end{frame}

\begin{frame}{Beispiele: Grenzübergänge}
\begin{itemize}
\item $x^n \rightarrow 0$ auf dem Intervall $(-1,1)$.
\item $\left( 1+ \frac{x}{n} \right)^n \rightarrow \exp(x)$ auf $\mathbb{R}$.
\item Potenzreihen konvergieren innerhalb ihres Konvergenzradius.
\item {\color{red} Warnung} zum Vertauschen der Grenzprozesse für $x \in (0,1)$:
\[ \lim_{x \rightarrow 1} \lim_{n \rightarrow \infty} x^n =0 \neq 1 = 
  \lim_{n \rightarrow \infty} \lim_{x \rightarrow 1} x^n.\]  
\end{itemize}
\end{frame}

\begin{frame}[fragile]{Gleichmäßige Konvergenz}
\begin{Definition}
$(f_n)_n$ konvergiert {\color{red} gleichmäßig} auf $D$ gegen $f$, wenn es zu
jedem $\varepsilon >0$ ein $n_0 \in \mathbb{N}$ gibt, so dass für alle
$x \in D$ und $n\geq n_0$ gilt:
\[ |f_n(x) -f(x)| < \varepsilon.\]
\end{Definition}

\begin{Satz}
 Konvergiert $(f_n)_n$ gleichmäßig auf $D$ und existiert $\lim_{x
\rightarrow a} f_n(x)$ für $a\in D$, so gilt:
\[ \lim_{x \rightarrow a} \lim_{n \rightarrow \infty} f_n(x) = \lim_{n
\rightarrow \infty} \lim_{x \rightarrow a} f_n(x). \]
\end{Satz}
\end{frame}

\begin{frame}{Bemerkungen}
\begin{itemize}
\item Die Grenzfunktion einer gleichmäßig konvergenten Folge stetiger
Funktionen ist stetig.
\item Alle Aussagen übertragen sich analog auf Funktionenreihen: Ist $f_1, f_2, \ldots$, eine Folge von Funktionen auf $D \subseteq \mathbb{R}$ dann definiert
\[
 s := \sum_{n=1}^\infty f_n
\]
eine Funktionenreihe. Aussagen über die Funktionenreihe sind, analog zum Fall der \glqq normalen\grqq\ Reihen, Aussagen über die Folge der Partialsummen
\[
 s_k := \sum_{n=1}^k f_n.
\]
 
 
\end{itemize}
\end{frame}

%------------- ---------------------------
\section{Grafiken}
%----------------------------------------

\begin{frame}[fragile]{Grafiken}
\begin{itemize}
\item Die Funktionen \verb+plotfunc2d+ und \verb+plotfunc3d+ dienen
             zur Darstellung von Graphen von Funktionen mit einem
             bzw. zwei Argumenten. 
\item Grafiken werden im Notebook integriert.
\item Die Grafiken können interaktiv bearbeitet werden.
\item Die meisten Grafikbefehle sind in der Bibliothek \verb+plot+
             untergebracht. 
\end{itemize}
\end{frame} 

\begin{frame}[fragile]{plotfunc2d}
Skalare Funktionen $f1,f2,\ldots$, können durch den Befehl
\begin{sage}
>> plotfunc2d(f1,f2,...,x=a..b,Mesh=n)
\end{sage}
auf dem Intervall $[a,b]$ mit Hilfe von $n$ Gitterpunkten gezeichnet
werden. Dabei ist die Angabe  von
\verb+x=a..b+ und \verb+Mesh=n+ optional. Beispiele:
\begin{sage}
>> plotfunc2d(x^2-1)
>> plotfunc2d(sin(1/x),x=-1..1)
>> plotfunc2d(sin(1/x),x=-1..1,Mesh=21)
>> plotfunc2d(sin(x),cos(x),Mesh=10)
\end{sage}
\end{frame}

\begin{frame}[fragile]{plotfunc3d}
Es werden Funktionen
$f:\mathbb{R}^2 \ \rightarrow \mathbb{R}$
grafisch dargestellt. (Genauer gesagt wird auch hier der Funktionsgraph als Teilmenge des $\mathbb{R}^3$ gezeichnet). Der Funktionsaufruf ist
\begin{sage}
>> plotfunc3d(f1,f2,...,x=a..b,y=c..d,
   Mesh=[nx,ny])
\end{sage}
Hierdurch werden  Ausdrücke $f1, f2,\ldots$, auf 
$[a,b]\times [c,d]$ mit Hilfe  von $nx$ Gitterpunkten in $x$-Richtung
und $ny$ Punkten in $y$-Richtung dargestellt. Default ist $a=c=-5$,
$b=d=5$ und $nx=ny=20$.  
\end{frame}

\begin{frame}[fragile]{Beispiele}
Plot von $f(x,y)=\sin(y^2+x)-\cos(y-x^2)$ auf $[0,\pi]^2$:
\begin{sage}
>> f:=sin(y^2+x)-cos(y-x^2):
>> plotfunc3d(f,x=0..PI,y=0..PI)
>> plotfunc3d(f,x=0..PI,y=0..PI,Mesh=[60,30])
\end{sage}
\medskip
Plot von $f(x,y)=\cos (\mbox{20} \exp(-x^2-y^2 ))$ auf $[-1,1] \times [-1,1]$:
\begin{sage}
>> g:=cos(20*exp(-x^2-y^2)):
>> plotfunc3d(g,x=-1..1,y=-1..1)
>> plotfunc3d(g,x=-1..1,y=-1..1,Mesh=[60,60])
\end{sage}
\end{frame}

\begin{frame}[fragile]{Grafische Szene}
Die Herangehensweise zur Erstellung komplexer Grafiken ist etwas
ungewohnt. Grafiken werden nicht direkt erzeugt, d.h. ausgegeben. 

Anstatt dessen werden
zuerst {\color{red} grafische Objekte} wie Geraden, Funktionsgraphen oder Kurven als
grafische Objekte erzeugt. 

Erst zuletzt werden die Objekte zu einer gemeinsamen {\color{red} grafischen
Szene} zusammengefaßt und durch den Befehl \verb+plot+ gezeichnet.
\begin{sage}
>> plot(Objekt1,Objekt2,...,SzeneOption1, SzeneOption2,...)
\end{sage}
\end{frame}

\begin{frame}[fragile]{Optionen für grafische Szenen}
\begin{tabular}{lp{8cm}}
ViewingBox & Dargestellter Achsenbereich \newline
              {\color{blue} \verb+ViewingBox = [x1..x2,y1..y2]+}\\
Scaling    & Verhältnis der Achsen
             \verb+Constrained+ (verzerrt), \verb+Unconstrained+
              (unverzerrt)\newline
                  {\color{blue} \verb+Scaling = Unconstrained+}\\          
Header & Überschrift der Grafik \newline
{\color{blue} \verb+Header = ''Ein Titel''+}\\
GridVisible & Sichtbarkeit von Gitterlinien\newline
              {\color{blue} \verb+GridVisible = TRUE+}
\end{tabular}
\end{frame}

\begin{frame}[fragile]{Optionen für grafische Objekte}
\begin{tabular}{lp{8cm}}
LineStyle  & Darstellung von Linien\newline
           (\verb+Solid+ (Def.), \verb+Dashed+, \verb+Dotted+)\newline
                {\color{blue} \verb+LineStyle = Dashed+}\\
LineWidth  & Linienstärke in mm\newline
              {\color{blue} \verb+LineWidth = 4+}\\
Color      & Zuweisung einer Farbe\newline
              {\color{blue} \verb+Color = RGB::Red+}\\
Mesh       & Anzahl Stützstellen\newline
              {\color{blue} \verb+Mesh = [nx,ny]+} (2 Parameter)\\
Legend     & Legende\newline
     {\color{blue} \verb+Legend = "Ein Titel"+}\\
\end{tabular}
\end{frame}

\begin{frame}[fragile]{Zweidimensionale Kurven}
Eine Kurve des $\mathbb{R}^2$ in Parameterdarstellung sei gegeben durch Funktionen $x(t),y(t)$, also die Menge aller Punkte: 
\[
 \{(x(t),y(t)) \in \mathbb{R}^2 \;|\; t \in [a,b]\}.
\]
 Zum Beispiel ergibt sich der Graph einer Funktion $f(x)$, $x
\in [a,b]$ durch $t,f(t)$ mit  $t \in
[a,b]$. Der Befehl zum Erzeugen des Objekts ist 
\begin{sage}
>> Objekt:=plot::Curve2d([x,y],t=a..b)
\end{sage}
$x$ und $y$ sind Ausdrücke mit der Unbekannten $t$. Zusätzlich können
noch Optionen übergeben werden.
\end{frame}

\begin{frame}[fragile]{Beispiele}
\begin{sage}
>> f1:=plot::Curve2d([t,sin(t)],t=0..2*PI):
>> f2:=plot::Curve2d([t,cos(t)],t=0..2*PI):
>> f3:=plot::Curve2d([cos(t),sin(t)],t=0..2*PI):
>> SzeneOptionen:=Header="EinBeispiel",
   HeaderFont=[24],Width=20*unit::cm,
   Height=10*unit::cm: 
>> plot(f1,f2,f3,SzeneOptionen)
\end{sage}
\begin{sage}
>> vielecke:=plot::Curve2d([cos(t),sin(t)],
      t=0..2*PI,Mesh=2*i+1 ) $\text{dollar}$ i=2..10:
>> plot(vielecke)
\end{sage}
\end{frame}

\begin{frame}[fragile]{Dreidimensionale Kurven}
Das Erzeugen einer dreidimensionaler Kurve, d.h. einer Kurve des $\mathbb{R}^3$ geschieht analog durch die Angabe von drei Funktionen $x(t),y(t),z(t)$:
\[
 \{(x(t),y(t),z(t)) \in \mathbb{R}^3 \;|\; t \in [a,b] \}.
\]
Der Befehl ist  
\begin{sage}
Objekt:=plot::Curve3d([x,y,z],t=a..b)
\end{sage}
$x$, $y$ und $z$ sind Ausdrücke mit der Unbekannten $t$. Zusätzlich können
noch Optionen übergeben werden. Beispiel:
\begin{sage}
>> objekt:=plot::Curve3d(
   [(1-t*t)*cos(99*t), 
    (1-t*t)*sin(99*t),t], 
   t=0..1,Mesh=400):
>> plot(objekt)
\end{sage}
\end{frame}

\begin{frame}[fragile]{Flächen}
\begin{itemize}
\item Typische dreidimensionale Grafikobjekte sind {\it parametrisierte Flächen}.
\item Die $x,y,z$ Koordinaten sind als Funktionen $x(t_1,t_2)$,
   $y(t_1,t_2)$, $z(t_1,t_2)$ zweier Parameter $t_1 \in [a,b]$ und $t_2 \in [c,d]$
   definiert. 
\end{itemize}
\end{frame}

\begin{frame}[fragile]{Flächen}
\begin{itemize}
\item Beispielsweise lassen sich Graphen von Funktionen $f:[a
   ,b]\times [c,d]
   \ \rightarrow \ \mathbb{R}$ als Flächen
\[ x=t_1, \ y=t_2, \ z=f(t_1,t_2) \]
mit $a \leq t_1 \leq b, \ c \leq
   t_2 \leq d$ erklären.
\item Die Oberfläche einer Kugel mit Radius $r$ ist gegeben durch:
\[ x=r \cos(t_1) \sin(t_2), \ y=r \sin (t_1) \sin (t_2), z =r
  \cos(t_2) \]
 mit $0 \leq t_1 \leq 2 \pi, 0 \leq t_2 \leq \pi.$ 
\end{itemize}
\end{frame}

\begin{frame}[fragile]{Flächen}
Flächen werden in MuPAD erzeugt durch
\begin{sage}
plot::Surface(
  [x(t1,t2), y(t1,t2), z(t1,t2)],
  t1=a..b, t2=c..d, Optionen):
\end{sage}
\bigskip
Beispiel: Kugeloberfläche
\begin{sage}
>> x:=r*cos(t1)*sin(t2):
>> y:=r*sin(t1)*sin(t2):
>> z:=r*cos(t2):
>> r:=1
>> Objekt:=plot::Surface(
   [x,y,z], t1=0..2*PI, t2=0..PI,
  MeshVisible=TRUE, Filled=FALSE,
  Mesh=[10,30])
>> plot(Objekt)
\end{sage}
\end{frame}

\begin{frame}[fragile]{Konturen}
\begin{itemize}
\item Man kann für Funktionen $f:\mathbb{R}^2 \rightarrow \mathbb{R}$
auch zweidimensionale Grafiken erstellen.
\item Man zeichnet die Niveaulinien (wie die Höhenmeter auf einer
Landkarte), d.h. man sucht zu $c \in \mathbb{R}$ die Menge von Punkten
$(x,y)$ mit $f(x,y)=c$.
\item In MuPAD geschieht das mit dem Befehl
\begin{sage}
>> plot::Implicit2d(f,x=a..b,y=a..b,
    Contours=[c1,c2,...], Optionen)
\end{sage}
Dabei geben $c1,c2,\ldots$ die entsprechenden Niveaulinien an.
\end{itemize}
\end{frame}

\begin{frame}[fragile]{Beispiel}
Wir betrachten die Funktion $\sin(4\pi x)y$ und zeichnen die Niveaulinien für $-0.5, 0, 0.5$. 
\begin{sage}
>> objekt:=plot::Implicit2d(
  sin(PI*4*x)*y,x=-1..1,y=-1..1,
       Mesh=[10,10],
       Contours = [-0.5, 0, 0.5])
>> plot(objekt)
\end{sage}
Die Angabe vieler Stützstellen mit Hilfe von \verb+Mesh+ erhöht hier die Genauigkeit (und die Rechenzeit) der Niveaulinien zum Teil gravierend!
\end{frame}

\begin{frame}[fragile]{Punkte zeichnen}
Mittels {\color{blue} \verb+plot::PointList2d+} 
können Punkte gezeichnet werden.

Beispiel:
\begin{sage}
>> Objekt:=plot::PointList2d(
 [[i,sin(i*6.28/50)] $\text{dollar}$ i=0..50],
 PointStyle=FilledSquares, 
 PointSize=2,
 Color=RGB::Blue): 
>> plot(Objekt) 
\end{sage}
\end{frame}

\begin{frame}{Ein Beispiel: Das Collatz Problem}

Sei $x_0\in \mathbb{N}$. Dann definiert man die folgende Folge
\[ x_n:= \left \{ \begin{array}{ll}
 x_{n-1}/2, & \mbox{falls } x_{n-1} \mbox{ gerade ist} \\
3x_{n-1}+1 & \mbox{falls } x_{n-1} \mbox{ ungerade ist} 
\end{array} \right. . \] 
Man kann zeigen, dass für alle Startwerte ein $N_0$ existiert mit $x_{N_0}=1$.
\end{frame}


\begin{frame}[fragile]{Programm}
\begin{small}
\begin{sage}
collatz:=proc(n)
begin /* Collatz problem */
   sequence := [n]; next_value := n;
   while next_value > 1 do
         if next_value mod 2 =0 then
            next_value := next_value/2;
         else
             next_value := 3*next_value+1;
         end_if;
         sequence := sequence.[next_value];
   end_while;
   Objekt:=plot::PointList2d([[i,sequence[i]] 
           $\text{dollar}$ i=1..nops(sequence)],
           PointStyle=FilledSquares,PointSize=1): 
   plot(Objekt);   
   return(sequence);
end_proc
\end{sage}
\end{small}
\end{frame}

\begin{frame}[fragile]{Animationen}
Es ist mit MuPAD sehr einfach Animationen zu erstellen. Man betrachte die
folgenden beiden Beispiele:
\begin{itemize}
\item Zweidimensionales Beispiel
\begin{sage}
>> plotfunc2d(a*x^2, x=-5..5, a=-10..10)
\end{sage}
\item Dreidimensionales Beispiel
\begin{sage}
>> plotfunc3d(cos(j^0.5*PI*exp(-x^2-y^2)), 
   x=-1..1, y=-1..1, j=1..30, Mesh=[40,40])
\end{sage}
\end{itemize}
\end{frame}
\end{document}
