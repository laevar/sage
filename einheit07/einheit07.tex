\documentclass[notes=hide,hyperref={dvipdfmx,pdfpagelabels=false}]{beamer}
\usepackage[xetex,dvipdfmx]{animate}
\title{Einführung in Sage - Einheit 7}
\subtitle{Funktionen, Grenzwerte, Funktionenfolgen, Grafiken}
\mode<article>
{
  \usepackage{fullpage}
  \usepackage{pgf}
  \usepackage[xetex]{hyperref}
  \setjobnamebeamerversion{beamer}
}

\mode<presentation>
{
  %\usetheme{Frankfurt}
 %\usetheme{My}
  \usetheme{Madrid}
  % or ...
%\usecolortheme{seagull}
  %\setbeamercovered{transparent}
  %\setbeamercovered{dynamic}
  % or whatever (possibly just delete it)
}
\usenavigationsymbolstemplate{}
\usefonttheme{structurebold}
\usepackage{multimedia}
%\usepackage{tikz}
\usepackage{fontspec,xunicode,xltxtra}

%\usepackage{polyglossia}
%\setdefaultlanguage[spelling=new, latesthyphen=true]{german}
%\setsansfont{DejaVu Sans}
%\setsansfont{Verdana}
%\setsansfont{Arial}
%\setromanfont{Linux Libertine O}
%\setsansfont{Linux Biolinum O}

\setbeamertemplate{footline}
{
\leavevmode
%\hbox{\begin{beamercolorbox}[wd=.5\paperwidth,ht=2.5ex,dp=1.125ex,
%leftskip=.3cm plus1fill,rightskip=.3cm]{author in head/foot}%
%    \usebeamerfont{author in head/foot}\insertshortauthor
%  \end{beamercolorbox}%
%  \begin{beamercolorbox}[wd=.5\paperwidth,ht=2.5ex,dp=1.125ex,leftskip=.3cm,
%rightskip=.3cm plus1fil]{title in head/foot}%
%    \usebeamerfont{title in head/foot}\insertshorttitle\hfill

\hfill\insertframenumber  \hspace{3pt}

%\inserttotalframenumber
%\hspace*{2ex}
%  \end{beamercolorbox}}%
  \vskip3pt%
}

\usepackage[ngerman]{babel}
\selectlanguage{ngerman}

%
% math/symbols
%
\usepackage{amssymb}
\usepackage{amsthm}
% \usepackage{latexsym}
\usepackage{amsmath}
%\usepackage{amsxtra} %Weitere Extrasymbole
%\usepackage{empheq} %Gleichungen hervorheben
%\usepackage{bm}
 %\bm{A} Boldface im Mathemodus

\usepackage{cellspace}
\setlength{\cellspacetoplimit}{2pt}
\setlength{\cellspacebottomlimit}{2pt}

%%%%%%%%%%%%%%%%%% Fuer Frames [fragile]-Option verwenden!
%Programm-Listing
%%%%%%%%%%%%%%%%%%
%Listingsumgebung fuer verbatim
%Grauhinterlegeter Text
%Automatischer Zeilenumbruch ist aktiviert
\usepackage{listings}
\definecolor{lgray}{gray}{0.80}
%\lstset{backgroundcolor=\color{lgray}, frame=single, basicstyle=\ttfamily, breaklines=true}
\lstnewenvironment{sage}{\lstset{backgroundcolor=\color{lgray},language=Python, emphstyle=\color{red}, frame=single, basicstyle=\ttfamily, breaklines=true,mathescape =true,escapechar=§}}{}


\usepackage{mydef}
\usepackage{cmap} % you can search in the pdf for umlauts and ligatures
\usepackage{colonequals} %corrects the definition-symbols \colonequals (besides others)
\title{Einführung in Sage}
%
%\subtitle{Disputation} % (optional)

\author{Jochen Schulz}
% - Use the \inst{?} command only if the authors have different
%   affiliation.

\institute{Georg-August Universit\"at G\"ottingen \pgfimage[height=0.5cm]{../figures/unilogo3}}
% - Use the \inst command only if there are several affiliations.
% - Keep it simple, no one is interested in your street address.

\date{\today}

\subject{Sage}
% This is only inserted into the PDF information catalog. Can be left
% out. 

% If you have a file called "university-logo-filename.xxx", where xxx
% is a graphic format that can be processed by latex or pdflatex,
% resp., then you can add a logo as follows:

%\logo{\pgfimage[height=0.5cm]{figures/unilogo3}}


% Delete this, if you do not want the table of contents to pop up at
% the beginning of each subsection:

\AtBeginSection[]
{
  \begin{frame}<beamer>
    \frametitle{Aufbau}
    \tableofcontents[currentsection,currentsubsection]
  \end{frame}
}

\AtBeginSubsection[]
{
  \begin{frame}<beamer>
    \frametitle{Aufbau}
    \tableofcontents[currentsection,currentsubsection]
  \end{frame}
}



%%%%%%%%%%%%%%%%%%%
%Neue Definitionen
%%%%%%%%%%%%%%%%%%%

%Newcommands
\newcommand{\Fun}[1]{\mathcal{#1}}      %Mathcal fuer Funktoren
\newcommand{\field}[1]{\mathbb{#1}}     %Grundkoerper ?? in mathds

\newcommand{\A}{\field{A}}              %Affines A
\newcommand{\C}{\field{C}}              %Complexes C
\newcommand{\Fp}{\field{F}_{\!p}}       %Endlicher Koerper mit p Elementen
\newcommand{\Fq}{\field{F}_{\!q}}       %Endlicher Koerper mit q Elementen
\newcommand{\Ga}{\field{G}_{a}}         %Add Gruppenschema
\newcommand{\K}{\field{K}}              %Generischer Koerper 
\newcommand{\N}{\field{N}}              %Nat Zahlen
\newcommand{\Pj}{\field{P}}             %Projektives P
\newcommand{\R}{\field{R}} 		%Reelle Zahlen
\newcommand{\Q}{\field{Q}}              %Rationale Zahlen  
\newcommand{\Qt}{\field{H}}             %Quaternionen 
\newcommand{\V}{\field{V}}              %Vektorbuendel V
\newcommand{\Z}{\field{Z}}              %Ganze Zahlen

\newcommand{\fdg}{\;|\;}                 %fuer die gilt

%Operatoren
\DeclareMathOperator{\Abb}{Abb}
%\usepackage{sagetex}

\begin{document}
\lstset{basicstyle={\lstbasicfont\footnotesize}}


\begin{document}
\maketitle

\begin{frame}{Aufbau}
\tableofcontents
\end{frame}


%===================================================
\section{Funktionen (mathematische)}
%==================================================

\begin{frame}{Funktionen}
(reelle) {\color{red} Funktion}:  Abbildung
\[f: \ D \subset \mathbb{R} \ \rightarrow \ \mathbb{R}.\]
die jedem Element aus $D$ eindeutig genau ein Element aus $\mathbb{R}$ zuordnet.
\begin{itemize}
 \item {\color{red} Definitionsbereich}: $D \subset \mathbb{R}$, $D \neq \emptyset$.
\item \alert{Wertebereich}: Die Menge $f(D)$ aller rellen Zahlen, die als Werte der
Funktion vorkommen.
\item {\color{red} Graph} einer Funktion: ist die Menge aller Punkte 
\[ \{ (x,f(x)) \in \mathbb{R}^2 \;|\; x \in D\}. \]
\end{itemize}
\end{frame}

\begin{frame}[fragile]{Verknüpfungen}
Seien $f$ und $g$ Funktionen mit einem gemeinsamen Definitionsbereich. Dann
definiert man:
\begin{itemize}
\item Summe: $(f+g)(x):=f(x)+g(x)$
\item Differenz: $(f-g)(x):=f(x)-g(x)$
\item Produkt: $(f\cdot g)(x):=f(x) \cdot g(x)$
\item Quotient: $(\frac{f}{g})(x):=\frac{f(x)}{g(x)}$, falls $g(x) \neq
0$ für alle $x \in D$ 
\item Komposition: Mit $f:D_f \rightarrow \mathbb{R}$ und $g:D_g \rightarrow \mathbb{R}$
mit $f(D_f) \subset D_g$ 
\[(g \circ f) (x):=g(f(x)).\] 
\end{itemize}

\end{frame}

\begin{frame}{Funktionen mehrerer Veränderlicher}
Ist $D \subseteq \mathbb{R}^n$ und $f : D \Rightarrow \mathbb{R}$ dann spricht man von
einer reellen Funktion in \alert{mehreren Veränderlichen}. Das Studium dieser Funktionen ist einer der Hauptinhalte der Diff2-Vorlesung.

\bigskip

Weiterhin können Funktionen auch Wertebereiche außerhalb der reellen Zahlen haben.
Z.B. 
\[f : D \Rightarrow \mathbb{R}^m.\]
 Im physikalischen Umfeld spricht man für $m=1$ dann von \alert{skalarwertigen Funktionen} und für $m>1$ von \alert{vektorwertigen Funktionen} oder \alert{Vektorfeldern}.
 
\end{frame}


\begin{frame}{Sage}
    \begin{center}
        \url{https://sage.math.uni-goettingen.de/home/pub/42/}
    \end{center}
\end{frame}



%----------------------------------------
\section{Grenzwerte und Stetigkeit}
%----------------------------------------

\begin{frame}{Grenzwerte von Funktionen}

{\color{red} Grenzwert}: Sei $f$ eine Funktion mit Definitionsbereich $D$ und $a\in D$.\\
$f$ strebt für $x \rightarrow a$ gegen $b \in \mathbb{R}$, wenn es zu jedem $\varepsilon >0$ ein $\delta >0$ gibt, so
dass für alle $x \in D\smallsetminus\{a \}$ mit $|x-a|<\delta$ gilt 
\[ |f(x)-b| < \varepsilon .\]
Der Grenzwert $b$ ist eindeutig bestimmt und man schreibt
\[ \lim_{x \rightarrow a} f(x) =b \mbox{ oder } f(x) \rightarrow b
\mbox{ für } x \rightarrow a. \]
Die Aussage überträgt sich sinngemäß auf $a=\pm \infty$.
\end{frame}

\begin{frame}{Bemerkungen}
\begin{itemize}
\item \alert{Folgenkriterium}: Es gilt $ \lim_{x \rightarrow a} f(x) =b$
genau dann, wenn für jede Folge $a_n \in D$ mit $a_n \neq a$ und $a_n \rightarrow a$
gilt $\lim_{n \rightarrow \infty} f(a_n)=b$.
\item Es gelten die üblichen Rechenregeln:
\begin{eqnarray*}
\lim_{x \rightarrow a}(f(x)+g(x)) &=&\lim_{x \rightarrow a} f(x) +
\lim_{x \rightarrow a} g(x) \\
\lim_{x \rightarrow a}(f(x) \cdot g(x)) &=& \lim_{x \rightarrow a}
f(x) \cdot \lim_{x \rightarrow a} g(x)
\end{eqnarray*}
wenn $\lim_{x \rightarrow a} f(x)$ und $\lim_{x \rightarrow a}g(x)$
existieren. 
\item Gilt $\lim_{x \rightarrow a} f(x)=b$, $\lim_{x \rightarrow b}
g(x)=c$ bei entsprechenden Definitionsgebieten für $f$ und $g$, so
folgt $\lim_{x \rightarrow a} g(f(x)) =c$.
\end{itemize}
\end{frame}



\begin{frame}{Stetigkeit}
Eine Funktion $f:D \ \rightarrow  \ \mathbb{R}$ heißt {\color{red} stetig an
der Stelle $x_0 \in D$}, wenn es zu jedem $\varepsilon>0$ ein $\delta>0$
gibt, so dass für alle $x \in D$ mit $|x - x_0| < \delta$ gilt
\[ |f(x)-f(x_0) | < \varepsilon .\]
Man sagt, dass $f$ {\color{red} stetig} ist, wenn $f$ an jeder Stelle $x_0
\in D$ stetig ist. \\
Sind $f$ und $g$ an $x_0$ stetig, so auch $f+g$, $f-g$, $f \cdot g$
und $\frac{f}{g}$ (falls $g(x_0) \neq 0$). 
\end{frame}

\begin{frame}{Wichtige Sätze I}
\begin{itemize}
\item Sei $f$ auf einem offenen Intervall $I$ definiert. $f$ ist an
$x_0 \in I$ genau dann stetig, wenn gilt
\[ \lim_{x \rightarrow x_0} f(x) = f(x_0). \]
\item Für $f:I \rightarrow \mathbb{R}$ und $g:J \rightarrow
\mathbb{R}$ gelte $f(I) \subset J$ und es seien $f$ an $x_0 \in I$ und
$g$ an $y_0=f(x_0)$ stetig. Dann ist $g \circ f$ an $x_0$ stetig.
\item Eine Funktion $f: D \rightarrow \mathbb{R}$ ist {\color{red}
linksstetig} bzw. {\color{red} rechtsstetig}, wenn $f|_{D\cap (-\infty,x_0)}$
bzw  $f|_{D\cap (x_0,\infty)}$ an $x_0$ stetig ist. Eine Funktion $f$
ist dann an $x_0$ stetig, genau dann wenn $f$ links- und rechtsstetig
an $x_0$ ist.
\end{itemize}
\end{frame}

\begin{frame}{Wichtige Sätze II}
\begin{itemize}
\item Eine stetige Funktion auf einem abgeschlossenen Intervall $I=[a,b]$
besitzt ein Maximum und ein Minimum.
\item Eine stetige Funktion $f$ auf einem abgeschlossenen  Intervall
$[a,b]$ nimmt in $I$ jeden Wert zwischen $f(a)$ und $f(b)$ an.
\item Potenzreihen $f(x)=\sum_{n=0}^\infty a_n (x-x_0)^n$ sind stetig
innerhalb ihres Konvergenzintervalls.
\end{itemize}
\end{frame}


\begin{frame}{Gleichmäßige Stetigkeit}
$f: D \rightarrow \mathbb{R}$ heißt {\color{red} gleichmäßig stetig auf $D$},
wenn es zu jedem $\varepsilon >0$ ein $\delta>0$ gibt, so dass für alle
Paare $x,x_0 \in D$ mit $|x - x_0|< \delta$ gilt
\[ | f(x)-f(x_0)| < \varepsilon. \]
\begin{itemize}
\item Die Exponentialfunktion ist auf jedem kompakten Intervall
gleichmäßig stetig (aber nicht auf ganz $\mathbb{R}$). 
\item $\log:(0,1) \rightarrow \mathbb{R}$ ist stetig aber nicht
gleichmäßig stetig.
\end{itemize}
\end{frame}


\begin{frame}{Sage}
    \begin{center}
        \url{https://sage.math.uni-goettingen.de/home/pub/43/}
    \end{center}
\end{frame}

% (Komplexe) Unstetigkeitsstellen oder Definitionslücken können mittels
% \isage{discont} aufgespürt werden:
% \begin{sagein}
% discont(sin(1/x)*x,x)
% \end{sagein}
% \begin{sage}
%   {0}
% \end{sage}
% \begin{sagein}
% discont(exp(x),x)
% \end{sagein}
% \begin{sage}
%   {}
% \end{sage}
% \begin{sagein}
% discont(tan(x),x)
% \end{sagein}
% \begin{sage}
%   { 1/2*PI + X1*PI |  X1 in Z_ }
% \end{sage}
% \end{frame}
% 
% \begin{frame}[fragile]{Stetigkeit in MuPAD}
% \begin{sagein}
% discont(1/sin(x),x=-1..10)
% \end{sagein}
% \begin{sage}
%   {PI, 0, 2 PI, 3 PI}
% \end{sage}
% Vorsicht! Es werden komplexe Unstetigkeitsstellen gesucht!
% \begin{sagein}
% discont(ln(x),x)
% \end{sagein}
% \begin{sage}
%   (-infinity, 0]
% \end{sage}
% \begin{sagein}
% discont(ln(x),x,Dom::Real)
% \end{sagein}
% \begin{sage}
%   {0}
% \end{sage}
% \begin{sagein}
% ln(-2)
% \end{sagein}
% \begin{sage}
%   I PI + ln(2)
% \end{sage}
% \end{frame}

%----------------------------------------
\section{Funktionenfolgen}
%----------------------------------------

\begin{frame}{Funktionenfolgen}
Seien $f_n: D \ \rightarrow \ \mathbb{R}$, $n \in
\mathbb{N}$  rellwertige Funktionen auf  $D \subset \mathbb{R}$.
\begin{itemize}
\item $(f_n)_n$ heißt {\color{red} Funktionenfolge.}
\item Ist für jedes $x\in D$ die Folge $(f_n(x))_n$ konvergent, so wird durch 
\[ f(x):= \lim_{n \rightarrow \infty} f_n(x), \quad x \in D \]
die {\color{red} Grenzfunktion} $f:D \ \rightarrow \ \mathbb{R}$ definiert.
\item Man sagt $f_n$ strebe {\color{red} punktweise} auf $D$ gegen $f$.  
\item Durch $\sum_{i=1}^\infty f_i$ definierte {\color{red} Funktionenreihen}
sind spezielle Funktionenfolgen.
\end{itemize}  
\end{frame}

\begin{frame}{Beispiele: Grenzübergänge}
\begin{itemize}
\item $x^n \rightarrow 0$ auf dem Intervall $(-1,1)$.
\item $\left( 1+ \frac{x}{n} \right)^n \rightarrow \exp(x)$ auf $\mathbb{R}$.
\item Potenzreihen konvergieren innerhalb ihres Konvergenzradius.
\item {\color{red} Warnung} zum Vertauschen der Grenzprozesse für $x \in (0,1)$:
\[ \lim_{x \rightarrow 1} \lim_{n \rightarrow \infty} x^n =0 \neq 1 = 
  \lim_{n \rightarrow \infty} \lim_{x \rightarrow 1} x^n.\]  
\end{itemize}
\end{frame}

\begin{frame}[fragile]{Gleichmäßige Konvergenz}
\begin{definition}
$(f_n)_n$ konvergiert {\color{red} gleichmäßig} auf $D$ gegen $f$, wenn es zu
jedem $\varepsilon >0$ ein $n_0 \in \mathbb{N}$ gibt, so dass für alle
$x \in D$ und $n\geq n_0$ gilt:
\[ |f_n(x) -f(x)| < \varepsilon.\]
\end{definition}

\begin{Satz}
 Konvergiert $(f_n)_n$ gleichmäßig auf $D$ und existiert $\lim_{x
\rightarrow a} f_n(x)$ für $a\in D$, so gilt:
\[ \lim_{x \rightarrow a} \lim_{n \rightarrow \infty} f_n(x) = \lim_{n
\rightarrow \infty} \lim_{x \rightarrow a} f_n(x). \]
\end{Satz}
\end{frame}

\begin{frame}{Bemerkungen}
\begin{itemize}
\item Die Grenzfunktion einer gleichmäßig konvergenten Folge stetiger
Funktionen ist stetig.
\item \alert{Funktionenreihen}: Ist $f_1, f_2, \ldots$, eine Folge von Funktionen auf $D \subseteq \mathbb{R}$ dann definiert
\[
 s := \sum_{n=1}^\infty f_n
\]
eine Funktionenreihe. 

Alle Aussagen übertragen sich analog; ebenso die Aussagen über die Folge der Partialsummen
\[
 s_k := \sum_{n=1}^k f_n.
\]
 
 
\end{itemize}
\end{frame}

%------------- ---------------------------
\section{Grafiken}
%----------------------------------------



\begin{frame}{Sage}
    \begin{center}
        \url{https://sage.math.uni-goettingen.de/home/pub/44/}
    \end{center}
\end{frame}




% \begin{frame}[fragile]{parametric\_plot()}
% \begin{itemize}
% \item Darstellung von Graphen von Funktionen mit einem bzw. zwei Argumenten.
% \begin{sagein}
% parametric_plot([f_x, f_y, f_z)], (u,  u_min, u_max),(v,  v_min, v_max), ...)  
% \end{sagein}
% \end{itemize}
% \end{frame} 








% \item Beispielsweise lassen sich Graphen von Funktionen $f:[a
%    ,b]\times [c,d]
%    \ \rightarrow \ \mathbb{R}$ als Flächen
% \[ x=t_1, \ y=t_2, \ z=f(t_1,t_2) \]
% mit $a \leq t_1 \leq b, \ c \leq
%    t_2 \leq d$ erklären.






% \item Dreidimensionales Beispiel
% \begin{sagein}
% plotfunc3d(cos(j^0.5*PI*exp(-x^2-y^2)), 
%    x=-1..1, y=-1..1, j=1..30, Mesh=[40,40])
% \end{sagein}
\end{document}
