\documentclass[a4paper,12pt,DIV15]{scrartcl}
\usepackage[xetex,bookmarks=true,pdfstartview=FitH,bookmarksopen=true,
    colorlinks,citecolor=Blue,linkcolor=DarkBlue,urlcolor=Green,
    pagebackref=true,plainpages=false,pdfpagelabels=true,unicode=true,
    breaklinks=true,naturalnames=false,setpagesize=true,a4paper=true,hyperindex]{hyperref}
\usepackage[svgnames,hyperref]{xcolor} %color definition
\usepackage{tikz}

%\usepackage{fontspec,xunicode}
% %\usepackage{polyglossia}
%\setdefaultlanguage[spelling=new, latesthyphen=true]{german}
%\setsansfont{DejaVu Sans}
%\setsansfont{Verdana}
%\setsansfont{Arial}
%\setromanfont[Mapping=tex-text]{Linux Libertine}
%\setsansfont[Mapping=tex-text]{Myriad Pro}
%\setmonofont[Mapping=tex-text]{Courier New}

%\setsansfont{Linux Biolinum}

\usepackage[ngerman]{babel}
\selectlanguage{ngerman}

%
% math/symbols
%
\usepackage{amssymb}
\usepackage{amsthm}
% \usepackage{latexsym}
\usepackage{amsmath}
%\usepackage{amsxtra} %Weitere Extrasymbole
%\usepackage{empheq} %Gleichungen hervorheben
%\usepackage{bm}
 %\bm{A} Boldface im Mathemodus
\usepackage{fontspec,xunicode,xltxtra}

\usepackage{multimedia}
%\usepackage{tikz}

\usepackage{cellspace}
\setlength{\cellspacetoplimit}{2pt}
\setlength{\cellspacebottomlimit}{2pt}

%%%%%%%%%%%%%%%%%% Fuer Frames [fragile]-Option verwenden!
%Programm-Listing
%%%%%%%%%%%%%%%%%%
%Listingsumgebung fuer verbatim
%Grauhinterlegeter Text
%Automatischer Zeilenumbruch ist aktiviert
%\usepackage{listings}
\usepackage[framed]{mcode}
%\usepackage{mcode}
% This command allows you to typeset syntax highlighted Matlab
% code ``inline''.
% mcode fuer matlab

\definecolor{lgray}{gray}{0.80}
\definecolor{gray}{gray}{0.3}
\definecolor{darkgreen}{rgb}{0,0.4,0}
\definecolor{darkblue}{rgb}{0,0,0.8}
\definecolor{key}{rgb}{0,0.5,0} 
\definecolor{NU0}{RGB}{68,85,136} % #458
\definecolor{KW3}{RGB}{85,68,136}
\definecolor{KW4}{RGB}{153,0,0}
\definecolor{dred}{RGB}{221,17,68} % #d14
\definecolor{BG}{RGB}{240,240,240}
%\lstset{backgroundcolor=\color{lgray}, frame=single, basicstyle=\ttfamily, breaklines=true}
%\lstnewenvironment{sage}{\lstset{,language=python, keywordstyle=color{blue},    commentstyle=color{green}, emphstyle=\color{red}, %frame=single, stringstyle=\color{red}, basicstyle=\ttfamily, ,mathescape =true,escapechar=§}}{}

\lstdefinelanguage{fooHaskell} {%
  basicstyle=\footnotesize\ttfamily,%
  commentstyle=\slshape\color{gray},%
  keywordstyle=\bfseries,%\color{KW4},
  breaklines=true,
  sensitive=true,
  xleftmargin=1pc,
  emph={[1]
    FilePath,IOError,abs,acos,acosh,all,and,any,appendFile,approxRational,asTypeOf,asin,
    asinh,atan,atan2,atanh,basicIORun,break,catch,ceiling,chr,compare,concat,concatMap,
    const,cos,cosh,curry,cycle,decodeFloat,denominator,digitToInt,div,divMod,drop,
    dropWhile,either,elem,encodeFloat,enumFrom,enumFromThen,enumFromThenTo,enumFromTo,
    error,even,exp,exponent,fail,filter,flip,floatDigits,floatRadix,floatRange,floor,
    fmap,foldl,foldl1,foldr,foldr1,fromDouble,fromEnum,fromInt,fromInteger,fromIntegral,
    fromRational,fst,gcd,getChar,getContents,getLine,head,id,inRange,index,init,intToDigit,
    interact,ioError,isAlpha,isAlphaNum,isAscii,isControl,isDenormalized,isDigit,isHexDigit,
    isIEEE,isInfinite,isLower,isNaN,isNegativeZero,isOctDigit,isPrint,isSpace,isUpper,iterate,
    last,lcm,length,lex,lexDigits,lexLitChar,lines,log,logBase,lookup,map,mapM,mapM_,max,
    maxBound,maximum,maybe,min,minBound,minimum,mod,negate,not,notElem,null,numerator,odd,
    or,ord,otherwise,pi,pred,primExitWith,print,product,properFraction,putChar,putStr,putStrLn,quot,
    quotRem,range,rangeSize,read,readDec,readFile,readFloat,readHex,readIO,readInt,readList,readLitChar,
    readLn,readOct,readParen,readSigned,reads,readsPrec,realToFrac,recip,rem,repeat,replicate,return,
    reverse,round,scaleFloat,scanl,scanl1,scanr,scanr1,seq,sequence,sequence_,show,showChar,showInt,
    showList,showLitChar,showParen,showSigned,showString,shows,showsPrec,significand,signum,sin,
    sinh,snd,span,splitAt,sqrt,subtract,succ,sum,tail,take,takeWhile,tan,tanh,threadToIOResult,toEnum,
    toInt,toInteger,toLower,toRational,toUpper,truncate,uncurry,undefined,unlines,until,unwords,unzip,
    unzip3,userError,words,writeFile,zip,zip3,zipWith,zipWith3,listArray,doParse
  },%
  emphstyle={[1]\color{NU0}},%
  emph={[2]
    Bool,Char,Double,Either,Float,IO,Integer,Int,Maybe,Ordering,Rational,Ratio,ReadS,Show,ShowS,String,
    Word8,InPacket
  },%
  emphstyle={[2]\bfseries\color{KW4}},%
  emph={[3]
    case,class,data,deriving,do,else,if,import,in,infixl,infixr,instance,let,
    module,of,primitive,then,type,where
  },
  emphstyle={[3]\color{darkblue}},
  emph={[4]
    quot,rem,div,mod,elem,notElem,seq
  },
  emphstyle={[4]\color{NU0}\bfseries},
  emph={[5]
    EQ,False,GT,Just,LT,Left,Nothing,Right,True,Show,Eq,Ord,Num
  },
  emphstyle={[5]\color{KW4}\bfseries},
  morestring=[b]",%
  morestring=[b]',%
  stringstyle=\color{darkgreen},%
  showstringspaces=false
}
\lstnewenvironment{hs}
{\lstset{language=fooHaskell,backgroundcolor=\color{BG}}}
{\smallskip}
\newcommand{\ihs}[1]{\lstset{language=fooHaskell,basicstyle=\color[gray]{0.6}}\lstinline|#1|}


\lstdefinelanguage{fooMatlab} {%
backgroundcolor=\color[gray]{0.9},
breaklines=true,
basicstyle=\ttfamily\small,
%otherkeywords={ =},
%keywordstyle=\color{blue},
%stringstyle=\color{darkgreen},
showstringspaces=false,
%emph={for, while, if, elif, else, not, and, or, printf, break, continue, return, end, function},
%emphstyle=\color{blue},
%emph={[2]True, False, None, self, NaN, NULL},
%emphstyle=[2]\color{key},
%emph={[3]from, import, as},
%emphstyle=[3]\color{blue},
%upquote=true,
%morecomment=[s]{"""}{"""},
%commentstyle=\color{gray}\slshape,
%framexleftmargin=1mm, framextopmargin=1mm, 
%title=\tiny matlab,
frame=single,
%mathescape =true,
%escapechar=§
}
\newcommand{\imatlab}[1]{\lstset{language=fooMatlab,basicstyle=\color[gray]{0.6}}\lstinline|#1|}
\lstnewenvironment{matlab}[1][]{\lstset{language=fooMatlab,xleftmargin=0.2cm,frame=none,backgroundcolor=\color{white},basicstyle=\color{darkblue}\ttfamily\small,#1}}{} 
\lstnewenvironment{matlabin}[1][]{\lstset{language=fooMatlab,#1}}{} 
\newcommand{\matinput}[1]{\lstset{language=fooMatlab}\lstinputlisting{#1}}

\lstdefinelanguage{fooPython} {%
language=python,
backgroundcolor=\color[gray]{0.7},
breaklines=true,
basicstyle=\ttfamily\small,
%otherkeywords={ =},
keywordstyle=\color{blue},
stringstyle=\color{darkgreen},
morestring=[b]",%
morestring=[b]',%
showstringspaces=false,
emph={class, pass, in, for, while, if, is, elif, else, not, and, or,
def, print, exec, break, continue, return, import, from, lambda, null},
emphstyle=\color{blue},
emph={[2]True, False, None, self},
emphstyle=[2]\color{key},
emph={[3]from, import, as},
emphstyle=[3]\color{blue},
upquote=true,
morecomment=[s]{"""}{"""},
comment=[l]{\#},
commentstyle=\color{gray},
%framexleftmargin=1mm, framextopmargin=1mm, 
%title=\tiny python,
%caption=python,
frame=single
%frameround=tttt,
%mathescape =true,
%escapechar=§
}

\newcommand{\pyinput}[1]{\lstset{language=fooPython}\lstinputlisting{#1}}
\newcommand{\isage}[1]{{\lstset{language=fooPython,basicstyle=\color[gray]{0.3}}\lstinline|#1|}}

\lstnewenvironment{pyout}[1][]{\lstset{language=fooPython,xleftmargin=0.2cm,frame=none,backgroundcolor=\color{white},basicstyle=\color{darkblue}\ttfamily\small,#1}}{}
\lstnewenvironment{pyin}[1][]{\lstset{language=fooPython,#1}}{}
\lstnewenvironment{sageout}[1][]{\lstset{language=fooPython,xleftmargin=0.2cm,frame=none,backgroundcolor=\color{white},basicstyle=\color{darkblue}\ttfamily\small,#1}}{}
\lstnewenvironment{sagein}[1][]{\lstset{language=fooPython,#1}}{}

%\usepackage{caption}
%\DeclareCaptionFont{white}{ \color{white} }
%\DeclareCaptionFormat{listing}{
%  \colorbox[cmyk]{0.43, 0.35, 0.35,0.01 }{
%      \parbox{\textwidth}{\hspace{15pt}#1#2#3}
%        }
%        }
%        \captionsetup[lstlisting]{ format=listing, labelfont=white, textfont=white, singlelinecheck=false, margin=0pt, font={bf,footnotesize} }


\usepackage{mydef}
%\usepackage{cmap} % you can search in the pdf for umlauts and ligatures
\usepackage{colonequals} %corrects the definition-symbols \colonequals (besides others)

\usepackage{ifthen}

%%%%%%%%%%%%%%%%%%%
%Neue Definitionen
%%%%%%%%%%%%%%%%%%%

%Newcommands
\newcommand{\Fun}[1]{\mathcal{#1}}      %Mathcal fuer Funktoren
\newcommand{\field}[1]{\mathbb{#1}}     %Grundkoerper ?? in mathds

\newcommand{\A}{\field{A}}              %Affines A
\newcommand{\Fp}{\field{F}_{\!p}}       %Endlicher Koerper mit p Elementen
\newcommand{\Fq}{\field{F}_{\!q}}       %Endlicher Koerper mit q Elementen
\newcommand{\Ga}{\field{G}_{a}}         %Add Gruppenschema
\newcommand{\K}{\field{K}}              %Generischer Koerper 
\newcommand{\N}{\field{N}}              %Nat Zahlen
\newcommand{\Pj}{\field{P}}             %Projektives P
\newcommand{\R}{\field{R}} 		%Reelle Zahlen
\newcommand{\Q}{\field{Q}}              %Rationale Zahlen  
\newcommand{\Qt}{\field{H}}             %Quaternionen 
\newcommand{\V}{\field{V}}              %Vektorbuendel V
\newcommand{\Z}{\field{Z}}              %Ganze Zahlen
\DeclareMathOperator{\Real}{Re}

\newcommand{\fdg}{\;|\;}                 %fuer die gilt

%Operatoren
\DeclareMathOperator{\Abb}{Abb}
%\usepackage{sagetex}


%
% Aufgaben
%
\parindent0cm % Abs�tze nicht einr�cken 
% Definieren einer neuen Farbe
\definecolor{light-gray}{gray}{.9}

\newcounter{zaehler}     % neuen Z�hler einf�hren
\newenvironment{aufgn}[2][0]
%---- Header
{\begin{samepage}%
%\colorbox{light-gray}{%                         % Box in gray
% \makebox[\textwidth]{%                           % Box in linewidth
%\textbf{Aufgabe \arabic{zaehler} } }\hspace{-\textwidth}\makebox[\textwidth]{\hfill #1 Punkte} }\\[0.05cm]       % Header
\dotfill\\
{\large\textbf{Aufgabe \arabic{zaehler} \ifthenelse{ \equal{#2}{} }{}{: \emph{ #2 } }}\ifthenelse{-1=#1}{(testierbar)}{}\ifthenelse{0=#1 \or -1=#1}{}{\hfill #1 Punkte} }\\[0.4cm]
%{\large\textbf{Exercise \arabic{zaehler}  #2 }\ifthenelse{-1=#1}{(testierbar)}{}\ifthenelse{0=#1 \or -1=#1}{}{\hfill #1 Punkte} }\\[0.4cm]
\begin{minipage}{\textwidth}%
}%
%-----  foot
{\end{minipage}\nopagebreak%\begin{minipage}{1cm} \end{minipage}
%\\ 
%\begin{minipage}{0.1cm} \end{minipage} 
%\hrulefill \begin{minipage}{1cm} \end{minipage}\\[1cm]  
\stepcounter{zaehler}                           % increase counter
\end{samepage}%
\\%
\bigskip%
}


\newenvironment{aufg}[1][0]
%---- Header
{\begin{samepage}%
\refstepcounter{zaehler}% increase counter
%\colorbox{light-gray}{%                         % Box in gray
% \makebox[\textwidth]{%                           % Box in linewidth
%\textbf{Aufgabe \arabic{zaehler} } }\hspace{-\textwidth}\makebox[\textwidth]{\hfill #1 Punkte} }\\[0.05cm]       % Header
\dotfill\\
{\large\textbf{Aufgabe \arabic{zaehler} }\ifthenelse{-1=#1}{(testierbar)}{}\ifthenelse{0=#1 \or -1=#1}{}{\hfill #1 Punkte} }\\[0.4cm]
\begin{minipage}{\textwidth}%
}%
%-----  foot
{\end{minipage}\nopagebreak%\begin{minipage}{1cm} \end{minipage}
%\\ 
%\begin{minipage}{0.1cm} \end{minipage} 
%\hrulefill \begin{minipage}{1cm} \end{minipage}\\[1cm]  
\end{samepage}%
\\%
\bigskip%
}

\begin{document}
\begin{center}
    \textbf{\LARGE Einführung in Sage}\\
    {\large Einheit 07}\\
    {\large Zusammenfassung: Funktionen und Funktionenfolgen}
\end{center}

\section{Funktionen}
\begin{defn}[Funktion]
(reelle) {\color{red} Funktion}:  Abbildung
\[f: \ D \subset \mathbb{R} \ \rightarrow \ \mathbb{R}.\]
die jedem Element aus $D$ eindeutig genau ein Element aus $\mathbb{R}$ zuordnet.
\begin{itemize}
 \item {\color{red} Definitionsbereich}: $D \subset \mathbb{R}$, $D \neq \emptyset$.
\item {\color{red}Wertebereich}: Die Menge $f(D)$ aller rellen Zahlen, die als Werte der
Funktion vorkommen.
\item {\color{red} Graph} einer Funktion: ist die Menge aller Punkte 
\[ \{ (x,f(x)) \in \mathbb{R}^2 \;|\; x \in D\}. \]
\end{itemize}
\end{defn}

\begin{defn}[Verknüpfungen]
Seien $f$ und $g$ Funktionen mit einem gemeinsamen Definitionsbereich. Dann
definiert man:
\begin{itemize}
\item Summe: $(f+g)(x):=f(x)+g(x)$
\item Differenz: $(f-g)(x):=f(x)-g(x)$
\item Produkt: $(f\cdot g)(x):=f(x) \cdot g(x)$
\item Quotient: $(\frac{f}{g})(x):=\frac{f(x)}{g(x)}$, falls $g(x) \neq
0$ für alle $x \in D$ 
\item Komposition: Mit $f:D_f \rightarrow \mathbb{R}$ und $g:D_g \rightarrow \mathbb{R}$
mit $f(D_f) \subset D_g$ 
\[(g \circ f) (x):=g(f(x)).\] 
\end{itemize}

\end{defn}

\paragraph{Funktionen mehrerer Veränderlicher:}
Ist $D \subseteq \mathbb{R}^n$ und $f : D \Rightarrow \mathbb{R}$ dann spricht man von
einer reellen Funktion in {\color{red}mehreren Veränderlichen}. Das Studium dieser Funktionen ist einer der Hauptinhalte der Diff2-Vorlesung.

\bigskip

Weiterhin können Funktionen auch Wertebereiche außerhalb der reellen Zahlen haben.
Z.B. 
\[f : D \Rightarrow \mathbb{R}^m.\]
 Im physikalischen Umfeld spricht man für $m=1$ dann von {\color{red}skalarwertigen Funktionen} und für $m>1$ von {\color{red}vektorwertigen Funktionen} oder {\color{red}Vektorfeldern}.
 
\bigskip
Sage-Notebook über Funktionen:        
\url{https://sage.math.uni-goettingen.de/home/pub/42/}

\section{Grenzwerte von Funktionen}

\begin{defn}[Grenzwert]
{\color{red} Grenzwert}: Sei $f$ eine Funktion mit Definitionsbereich $D$ und $a\in D$.\\
$f$ strebt für $x \rightarrow a$ gegen $b \in \mathbb{R}$, wenn es zu jedem $\varepsilon >0$ ein $\delta >0$ gibt, so
dass für alle $x \in D\smallsetminus\{a \}$ mit $|x-a|<\delta$ gilt 
\[ |f(x)-b| < \varepsilon .\]
Der Grenzwert $b$ ist eindeutig bestimmt und man schreibt
\[ \lim_{x \rightarrow a} f(x) =b \mbox{ oder } f(x) \rightarrow b
\mbox{ für } x \rightarrow a. \]
Die Aussage überträgt sich sinngemäß auf $a=\pm \infty$.
\end{defn}

\paragraph{Bemerkungen:}
\begin{itemize}
\item {\color{red}Folgenkriterium}: Es gilt $ \lim_{x \rightarrow a} f(x) =b$
genau dann, wenn für jede Folge $a_n \in D$ mit $a_n \neq a$ und $a_n \rightarrow a$
gilt $\lim_{n \rightarrow \infty} f(a_n)=b$.
\item Es gelten die üblichen Rechenregeln:
\begin{eqnarray*}
\lim_{x \rightarrow a}(f(x)+g(x)) &=&\lim_{x \rightarrow a} f(x) +
\lim_{x \rightarrow a} g(x) \\
\lim_{x \rightarrow a}(f(x) \cdot g(x)) &=& \lim_{x \rightarrow a}
f(x) \cdot \lim_{x \rightarrow a} g(x)
\end{eqnarray*}
wenn $\lim_{x \rightarrow a} f(x)$ und $\lim_{x \rightarrow a}g(x)$
existieren. 
\item Gilt $\lim_{x \rightarrow a} f(x)=b$, $\lim_{x \rightarrow b}
g(x)=c$ bei entsprechenden Definitionsgebieten für $f$ und $g$, so
folgt $\lim_{x \rightarrow a} g(f(x)) =c$.
\end{itemize}



\begin{defn}[Stetigkeit]
Eine Funktion $f:D \ \rightarrow  \ \mathbb{R}$ heißt {\color{red} stetig an
der Stelle $x_0 \in D$}, wenn es zu jedem $\varepsilon>0$ ein $\delta>0$
gibt, so dass für alle $x \in D$ mit $|x - x_0| < \delta$ gilt
\[ |f(x)-f(x_0) | < \varepsilon .\]
Man sagt, dass $f$ {\color{red} stetig} ist, wenn $f$ an jeder Stelle $x_0
\in D$ stetig ist. \\
Sind $f$ und $g$ an $x_0$ stetig, so auch $f+g$, $f-g$, $f \cdot g$
und $\frac{f}{g}$ (falls $g(x_0) \neq 0$). 
\end{defn}

\begin{thm}[Wichtige Sätze]
\begin{itemize}
\item Sei $f$ auf einem offenen Intervall $I$ definiert. $f$ ist an
$x_0 \in I$ genau dann stetig, wenn gilt
\[ \lim_{x \rightarrow x_0} f(x) = f(x_0). \]
\item Für $f:I \rightarrow \mathbb{R}$ und $g:J \rightarrow
\mathbb{R}$ gelte $f(I) \subset J$ und es seien $f$ an $x_0 \in I$ und
$g$ an $y_0=f(x_0)$ stetig. Dann ist $g \circ f$ an $x_0$ stetig.
\item Eine Funktion $f: D \rightarrow \mathbb{R}$ ist {\color{red}
linksstetig} bzw. {\color{red} rechtsstetig}, wenn $f|_{D\cap (-\infty,x_0)}$
bzw  $f|_{D\cap (x_0,\infty)}$ an $x_0$ stetig ist. Eine Funktion $f$
ist dann an $x_0$ stetig, genau dann wenn $f$ links- und rechtsstetig
an $x_0$ ist.
\item Eine stetige Funktion auf einem abgeschlossenen Intervall $I=[a,b]$
besitzt ein Maximum und ein Minimum.
\item Eine stetige Funktion $f$ auf einem abgeschlossenen  Intervall
$[a,b]$ nimmt in $I$ jeden Wert zwischen $f(a)$ und $f(b)$ an.
\item Potenzreihen $f(x)=\sum_{n=0}^\infty a_n (x-x_0)^n$ sind stetig
innerhalb ihres Konvergenzintervalls.
\end{itemize}
\end{thm}


\begin{defn}[Gleichmäßige Stetigkeit]
$f: D \rightarrow \mathbb{R}$ heißt {\color{red} gleichmäßig stetig auf $D$},
wenn es zu jedem $\varepsilon >0$ ein $\delta>0$ gibt, so dass für alle
Paare $x,x_0 \in D$ mit $|x - x_0|< \delta$ gilt
\[ | f(x)-f(x_0)| < \varepsilon. \]
\begin{itemize}
\item Die Exponentialfunktion ist auf jedem kompakten Intervall
gleichmäßig stetig (aber nicht auf ganz $\mathbb{R}$). 
\item $\log:(0,1) \rightarrow \mathbb{R}$ ist stetig aber nicht
gleichmäßig stetig.
\end{itemize}
\end{defn}

\bigskip
Sage-Notebook über Grenzwerte von Funktionen:        
        \url{https://sage.math.uni-goettingen.de/home/pub/43/}


\section{Funktionenfolgen}
\begin{defn}[Funktionenfolge]
Seien $f_n: D \ \rightarrow \ \mathbb{R}$, $n \in
\mathbb{N}$  rellwertige Funktionen auf  $D \subset \mathbb{R}$.
\begin{itemize}
\item $(f_n)_n$ heißt {\color{red} Funktionenfolge.}
\item Ist für jedes $x\in D$ die Folge $(f_n(x))_n$ konvergent, so wird durch 
\[ f(x):= \lim_{n \rightarrow \infty} f_n(x), \quad x \in D \]
die {\color{red} Grenzfunktion} $f:D \ \rightarrow \ \mathbb{R}$ definiert.
\item Man sagt $f_n$ strebe {\color{red} punktweise} auf $D$ gegen $f$.  
\item Durch $\sum_{i=1}^\infty f_i$ definierte {\color{red} Funktionenreihen}
sind spezielle Funktionenfolgen.
\end{itemize}  
\end{defn}

\paragraph{Beispiele Grenzübergänge:}
\begin{itemize}
\item $x^n \rightarrow 0$ auf dem Intervall $(-1,1)$.
\item $\left( 1+ \frac{x}{n} \right)^n \rightarrow \exp(x)$ auf $\mathbb{R}$.
\item Potenzreihen konvergieren innerhalb ihres Konvergenzradius.
\item {\color{red} Warnung} zum Vertauschen der Grenzprozesse für $x \in (0,1)$:
\[ \lim_{x \rightarrow 1} \lim_{n \rightarrow \infty} x^n =0 \neq 1 = 
  \lim_{n \rightarrow \infty} \lim_{x \rightarrow 1} x^n.\]  
\end{itemize}

\begin{defn}[Gleichmäßige Konvergenz]
$(f_n)_n$ konvergiert {\color{red} gleichmäßig} auf $D$ gegen $f$, wenn es zu
jedem $\varepsilon >0$ ein $n_0 \in \mathbb{N}$ gibt, so dass für alle
$x \in D$ und $n\geq n_0$ gilt:
\[ |f_n(x) -f(x)| < \varepsilon.\]
\end{defn}

\begin{thm}
 Konvergiert $(f_n)_n$ gleichmäßig auf $D$ und existiert $\lim_{x
\rightarrow a} f_n(x)$ für $a\in D$, so gilt:
\[ \lim_{x \rightarrow a} \lim_{n \rightarrow \infty} f_n(x) = \lim_{n
\rightarrow \infty} \lim_{x \rightarrow a} f_n(x). \]
\end{thm}

\paragraph{Bemerkungen:}
\begin{itemize}
\item Die Grenzfunktion einer gleichmäßig konvergenten Folge stetiger
Funktionen ist stetig.
\item {\color{red}Funktionenreihen}: Ist $f_1, f_2, \ldots$, eine Folge von Funktionen auf $D \subseteq \mathbb{R}$ dann definiert
\[
 s := \sum_{n=1}^\infty f_n
\]
eine Funktionenreihe. 

Alle Aussagen übertragen sich analog; ebenso die Aussagen über die Folge der Partialsummen
\[
 s_k := \sum_{n=1}^k f_n.
\]
 
 
\end{itemize}
\end{document}
