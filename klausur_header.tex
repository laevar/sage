\usepackage[psamsfonts]{amssymb}
\usepackage{amsmath}
%\usepackage{latexsym}
\usepackage{theorem}


\usepackage{fontspec,xunicode,xltxtra}
\usepackage[ngerman]{babel}
\selectlanguage{ngerman}


\usepackage[svgnames]{xcolor} %color definitions
%\usepackage{tikz}
%\usetikzlibrary{shadows}
%\usetikzlibrary{fit}
%\usetikzlibrary{shapes}
%\usetikzlibrary{backgrounds}

\usepackage{cellspace}
\setlength{\cellspacetoplimit}{8pt}
\setlength{\cellspacebottomlimit}{8pt}

\usepackage{mcode}
%\usepackage{pstricks,pst-node,pst-text,pst-3d}
\parindent0cm % Abs�tze nicht einr�cken 

%---- neue Umgebung f�r Aufgaben
\theoremstyle{break}
\theoremheaderfont{\Large \bf}
\theorembodyfont{\normalfont}


% Definieren einer neuen Farbe
\definecolor{light-gray}{gray}{.9}

\newcounter{zaehler}     % neuen Z�hler einf�hren
\stepcounter{zaehler}    % Z�hler einen hochz�hlen

\newenvironment{aufg}[1]
%---- Header
{\begin{samepage}%
\colorbox{light-gray}{%                         % Box in gray
 \makebox[\textwidth]{%                           % Box in linewidth
\textbf{Aufgabe \arabic{zaehler} } }\hspace{-\textwidth}\makebox[\textwidth]{\hfill #1 Punkte} }\\[0.1cm]       % Header
\begin{minipage}{\textwidth}}
%-----  foot
{\end{minipage} \nopagebreak %\begin{minipage}{1cm} \end{minipage}
\\[0.1cm] 
%\begin{minipage}{0.1cm} \end{minipage} 
%\hrulefill \begin{minipage}{1cm} \end{minipage}\\[1cm]  
\stepcounter{zaehler}                           % increase counter
 \end{samepage}%
}

%-------------------------------------------------------------------------------
\begin{document}
%-------------------------------------------------------------------------------

%--------------------------------------------------- Header
\begin{center}
\textbf{\LARGE Einf\"uhrung in MATLAB }\\
\end{center}
\begin{minipage}{6cm}
Dr. J. Schulz
\end{minipage}\hfill
\begin{minipage}{4cm}
%\textbf{Klausur}\\
%30.07.2007
18.09.2009
\end{minipage}\\[1cm]
%------------------------------

\begin{center}
\Huge \textbf{Klausur}
\end{center}
\bigskip\bigskip\bigskip
\Large
\begin{center}
\begin{tabular}{|Sl||p{0.5cm}|p{0.5cm}|p{0.5cm}|p{0.5cm}|p{0.5cm}|p{0.5cm}|p{0.5cm}|p{0.5cm}|p{0.5cm}|p{0.5cm}|p{0.5cm}|p{0.5cm}|p{0.5cm}||p{1cm}|}
\hline
Aufgabe & 1 & 2 & 3 & 4 & 5 & 6 & 7 & 8 & 9 & 10   & 11 & $\sum$ \\
\hline
Punkte &    &   &   &   &   &   &   &   &   &    &  &    \\
\hline
\end{tabular}
\end{center}

\bigskip\bigskip\bigskip
Bitte eintragen:\\
\begin{center}
\begin{tabular}{|Sl|p{8cm}|}
\hline
Nachname: & \\
\hline
Vorname: & \\
\hline
Studiengang: & \\
\hline 
Semester: & \\
\hline 
Immatrikulationsnummer: & \\
\hline
\end{tabular}\\[1cm]
\textbf{Hinweise:}
\begin{itemize}
\item Die Klausur beginnt um 10.00 Uhr und endet um 12.00 Uhr.
\item Ben\"otigte Hilfsmittel sind Stift und Papier.
\item Erlaubte Hilfsmittel sind gedruckte sowie handgeschriebene Notizen oder Skripte. 
\end{itemize}
\end{center}

\newpage
\normalsize
