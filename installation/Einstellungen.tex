\documentclass[a4paper,10pt]{article}
\usepackage[utf8x]{inputenc}
\usepackage [ngerman]{babel}
%Für erweiterte Enumerationseinstellungen
\usepackage{enumerate}



\usepackage{xcolor}
\definecolor{white}{rgb}{1,1,1}
\usepackage[linkbordercolor=white,urlbordercolor=white ,citebordercolor=white]{hyperref}
%opening
\title{Zusätzliche Einstellungen der virtuellen Maschine}


\begin{document}
\maketitle
\section{Tastaturlayout}
Versucht man in der virtuellen Maschine zum Beispiel in der Konsole zu arbeiten, oder auch in Sage. So stellt man schnell fest, 
dass als Standard das englische Tastaturlayout eingestellt ist. Leicht zu erkennen daran, dass \emph{z} und \emph{y} vertauscht sind.
Dies kann man leicht umstellen. Im Menü \emph{System $\rightarrow$ Preferences $\rightarrow$ Keyboard}. im Reiter \emph{Layout} kann 
über den Button \emph{add} das deutsche Tastaturlayout hinzugefügt werden. Dann kann noch der Punkt bei \emph{default} gesetzt werden,
sodass dieses auch voreingestellt ist. Sollten laufende Programme immer noch nicht auf das veränderte Layout reagieren, so hilft es 
dieses einfach mal neuzustarten. 
\section{Bildschirmauflösung}
Sollte in der virtuellen Maschine gearbeitet werden, so kann es sinnvoll sein, die Bildschirmauflösung etwas größer zu wählen, als
die Vorgebenen 800x600. Dafür müssen, die so genannten Gasterweiterungen installiert sein. Dafür einfach die virtuelle Maschine
starten. Sobald diese gestartet ist im Reiter \emph{Geräte} den Eintrag \emph{Gasterweiterungen installieren} auswählen.
Sobald dies geschehen ist, sollte auf dem Desktop ein CD Symbol zu sehen sein. Dies mit einem Doppelklick öffnen und mit einem weiteren
die Datei \emph{autorun.sh} öffnen. Dann \emph{Run in Terminal} auswählen. Nun werden die Erweiterungen installiert. Falls das 
Systempasswort verlangt wird, dieses ist \textquotedblleft\emph{sage}\textquotedblright. Mit \emph{Return (Enter)} wird die Installation abgeschlossen.

Sind die Erweiterungen installiert, muss der Computer (der virtuelle) noch neu gestartet werden. Dazu 
\emph{System$\rightarrow$ Shut Dowm $\rightarrow$ Restart}
auswählen. Nach dem Neustart kann die Bildschirmauflösung (falls nicht automatisch geschehen) über den Eintrag 
\emph{System$\rightarrow$ Preferences $\rightarrow$ Display} geändert werden. 
\end{document}
