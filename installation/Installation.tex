\documentclass[a4paper,10pt]{article}
\usepackage[utf8x]{inputenc}
\usepackage [ngerman]{babel}
%Für erweiterte Enumerationseinstellungen
\usepackage{enumerate}



\usepackage{xcolor}
\definecolor{white}{rgb}{1,1,1}
\usepackage[linkbordercolor=white,urlbordercolor=white ,citebordercolor=white]{hyperref}
%opening
\title{Installation von Sage}
\author{Thorsten Groth}

\begin{document}
\maketitle
\section{Windows}
Leider kann man Sage nicht so unter Windows installieren, wie man es sonst gewohnt ist. Um es trotzdem in seiner gewohnten Windows
Umgebung nutzen zu können, muss man eine virtuelle Maschiene erstellen, auf der eine Linuxversion installiert ist, auf der dann 
Sage läuft. 
Zum Glück gibt es das auch schon fertig installiert. 

\subsection{Virtual Box}
Um diese diese Methode benutzen zu können, muss zunächst das kostenlose Programm \emph{Virtual Box} heruntergeladen und installiert werden. 
Das Programm findet sich hier: \url{http://www.virtualbox.org/wiki/Downloads}. Dort die Datei \emph{VirtualBox for Windows host} herunterladen.
Alternativ geht auch das Programm \emph{VMware} aber 
da man hier für den Download erst die eigene Emailadresse herausrücken muss, ist wohl Virtualbox die einfacher Wahl.
\paragraph{}
Das Programm zu installieren sollte keine Schwierigkeiten bereiten. Einfach die Abfragen mit weiter bestätigen. Unter Windows XP bekommt
man unter Umständen die mehrmals Mitteilung die Software hätte den Windows Logo Test nicht bestanden. Diese Mitteilung kann man 
getrost ignorieren und \emph{Installation fortsetzen} klicken. Zwischendurch wird außerdem gefragt ob man einige Gerätetreiber mit installieren will.
Diese Abfrage unbedingt mit \emph{Ja} beantworten, da sonst die Virtuelle Maschiene keinen Zugriff auf USB Geräte und Netzwerk hat.
\paragraph{}
 Nun benötigt man noch die Datei die ein intalliertes Linux und Sage enthält. Diese bekommt man unter der Adresse:
 \url{ftp://ftp.fu-berlin.de/unix/misc/sage/win/index.html}. Die etwa 1.5GB große .ova Datei herunterladen und speichern.
Möglichst nicht irgendwo auf dem Desktop, wo man sie vermutlich demnächst versehentlich löscht. 
\paragraph{}
%Nun müssen wir diese Datei in Virtualbox einlesen. Dafür starten wir das Programm und klicken auf das Zahnradsymbol 
%\emph{neu}. Die Erklärung, dass uns der Dialog führen wird, quitieren wir mit \emph{weiter}.

%Jetzt kann ein Name gewählt werden. 
%Mein Vorschlag \textquotedblleft sage\textquotedblright. Als Betriebsystem wählen wir \emph{Linux} und als Version \emph{ubuntu}.
%Mit \emph{weiter} bestätigen wir diese Angaben. 

%Nun gilt es die Menge an Speicher auszuwählen. Dies hängt jetzt etwas vom eigenem 
%System ab. Es sollten schon mindestens 512MB sein und solange der Rechner nicht allzu alt ist kann gerne 1024MB
%ausgewählt werden. Mit einem Klick auf \emph{weiter} sind wir im nächsten Dialogteil.

%Den Hacken bei Bootfestplatte lassen wir gesetzt. Aber wählen den Punkt \emph{Festplatte benutzen} aus. Mit einem Klick auf das 
%Ordnersymbol rechts daneben navigieren wir zu dem Ort, an den wir die .ova Datei gespeicehrt haben. Diese wird ausgewählt. Mit 
%\emph{abschließen} wird der Vorgang abgeschlossen (Wer hätte es gedacht).
%\paragraph{}
Nun müssen wir diese Datei in Virtualbox einlesen. Dafür starten wir das Programm und wählen den Eintrag \emph{Datei $\to$ Appliance importieren}. 
Ein Klick auf \emph{Auswählen} öffnet einen Dateiauswahldialog. Dort zum Verzeichnis, in dem die .ova Datei gespeichert ist, navigieren und diese auswählen.
Ein weiterer Klick auf \emph{weiter} führt zu einer Übersicht des System, das eingebunden werden soll.
Mit einem Klick auf \emph{Importieren} wird dieser Vorgang abgeschlossen. Dies kann unter Umständen einige Zeit dauern. Danach sind wir allerdings fertig.
Mit einem Klick auf den grünen Pfeil starten wir nun den Linuxrechner.

In einem neuen Fenster sieht man das vertraute Bild eines startenden PCs (In diesem Fall mit einem Fedora Betriebsystem). 
Ist dieser fertig gestartet sollte auf schwarzem Grund von Sternchen umrandet stehen: 
\textquotedblleft Open your web browser to http://localhost:8000  \textquotedblright.
Das sollte nun auch getan werden. Also einen Internet Browser der eigenen Wahl öffnen und die Adresse eingeben. Nun sollte die Oberfläche des
Sagenotebooks zu sehen sein. 

In Ausnamfällen kann es sein, dass das nicht funktioniert, weil keine Netzwerkverbindung erkannt wird. Für diesen Fall habe ich eine
DVD mit einem etwas großzügerem Ubuntu und fertig installiertem Sage vorbereitet. Diese kann sich gerne in einer der ersten Übungen
ausgeliehen werden.
\paragraph{}
Außerdem besteht natürlich auch immer die Möglichkeit sich auf einer Partition Ubuntu zu installieren und Sage darauf laufen zu lassen.




\section{Linux(Ubuntu)}
Ich erkläre hier einmal die Vorgehensweise für Ubuntu, für andere Distributionen sollte es aber genauso gehen.
Zunächst muss das Programm heruntergeladen werden. \url{ftp://ftp.fu-berlin.de/unix/misc/sage/linux/index.html}. Hier wählt man die 
entprechende Version des Betriebsystems aus. Sollte nicht klar sein, ob man ein 32 oder ein 64Bit System verwendet, funktioniert
im Zweifelsfall die 32Bit Version auch unter 64Bit. 
\paragraph{}
Ist das Archiv erstmal heruntergeladen, gilt es dieses zu entpacken. Am besten an einen Ort, wo der Ordner auch weiterhin bestehen kann
und nicht irgendwo im Download Verzeichnis. Dann am besten noch den Ordnernamen in einen einfachen Titel umbennen, da wir den Namen 
auch mal eintippen müssen auf Leerzeichen und Umlaute sollte man dabei aber besser verzichten.
\paragraph{}
Zum Test ob Sage denn jetzt schon funktioniert, können wir in der Konsole einfach einmal eingeben \verb cd/Pfad/zum/Ordner/Sage  und 
 \verb ./sage  Das sollte Sage starten. Dann (wie auch angezeigt) muss man einen Augenblick warten. Bis die Eingabe erwartet
wird \emph{sage:}. Hier gibt man nun \verb notebook()  ein. Beim ersten Start wird dann ein Passwort festgelegt. Nun könnte man Sage
benutzen. 
\paragraph{}
Aber wir können noch eine Abkürzung zum Starten des Programms definieren, denn wer will schon immer den ganzen Pfad mit 
eingeben.

Dafür müssen dann einfach die folgenden Befehle in die Konsole eingeben werden: \verb PATH=\$PATH:/Pfad/zum/Ordner/Sage  und
\verb export  \verb PATH  Das war es dann auch schon
nun kann Sage einfach über den Konsolenaufruf \verb sage  gestartet werden.

Wer will kann sich ja noch eine Desktopverknüpfung oder einen Menüeintrag anlegen. 


\section{OS-X}
Um Sage unter OS-X zu installieren, muss zunächst einmal das .dmg Abbild von Sage heruntergeladen werden. Dieses ist hier
\url{ftp://ftp.fu-berlin.de/unix/misc/sage/osx/index.html} zu finden. Erst mal den Prozessortyp auswählen. Sollte nicht klar 
sein, welcher Prozessortyp in dem Mac steckt, hilft ein Blick in \emph{Über diesen Mac} im Finder. Dort sollte das stehen. Dann ist 
auch gleich klar, ob man einen 32 oder 64Bit Prozessor hat.

Ist die Datei erst einmal heruntergeladen, reicht es die .dmg Datei zu öffnen. Die Dateien werden dann entpackt und man kann sie an 
die gewünschte Stelle verschieben. Gestartetet wird dann über die Konsole. Zu dem Ordner navigieren und mit \verb ./sage  wird das
Programm gestartet. Eingabe von \verb notebook()  startet dann das Web-Frontend. Bei der ersten Ausführung sollte man ein wenig
Geduld aufbringen und warten, bis die Dateipfade richtig gesetzt sind. Beim Start des notebooks muss dann noch ein Passwort vergeben
werden.
\end{document}
