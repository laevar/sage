\documentclass[a4paper,9pt,DIV15,twocolumn]{scrartcl}
\usepackage{geometry}
\geometry{a4paper,left=15mm,right=15mm, top = 15mm, bottom=15mm}

\usepackage[svgnames]{xcolor} %color definition
\usepackage[xetex,bookmarks=true,pdfstartview=FitH,bookmarksopen=true,
    colorlinks,citecolor=Blue,linkcolor=DarkBlue,urlcolor=Green,
    pagebackref=true,plainpages=false,pdfpagelabels=true,unicode=true,
    breaklinks=true,naturalnames=false,setpagesize=true,a4paper=true,hyperindex]{hyperref}

%\usepackage{fontspec,xunicode}
% %\usepackage{polyglossia}
%\setdefaultlanguage[spelling=new, latesthyphen=true]{german}
%\setsansfont{DejaVu Sans}
%\setsansfont{Verdana}
%\setsansfont{Arial}
%\setromanfont[Mapping=tex-text]{Linux Libertine}
%\setsansfont[Mapping=tex-text]{Myriad Pro}
%\setmonofont[Mapping=tex-text]{Courier New}

%\setsansfont{Linux Biolinum}

\usepackage[ngerman]{babel}
\selectlanguage{ngerman}

%
% math/symbols
%
\usepackage{amssymb}
\usepackage{amsthm}
% \usepackage{latexsym}
\usepackage{amsmath}
%\usepackage{amsxtra} %Weitere Extrasymbole
%\usepackage{empheq} %Gleichungen hervorheben
%\usepackage{bm}
 %\bm{A} Boldface im Mathemodus
\usepackage{fontspec,xunicode,xltxtra}

\usepackage{multimedia}
%\usepackage{tikz}

\usepackage{cellspace}
\setlength{\cellspacetoplimit}{2pt}
\setlength{\cellspacebottomlimit}{2pt}

%%%%%%%%%%%%%%%%%% Fuer Frames [fragile]-Option verwenden!
%Programm-Listing
%%%%%%%%%%%%%%%%%%
%Listingsumgebung fuer verbatim
%Grauhinterlegeter Text
%Automatischer Zeilenumbruch ist aktiviert
%\usepackage{listings}
\usepackage[framed]{mcode}
%\usepackage{mcode}
% This command allows you to typeset syntax highlighted Matlab
% code ``inline''.
% mcode fuer matlab

\definecolor{lgray}{gray}{0.80}
\definecolor{gray}{gray}{0.3}
\definecolor{darkgreen}{rgb}{0,0.4,0}
\definecolor{darkblue}{rgb}{0,0,0.8}
\definecolor{key}{rgb}{0,0.5,0} 
\definecolor{NU0}{RGB}{68,85,136} % #458
\definecolor{KW3}{RGB}{85,68,136}
\definecolor{KW4}{RGB}{153,0,0}
\definecolor{dred}{RGB}{221,17,68} % #d14
\definecolor{BG}{RGB}{240,240,240}
%\lstset{backgroundcolor=\color{lgray}, frame=single, basicstyle=\ttfamily, breaklines=true}
%\lstnewenvironment{sage}{\lstset{,language=python, keywordstyle=color{blue},    commentstyle=color{green}, emphstyle=\color{red}, %frame=single, stringstyle=\color{red}, basicstyle=\ttfamily, ,mathescape =true,escapechar=§}}{}

\lstdefinelanguage{fooHaskell} {%
  basicstyle=\footnotesize\ttfamily,%
  commentstyle=\slshape\color{gray},%
  keywordstyle=\bfseries,%\color{KW4},
  breaklines=true,
  sensitive=true,
  xleftmargin=1pc,
  emph={[1]
    FilePath,IOError,abs,acos,acosh,all,and,any,appendFile,approxRational,asTypeOf,asin,
    asinh,atan,atan2,atanh,basicIORun,break,catch,ceiling,chr,compare,concat,concatMap,
    const,cos,cosh,curry,cycle,decodeFloat,denominator,digitToInt,div,divMod,drop,
    dropWhile,either,elem,encodeFloat,enumFrom,enumFromThen,enumFromThenTo,enumFromTo,
    error,even,exp,exponent,fail,filter,flip,floatDigits,floatRadix,floatRange,floor,
    fmap,foldl,foldl1,foldr,foldr1,fromDouble,fromEnum,fromInt,fromInteger,fromIntegral,
    fromRational,fst,gcd,getChar,getContents,getLine,head,id,inRange,index,init,intToDigit,
    interact,ioError,isAlpha,isAlphaNum,isAscii,isControl,isDenormalized,isDigit,isHexDigit,
    isIEEE,isInfinite,isLower,isNaN,isNegativeZero,isOctDigit,isPrint,isSpace,isUpper,iterate,
    last,lcm,length,lex,lexDigits,lexLitChar,lines,log,logBase,lookup,map,mapM,mapM_,max,
    maxBound,maximum,maybe,min,minBound,minimum,mod,negate,not,notElem,null,numerator,odd,
    or,ord,otherwise,pi,pred,primExitWith,print,product,properFraction,putChar,putStr,putStrLn,quot,
    quotRem,range,rangeSize,read,readDec,readFile,readFloat,readHex,readIO,readInt,readList,readLitChar,
    readLn,readOct,readParen,readSigned,reads,readsPrec,realToFrac,recip,rem,repeat,replicate,return,
    reverse,round,scaleFloat,scanl,scanl1,scanr,scanr1,seq,sequence,sequence_,show,showChar,showInt,
    showList,showLitChar,showParen,showSigned,showString,shows,showsPrec,significand,signum,sin,
    sinh,snd,span,splitAt,sqrt,subtract,succ,sum,tail,take,takeWhile,tan,tanh,threadToIOResult,toEnum,
    toInt,toInteger,toLower,toRational,toUpper,truncate,uncurry,undefined,unlines,until,unwords,unzip,
    unzip3,userError,words,writeFile,zip,zip3,zipWith,zipWith3,listArray,doParse
  },%
  emphstyle={[1]\color{NU0}},%
  emph={[2]
    Bool,Char,Double,Either,Float,IO,Integer,Int,Maybe,Ordering,Rational,Ratio,ReadS,Show,ShowS,String,
    Word8,InPacket
  },%
  emphstyle={[2]\bfseries\color{KW4}},%
  emph={[3]
    case,class,data,deriving,do,else,if,import,in,infixl,infixr,instance,let,
    module,of,primitive,then,type,where
  },
  emphstyle={[3]\color{darkblue}},
  emph={[4]
    quot,rem,div,mod,elem,notElem,seq
  },
  emphstyle={[4]\color{NU0}\bfseries},
  emph={[5]
    EQ,False,GT,Just,LT,Left,Nothing,Right,True,Show,Eq,Ord,Num
  },
  emphstyle={[5]\color{KW4}\bfseries},
  morestring=[b]",%
  morestring=[b]',%
  stringstyle=\color{darkgreen},%
  showstringspaces=false
}
\lstnewenvironment{hs}
{\lstset{language=fooHaskell,backgroundcolor=\color{BG}}}
{\smallskip}
\newcommand{\ihs}[1]{\lstset{language=fooHaskell,basicstyle=\color[gray]{0.6}}\lstinline|#1|}


\lstdefinelanguage{fooMatlab} {%
backgroundcolor=\color[gray]{0.9},
breaklines=true,
basicstyle=\ttfamily\small,
%otherkeywords={ =},
%keywordstyle=\color{blue},
%stringstyle=\color{darkgreen},
showstringspaces=false,
%emph={for, while, if, elif, else, not, and, or, printf, break, continue, return, end, function},
%emphstyle=\color{blue},
%emph={[2]True, False, None, self, NaN, NULL},
%emphstyle=[2]\color{key},
%emph={[3]from, import, as},
%emphstyle=[3]\color{blue},
%upquote=true,
%morecomment=[s]{"""}{"""},
%commentstyle=\color{gray}\slshape,
%framexleftmargin=1mm, framextopmargin=1mm, 
%title=\tiny matlab,
frame=single,
%mathescape =true,
%escapechar=§
}
\newcommand{\imatlab}[1]{\lstset{language=fooMatlab,basicstyle=\color[gray]{0.6}}\lstinline|#1|}
\lstnewenvironment{matlab}[1][]{\lstset{language=fooMatlab,xleftmargin=0.2cm,frame=none,backgroundcolor=\color{white},basicstyle=\color{darkblue}\ttfamily\small,#1}}{} 
\lstnewenvironment{matlabin}[1][]{\lstset{language=fooMatlab,#1}}{} 
\newcommand{\matinput}[1]{\lstset{language=fooMatlab}\lstinputlisting{#1}}

\lstdefinelanguage{fooPython} {%
language=python,
backgroundcolor=\color[gray]{0.7},
breaklines=true,
basicstyle=\ttfamily\small,
%otherkeywords={ =},
keywordstyle=\color{blue},
stringstyle=\color{darkgreen},
morestring=[b]",%
morestring=[b]',%
showstringspaces=false,
emph={class, pass, in, for, while, if, is, elif, else, not, and, or,
def, print, exec, break, continue, return, import, from, lambda, null},
emphstyle=\color{blue},
emph={[2]True, False, None, self},
emphstyle=[2]\color{key},
emph={[3]from, import, as},
emphstyle=[3]\color{blue},
upquote=true,
morecomment=[s]{"""}{"""},
comment=[l]{\#},
commentstyle=\color{gray},
%framexleftmargin=1mm, framextopmargin=1mm, 
%title=\tiny python,
%caption=python,
frame=single
%frameround=tttt,
%mathescape =true,
%escapechar=§
}

\newcommand{\pyinput}[1]{\lstset{language=fooPython}\lstinputlisting{#1}}
\newcommand{\isage}[1]{{\lstset{language=fooPython,basicstyle=\color[gray]{0.3}}\lstinline|#1|}}

\lstnewenvironment{pyout}[1][]{\lstset{language=fooPython,xleftmargin=0.2cm,frame=none,backgroundcolor=\color{white},basicstyle=\color{darkblue}\ttfamily\small,#1}}{}
\lstnewenvironment{pyin}[1][]{\lstset{language=fooPython,#1}}{}
\lstnewenvironment{sageout}[1][]{\lstset{language=fooPython,xleftmargin=0.2cm,frame=none,backgroundcolor=\color{white},basicstyle=\color{darkblue}\ttfamily\small,#1}}{}
\lstnewenvironment{sagein}[1][]{\lstset{language=fooPython,#1}}{}

%\usepackage{caption}
%\DeclareCaptionFont{white}{ \color{white} }
%\DeclareCaptionFormat{listing}{
%  \colorbox[cmyk]{0.43, 0.35, 0.35,0.01 }{
%      \parbox{\textwidth}{\hspace{15pt}#1#2#3}
%        }
%        }
%        \captionsetup[lstlisting]{ format=listing, labelfont=white, textfont=white, singlelinecheck=false, margin=0pt, font={bf,footnotesize} }


\usepackage{mydef}
%\usepackage{cmap} % you can search in the pdf for umlauts and ligatures
\usepackage{colonequals} %corrects the definition-symbols \colonequals (besides others)

\usepackage{ifthen}

%%%%%%%%%%%%%%%%%%%
%Neue Definitionen
%%%%%%%%%%%%%%%%%%%

%Newcommands
\newcommand{\Fun}[1]{\mathcal{#1}}      %Mathcal fuer Funktoren
\newcommand{\field}[1]{\mathbb{#1}}     %Grundkoerper ?? in mathds

\newcommand{\A}{\field{A}}              %Affines A
\newcommand{\Fp}{\field{F}_{\!p}}       %Endlicher Koerper mit p Elementen
\newcommand{\Fq}{\field{F}_{\!q}}       %Endlicher Koerper mit q Elementen
\newcommand{\Ga}{\field{G}_{a}}         %Add Gruppenschema
\newcommand{\K}{\field{K}}              %Generischer Koerper 
\newcommand{\N}{\field{N}}              %Nat Zahlen
\newcommand{\Pj}{\field{P}}             %Projektives P
\newcommand{\R}{\field{R}} 		%Reelle Zahlen
\newcommand{\Q}{\field{Q}}              %Rationale Zahlen  
\newcommand{\Qt}{\field{H}}             %Quaternionen 
\newcommand{\V}{\field{V}}              %Vektorbuendel V
\newcommand{\Z}{\field{Z}}              %Ganze Zahlen
\DeclareMathOperator{\Real}{Re}

\newcommand{\fdg}{\;|\;}                 %fuer die gilt

%Operatoren
\DeclareMathOperator{\Abb}{Abb}
%\usepackage{sagetex}


%
% Aufgaben
%
\parindent0cm % Abs�tze nicht einr�cken 
% Definieren einer neuen Farbe
\definecolor{light-gray}{gray}{.9}

\newcounter{zaehler}     % neuen Z�hler einf�hren
\newenvironment{aufgn}[2][0]
%---- Header
{\begin{samepage}%
%\colorbox{light-gray}{%                         % Box in gray
% \makebox[\textwidth]{%                           % Box in linewidth
%\textbf{Aufgabe \arabic{zaehler} } }\hspace{-\textwidth}\makebox[\textwidth]{\hfill #1 Punkte} }\\[0.05cm]       % Header
\dotfill\\
{\large\textbf{Aufgabe \arabic{zaehler} \ifthenelse{ \equal{#2}{} }{}{: \emph{ #2 } }}\ifthenelse{-1=#1}{(testierbar)}{}\ifthenelse{0=#1 \or -1=#1}{}{\hfill #1 Punkte} }\\[0.4cm]
%{\large\textbf{Exercise \arabic{zaehler}  #2 }\ifthenelse{-1=#1}{(testierbar)}{}\ifthenelse{0=#1 \or -1=#1}{}{\hfill #1 Punkte} }\\[0.4cm]
\begin{minipage}{\textwidth}%
}%
%-----  foot
{\end{minipage}\nopagebreak%\begin{minipage}{1cm} \end{minipage}
%\\ 
%\begin{minipage}{0.1cm} \end{minipage} 
%\hrulefill \begin{minipage}{1cm} \end{minipage}\\[1cm]  
\stepcounter{zaehler}                           % increase counter
\end{samepage}%
\\%
\bigskip%
}


\newenvironment{aufg}[1][0]
%---- Header
{\begin{samepage}%
\refstepcounter{zaehler}% increase counter
%\colorbox{light-gray}{%                         % Box in gray
% \makebox[\textwidth]{%                           % Box in linewidth
%\textbf{Aufgabe \arabic{zaehler} } }\hspace{-\textwidth}\makebox[\textwidth]{\hfill #1 Punkte} }\\[0.05cm]       % Header
\dotfill\\
{\large\textbf{Aufgabe \arabic{zaehler} }\ifthenelse{-1=#1}{(testierbar)}{}\ifthenelse{0=#1 \or -1=#1}{}{\hfill #1 Punkte} }\\[0.4cm]
\begin{minipage}{\textwidth}%
}%
%-----  foot
{\end{minipage}\nopagebreak%\begin{minipage}{1cm} \end{minipage}
%\\ 
%\begin{minipage}{0.1cm} \end{minipage} 
%\hrulefill \begin{minipage}{1cm} \end{minipage}\\[1cm]  
\end{samepage}%
\\%
\bigskip%
}

\lstnewenvironment{sageinsmall}[1][]{\lstset{language=fooPython,belowskip=0em,aboveskip=0.4em,xleftmargin=1em,xrightmargin=1em,#1}}{}

\begin{document}
\begin{center}
        \textbf{\LARGE Einf\"uhrung in Sage}\\
        {\large Kurzreferenz}
\end{center}

\paragraph{Über die Kurzreferenz}
In der Kurzreferenz werden an vielen Stellen Platzhalter benutzt. Diese sind durch spitze Klammern gekennzeichnet. Zum Beispiel 
{\verb <Ausd> }. Wenn in der Referenz steht, dass Zuweisungen in der Form 
\begin{sageinsmall}
  <Bezeichner> = <Ausdruck>
\end{sageinsmall}
gemacht werden, dann ist ein Beispiel für einen solchen Code:\\ \isage{Vier = 2 + 2}
\paragraph{Überlebensregeln}
\begin{itemize}
%\item Mehrere Befehle in einer Zeile trennen: {\color{blue}\isage{;}}
%\item Bei Eingaben, die über mehrere Zeilen gehen, kann ein
%  Zeilenumbruch durch {\color{blue} \verb~<ENTER>~} erreicht werden.
\item Das Auswerten eines Blocks erfolgt mit {\color{blue} \verb~<SHIFT>+<ENTER>~}.
\item Ein neues Eingabefeld erhält man durch Klicken auf den blauen, horizontalen Balken oder das Plussymbol unter jeder Zelle.
\item Bei Python (und damit auch Sage) ist das Einrücken von Codezeilen von Bedeutung. Es werden damit die bei anderen 
      Programmiersprachen üblichen Klammern ersetzt.
\end{itemize}
\paragraph{Nützliches}
\begin{itemize}
%\item {\color{blue} \isage{_} } refenziert die letzte Ausgabe (Warnung: unübersichtlich!).
\item Löschen aller eigenen Variablen und Zurücksetzen auf den Anfangsstatus: {\color{blue} \isage{reset()}}
%\item Anzeigen aller definierten Variablen: {\color{blue} }
%\item Anzeigen aller selbst definierten Variablen: {\color{blue} }
\item Aktivieren des Feldes \emph{Typeset} lässt alle Ausgaben von \LaTeX{} rendern.
\item Kommentare werden mit \# eingeleitet.
%\item profiler und debugger -> commandline
\item Dokumentation im Notebook mit HTML und \LaTeX{}-Formeln: Durch Klick auf die Sprechblase neben dem Plusymbol unter jeder Zelle startet einen WYSIWYG Editor.
\item Publish: Im Notebook kann durch Klicken des \emph{Publish}-Reiters das Notebook für alle offen gelegt werden. 
%\item Unter dem Menupunkt \emph{File} kann man 
\end{itemize}

\paragraph{Hilfefunktionen}
\begin{itemize}
\item {\color{blue} Autovervollständigung :} Mit der {\color{blue} \verb~<TAB>~}-Taste erhält man alle möglichen Funktions- und/oder Variablen-Namen im gegebenen Kontext.\\
Dies gilt insbesondere auch für Objektfunktionen\\ (\isage{<Objekt>.<Funktion()>}).
\item {\color{blue} \isage{<Befehl>?} :} Gibt ausführliche Hilfe zu \isage{Befehl} an.
\item {\color{blue} \isage{<Befehl>??} :} Gibt den Quellcode von \isage{Befehl} an.
\item {\color{blue} \isage{help(<Befehl>)} :} Öffnet ein Hilfefenster zu \isage{Befehl}.
\item {\color{blue} \isage{search_doc('<Begriff>')} :} Sucht in der Hilfe nach <Begriff>.
\item Dokumentation:
\begin{itemize}
    \item Sage (lokal): {\small\url{file:///usr/local/sage-5.5/devel/sage-main/doc/output/html/en/index.html}}
    \item Sage (Hauptseite): {\small\url{http://www.sagemath.org/doc/index.html}}
\item Python: \url{http://docs.python.org/}
\end{itemize}
\end{itemize}

\paragraph{Objektorientierung}
Herausfinden der Klasse eines Objekts:
\begin{sageinsmall}
type(<object>)
\end{sageinsmall}
\paragraph{Datentypen}
\textbf{Liste/Tuple:} 	{\href{http://docs.python.org/library/functions.html#list}{\textbf{list(),} } \href{http://docs.python.org/library/functions.html#tuple}{\textbf{tuple()}}
\begin{itemize}
 \item Konstruktion
\begin{sageinsmall}
<Bez> = [<Wert1>,<Wert2>,...] #Liste
<Bez> = (<Wert1>,<Wert2>,...) #Tupel
\end{sageinsmall}
\end{itemize}
\paragraph{Dictionaries:}		\href{http://docs.python.org/library/stdtypes.html?highlight=.update#mapping-types-dict}{\textbf{dictionaries}}
\begin{itemize}
 \item Deklarieren eines Dictionaries:
\begin{sageinsmall}
<Bez> = {<Index1>:<Wert1>,...}
\end{sageinsmall}
 \item Beispiel:
\begin{sageinsmall}
Auto = {'Marke':'VW','Typ':'Up','Km':150000}
Auto['Marke']		# gibt 'VW' aus
\end{sageinsmall}
\end{itemize}

\paragraph{map() und map\_threaded():} \href{https://sage.math.uni-goettingen.de/doc/static/reference/sage/combinat/generator.html?highlight=map#sage.combinat.generator.map}{\textbf{map()}}\\
(Rekursive) Auswertung der einstelligen Funktion auf eine (verschachtelte) Liste.
\begin{sageinsmall}
 map_threaded(<Funktion>,<Menge oder Liste>)
\end{sageinsmall}
Beispiel:
\begin{sageinsmall}
 map(sqrt,[4,144,16]) 			#[2,12,4]
 map_threaded(sqrt,[[25,16],9]	#[[5,4],3]
\end{sageinsmall}

\paragraph{filter:} \href{https://sage.math.uni-goettingen.de/doc/static/reference/sage/combinat/combinat.html?highlight=filter#sage.combinat.combinat.CombinatorialClass.filter}{\textbf{filter()}}\\
Filtert nach Wahrheitswert der übergebenen Funktion. 
\begin{sageinsmall}
filter(<Funktion>,<Menge oder Liste>)
\end{sageinsmall}

\paragraph{Zahlen}
\begin{itemize}
    \item Zahlenmengen/Körper:\\
    \begin{tabular}{|ll|}
\hline 
{\isage{ZZ}} & Ganze Zahlen\\
{\isage{QQ}} & Rationale Zahlen\\
{\isage{RR}} & Reelle Zahlen \\
{\isage{CC}} & Komplexe Zahlen \\
{\isage{GF(2)}} & Körper mit zwei Elementen\\
%{\isage{parent}} & Vaterobjekt; Gruppe der Zahl\\
\hline
\end{tabular}
   \item Einige wichtige Funktionen für Zahlen:\\
\begin{tabular}{|ll|}
\hline 
{\isage{abs}} & Absolutbetrag\\
{\isage{sign}} & Vorzeichen\\
{\isage{arg}} & Argument\\
{\isage{sqrt}} & Wurzel \\
%{\isage{parent}} & Vaterobjekt; Gruppe der Zahl\\
\hline
\end{tabular}
\begin{tabular}{|ll|}
\hline 
{\isage{ceil}} & Aufrunden\\
{\isage{floor}} & Abrunden\\
{\isage{round}} & Runden \\
{\isage{n}} & num. Näherung \\
%{\isage{parent}} & Vaterobjekt; Gruppe der Zahl\\
\hline
\end{tabular}
\item Annahmen: \href{https://sage.math.uni-goettingen.de/doc/static/reference/sage/symbolic/expression.html?highlight=assume#sage.symbolic.expression.Expression.assume}{\textbf{assume()}}
    \begin{sageinsmall}
assume(<Annahme>)        
    \end{sageinsmall}
    Achtung: Annahmen werden mit \isage{reset()} nicht wieder zurückgesetzt. Dafür gibt es den Befehl \isage{forget()}.
\end{itemize}

\paragraph{Matrix:}\href{https://sage.math.uni-goettingen.de/doc/static/reference/sage/matrix/constructor.html#sage.matrix.constructor.Matrix}{\textbf{matrix()}}
\begin{itemize}
\item Deklaration
\begin{sageinsmall}
matrix(<Koerper>,[[a11,...],[a21,..],..])
\end{sageinsmall}
Dabei ist die Angabe des Körpers/Gruppe meist optional. Beispiel:
\begin{sageinsmall}
matrix([[1,2],[3,4]])
\end{sageinsmall} 
Einige Funktionen für Matrizen:\\
\begin{tabular}{|ll|}
\hline 
{\isage{det}} & Determinante\\
{\isage{eigenvalues}} & Eigenwerte\\
{\isage{inverse}} & Inverse berechnen\\
{\isage{rank}} & Rang der Matrix bestimmen \\
{\isage{right_kernel}} & Kern der Matrix bestimmen \\
\hline
\end{tabular}

\end{itemize}

\paragraph{Vektor:}\href{https://sage.math.uni-goettingen.de/doc/static/reference/sage/modules/free_module_element.html#sage.modules.free_module_element.vector}{\textbf{vector()}}
\begin{itemize}
 \item Deklaration
\begin{sageinsmall}
vector([v1,v2,..]) 
\end{sageinsmall}
\end{itemize}
\paragraph{Vektorräume:} \href{https://sage.math.uni-goettingen.de/doc/static/reference/modules/sage/modules/free_module.html#sage.modules.free_module.VectorSpace}{\textbf{vectorspace()}}
\begin{itemize}
 \item Deklaration
\begin{sageinsmall}
vectorspace(<Koerper>,<Dimension>)
\end{sageinsmall}
 \item Lineare Hülle:
\begin{sageinsmall}
span([<Vec1>,<Vec2>,...],<Koerper>)
\end{sageinsmall}
\end{itemize}
\paragraph{Abfragen:} \href{http://docs.python.org/reference/compound_stmts.html#the-if-statement}{\textbf{if}}
\begin{itemize}
 \item  Syntax: \begin{sageinsmall}
if <Boolscher Ausdruck>:	#z.B. x==2
    <Code-Block>
else:
    <Code-Block>
\end{sageinsmall}
\end{itemize}

\paragraph{Schleifen}
\begin{itemize}
\item Einzeilige for-Schleife
    \begin{sageinsmall}
[<Ausd(Bez)> for <Bez> in <Liste> if <Bed>]
    \end{sageinsmall}
\item for-Schleife: \href{http://docs.python.org/tutorial/controlflow.html#for-statements}{\textbf{for}}
    \begin{sageinsmall}
for <Bezeichner> in <liste>:
    <Code-Block>
    \end{sageinsmall}
  \item while-Schleife: \href{http://docs.python.org/reference/compound_stmts.html#the-while-statement}{\textbf{while()}}
\begin{sageinsmall}
while <Bedingung>:
    <Code-Block>
\end{sageinsmall}
\end{itemize}

\paragraph{Funktionen}
\begin{itemize}
 \item Mathematische Funktionen (Ausdrücke)
\begin{sageinsmall}
<Bez>(<Arg1>,<Arg2>,...) = <Ausdruck>
\end{sageinsmall}
    \item einzeilige Deklaration: \href{http://docs.python.org/reference/compound_stmts.html#function-definitions}{\textbf{def}}     
        \begin{sageinsmall}
def <Bez> (<Arg1>,..): return <Rückgabewert>            
        \end{sageinsmall}
 \item normale Deklaration:
\begin{sageinsmall}
def <Bez><(<arg1>,<arg2>,..)>:
    <Code-Block>
    return <Rückgabewert>
\end{sageinsmall}
Beispiele:
\begin{sageinsmall}
Summe(x,y) = x+y

def Summe(x,y): return x+y

def Summe(x,y):
    s = x+y
    return s
\end{sageinsmall}
\end{itemize}

\paragraph{Grafik:}\href{https://sage.math.uni-goettingen.de/doc/static/reference/sage/combinat/e_one_star.html?highlight=.plot#sage.combinat.e_one_star.Patch.plot}{\textbf{plot()}}	/	\href{https://sage.math.uni-goettingen.de/doc/static/reference/sage/combinat/e_one_star.html?highlight=.plot#sage.combinat.e_one_star.Patch.plot3d}{\textbf{plot3d()}}
\begin{itemize}
\item 2D/3D Plot
\begin{sageinsmall}
plot(<Funktion>,(x,a,b),<Optionen>,...)
plot3d(<Funk>,(x,a,b),(y,c,d),<Optionen>,...)
\end{sageinsmall}
Einige mögliche Optionen: \\
\begin{tabular}{|ll|}
\hline 
{\isage{color}} & Farbe z.B. 'red', '\#FF0000', (1,0,0)\\
{\isage{plot_points}} & Bildauflösung\\
%{\isage{xmin, xmax}} & Achsendefinitionsbereich\\
{\isage{opacity}} & Transparenz (bei 3D Plots) \\
{\isage{aspect_ratio}} & Seitenverhältnis der Achsen \\
\hline
\end{tabular}
Beispiel: 
\begin{sageinsmall}
plot(x^2,(x,-2,2),color='red')
\end{sageinsmall}
%\item Parametrisierte Funktionen \href{https://sage.math.uni-goettingen.de/doc/static/reference/sage/combinat/e_one_star.html?highlight=.plot#sage.combinat.e_one_star.Patch.plot}{\textbf{plot()}}	/	\href{https://sage.math.uni-goettingen.de/doc/static/reference/sage/combinat/e_one_star.html?highlight=.plot#sage.combinat.e_one_star.Patch.plot3d}{\textbf{parametric_plot()}}
%\begin{sageinsmall}
%parametric_plot([<F1>,<F2>],(x,a,b))
%\end{sageinsmall}
\end{itemize}
\paragraph{Summen:} \href{https://sage.math.uni-goettingen.de/doc/static/reference/calculus/sage/calculus/calculus.html#sage.calculus.calculus.symbolic_sum}{\textbf{sum()}}

\begin{itemize}
 \item Aufaddieren von Zahlen:
 \begin{sageinsmall}
add([<Summand1>,<Summand2>,..]) 
\end{sageinsmall}
 \item Symbolischer Summenausdruck
 \begin{sageinsmall}
sum(<Ausdr>,<Var>,<Start>,<Stop>) 
\end{sageinsmall}
Achtung: Der symbolische Summenausdruck kann von Sage nicht immer in einen Zahlwert umgewandelt werden.\\
Der symbolische Summenoperator kann auch Reihen vereinfachen. Beispiel:
 \begin{sageinsmall}
sum(x^(-2),x,1,oo)		#1/6*pi^2
\end{sageinsmall}
\end{itemize}

\paragraph{Grenzwerte:} \href{https://sage.math.uni-goettingen.de/doc/static/reference/calculus/sage/calculus/calculus.html#sage.calculus.calculus.limit}{\textbf{limit()}}
\begin{itemize}
 \item Verhalten von Funktionen an Grenzwerten:
 \begin{sageinsmall}
limit(<Ausdr>,<Variable>=<Grenzwert>,dir=<Richtung>)
\end{sageinsmall}
Beispiel:
 \begin{sageinsmall}
limit(e^(-1/x), x=0, dir='right')
 \end{sageinsmall}
 \end{itemize}
 
\paragraph{Differentiation:} \href{https://sage.math.uni-goettingen.de/doc/static/reference/sage/calculus/functional.html?highlight=diff#sage.calculus.functional.diff}{\textbf{diff()}}
\begin{itemize}
 \item Ableitungen: 
\begin{sageinsmall}
diff(<Ausdruck>,<var1>,<var2>,<var3>,...)
diff(<Ausdruck>,<var>,<anzahl>) 
\end{sageinsmall}
Beispiele:
\begin{sageinsmall}
diff(x^2*y^2,x,y) 	#6*x^2*y 
diff(x^10,x,3) 		#720*x^7 
\end{sageinsmall}
\end{itemize}

\paragraph{Taylorformel:} \href{https://sage.math.uni-goettingen.de/doc/static/reference/sage/calculus/functional.html?highlight=diff#sage.calculus.functional.integral}{\textbf{taylor()}}
\begin{itemize}
 \item Taylorapproximation: 

\begin{sageinsmall}
taylor(<funktion>,<var>,<stelle>,<grad>)
\end{sageinsmall}
\end{itemize}
\paragraph{Gleichungen:}
\begin{itemize}
 \item Exaktes Lösen von Gleichungen: \href{https://sage.math.uni-goettingen.de/doc/static/reference/calculus/sage/symbolic/relation.html#sage.symbolic.relation.solve}{\textbf{solve()}}
  \begin{sageinsmall}
solve([<Gleichung1>,<Gleichung2>,...],<Var>)
\end{sageinsmall}
Bei nur einer Gleichung, kann die Liste auch weggelassen werden. Beispiel:
  \begin{sageinsmall}
S=solve(x^2-4 == 0,x) #Ergebnis: [x==2,x==-2]
\end{sageinsmall}
Zugreifen auf die Lösung:
  \begin{sageinsmall}
S[0].rhs() #Ergebnis: 2
S[1].rhs() #Ergebnis: -2
\end{sageinsmall}
\item Numerisches Lösen: \href{https://sage.math.uni-goettingen.de/doc/static/reference/numerical/sage/numerical/optimize.html#sage.numerical.optimize.find_root}{\textbf{find\_root()}}

   \begin{sageinsmall}
find_root(<Gleichung>,<uG>,<oG>,<Toleranz>)
\end{sageinsmall}
Findet Lösungen im Intervall  $[<uG>,<oG>]$.\\
Beispiel:
  \begin{sageinsmall}
find_root(cos(x)==sin(x),0,2)
\end{sageinsmall}
\end{itemize}

\paragraph{Integrale:} \href{https://sage.math.uni-goettingen.de/doc/static/reference/sage/calculus/functional.html?highlight=diff#sage.calculus.functional.integral}{\textbf{integrate()}}
\begin{itemize}
\item bestimmte/unbestimmte Integrale: 
\begin{sageinsmall}
integrate(<funktion>,<var>,[<uG>,<oG>]) 
\end{sageinsmall}
\end{itemize}

\paragraph{Strings/Zeichenketten und Ausgabe:}\href{http://docs.python.org/library/string.html?highlight=string.replace#string-constants}{\textbf{string}}

\begin{itemize}
 \item Deklaration:
\begin{sageinsmall}
<Bezeichner> = '<Inhalt>'
\end{sageinsmall}
\item Zu Strings konvertieren: \href{http://docs.python.org/library/functions.html?highlight=print#str}{\textbf{str()}}
\begin{sageinsmall}
str(<vorher kein String>)
\end{sageinsmall}

\item Stringformatierung: \href{http://docs.python.org/2/library/string.html#format-string-syntax}{\textbf{format}}
    \begin{sageinsmall}
print ("Text {<format>} und {<format>}... ".format(x,y,...))
    \end{sageinsmall}
    wichtigsten Formate:
    \begin{itemize}
        \item \isage{:d} : integer (Ganze Zahl)
        \item \isage{:f} : Nachkommastellen-Notation
        \item \isage{:e} : Exponential-Notation 
    \end{itemize}
\end{itemize}


\end{document}

