\documentclass[a4paper,9pt,DIV15,twocolumn]{scrartcl}
\usepackage[psamsfonts]{amssymb}
\usepackage{amsmath}
\usepackage[svgnames]{xcolor} %color definition
\usepackage{fontspec,xunicode,xltxtra}
%\usepackage{fontspec,xunicode}
%\usepackage{polyglossia}
%\setdefaultlanguage[spelling=new, latesthyphen=true]{german}
%\setsansfont{DejaVu Sans}
%\setsansfont{Verdana}
%\setsansfont{Arial}
%\setromanfont[Mapping=tex-text]{Linux Libertine}
%\setsansfont[Mapping=tex-text]{Myriad Pro}
%\setmonofont[Mapping=tex-text]{Courier New}

%\setsansfont{Linux Biolinum}

\usepackage[ngerman]{babel}
\selectlanguage{ngerman}

%
% math/symbols
%
\usepackage{amssymb}
\usepackage{amsthm}
% \usepackage{latexsym}
\usepackage{amsmath}
%\usepackage{amsxtra} %Weitere Extrasymbole
%\usepackage{empheq} %Gleichungen hervorheben
%\usepackage{bm}
 %\bm{A} Boldface im Mathemodus

\usepackage{multimedia}
%\usepackage{tikz}

\usepackage{cellspace}
\setlength{\cellspacetoplimit}{2pt}
\setlength{\cellspacebottomlimit}{2pt}

%%%%%%%%%%%%%%%%%% Fuer Frames [fragile]-Option verwenden!
%Programm-Listing
%%%%%%%%%%%%%%%%%%
%Listingsumgebung fuer verbatim
%Grauhinterlegeter Text
%Automatischer Zeilenumbruch ist aktiviert
\usepackage{listings}
% This command allows you to typeset syntax highlighted Matlab
% code ``inline''.
\newcommand{\isage}[1]{\lstinline|#1|}

\definecolor{lgray}{gray}{0.80}
\definecolor{gray}{gray}{0.3}
\definecolor{darkgreen}{rgb}{0,0.4,0}
\definecolor{darkblue}{rgb}{0,0,0.8}
\definecolor{key}{rgb}{0,0.5,0} 
%\lstset{backgroundcolor=\color{lgray}, frame=single, basicstyle=\ttfamily, breaklines=true}
\lstnewenvironment{sage}[1][]{\lstset{xleftmargin=0.2cm,frame=none,backgroundcolor=\color{white},basicstyle=\color{darkblue}\ttfamily\small,#1}}{} 
\lstnewenvironment{sagein}[1][]{\lstset{#1}}{} 
%\lstnewenvironment{sage}{\lstset{,language=python, keywordstyle=color{blue},    commentstyle=color{green}, emphstyle=\color{red}, %frame=single, stringstyle=\color{red}, basicstyle=\ttfamily, ,mathescape =true,escapechar=§}}{}

\lstset{
language=python,
backgroundcolor=\color{lgray},
breaklines=true,
basicstyle=\ttfamily\small,
%otherkeywords={ =},
keywordstyle=\color{blue},
stringstyle=\color{darkgreen},
showstringspaces=false,
emph={class, pass, in, for, while, if, is, elif, else, not, and, or,
def, print, exec, break, continue, return},
emphstyle=\color{blue},
emph={[2]True, False, None, self},
emphstyle=[2]\color{key},
emph={[3]from, import, as},
emphstyle=[3]\color{blue},
upquote=true,
morecomment=[s]{"""}{"""},
commentstyle=\color{gray}\slshape,
%framexleftmargin=1mm, framextopmargin=1mm, 
frame=single,
mathescape =true,
escapechar=§
}


\usepackage{mydef}
%\usepackage{cmap} % you can search in the pdf for umlauts and ligatures
\usepackage{colonequals} %corrects the definition-symbols \colonequals (besides others)
\usepackage{ifthen}
%%%%%%%%%%%%%%%%%%%
%Neue Definitionen
%%%%%%%%%%%%%%%%%%%

%Newcommands
\newcommand{\Fun}[1]{\mathcal{#1}}      %Mathcal fuer Funktoren
\newcommand{\field}[1]{\mathbb{#1}}     %Grundkoerper ?? in mathds

\newcommand{\A}{\field{A}}              %Affines A
\newcommand{\C}{\field{C}}              %Complexes C
\newcommand{\Fp}{\field{F}_{\!p}}       %Endlicher Koerper mit p Elementen
\newcommand{\Fq}{\field{F}_{\!q}}       %Endlicher Koerper mit q Elementen
\newcommand{\Ga}{\field{G}_{a}}         %Add Gruppenschema
\newcommand{\K}{\field{K}}              %Generischer Koerper 
\newcommand{\N}{\field{N}}              %Nat Zahlen
\newcommand{\Pj}{\field{P}}             %Projektives P
\newcommand{\R}{\field{R}} 		%Reelle Zahlen
\newcommand{\Q}{\field{Q}}              %Rationale Zahlen  
\newcommand{\Qt}{\field{H}}             %Quaternionen 
\newcommand{\V}{\field{V}}              %Vektorbuendel V
\newcommand{\Z}{\field{Z}}              %Ganze Zahlen
\DeclareMathOperator{\Real}{Re}

\newcommand{\fdg}{\;|\;}                 %fuer die gilt

%Operatoren
\DeclareMathOperator{\Abb}{Abb}
%\usepackage{sagetex}

%
% Aufgaben
%
\parindent0cm % Abs�tze nicht einr�cken 
% Definieren einer neuen Farbe
\definecolor{light-gray}{gray}{.9}

\newcounter{zaehler}     % neuen Z�hler einf�hren
\stepcounter{zaehler}    % Z�hler einen hochz�hlen

\newenvironment{aufg}[1][0]
%---- Header
{\begin{samepage}%
%\colorbox{light-gray}{%                         % Box in gray
% \makebox[\textwidth]{%                           % Box in linewidth
%\textbf{Aufgabe \arabic{zaehler} } }\hspace{-\textwidth}\makebox[\textwidth]{\hfill #1 Punkte} }\\[0.05cm]       % Header
\dotfill\\
{\large\textbf{Aufgabe \arabic{zaehler} }\ifthenelse{0=#1}{}{\hfill #1 Punkte} }\\[0.4cm]
\begin{minipage}{\textwidth}
}
%-----  foot
{\end{minipage} \nopagebreak %\begin{minipage}{1cm} \end{minipage}
%\\ 
%\begin{minipage}{0.1cm} \end{minipage} 
%\hrulefill \begin{minipage}{1cm} \end{minipage}\\[1cm]  
\stepcounter{zaehler}                           % increase counter
\end{samepage}%
\\%
\bigskip%
}

\begin{document}
%--------------------------------------------------- Header
\begin{center}
    \textbf{\LARGE Einf\"uhrung in Sage}\\
    {\large Zusammenfassung Einheit 01}
\end{center}
\textsl{Hinweis:} Textbausteine mit \isage{<name>} weisen darauf hin, das anstatt des Ausdrucks eine passende Variable eingefügt werden muss.

\medskip
\textbf{Kurvendiskussion}
\begin{itemize}
 \item Deklarieren von Variablen mit 
\begin{sagein}
var('<varname>')
\end{sagein}
\item Definieren von Variablen 
    \begin{sagein}
<varname>=<value>        
    \end{sagein}
\item Definieren von (mathematischen) Funktionen 
    \begin{sagein}
<functionname>(<arguments>) = <expr>
    \end{sagein}
%\item Symbolisches Rechnen 
%\begin{itemize}
\item Grenzwertbestimmung:
    \begin{sagein}
<expr>.limit(x=<a>, dir='<plus|minus>')
    \end{sagein}
\item Bilden von Ableitungen
\begin{sagein}
<expr>.differentiate(<variable>)
\end{sagein}

%\end{itemize}  
%\end{itemize}
%\begin{itemize}
\item Lösen von Gleichungen
\begin{sagein}
solve( f(x)==0, x)
\end{sagein}

\item Berechnen numerischer Approximationen
\begin{sagein}
float(<expr>)
\end{sagein}
\item Plotten einer Funktion 
\begin{sagein}
plot(<function>,(<lowerbound>,<upperbound>))
\end{sagein}
\end{itemize}

\textbf{Symbolisches Rechnen}
\begin{itemize}
\item Symbolisch Integrieren
\begin{sagein}
integrate(<expr>,<variable>)
\end{sagein}
 \item Numerisch Integrieren 
\begin{sagein}
integrate(<expr>,<variable>,<lower>,<upper>)
\end{sagein}
\item Faktorisieren 
\begin{sagein}
factor(<expr>)
\end{sagein}
\item Sortieren 
\begin{sagein}
<expr>.collect(x)
\end{sagein}
\item Partialbruchzerlegung 
\begin{sagein}
<expr>.partial_fraction()
\end{sagein}
\item (vollständiges) Vereinfachen 
\begin{sagein}
<expr>.full_simplify()
\end{sagein}
\item Vereinfachen mit radicals 
\begin{sagein}
<expr>.radical_simplify()
\end{sagein}
\end{itemize}

\textbf{AGLA}
\begin{itemize}
\item Matrix eingeben 
\begin{sagein}
matrix([ [<z1s1>,<z1s2>],[<z2s1>,<z2s2>] ])
\end{sagein}
 \item Vektor eingeben 
\begin{sagein}
vector([<a>,<b>,<c>])
\end{sagein}
\item LGS lösen 
\begin{sagein}
A\b
\end{sagein}
\item Matrixoperationen 
\begin{sagein}
A+B, A-B, A*B
\end{sagein}
\item Matrix invertieren 
\begin{sagein}
A^(-1); A.inverse()
\end{sagein}
\item Substitutieren 
\begin{sagein}
<expr>.subs(<variable>=<subs>)
\end{sagein}
\end{itemize}

\textbf{Etwas Programmieren}
\begin{itemize}
    \item Listen (geordnet)
        \begin{sagein}
[a,b,c,..]
        \end{sagein}
    \item Tuple
        \begin{sagein}
(a,b,c,..)
        \end{sagein}
    \item (Nicht-mathematische) Funktionen
\begin{sagein}
def <function>(<argument>): <befehle> return <rueckgabewert>
\end{sagein}
    \item Einzeilige Schleifen
\begin{sagein}
[<expr(var)> for <var> in <range|liste> if <expr>]
\end{sagein}
\end{itemize}

\textbf{Zahlentheorie}
\begin{itemize}
    \item Teiler 
\begin{sagein}
divisors(<number>)
\end{sagein}
\item Anzahl Teiler
\begin{sagein}
number_of_divisors(<number>)
\end{sagein}
    \item Primzahl-Überprüfung 
\begin{sagein} 
is_prime(<number>)
\end{sagein}
\end{itemize}



\end{document}

