\documentclass[a4paper,9pt,DIV15,twocolumn]{scrartcl}
%\usepackage[psamsfonts]{amssymb}
%\usepackage{amsmath}
\usepackage[svgnames,hyperref]{xcolor} %color definition
\usepackage[xetex,bookmarks=true,pdfstartview=FitH,bookmarksopen=true,
    colorlinks,citecolor=Blue,linkcolor=DarkBlue,urlcolor=Green,
    pagebackref=true,plainpages=false,pdfpagelabels=true,unicode=true,
    breaklinks=true,naturalnames=false,setpagesize=true,a4paper=true,hyperindex]{hyperref}
%\usepackage[xetex]{hyperref}

%\usepackage{fontspec,xunicode}
% %\usepackage{polyglossia}
%\setdefaultlanguage[spelling=new, latesthyphen=true]{german}
%\setsansfont{DejaVu Sans}
%\setsansfont{Verdana}
%\setsansfont{Arial}
%\setromanfont[Mapping=tex-text]{Linux Libertine}
%\setsansfont[Mapping=tex-text]{Myriad Pro}
%\setmonofont[Mapping=tex-text]{Courier New}

%\setsansfont{Linux Biolinum}

\usepackage[ngerman]{babel}
\selectlanguage{ngerman}

%
% math/symbols
%
\usepackage{amssymb}
\usepackage{amsthm}
% \usepackage{latexsym}
\usepackage{amsmath}
%\usepackage{amsxtra} %Weitere Extrasymbole
%\usepackage{empheq} %Gleichungen hervorheben
%\usepackage{bm}
 %\bm{A} Boldface im Mathemodus
\usepackage{fontspec,xunicode,xltxtra}

\usepackage{multimedia}
%\usepackage{tikz}

\usepackage{cellspace}
\setlength{\cellspacetoplimit}{2pt}
\setlength{\cellspacebottomlimit}{2pt}

%%%%%%%%%%%%%%%%%% Fuer Frames [fragile]-Option verwenden!
%Programm-Listing
%%%%%%%%%%%%%%%%%%
%Listingsumgebung fuer verbatim
%Grauhinterlegeter Text
%Automatischer Zeilenumbruch ist aktiviert
%\usepackage{listings}
\usepackage[framed]{mcode}
%\usepackage{mcode}
% This command allows you to typeset syntax highlighted Matlab
% code ``inline''.
% mcode fuer matlab

\definecolor{lgray}{gray}{0.80}
\definecolor{gray}{gray}{0.3}
\definecolor{darkgreen}{rgb}{0,0.4,0}
\definecolor{darkblue}{rgb}{0,0,0.8}
\definecolor{key}{rgb}{0,0.5,0} 
\definecolor{NU0}{RGB}{68,85,136} % #458
\definecolor{KW3}{RGB}{85,68,136}
\definecolor{KW4}{RGB}{153,0,0}
\definecolor{dred}{RGB}{221,17,68} % #d14
\definecolor{BG}{RGB}{240,240,240}
%\lstset{backgroundcolor=\color{lgray}, frame=single, basicstyle=\ttfamily, breaklines=true}
%\lstnewenvironment{sage}{\lstset{,language=python, keywordstyle=color{blue},    commentstyle=color{green}, emphstyle=\color{red}, %frame=single, stringstyle=\color{red}, basicstyle=\ttfamily, ,mathescape =true,escapechar=§}}{}

\lstdefinelanguage{fooHaskell} {%
  basicstyle=\footnotesize\ttfamily,%
  commentstyle=\slshape\color{gray},%
  keywordstyle=\bfseries,%\color{KW4},
  breaklines=true,
  sensitive=true,
  xleftmargin=1pc,
  emph={[1]
    FilePath,IOError,abs,acos,acosh,all,and,any,appendFile,approxRational,asTypeOf,asin,
    asinh,atan,atan2,atanh,basicIORun,break,catch,ceiling,chr,compare,concat,concatMap,
    const,cos,cosh,curry,cycle,decodeFloat,denominator,digitToInt,div,divMod,drop,
    dropWhile,either,elem,encodeFloat,enumFrom,enumFromThen,enumFromThenTo,enumFromTo,
    error,even,exp,exponent,fail,filter,flip,floatDigits,floatRadix,floatRange,floor,
    fmap,foldl,foldl1,foldr,foldr1,fromDouble,fromEnum,fromInt,fromInteger,fromIntegral,
    fromRational,fst,gcd,getChar,getContents,getLine,head,id,inRange,index,init,intToDigit,
    interact,ioError,isAlpha,isAlphaNum,isAscii,isControl,isDenormalized,isDigit,isHexDigit,
    isIEEE,isInfinite,isLower,isNaN,isNegativeZero,isOctDigit,isPrint,isSpace,isUpper,iterate,
    last,lcm,length,lex,lexDigits,lexLitChar,lines,log,logBase,lookup,map,mapM,mapM_,max,
    maxBound,maximum,maybe,min,minBound,minimum,mod,negate,not,notElem,null,numerator,odd,
    or,ord,otherwise,pi,pred,primExitWith,print,product,properFraction,putChar,putStr,putStrLn,quot,
    quotRem,range,rangeSize,read,readDec,readFile,readFloat,readHex,readIO,readInt,readList,readLitChar,
    readLn,readOct,readParen,readSigned,reads,readsPrec,realToFrac,recip,rem,repeat,replicate,return,
    reverse,round,scaleFloat,scanl,scanl1,scanr,scanr1,seq,sequence,sequence_,show,showChar,showInt,
    showList,showLitChar,showParen,showSigned,showString,shows,showsPrec,significand,signum,sin,
    sinh,snd,span,splitAt,sqrt,subtract,succ,sum,tail,take,takeWhile,tan,tanh,threadToIOResult,toEnum,
    toInt,toInteger,toLower,toRational,toUpper,truncate,uncurry,undefined,unlines,until,unwords,unzip,
    unzip3,userError,words,writeFile,zip,zip3,zipWith,zipWith3,listArray,doParse
  },%
  emphstyle={[1]\color{NU0}},%
  emph={[2]
    Bool,Char,Double,Either,Float,IO,Integer,Int,Maybe,Ordering,Rational,Ratio,ReadS,Show,ShowS,String,
    Word8,InPacket
  },%
  emphstyle={[2]\bfseries\color{KW4}},%
  emph={[3]
    case,class,data,deriving,do,else,if,import,in,infixl,infixr,instance,let,
    module,of,primitive,then,type,where
  },
  emphstyle={[3]\color{darkblue}},
  emph={[4]
    quot,rem,div,mod,elem,notElem,seq
  },
  emphstyle={[4]\color{NU0}\bfseries},
  emph={[5]
    EQ,False,GT,Just,LT,Left,Nothing,Right,True,Show,Eq,Ord,Num
  },
  emphstyle={[5]\color{KW4}\bfseries},
  morestring=[b]",%
  morestring=[b]',%
  stringstyle=\color{darkgreen},%
  showstringspaces=false
}
\lstnewenvironment{hs}
{\lstset{language=fooHaskell,backgroundcolor=\color{BG}}}
{\smallskip}
\newcommand{\ihs}[1]{\lstset{language=fooHaskell,basicstyle=\color[gray]{0.6}}\lstinline|#1|}


\lstdefinelanguage{fooMatlab} {%
backgroundcolor=\color[gray]{0.9},
breaklines=true,
basicstyle=\ttfamily\small,
%otherkeywords={ =},
%keywordstyle=\color{blue},
%stringstyle=\color{darkgreen},
showstringspaces=false,
%emph={for, while, if, elif, else, not, and, or, printf, break, continue, return, end, function},
%emphstyle=\color{blue},
%emph={[2]True, False, None, self, NaN, NULL},
%emphstyle=[2]\color{key},
%emph={[3]from, import, as},
%emphstyle=[3]\color{blue},
%upquote=true,
%morecomment=[s]{"""}{"""},
%commentstyle=\color{gray}\slshape,
%framexleftmargin=1mm, framextopmargin=1mm, 
%title=\tiny matlab,
frame=single,
%mathescape =true,
%escapechar=§
}
\newcommand{\imatlab}[1]{\lstset{language=fooMatlab,basicstyle=\color[gray]{0.6}}\lstinline|#1|}
\lstnewenvironment{matlab}[1][]{\lstset{language=fooMatlab,xleftmargin=0.2cm,frame=none,backgroundcolor=\color{white},basicstyle=\color{darkblue}\ttfamily\small,#1}}{} 
\lstnewenvironment{matlabin}[1][]{\lstset{language=fooMatlab,#1}}{} 
\newcommand{\matinput}[1]{\lstset{language=fooMatlab}\lstinputlisting{#1}}

\lstdefinelanguage{fooPython} {%
language=python,
backgroundcolor=\color[gray]{0.7},
breaklines=true,
basicstyle=\ttfamily\small,
%otherkeywords={ =},
keywordstyle=\color{blue},
stringstyle=\color{darkgreen},
morestring=[b]",%
morestring=[b]',%
showstringspaces=false,
emph={class, pass, in, for, while, if, is, elif, else, not, and, or,
def, print, exec, break, continue, return, import, from, lambda, null},
emphstyle=\color{blue},
emph={[2]True, False, None, self},
emphstyle=[2]\color{key},
emph={[3]from, import, as},
emphstyle=[3]\color{blue},
upquote=true,
morecomment=[s]{"""}{"""},
comment=[l]{\#},
commentstyle=\color{gray},
%framexleftmargin=1mm, framextopmargin=1mm, 
%title=\tiny python,
%caption=python,
frame=single
%frameround=tttt,
%mathescape =true,
%escapechar=§
}

\newcommand{\pyinput}[1]{\lstset{language=fooPython}\lstinputlisting{#1}}
\newcommand{\isage}[1]{{\lstset{language=fooPython,basicstyle=\color[gray]{0.3}}\lstinline|#1|}}

\lstnewenvironment{pyout}[1][]{\lstset{language=fooPython,xleftmargin=0.2cm,frame=none,backgroundcolor=\color{white},basicstyle=\color{darkblue}\ttfamily\small,#1}}{}
\lstnewenvironment{pyin}[1][]{\lstset{language=fooPython,#1}}{}
\lstnewenvironment{sageout}[1][]{\lstset{language=fooPython,xleftmargin=0.2cm,frame=none,backgroundcolor=\color{white},basicstyle=\color{darkblue}\ttfamily\small,#1}}{}
\lstnewenvironment{sagein}[1][]{\lstset{language=fooPython,#1}}{}

%\usepackage{caption}
%\DeclareCaptionFont{white}{ \color{white} }
%\DeclareCaptionFormat{listing}{
%  \colorbox[cmyk]{0.43, 0.35, 0.35,0.01 }{
%      \parbox{\textwidth}{\hspace{15pt}#1#2#3}
%        }
%        }
%        \captionsetup[lstlisting]{ format=listing, labelfont=white, textfont=white, singlelinecheck=false, margin=0pt, font={bf,footnotesize} }


\usepackage{mydef}
%\usepackage{cmap} % you can search in the pdf for umlauts and ligatures
\usepackage{colonequals} %corrects the definition-symbols \colonequals (besides others)

\usepackage{ifthen}

%%%%%%%%%%%%%%%%%%%
%Neue Definitionen
%%%%%%%%%%%%%%%%%%%

%Newcommands
\newcommand{\Fun}[1]{\mathcal{#1}}      %Mathcal fuer Funktoren
\newcommand{\field}[1]{\mathbb{#1}}     %Grundkoerper ?? in mathds

\newcommand{\A}{\field{A}}              %Affines A
\newcommand{\Fp}{\field{F}_{\!p}}       %Endlicher Koerper mit p Elementen
\newcommand{\Fq}{\field{F}_{\!q}}       %Endlicher Koerper mit q Elementen
\newcommand{\Ga}{\field{G}_{a}}         %Add Gruppenschema
\newcommand{\K}{\field{K}}              %Generischer Koerper 
\newcommand{\N}{\field{N}}              %Nat Zahlen
\newcommand{\Pj}{\field{P}}             %Projektives P
\newcommand{\R}{\field{R}} 		%Reelle Zahlen
\newcommand{\Q}{\field{Q}}              %Rationale Zahlen  
\newcommand{\Qt}{\field{H}}             %Quaternionen 
\newcommand{\V}{\field{V}}              %Vektorbuendel V
\newcommand{\Z}{\field{Z}}              %Ganze Zahlen
\DeclareMathOperator{\Real}{Re}

\newcommand{\fdg}{\;|\;}                 %fuer die gilt

%Operatoren
\DeclareMathOperator{\Abb}{Abb}
%\usepackage{sagetex}


%
% Aufgaben
%
\parindent0cm % Abs�tze nicht einr�cken 
% Definieren einer neuen Farbe
\definecolor{light-gray}{gray}{.9}

\newcounter{zaehler}     % neuen Z�hler einf�hren
\newenvironment{aufgn}[2][0]
%---- Header
{\begin{samepage}%
%\colorbox{light-gray}{%                         % Box in gray
% \makebox[\textwidth]{%                           % Box in linewidth
%\textbf{Aufgabe \arabic{zaehler} } }\hspace{-\textwidth}\makebox[\textwidth]{\hfill #1 Punkte} }\\[0.05cm]       % Header
\dotfill\\
{\large\textbf{Aufgabe \arabic{zaehler} \ifthenelse{ \equal{#2}{} }{}{: \emph{ #2 } }}\ifthenelse{-1=#1}{(testierbar)}{}\ifthenelse{0=#1 \or -1=#1}{}{\hfill #1 Punkte} }\\[0.4cm]
%{\large\textbf{Exercise \arabic{zaehler}  #2 }\ifthenelse{-1=#1}{(testierbar)}{}\ifthenelse{0=#1 \or -1=#1}{}{\hfill #1 Punkte} }\\[0.4cm]
\begin{minipage}{\textwidth}%
}%
%-----  foot
{\end{minipage}\nopagebreak%\begin{minipage}{1cm} \end{minipage}
%\\ 
%\begin{minipage}{0.1cm} \end{minipage} 
%\hrulefill \begin{minipage}{1cm} \end{minipage}\\[1cm]  
\stepcounter{zaehler}                           % increase counter
\end{samepage}%
\\%
\bigskip%
}


\newenvironment{aufg}[1][0]
%---- Header
{\begin{samepage}%
\refstepcounter{zaehler}% increase counter
%\colorbox{light-gray}{%                         % Box in gray
% \makebox[\textwidth]{%                           % Box in linewidth
%\textbf{Aufgabe \arabic{zaehler} } }\hspace{-\textwidth}\makebox[\textwidth]{\hfill #1 Punkte} }\\[0.05cm]       % Header
\dotfill\\
{\large\textbf{Aufgabe \arabic{zaehler} }\ifthenelse{-1=#1}{(testierbar)}{}\ifthenelse{0=#1 \or -1=#1}{}{\hfill #1 Punkte} }\\[0.4cm]
\begin{minipage}{\textwidth}%
}%
%-----  foot
{\end{minipage}\nopagebreak%\begin{minipage}{1cm} \end{minipage}
%\\ 
%\begin{minipage}{0.1cm} \end{minipage} 
%\hrulefill \begin{minipage}{1cm} \end{minipage}\\[1cm]  
\end{samepage}%
\\%
\bigskip%
}

\begin{document}
%--------------------------------------------------- Header
\begin{center}
    \textbf{\LARGE Einf\"uhrung in Sage}\\
    {\large Zusammenfassung Einheit 01}
\end{center}
\textsl{Hinweis:} Textbausteine mit \isage{<name>} weisen darauf hin, das anstatt des Ausdrucks eine passende Variable eingefügt werden muss.

\medskip
\textbf{Kurvendiskussion}
\begin{itemize}
 \item Deklarieren von Variablen mit- \href{}{}
\begin{sagein}
var('<varname>')
\end{sagein}
\item Definieren von Variablen 
    \begin{sagein}
<varname>=<value>        
    \end{sagein}
\item Definieren von (mathematischen) Funktionen 
    \begin{sagein}
<functionname>(<arguments>) = <expr>
    \end{sagein}
%\item Symbolisches Rechnen 
%\begin{itemize}
\item Grenzwertbestimmung- \href{https://sage.math.uni-goettingen.de/doc/static/reference/sage/calculus/functional.html?highlight=function#sage.calculus.functional.limit}{limit()}
    \begin{sagein}
<expr>.limit(x=<a>, dir='<plus|minus>')
    \end{sagein}
\item Bilden von Ableitungen- \href{https://sage.math.uni-goettingen.de/doc/static/reference/sage/symbolic/expression.html?highlight=differentiate#sage.symbolic.expression.Expression.differentiate}{differentiate()}
\begin{sagein}
<expr>.differentiate(<variable>)
\end{sagein}

%\end{itemize}  
%\end{itemize}
%\begin{itemize}
\item Lösen von Gleichungen- \href{https://sage.math.uni-goettingen.de/doc/static/reference/sage/symbolic/relation.html?highlight=symbolic.relation#sage.symbolic.relation.solve}{solve()}:
\begin{sagein}
solve( f(x)==0, x)
\end{sagein}

\item Berechnen numerischer Approximationen- \href{http://docs.python.org/library/functions.html#float}{float()}
\begin{sagein}
float(<expr>)
\end{sagein}
\item Plotten einer Funktion- \href{https://sage.math.uni-goettingen.de/doc/static/reference/sage/plot/plot.html?highlight=.plot#d-plotting}{plotting}
\begin{sagein}
plot(<function>,(<lowerbound>,<upperbound>))
\end{sagein}
\end{itemize}

\textbf{Symbolisches Rechnen}
\begin{itemize}
\item Symbolisch Integrieren- \href{https://sage.math.uni-goettingen.de/doc/static/reference/sage/calculus/functional.html?highlight=diff#sage.calculus.functional.integral}{integrate()}
\begin{sagein}
integrate(<expr>,<variable>)
\end{sagein}
 \item Numerisch Integrieren 
\begin{sagein}
integrate(<expr>,<variable>,<lower>,<upper>)
\end{sagein}
\item Faktorisieren- \href{https://sage.math.uni-goettingen.de/doc/static/reference/sage/symbolic/expression.html?highlight=simplify_full#sage.symbolic.expression.Expression.factor}{factor()} 
\begin{sagein}
factor(<expr>)
\end{sagein}
\item Sortieren- \href{https://sage.math.uni-goettingen.de/doc/static/reference/sage/symbolic/expression.html?highlight=simplify_full#sage.symbolic.expression.Expression.collect}{collect()}
\begin{sagein}
<expr>.collect(x)
\end{sagein}
\item Partialbruchzerlegung- \href{https://sage.math.uni-goettingen.de/doc/static/reference/sage/symbolic/expression.html?highlight=simplify_full#sage.symbolic.expression.Expression.partial_fraction}{partial\_fraction()}
\begin{sagein}
<expr>.partial_fraction()
\end{sagein}
\item (vollständiges) Vereinfachen- \href{https://sage.math.uni-goettingen.de/doc/static/reference/sage/symbolic/expression.html?highlight=simplify_full#sage.symbolic.expression.Expression.simplify_full}{simplify\_full()}
\begin{sagein}
<expr>.full_simplify()
\end{sagein}
\item Vereinfachen mit radicals- \href{https://sage.math.uni-goettingen.de/doc/static/reference/sage/symbolic/expression.html?highlight=simplify_full#sage.symbolic.expression.Expression.simplify_radical}{simplify\_radical()}
\begin{sagein}
<expr>.radical_simplify()
\end{sagein}
\end{itemize}

\textbf{AGLA}
\begin{itemize}
\item Matrix eingeben- \href{https://sage.math.uni-goettingen.de/doc/static/reference/sage/matrix/constructor.html#sage.matrix.constructor.Matrix}{matrix()}
\begin{sagein}
matrix([ [<z1s1>,<z1s2>],[<z2s1>,<z2s2>] ])
\end{sagein}
 \item Vektor eingeben- \href{https://sage.math.uni-goettingen.de/doc/static/reference/sage/modules/free_module_element.html#sage.modules.free_module_element.vector}{vector()}
\begin{sagein}
vector([<a>,<b>,<c>])
\end{sagein}
\item LGS lösen 
\begin{sagein}
A\b
\end{sagein}
\item Matrixoperationen 
\begin{sagein}
A+B, A-B, A*B
\end{sagein}
\item Matrix invertieren- \href{https://sage.math.uni-goettingen.de/doc/static/reference/sage/matrix/matrix2.html?highlight=matrix.inverse#sage.matrix.matrix2.Matrix.inverse}{inverse()}
\begin{sagein}
A^(-1); A.inverse()
\end{sagein}
\item Substitutieren- \href{https://sage.math.uni-goettingen.de/doc/static/reference/sage/crypto/mq/mpolynomialsystem.html#sage.crypto.mq.mpolynomialsystem.MPolynomialRoundSystem_generic.subs}{subs()}
\begin{sagein}
<expr>.subs(<variable>=<subs>)
\end{sagein}
\end{itemize}

\textbf{Etwas Programmieren}
\begin{itemize}
    \item Listen (geordnet)- \href{http://docs.python.org/library/functions.html#list}{list()}
        \begin{sagein}
[a,b,c,..]
        \end{sagein}
    \item Tuple- \href{http://docs.python.org/library/functions.html#tuple}{tuple()}
        \begin{sagein}
(a,b,c,..)
        \end{sagein}
    \item (Nicht-mathematische) Funktionen- \href{http://docs.python.org/reference/compound_stmts.html#function-definitions}{def}
\begin{sagein}
def <function>(<argument>): <befehle> return <rueckgabewert>
\end{sagein}
    \item Einzeilige Schleifen- \href{http://docs.python.org/tutorial/controlflow.html#for-statements}{for}
\begin{sagein}
[<expr(var)> for <var> in <range|liste> if <expr>]
\end{sagein}
\end{itemize}

\textbf{Zahlentheorie}
\begin{itemize}
    \item Teiler- \href{https://sage.math.uni-goettingen.de/doc/static/reference/sage/rings/arith.html#sage.rings.arith.divisors}{divisors()}
\begin{sagein}
divisors(<number>)
\end{sagein}
\item Anzahl Teiler- \href{https://sage.math.uni-goettingen.de/doc/static/reference/sage/rings/arith.html#sage.rings.arith.number_of_divisors}{number\_of\_divisors()}
\begin{sagein}
number_of_divisors(<number>)
\end{sagein}
    \item Primzahl-Überprüfung- \href{https://sage.math.uni-goettingen.de/doc/static/reference/sage/rings/arith.html#sage.rings.arith.is_prime}{is\_prime()}
\begin{sagein} 
is_prime(<number>)
\end{sagein}
\end{itemize}



\end{document}

