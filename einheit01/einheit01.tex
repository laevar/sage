\documentclass[notes=hide,hyperref={dvipdfmx,pdfpagelabels=false}]{beamer}
\mode<article>
{
  \usepackage{fullpage}
  \usepackage{pgf}
  \usepackage[xetex]{hyperref}
  \setjobnamebeamerversion{beamer}
}

\mode<presentation>
{
  %\usetheme{Frankfurt}
 %\usetheme{My}
  \usetheme{Madrid}
  % or ...
%\usecolortheme{seagull}
  %\setbeamercovered{transparent}
  %\setbeamercovered{dynamic}
  % or whatever (possibly just delete it)
}
\usenavigationsymbolstemplate{}
\usefonttheme{structurebold}
\usepackage{multimedia}
%\usepackage{tikz}
\usepackage{fontspec,xunicode,xltxtra}

%\usepackage{polyglossia}
%\setdefaultlanguage[spelling=new, latesthyphen=true]{german}
%\setsansfont{DejaVu Sans}
%\setsansfont{Verdana}
%\setsansfont{Arial}
%\setromanfont{Linux Libertine O}
%\setsansfont{Linux Biolinum O}

\setbeamertemplate{footline}
{
\leavevmode
%\hbox{\begin{beamercolorbox}[wd=.5\paperwidth,ht=2.5ex,dp=1.125ex,
%leftskip=.3cm plus1fill,rightskip=.3cm]{author in head/foot}%
%    \usebeamerfont{author in head/foot}\insertshortauthor
%  \end{beamercolorbox}%
%  \begin{beamercolorbox}[wd=.5\paperwidth,ht=2.5ex,dp=1.125ex,leftskip=.3cm,
%rightskip=.3cm plus1fil]{title in head/foot}%
%    \usebeamerfont{title in head/foot}\insertshorttitle\hfill

\hfill\insertframenumber  \hspace{3pt}

%\inserttotalframenumber
%\hspace*{2ex}
%  \end{beamercolorbox}}%
  \vskip3pt%
}

\usepackage[ngerman]{babel}
\selectlanguage{ngerman}

%
% math/symbols
%
\usepackage{amssymb}
\usepackage{amsthm}
% \usepackage{latexsym}
\usepackage{amsmath}
%\usepackage{amsxtra} %Weitere Extrasymbole
%\usepackage{empheq} %Gleichungen hervorheben
%\usepackage{bm}
 %\bm{A} Boldface im Mathemodus

\usepackage{cellspace}
\setlength{\cellspacetoplimit}{2pt}
\setlength{\cellspacebottomlimit}{2pt}

%%%%%%%%%%%%%%%%%% Fuer Frames [fragile]-Option verwenden!
%Programm-Listing
%%%%%%%%%%%%%%%%%%
%Listingsumgebung fuer verbatim
%Grauhinterlegeter Text
%Automatischer Zeilenumbruch ist aktiviert
\usepackage{listings}
\definecolor{lgray}{gray}{0.80}
%\lstset{backgroundcolor=\color{lgray}, frame=single, basicstyle=\ttfamily, breaklines=true}
\lstnewenvironment{sage}{\lstset{backgroundcolor=\color{lgray},language=Python, emphstyle=\color{red}, frame=single, basicstyle=\ttfamily, breaklines=true,mathescape =true,escapechar=§}}{}


\usepackage{mydef}
\usepackage{cmap} % you can search in the pdf for umlauts and ligatures
\usepackage{colonequals} %corrects the definition-symbols \colonequals (besides others)
\title{Einführung in Sage}
%
%\subtitle{Disputation} % (optional)

\author{Jochen Schulz}
% - Use the \inst{?} command only if the authors have different
%   affiliation.

\institute{Georg-August Universit\"at G\"ottingen \pgfimage[height=0.5cm]{../figures/unilogo3}}
% - Use the \inst command only if there are several affiliations.
% - Keep it simple, no one is interested in your street address.

\date{\today}

\subject{Sage}
% This is only inserted into the PDF information catalog. Can be left
% out. 

% If you have a file called "university-logo-filename.xxx", where xxx
% is a graphic format that can be processed by latex or pdflatex,
% resp., then you can add a logo as follows:

%\logo{\pgfimage[height=0.5cm]{figures/unilogo3}}


% Delete this, if you do not want the table of contents to pop up at
% the beginning of each subsection:

\AtBeginSection[]
{
  \begin{frame}<beamer>
    \frametitle{Aufbau}
    \tableofcontents[currentsection,currentsubsection]
  \end{frame}
}

\AtBeginSubsection[]
{
  \begin{frame}<beamer>
    \frametitle{Aufbau}
    \tableofcontents[currentsection,currentsubsection]
  \end{frame}
}



%%%%%%%%%%%%%%%%%%%
%Neue Definitionen
%%%%%%%%%%%%%%%%%%%

%Newcommands
\newcommand{\Fun}[1]{\mathcal{#1}}      %Mathcal fuer Funktoren
\newcommand{\field}[1]{\mathbb{#1}}     %Grundkoerper ?? in mathds

\newcommand{\A}{\field{A}}              %Affines A
\newcommand{\C}{\field{C}}              %Complexes C
\newcommand{\Fp}{\field{F}_{\!p}}       %Endlicher Koerper mit p Elementen
\newcommand{\Fq}{\field{F}_{\!q}}       %Endlicher Koerper mit q Elementen
\newcommand{\Ga}{\field{G}_{a}}         %Add Gruppenschema
\newcommand{\K}{\field{K}}              %Generischer Koerper 
\newcommand{\N}{\field{N}}              %Nat Zahlen
\newcommand{\Pj}{\field{P}}             %Projektives P
\newcommand{\R}{\field{R}} 		%Reelle Zahlen
\newcommand{\Q}{\field{Q}}              %Rationale Zahlen  
\newcommand{\Qt}{\field{H}}             %Quaternionen 
\newcommand{\V}{\field{V}}              %Vektorbuendel V
\newcommand{\Z}{\field{Z}}              %Ganze Zahlen

\newcommand{\fdg}{\;|\;}                 %fuer die gilt

%Operatoren
\DeclareMathOperator{\Abb}{Abb}
%\usepackage{sagetex}

\begin{document}
\lstset{basicstyle={\lstbasicfont\footnotesize}}


\subtitle{Einheit 1}
\maketitle

%%%%%%%%%%%%%%%%%%%%%%%%%%%%%%%%%%%%%%
\section*{Organisatorisches}
%%%%%%%%%%%%%%%%%%%%%%%%%%%%%%%%%%%%%

\begin{frame}{Organisatorisches}
\begin{itemize}
\item Anmeldungen zu der Veranstaltung über StudIP \\
      \url{https://www.studip.uni-goettingen.de/}

{\color{blue}{Einführung in MuPAD (Mathematische Anwendersysteme) (WS 2008/09)}}
\item Alle Unterlagen können von der  StudIP-Seite (Reiter Dateien) heruntergeladen werden 
\pause
\item Dozent: Stefan Wiedmann \texttt{wiedmann@uni-math.gwdg.de}
\item Büro: Zi. 006 (Erdgescho{\ss}), MI
\item Telefon: 39-7754
\end{itemize}
\end{frame}

\begin{frame}{Accounts}
\begin{itemize}
\item MuPAD ist in Version 4.0 auf allen Rechnern der Mathematischen Fakultät installiert
\item Kiosk-System im Übungssaal und im Computerraum -- kein Login aber auch keine Speicherung von eigenen Dateien
\item Accounts im CIP-Pools der Mathematischen Fakultät:
\begin{itemize}
\item Mit \alert{Stud.it-Account}: Formular unter \url{https://xldap.uni-math.gwdg.de/register.php} ausfüllen
\item Ohne \alert{Stud.it-Account}: An Herrn Jochen Schulz wenden
\end{itemize}
\item Mit Account: Möglichkeit via {\color{blue}ssh} in der Shell mupad zu starten:

\texttt{ssh -X username@l1.num.math.uni-goettingen.de}

\texttt{ssh -X username@s1.math.uni-goettingen.de}
\item Alternative Rechner: \texttt{l2 -- l14}, bzw. \texttt{s2 -- s6}
\end{itemize}
\end{frame}

\begin{frame}[fragile]{Unterlagen}
\begin{itemize}
\item {Aufgabenblätter:} {\color{blue}StudIP}
\item {Folien zur Vorlesung}: {\color{blue}StudIP}
\item {Folien der Vorlesung zum Ausdrucken}: {\color{blue}StudIP}
\item {Notebooks zur Vorlesung}: {\color{blue}StudIP}
\item {MuPAD Buch:} Preis: 19.95 Euro,\\
{Rapin, Wassong, Wiedmann, Koospal:

 MuPAD - Eine Einführung, 

Springer 2007, ISBN 978-3-540-73475-8}
\end{itemize}
\end{frame}

\begin{frame}{Ablauf der Veranstaltung}
\begin{itemize}
\item Blockveranstaltung vom  16.2.-27.2.2008
\item Vorlesung: 9.15 Uhr bis 11.30 Uhr
\item Nachmittags: 4 Übungsgruppen \`a je 1h 15min
\begin{itemize}
\item 13:00-14:15 (Tutor: Schulz)	
\item 14:15-15:30 (Tutor: Wiedmann)
\item 15:30-16:45 (Tutor: Schulz)
\item 16:45-18:00 (Tutor: Wiedmann)
\end{itemize}

\item Scheinerwerb
\begin{itemize}
\item Regelmäßige Teilnahme an den Übungen
\item Vorführen von Aufgaben
\item Klausur am 27.2.2009; 9:15 -- 10:45; Anmeldung über FlexNow!
\end{itemize}
\end{itemize}
\end{frame}

\begin{frame}[<+->]{Inhalt der Vorlesung}
\begin{description}
 \item[1. Tag]Organisatorisches, Aufbau von MuPAD, Streifzug durch MuPAD
\item [2. Tag]Grundlagen MuPAD (grundlegende Datentypen, Ausdrücke), Lösen von Gleichungen, Symbolisches Rechnen
\item [3. Tag] Mengen,  natürliche, rationale, reelle und komplexe Zahlen, Gleitkommazahlen, Ungleichungen
\item [4. Tag] Vektoren und Matrizen, Lineare Algebra in MuPAD, Programmieren I
\item [5. Tag] Datencontainer in MuPAD, Lineare Abbildungen und Matrizen
\item [6. Tag] Folgen und Reihen
\item [7. Tag] Reelle Funktionen, Grafiken
\item [8. Tag] Differenzial- und Integralrechnung
\item [9. Tag] Grundlagen der Programmierung,  Zeichenketten (Strings)
\item [10. Tag] Klausur (9.15 - 10.45 Uhr)
\end{description}
\end{frame}

\begin{frame}[<+->]{Gliederung des heutigen Tages}
\begin{itemize}
\item Organisatorisches
\item Was ist MuPAD?
\item Ein Streifzug durch MuPAD
\begin{itemize}
\item Eine Kurvendiskussion
\item Symbolisches Rechnen
\item Etwas AGLA
\item Etwas Zahlentheorie
\item Nützliches und Hilfe
\end{itemize}
\end{itemize}
\end{frame}



%%%%%%%%%%%%%%%%%%%%%%%%%%%%%%
\section{Was ist Sage?}
%%%%%%%%%%%%%%%%%%%%%%%%%%%%%%%%%

\begin{frame}{Sage}
\begin{itemize}
\item MuPAD $\equiv$ \underline{Mu}lti \underline{P}rocessing \underline{A}lgebra \underline{D}ata tool
\item MuPAD ist ein Computeralgebra-System
\begin{itemize}
 \item Entwicklung von MuPAD seit 1990 an der Universität Paderborn
 \item Seit 1997 Teilausgliederung in die SciFace Sofware GmbH
 \item MuPAD wurde mitte des Jahres 2008 an Mathworks verkauft
 \item MuPAD ist seither eine Toolbox des Programms \alert{Matlab} (Symbolic Toolbox)
\end{itemize}
\item MuPAD ist in C/C++ geschrieben.
\item MuPAD ist objektorientiert.
\end{itemize}
\end{frame}

\begin{frame}{Computeralgebra-Systeme}

\begin{block}{Computeralgebra}
beschäftigt sich mit \alert{exakten} Berechnungen von
mathematischen Objekten
\end{block}
\bigskip

\begin{block}{Mathematische Objekte} 
Natürliche Zahlen, reelle Zahlen, Polynome, Funktionen,
Gruppen, Ringe, \ldots
\end{block}

\begin{block}{Numerischen Berechnungen}
Bei numerischen
Rechnungen (z.B. Taschenrechner) benutzt man Zahlen in 
{\color{blue}Gleitpunktdarstellung}, also i.A.~nur Näherungen an die
gesuchte Lösung
\end{block}
\end{frame}

\begin{frame}{Computeralgebra <-> Numerische Berechnung}
\begin{block}{Beispiel}
\begin{tabular}{ll}
 Mathematische Objekte & $\pi$, $\sqrt{2}$\\
 Gleitpunktdarstellung (8 Stellen)& $3.1415927$, $1.4142136$
\end{tabular}
\end{block}
\end{frame}

\begin{frame}{Andere Computeralgebra-Systeme}
\begin{small}
\begin{tabular}{cp{7cm}}
\alert{General purpose}: & Derive (Eingestellt 2006, TI)\\
& LiveMath (Maple)\\
& Maxima (Free, GPL)\\
& Reduce (sehr alt, in Lisp programmiert, free)\\
& Mathematica (Platzhirsch)\\
& Maple (Platzhirsch)\\
& Matlab/Octave (Für große Rechnungen)\\
& Magma (Spezielle mathematische Rechungen)\\
\alert{Special  purpose}: & Cadabra (Körpertheorie)\\
& PARI/GP (Zahlentheorie)\\
& GAP (Gruppentheorie) \\
& Macaulay (Algebraische Geometrie)\\
& Singular (Algebraische Geometrie)
\end{tabular}
\end{small}
\end{frame}

\begin{frame}{Neuere Entwicklungen}
Neue Entwicklungen/Bibliotheken:
\begin{description}
\item[Sage] Sehr ehrgeiziges Projekt, komplett in Phython geschrieben
\item[SymPy] Phython-Bibliotheken als CAS-Verwendbar
\item[SymbolicC++] Bibliotheken zur CA in C++
\end{description}
Überblick:

 \url{http://en.wikipedia.org/wiki/Comparison_of_computer_algebra_systems}
\end{frame}


\begin{frame}[<+->]{MuPAD - Stärken}

\begin{itemize}
\item Objektorientiertes Konzept (definieren eigener Datentypen,
  überladen von Operatoren möglich)
\item interaktiver Quellcode-Debugger
\item umfangreiches Hilfesystem
\item Einfaches Einbinden von C/C++ Routinen (dynamische Module)
\item Teil einer großen Software (Weiterenwicklung gesichert)
\item Viele freie (Unterrichts-)materialien im Netz 
\end{itemize}
\end{frame}

\begin{frame}[<+->]{MuPAD - Schwächen}
\begin{itemize}
\item Befehlsumfang nicht so mächtig wie bei Maple oder Mathematica
\item Geringerer Verbreitungsgrad 
\item Benutzung ist nicht so intuitiv wie bei anderen Systemen
\item Teil einer großen Software (Matlab muß installiert/erworben werden)
\item Programmierumgebung wenig komfortabel (Es gibt in der Matlab-Version einen Editor)
\item Gültigkeitsberereich von Variablen in Funktionen ist global
\item Zum Teil nicht konsistent erscheinende Auswertungen von Variablen
\end{itemize}
\end{frame}

\begin{frame}{Struktur von MuPAD}
\begin{center}
\includegraphics[width=6cm]{figures/components.png}
\end{center}
\end{frame}

\begin{frame}{Kern von MuPAD}
\begin{itemize}
\item \alert{Parser:} Liest die Eingaben und überprüft die Syntax;
  Umwandlung in MuPAD-Datentyp
\item \alert{Auswerter:} Auswertung und Vereinfachung der
  Ergebnisse
\item \alert{Speicherverwaltung:} (MAMMUT $\equiv$ Memory
  Allocation Managment Unit) interne Verwaltung der
  MuPAD-Objekte
\item \alert{Kernfunktionen:}  Oft benötigte Funktionen
  werden aus Effizienzgründen im Kern auf C-Ebene implementiert.
\end{itemize}
\end{frame}

\begin{frame}{Literatur}
\begin{itemize}
\item K. Gehrs, F. Postel. MuPAD -- Eine praktische
  Einführung. SciFace. 2001.
\item Ch. Creutzig, J. Gerhard, W. Oevel, St. Wehmeier. Das MuPAD
  Tutorium. Springer. 2. Auflage. 2002.
\item M. Majewski. MuPAD Pro Computing Essentials. Springer. 2002.
\item Rolf Monnerjahn. Mathematische Anwendungen in Biologie, Chemie, Physik. MuPad im Mathematikunterricht: 5.-10. Schuljahr
\item  Gerd Rapin, Thomas Wassong, Stefan Wiedmann und Stefan Koospal. MuPAD: Eine Einführung
\end{itemize}
\end{frame}

%%%%%%%%%%%%%%%%%%%%%%%%%%%%%%%%%%
\section{Streifzug durch Sage}
%%%%%%%%%%%%%%%%%%%%%%%%%%%%%%%%%

\begin{frame}[fragile]{Sage als Taschenrechner}
Hier einige Beispiele:
\begin{sage}
>> 3+4*10+12 
\end{sage}
\begin{sage}
  55 
\end{sage}
\begin{sage}
>> sin(Pi) 
\end{sage}
\begin{sage}
  0
\end{sage}
\begin{sage}
>> float(Pi)
\end{sage}
\begin{sage}
  3.141592654 
\end{sage}
\begin{sage}
>> float(sqrt(2))
\end{sage}
\begin{sage}
  1.414213562 
\end{sage}
\end{frame}



%%%%%%%%%%%%%%%%%%%%%%%%%%%%%%
\section{Eine Kurvendiskussion}
%%%%%%%%%%%%%%%%%%%%%%%%%%%%%%%%%

\begin{frame}[fragile]{Kurvendiskussion I}
Betrachte die durch die reelle Zahl $a$ parametrisierte Funktionenschar:
\[ 
f: x \quad \mapsto \quad \frac{2x^2-20x+42}{x-1}+a, \quad
a \in \mathbb{R} 
\]

\begin{itemize}
\item Eingabe der Funktion
\begin{sage}
>> var('a')
>> f(x) = (2*x^2-20*x +42)/(x-1)+a
\end{sage}
\begin{sage}
  x |--> a + 2*(x^2 - 10*x + 21)/(x - 1)
\end{sage}
\end{itemize}
\end{frame}

\begin{frame}[fragile]{Kurvendiskussion II}
\begin{itemize}
\item Pol ?
\begin{sage}
>> f.limit(x=1, dir='minus')
\end{sage}
\begin{sage}
  x |--> -Infinity
\end{sage}
\begin{sage}
>> f.limit(x=1, dir='plus') 
\end{sage}
\begin{sage}
  x |--> +Infinity
\end{sage}
\item Umformen
\begin{sage}
>> f.full_simplify()
\end{sage}
\begin{sage}
x |--> ((a - 20)*x + 2*x^2 - a + 42)/(x - 1)
\end{sage}
\end{itemize}
\end{frame}

\begin{frame}[fragile]{Kurvendiskussion III}
\begin{itemize}
\item Nullstellen
\begin{sage}
>> solve(f==0,x)
\end{sage}
\begin{scriptsize}
\begin{sage}
[x == -1/4*a - 1/4*sqrt(a^2 - 32*a + 64) + 5, x == -1/4*a + 1/4*sqrt(a^2 - 32*a + 64) + 5]
\end{sage}
\end{scriptsize}
\item Berechnen der Ableitung
\begin{sage}
>> f.differentiate(x)
\end{sage}
\begin{sage} 
x |--> 4*(x - 5)/(x - 1) - 2*(x^2 - 10*x + 21)/(x - 1)^2
\end{sage}
\end{itemize}
\end{frame}

\begin{frame}[fragile]{Kurvendiskussion IV}
\begin{itemize}
\item Extremwerte
\begin{sage}
>> solve(f.differentiate(x)==0,x)
\end{sage}
\begin{sage}
[x == -2*sqrt(3) + 1, x == 2*sqrt(3) + 1]
\end{sage}
\item Lokale Minima und Maxima
\begin{sage}
>> float( ((f.diff(x)).diff(x))(-2*sqrt(3)+1) )
\end{sage}
\begin{sage}
-1.1547005383792501
\end{sage}
\begin{sage}
>> float( ((f.diff(x)).diff(x))(2*sqrt(3)+1) )
\end{sage}
\begin{sage}
1.1547005383792515
\end{sage}
\end{itemize}
\end{frame}

\begin{frame}[fragile]{Kurvendiskussion V}
\begin{itemize}
\item Verhalten von $f$ für große $x$
\begin{sage}
>> f.limit(x=oo); f.limit(x=-oo)
\end{sage}
\begin{sage}
x |--> +Infinity
x |--> -Infinity
\end{sage}
\item Definiere $f_{0}$, $f_{1}$, $f_{2}$
\begin{sage}
>>  f0 = f(x, a=0)
>>  f1 = f(x, a=-20)
>>  f2 = f(x, a=20);f0,f1,f2
\end{sage}
\begin{scriptsize}
\begin{sage}
(2*(x^2 - 10*x + 21)/(x - 1), 2*(x^2 - 10*x + 21)/(x - 1) - 20, 2*(x^2 - 10*x + 21)/(x - 1) + 20)
\end{sage}
\end{scriptsize}
\end{itemize}
\end{frame}

\begin{frame}[fragile]{Plot}
\begin{sage}
>> p = plot(f0,(x, 0, 10),ymin=-80,ymax=80,detect_poles='True',color='red')
p += plot(f1,(x, 0, 10),detect_poles='True',color='green')
p += plot(f2,(x, 0, 10),detect_poles='True',color='blue')
p.show()
\end{sage}
\begin{center}
\includegraphics[height=6cm]{figures/graphen.png}
\end{center}
\end{frame}

\begin{frame}[fragile]{Zusammenfassung I}
\begin{itemize}
\item Definieren von Variablen mit {\color{blue}':='}, z.B. {\verb~a:=3~} 
\item Löschen von Objekten mit {\color{blue}{\verb+delete+}}, z.B. {\verb+delete a+}
\item Definieren von Funktionen mit{\color{blue} '->'}, z.B. {\verb~f:= x -> x^2- 6*x~}
\item Symbolisches Rechnen 
\begin{itemize}
\item Unstetigkeitsstellen: {\color{blue} \verb~discont(f(x),x)~}
\item Grenzwertbestimmung: {\color{blue}   \verb~limit(f(x), x= 2, Left)~}
\item Vereinfachen: {\color{blue}  \verb~normal(f(x))~}
\item Bilden von Ableitungen {\color{blue} \verb~f'(x)~}
\end{itemize} 
\end{itemize}
\end{frame}

\begin{frame}[fragile]{Zusammenfassung II}
\begin{itemize}
\item Lösen von Gleichungen: {\color{blue} \verb~solve( f(x)=0, x)~}
\item Berechnen numerischer Approximationen:
  {\color{blue} \verb~float(f(sqrt(3)+ 4))~}
\item Plotten einer Funktion: {\color{blue} \verb~plotfunc2d(sin(x),x=0..4)~}
\end{itemize}
\end{frame}

%%%%%%%%%%%%%%%%%%%%%%%%%%%%%%
\section{Symbolisches Rechnen}
%%%%%%%%%%%%%%%%%%%%%%%%%%%%%%%%%

\begin{frame}[fragile]{Symbolisches Rechnen I}
\begin{itemize}
\item Integrieren von $\int_0^\infty x^4 e^{-x} dx$
\begin{sage}
>> int(x^(4)*exp(-x),x=0..infinity)
\end{sage}
\begin{sage}
  24
\end{sage}
\item Stammfunktion von $\frac{1+\sin (x)}{1+\cos(x)}$
\begin{sage}
>> f := x -> (1+sin(x))/(1+cos(x)): 
>> int(f(x),x)
\end{sage}
\begin{scriptsize}
\begin{sage}
    ln(2 cos(x) + 2) - sin(x) + cos(x) ln(2 cos(x) + 2)
  - ---------------------------------------------------
                         cos(x) + 1
\end{sage}
\end{scriptsize}
\end{itemize}
\end{frame}

\begin{frame}[fragile]{Symbolisches Rechnen II}
\begin{itemize}
\item Faktorisieren und Ausmultiplizieren 
\begin{sage}
>> expand((x-1)*(x-2)*(x-3)*(x-4))
\end{sage}
\begin{sage}
   4       3       2
  x  - 10 x  + 35 x  - 50 x + 24
\end{sage}
\begin{sage}
>> factor(%)
\end{sage}
\begin{sage}
  (x - 1) (x - 2) (x - 3) (x - 4)
\end{sage}
\item Sortieren eines Ausdrucks bezüglich einer Unbekannten
\begin{sage}
collect(x^2+2*x+b*x^2+sin(x)+a*x,x)
\end{sage}
\begin{sage}
            2
  (b + 1) x  + (a + 2) x + sin(x)
\end{sage}
\end{itemize}
\end{frame}

\begin{frame}[fragile]{Symbolisches Rechnen III}
\begin{itemize}
\item Partialbruchzerlegung
\begin{sage}
>> partfrac( x^ 2/( x^ 2- 1))
\end{sage}
\begin{sage}
      1           1
  --------- - --------- + 1
  2 (x - 1)   2 (x + 1)
\end{sage}
\item Vereinfachen von Ausdrücken ($\frac{e^x -1}{e^{(1/2)x}+1}$)
\begin{sage}
>> simplify((exp(x)-1)/(exp(x/2)+1))
\end{sage}
\begin{sage}
     / x \
  exp| - | - 1
     \ 2 /
\end{sage}
\end{itemize}
\end{frame}

\begin{frame}[fragile]{MuPAD unterscheidet 
strikt zwischen Funktionen und Ausdrücken I}

Beispiele:

\begin{sage}
>> f:= x -> sin(x)
\end{sage}
\begin{sage}
 x -> sin(x)
\end{sage}
\begin{sage}
>> g:=sin(x)
\end{sage}
\begin{sage}
  sin(x)
\end{sage}
\begin{sage}
>> f(1),g(1)
\end{sage}
\begin{sage}
 sin(1), sin(x)(1)
\end{sage}
\end{frame}

\begin{frame}[fragile]{MuPAD unterscheidet 
strikt zwischen Funktionen und Ausdrücken II}
\begin{sage}
>> int(f,x)
\end{sage}
\begin{sage}
  Error: Illegal integrand [int]
\end{sage}
\begin{sage}
>> int(f(x),x)
\end{sage}
\begin{sage}
  -cos(x)
\end{sage}
\begin{sage}
>> f(x)-g
\end{sage}
\begin{sage}
  0
\end{sage}
\begin{sage}
>> h:=fp::unapply(g)
\end{sage}
\begin{sage}
  x -> sin(x)
\end{sage}
\end{frame}


%%%%%%%%%%%%%%%%%%%%%%%%%%%%%%
\section{Etwas AGLA}
%%%%%%%%%%%%%%%%%%%%%%%%%%%%%%%%%


\begin{frame}{Analytische Geometrie und Lineare Algebra}

Berechnen des Schnittpunkts der Ebene 
\[ E: \vec{x}= 
\left ( \begin{array}{c}  2 \\ 1 \\ -1 \end{array} \right) +l 
\left ( \begin{array}{c}  1 \\ -1 \\ -1 \end{array} \right) +m
\left ( \begin{array}{c}  -3 \\ 1 \\ 4 \end{array} \right), \quad l,m
\in \mathbb{R}
\]
mit der Geraden 
\[
g: \vec{x}=
\left ( \begin{array}{c}  3 \\ 0 \\ 1 \end{array} \right) +k
\left ( \begin{array}{c}  4 \\ -1 \\ 2 \end{array} \right), \quad k \in \mathbb{R}
\]
\end{frame}

\begin{frame}[fragile]{Grafische Darstellung}
\begin{sage}
>> E1 := 2+l-3*m: E2:=1-l+m: E3:=-1-l+4*m:
\end{sage}
\begin{sage}
>> Ebene1 := plot::Surface([E1,E2,E3],
 l=-2..2,m=-2..2, Mesh=[20,20]):
\end{sage}
\begin{sage}
>> g1 := 3+4*k: g2 := -k: g3 := 1+2*k:
\end{sage}
\begin{sage}
>> Gerade1 := plot::Curve3d([g1, g2,  
 g3], k=-3..3):
\end{sage}
\begin{sage}
>> plot(Ebene1,Gerade1)
\end{sage}
\end{frame}

\begin{frame}{Grafische Darstellung}
\begin{center}
\includegraphics[width=10cm, height=6cm]{figures/ebene2.png}
\end{center}
\end{frame}

\begin{frame}{Analytische Lösung}
Gleichsetzen ergibt: 
\[ 
\left ( \begin{array}{c}  2 \\ 1 \\ -1 \end{array} \right) +l 
\left ( \begin{array}{c}  1 \\ -1 \\ -1 \end{array} \right) +m
\left ( \begin{array}{c}  -3 \\ 1 \\ 4 \end{array} \right) = \left ( \begin{array}{c}  3 \\ 0 \\ 1 \end{array} \right) +k
\left ( \begin{array}{c}  4 \\ -1 \\ 2 \end{array} \right)
\] oder {
\[ 
\underbrace{\left(   
\begin{array} {ccc} 
1 & -3 & -4\\
-1 & 1 & 1 \\
-1 & 4 & -2  
\end{array} \right)}_{\displaystyle =:A} 
\underbrace{\left ( \begin{array}{c}  l \\ m \\ k \end{array}
  \right)}_{\displaystyle =:L} = \underbrace{\left ( \begin{array}{c}  1 \\ -1 \\ 2
  \end{array} \right)}_{\displaystyle =:b}
\] }
oder $A L=b$.
\end{frame}

\begin{frame}[fragile]{Definieren und Lösen des LGS}
\begin{itemize}
\item Definieren der Matrix $A$
\begin{sage}
>> A:= matrix([[ 1, -3, -4], 
[- 1,1,1], [- 1,4,- 2]])
\end{sage}
\begin{sage}
  +-            -+
  |   1, -3, -4  |
  |              |
  |  -1,  1,  1  |
  |              |
  |  -1,  4, -2  |
  +-            -+
\end{sage}
\item Definieren des Vektors $b$
\begin{sage}
>> b:=matrix([1,-1,2]):
\end{sage}
\end{itemize}
\end{frame}

\begin{frame}[fragile]
\begin{itemize}
\item Lösen von  $A \ L=b$
\begin{small}
\begin{sage}
>> L:=linalg::matlinsolve(A,b)
\end{sage}
\end{small}
\begin{scriptsize}
\begin{sage}
  +-      -+
  |   6/5  |
  |        |
  |   3/5  |
  |        |
  |  -2/5  |
  +-      -+
\end{sage}
\end{scriptsize}
\item Einsetzen in die Geradengleichung
\begin{sage}
>> k:=L[3]: x_s:=matrix([g1,g2,g3])
\end{sage}
\begin{scriptsize}
\begin{sage}
  +-     -+
  |  7/5  |
  |       |
  |  2/5  |
  |       |
  |  1/5  |
  +-     -+
\end{sage}
\end{scriptsize}
\end{itemize}
\end{frame}

\begin{frame}[fragile]
\begin{itemize}
\item Matrizenoperationen
\begin{sage}
>> B:=matrix([[1,0,0],[0,1,1],[1,1,1]]):
>> A*B, A-B, A+B
\end{sage}
\begin{scriptsize}
\begin{sage}
  +-            -+  +-            -+  +-            -+
  |  -3, -7, -7  |  |   0, -3, -4  |  |   2, -3, -4  |
  |              |  |              |  |              |
  |   0,  2,  2  |, |  -1,  0,  0  |, |  -1,  2,  2  |
  |              |  |              |  |              |
  |  -3,  2,  2  |  |  -2,  3, -3  |  |   0,  5, -1  |
  +-            -+  +-            -+  +-            -+
\end{sage}
\end{scriptsize}
\item Berechnen der Inversen (mit Probe)
\begin{sage}
>> A^(-1), A*A^(-1)
\end{sage}
\begin{scriptsize}
\begin{sage}
  +-                     -+  +-         -+
  |  -2/5, -22/15,  1/15  |  |  1, 0, 0  |
  |                       |  |           |
  |  -1/5,  -2/5,   1/5   |, |  0, 1, 0  |
  |                       |  |           |
  |  -1/5,  -1/15, -2/15  |  |  0, 0, 1  |
  +-                     -+  +-         -+
\end{sage}
\end{scriptsize}
\end{itemize}
\end{frame}

%%%%%%%%%%%%%%%%%%%%%%%%%%%%%%
\section{Etwas Zahlentheorie}
%%%%%%%%%%%%%%%%%%%%%%%%%%%%%%%%%

\begin{frame}[fragile]{Etwas Zahlentheorie I}
Fermatsche Primzahlen: $F_n=2^{2^n} +1$. Finden Sie die kleinste
Zahl $F_n$, die keine Primzahl ist!
\begin{sage}
>> F:=2^(2^n)+1:
>> n:=1: F, isprime(F)
  5, TRUE
>> n:=2: F, isprime(F)
  17, TRUE
>> n:=3: F, isprime(F)
  257, TRUE
>> n:=4: F, isprime(F)
  65537, TRUE
>> n:=5: F, isprime(F)
  4294967297, FALSE
>> numlib::divisors(F)
  [1, 641, 6700417, 4294967297]
\end{sage}
\end{frame}

\begin{frame}[fragile]{Etwas Zahlentheorie II}
\begin{itemize}
\item Eine Liste der ersten  Primzahlen bis $100$
\begin{sage}
>> Menge:= [ i $\text{\dollar}$ i=1..100 ]:
>> select(Menge,isprime)
\end{sage}
\begin{sage}
  [2, 3, 5, 7, 11, 13, 17, 19, 23, 29,
   31, 37, 41, 43, 47, 53, 59, 61, 67,
   71, 73, 79, 83, 89, 97]
\end{sage}
\item Mersenne-Primzahlen $2^p-1$, $p$ Primzahl. Bestimmen der ersten
Mersenne Primzahlen im Bereich $\leq 200$.
\begin{sage}
>> Primes := select([i $\text{\dollar}$ i=1..200],isprime):
>> select(Primes,p->isprime(2^p-1))
\end{sage}
\begin{sage}
  [2,3,5,7,13,17,19,31,61,89,107,127]
\end{sage}
\begin{sage}
>> numlib::mersenne()
\end{sage} 
\end{itemize}
\end{frame}

\begin{frame}[fragile]{Etwas Zahlentheorie III}
Wir geben für die natürlichen Zahlen $\leq 1000$ an, wieviele Zahlen
$1,2,3,\dots $ Teiler haben.
\begin{sage}
>> Liste:=[i $\text{\dollar}$ i=1..1000]:
>> anz_teiler:= n -> nops(numlib::divisors(n)):
>> Liste1:=map(Liste,anz_teiler):
>> for i from 1 to 50 do
     print(i,nops(select(Liste1, x -> (x = i))))
   end_for:
\end{sage}

\begin{sage}
  1, 0
  2, 168
   ...
\end{sage}
Teiler der Zahl $840$:
\begin{sage}
>> numlib::divisors(840)
\end{sage}
\end{frame}

%%%%%%%%%%%%%%%%%%%%%%
\section{Nützliches und Hilfe}
%%%%%%%%%%%%%%%%%%%%%%%%%%%%%%%

\begin{frame}[fragile]{Überlebensregeln}
\begin{itemize}
\item Kommas zwischen Eingaben erzeugen den Datentyp einer Folge!
\item Mehrere Befehle in einer Zeile besser durch
{\color{blue} ';'} oder {\color{blue} ':'} 
  trennen. Ausgabe wird mit {\color{blue} ':'} am Ende unterdrückt
\item Die Eingabe {\color{blue} \%} ergibt die Ausgabe des letzten Befehls
\item Die Eingabe {\color{blue} \%n} ergibt die Ausgabe des $n$-letzten Befehls
\item Bei Eingaben, die über mehrere Zeilen gehen, kann ein
  Zeilenumbruch durch {\color{blue} <SHIFT>+<ENTER>} erreicht werden
\end{itemize}
\end{frame} 

\begin{frame}{Nützliches}
\begin{itemize}
\item Löschen aller eigenen Variablen und Zurücksetzen auf den Anfangsstatus: {\color{blue} reset()}
\item Anzeigen aller definierten Variablen: {\color{blue} anames(All)}
\item Anzeigen aller selbst definierten Variablen: {\color{blue} anames(All,User)}
\item Alle Ausgaben entfernen: Bearbeiten -> Alle Ausgaben entfernen
\item Matheklammer erzeugen: $\pi$
\item Lücke für Text erzeugen: \P
\end{itemize}
\end{frame}

\begin{frame}[fragile]{Starten von MuPAD}
\begin{itemize}
\item Kiosk: Lernprogramme -> Mathematisches -> MuPAD 4.0.0
\item Benutzung der Terminals:
\begin{itemize}
\item Einloggen auf einem Linuxrechner
(l1,\ldots,l14) mit

{\color{blue} \verb+ ssh -X l1.num.math.uni-goettingen.de+}
\item Starten von MuPAD:
{\color{blue} \verb+mupad &+}
\end{itemize}
\item Hilfefunktionen in MuPAD
\begin{itemize}
\item {\color{blue} ?} (startet menügesteuertes Hilfefenster)
\item {\color{blue} info({\it name})} gibt  eine Kurzhilfe zu {\it name}
\item {\color{blue} ? {\it name}} oder help(``{\it name}'') gibt ausführliche Hilfe.
\end{itemize}
\end{itemize}
\end{frame}
\end{document}





















