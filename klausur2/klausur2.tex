\documentclass[a4paper,12pt,DIV15]{scrartcl}
\usepackage[psamsfonts]{amssymb}
\usepackage{amsmath}
\usepackage[svgnames]{xcolor} %color definitions

\usepackage{fontspec,xunicode,xltxtra}
%\usepackage{fontspec,xunicode}
%\usepackage{polyglossia}
%\setdefaultlanguage[spelling=new, latesthyphen=true]{german}
%\setsansfont{DejaVu Sans}
%\setsansfont{Verdana}
%\setsansfont{Arial}
%\setromanfont[Mapping=tex-text]{Linux Libertine}
%\setsansfont[Mapping=tex-text]{Myriad Pro}
%\setmonofont[Mapping=tex-text]{Courier New}

%\setsansfont{Linux Biolinum}

\usepackage[ngerman]{babel}
\selectlanguage{ngerman}

%
% math/symbols
%
\usepackage{amssymb}
\usepackage{amsthm}
% \usepackage{latexsym}
\usepackage{amsmath}
%\usepackage{amsxtra} %Weitere Extrasymbole
%\usepackage{empheq} %Gleichungen hervorheben
%\usepackage{bm}
 %\bm{A} Boldface im Mathemodus

\usepackage{multimedia}
%\usepackage{tikz}

\usepackage{cellspace}
\setlength{\cellspacetoplimit}{2pt}
\setlength{\cellspacebottomlimit}{2pt}

%%%%%%%%%%%%%%%%%% Fuer Frames [fragile]-Option verwenden!
%Programm-Listing
%%%%%%%%%%%%%%%%%%
%Listingsumgebung fuer verbatim
%Grauhinterlegeter Text
%Automatischer Zeilenumbruch ist aktiviert
\usepackage{listings}
% This command allows you to typeset syntax highlighted Matlab
% code ``inline''.
\newcommand{\isage}[1]{\lstinline|#1|}

\definecolor{lgray}{gray}{0.80}
\definecolor{gray}{gray}{0.3}
\definecolor{darkgreen}{rgb}{0,0.4,0}
\definecolor{darkblue}{rgb}{0,0,0.8}
\definecolor{key}{rgb}{0,0.5,0} 
%\lstset{backgroundcolor=\color{lgray}, frame=single, basicstyle=\ttfamily, breaklines=true}
\lstnewenvironment{sage}[1][]{\lstset{xleftmargin=0.2cm,frame=none,backgroundcolor=\color{white},basicstyle=\color{darkblue}\ttfamily\small,#1}}{} 
\lstnewenvironment{sagein}[1][]{\lstset{#1}}{} 
%\lstnewenvironment{sage}{\lstset{,language=python, keywordstyle=color{blue},    commentstyle=color{green}, emphstyle=\color{red}, %frame=single, stringstyle=\color{red}, basicstyle=\ttfamily, ,mathescape =true,escapechar=§}}{}

\lstset{
language=python,
backgroundcolor=\color{lgray},
breaklines=true,
basicstyle=\ttfamily\small,
%otherkeywords={ =},
keywordstyle=\color{blue},
stringstyle=\color{darkgreen},
showstringspaces=false,
emph={class, pass, in, for, while, if, is, elif, else, not, and, or,
def, print, exec, break, continue, return},
emphstyle=\color{blue},
emph={[2]True, False, None, self},
emphstyle=[2]\color{key},
emph={[3]from, import, as},
emphstyle=[3]\color{blue},
upquote=true,
morecomment=[s]{"""}{"""},
commentstyle=\color{gray}\slshape,
%framexleftmargin=1mm, framextopmargin=1mm, 
frame=single,
mathescape =true,
escapechar=§
}


\usepackage{mydef}
%\usepackage{cmap} % you can search in the pdf for umlauts and ligatures
\usepackage{colonequals} %corrects the definition-symbols \colonequals (besides others)
\usepackage{ifthen}
%%%%%%%%%%%%%%%%%%%
%Neue Definitionen
%%%%%%%%%%%%%%%%%%%

%Newcommands
\newcommand{\Fun}[1]{\mathcal{#1}}      %Mathcal fuer Funktoren
\newcommand{\field}[1]{\mathbb{#1}}     %Grundkoerper ?? in mathds

\newcommand{\A}{\field{A}}              %Affines A
\newcommand{\C}{\field{C}}              %Complexes C
\newcommand{\Fp}{\field{F}_{\!p}}       %Endlicher Koerper mit p Elementen
\newcommand{\Fq}{\field{F}_{\!q}}       %Endlicher Koerper mit q Elementen
\newcommand{\Ga}{\field{G}_{a}}         %Add Gruppenschema
\newcommand{\K}{\field{K}}              %Generischer Koerper 
\newcommand{\N}{\field{N}}              %Nat Zahlen
\newcommand{\Pj}{\field{P}}             %Projektives P
\newcommand{\R}{\field{R}} 		%Reelle Zahlen
\newcommand{\Q}{\field{Q}}              %Rationale Zahlen  
\newcommand{\Qt}{\field{H}}             %Quaternionen 
\newcommand{\V}{\field{V}}              %Vektorbuendel V
\newcommand{\Z}{\field{Z}}              %Ganze Zahlen
\DeclareMathOperator{\Real}{Re}

\newcommand{\fdg}{\;|\;}                 %fuer die gilt

%Operatoren
\DeclareMathOperator{\Abb}{Abb}
%\usepackage{sagetex}

%
% Aufgaben
%
\parindent0cm % Abs�tze nicht einr�cken 
% Definieren einer neuen Farbe
\definecolor{light-gray}{gray}{.9}

\newcounter{zaehler}     % neuen Z�hler einf�hren
\stepcounter{zaehler}    % Z�hler einen hochz�hlen

\newenvironment{aufg}[1][0]
%---- Header
{\begin{samepage}%
%\colorbox{light-gray}{%                         % Box in gray
% \makebox[\textwidth]{%                           % Box in linewidth
%\textbf{Aufgabe \arabic{zaehler} } }\hspace{-\textwidth}\makebox[\textwidth]{\hfill #1 Punkte} }\\[0.05cm]       % Header
\dotfill\\
{\large\textbf{Aufgabe \arabic{zaehler} }\ifthenelse{0=#1}{}{\hfill #1 Punkte} }\\[0.4cm]
\begin{minipage}{\textwidth}
}
%-----  foot
{\end{minipage} \nopagebreak %\begin{minipage}{1cm} \end{minipage}
%\\ 
%\begin{minipage}{0.1cm} \end{minipage} 
%\hrulefill \begin{minipage}{1cm} \end{minipage}\\[1cm]  
\stepcounter{zaehler}                           % increase counter
\end{samepage}%
\\%
\bigskip%
}


%\usepackage{tikz}
%\usetikzlibrary{shadows}
%\usetikzlibrary{fit}
%\usetikzlibrary{shapes}
%\usetikzlibrary{backgrounds}

\parindent0cm % Abs�tze nicht einr�cken 

% Definieren einer neuen Farbe
\definecolor{light-gray}{gray}{.9}

%-------------------------------------------------------------------------------
\begin{document}
%-------------------------------------------------------------------------------

%--------------------------------------------------- Header
\begin{center}
\textbf{\LARGE Mathematische Anwendersysteme }\\
\textbf{\LARGE Einführung in Sage}\\\medskip
\end{center}
%\begin{minipage}{6cm}
%Jochen Schulz,\\
%Christoph Rügge
%\end{minipage}
\hfill
\begin{minipage}{4cm}
%\textbf{Klausur}\\
10.03.2011
\end{minipage}\\[1cm]
%------------------------------

\begin{center}
\Huge \textbf{Klausur}
\end{center}
\bigskip\bigskip\bigskip
\Large
\begin{center}
\begin{tabular}{|Sl||p{0.5cm}|p{0.5cm}|p{0.5cm}|p{0.5cm}|p{0.5cm}|p{0.5cm}|p{0.5cm}||Sc|}
\hline
Aufgabe & \textbf{1} & \textbf{2} & \textbf{3} & \textbf{4} & \textbf{5} & \textbf{6} & \textbf{7} & \textbf{Summe}\\
\hline
Mögl. Pkt. &  4  & 2  & 2  & 3  & 5  & 8  & 5  &  29  \\
\hline
Erreichte Pkt. &    &   &   &   &   &   &     &    \\
\hline
\end{tabular}
\end{center}

\bigskip\bigskip\bigskip
Bitte eintragen:\\
\begin{center}
\begin{tabular}{|Sl|p{8cm}|}
\hline
Nachname: & \\
\hline
Vorname: & \\
\hline
Studiengang: & \\
\hline 
Semester: & \\
\hline 
Immatrikulationsnummer: & \\
\hline
\end{tabular}\\[1cm]
\textbf{Hinweise:}
\begin{itemize}
\item Die Klausur beginnt um 10.30 Uhr und endet um 12.00 Uhr.
\item Ben\"otigte Hilfsmittel sind Stift und Papier.
\item Erlaubte Hilfsmittel sind gedruckte sowie handgeschriebene Notizen oder Skripte. 
\item Benutzen Sie zum Aufschreiben der Aufgaben möglichst exakten Sage-Syntax.
\end{itemize}
\end{center}

\newpage
\normalsize

%+1
% \begin{aufg}{4}
% \begin{itemize}
%  \item Definieren Sie in Sage eine Liste, ein Tuple, ein Dictionary und eine Menge (\isage{Set}).
% \item Geben Sie jeweils eine kurze Erklärung zu den genannten Datentypen. Geben Sie 4 Operationen an, die Sie 
% nur mit jeweils einen der genannten Datentypen durchführen können. 
% \end{itemize}
% \end{aufg}

%-------------------------------------------------------------------------
\begin{aufg}[4]
\begin{itemize}
\item Erstellen Sie, in Sage eine Liste, ein Tuple, ein
  Dictionary und eine Menge mit je einem Element Ihrer Wahl. Wie fügt man jeweils Elemente
  hinzu (sofern möglich) oder greift auf Elemente zu?
\item Welchen dieser Datentyen würden Sie wählen, wenn Sie Messungen zu äquidistanten Zeitintervallen darstellen wollen? Wäre Ihre Wahl eine andere, wenn 
  die Messungen zu zufälligen Zeitpunkten erfolgt sind? Begründen Sie!
\end{itemize}
\end{aufg}

%-------------------------------------------------------------------------
\begin{aufg}[2]
Nennen Sie Funktionen mit denen symbolische Ausdrücke bzgl. ihrer Struktur verändert werden können (z.B. Vereinfachen).
\end{aufg}
%-------------------------------------------------------------------------

\begin{aufg}[2]
Erklären Sie die Funktionsweise einer \isage{for}-Schleife anhand eines kurzen Beispiels.
\end{aufg}

\begin{aufg}[3]
Schreiben Sie eine Funktion mit \isage{def}, die folgende Funktion berechnet: 
\[ f(n) = 2^{(2^n)}+1\]
Gibt es noch eine andere Möglichkeit die Funktion zu definieren? Wenn ja, erklären sie kurz worin die Unterschiede zwischen den  Varianten liegen.
\end{aufg}

\begin{aufg}[4]
Schreiben Sie eine rekursive Funktion \isage{verfeinere(f, a, b, TOL)}, die gegeben eine Funktion $f$, ein Intervall $[a,\, b]$ und eine Genauigkeit $TOL$ Stützstellen bestimmt, so dass sich die Funktionswerte an benachbarten Stützstellen um weniger als $TOL$ unterscheiden. Gehen Sie dabei folgendermaßen vor:
\begin{itemize}
\item Wenn $|f(a)-f(b)| < TOL$ ist, geben Sie die Liste $[a,\, b]$ zurück.
\item Andernfalls nutzen Sie die Funktion rekursiv, um jeweils die linke Intervall-Hälfte $[a,\,\frac{a+b}{2}]$ und die rechte Intervall-Hälfte $\frac{a+b}{2},\, b]$ zu verfeinern. Fügen Sie die Resultate in eine Liste zusammen und gegeben Sie diese anschließend als Ergebnis zurück.
\end{itemize}
\end{aufg}

%\begin{aufg}[5]
%Schreiben Sie eine Abfolge von Befehlen die mit einer gegebenen unbekannten Funktion $f: \mathbb{R} \to \mathbb{R}$ eine 
%Kurvendiskussion durchführt, ohne dass Sie die Befehlsfolge noch an die Funktion anpassen müssten. Dabei sollen folgende Schritte bearbeitet werden:
%\begin{itemize}
%\item Untersuchen Sei das Verhalten von $f(x)$ für $x\r \rightarrow \pm \infty$.
%\item Finden Sie die Nullstellen und berechnen sie die jeweilige Steigung in dem Punkt.
%\item Geben sie mittels des \isage{print}-Befehls und einer Schleife die Nullstelle und ihre Steigung  aus.
%\end{itemize}
%\end{aufg}

% \begin{aufg}{5}
% Schreiben Sie eine Funktion mit Input-Variablen $x_0$ und $TOL$, die die Folge
% \[ x_{n+1} = x_n - \frac{x_n^2-x_n-1}{2 x_n-1}, \quad n \in \mathbb{N} \]
% berechnet und abbricht, wenn $|x_n^2-x_n-1| \leq TOL$ ist. Die Funktion soll
% $x_{n}$ und das zugeh\"orige $n$ zur\"uckgeben. 
% \end{aufg}



% \begin{aufg}{5}
%   Schreiben Sie eine Funktion, die den Binomialkoeffizienten $n\choose
%   k$ berechnet. Nutzen Sie dafür die Rekursionsbeziehung
%   \[ {n \choose k} = {n-1 \choose k-1} + {n-1 \choose k},
%      \qquad
%      {n \choose 0} = 1,\,\,
%      {0 \choose k} = 0.
%   \]
% \end{aufg}

% \begin{aufg}{3}
% Gegeben sei der Vektor \isage{xvec = [x0,...,xn]} für $n\in \mathbb{N}$.
% Mit welcher Befehlsfolge kann man die Vandermonde-Matrix 
% \[ 
% V:= \left(\begin{array}{ccccc} 
% 1 & x_0 & x_0^2 & \hdots & x_0^{n}\\
% 1 & x_1 & x_1^2 & \hdots & x_1^{n}\\
% \vdots & \vdots & \vdots & \vdots & \vdots\\
% 1 & x_n & x_n^2 & \hdots & x_n^{n}\\
% \end{array} \right) 
% \]
% erstellen?
% \end{aufg}


% \begin{aufg}{2}z
% Schreiben Sie eine MuPAD-Prozedur \isage{programm1} die für eine Zahl $n>10$ die Zeichenkette \isage{MuPAD ist super} und für $n\leq10$ die Zeichenkette   \isage{MuPAD ist Mist} zurückgibt (Typenüberprüfung braucht nicht durchgeführt zu werden).
% \end{aufg}
%-----------------------------------------------------------------------------------
%\begin{aufg}[5]
%Schreiben Sie ohne Verwendung der Funktion \isage{index()} eine Sage-Prozedur mit 2 Argumenten, die aus einer Liste von Zahlen den Index einer zu %suchenden Zahl, oder \isage{-1}, zurückliefert.
%\end{aufg}

%-----------------------------------------------------------------------------------
\begin{aufg}[4]
Schreiben Sie eine Funktion, die als Eingabe einen Rechenausdruck bestehend aus Multiplikationen und Additionen als String bekommt (zum Beispiel "3+17*5+12") und das Ergebnis zurückliefert. \emph{Hinweis:} Benutzen Sie die Funktion \isage{split()} um an die Operanden zu kommen und die Funktion \isage{int()} zum Umwandeln in Zahlen. Denken Sie auch an "`Punkt- vor Strichrechnung"'.
\end{aufg}
\end{document}

% Local Variables:
% TeX-command-default: "XeLaTeX"
% auto-fill-function: nil
% End:
