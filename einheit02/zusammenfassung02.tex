\documentclass[a4paper,9pt,DIV15,twocolumn]{scrartcl}
\usepackage[xetex,bookmarks=true,pdfstartview=FitH,bookmarksopen=true,
    colorlinks,citecolor=Blue,linkcolor=DarkBlue,urlcolor=Green,
    pagebackref=true,plainpages=false,pdfpagelabels=true,unicode=true,
    breaklinks=true,naturalnames=false,setpagesize=true,a4paper=true,hyperindex]{hyperref}
\usepackage[svgnames,hyperref]{xcolor} %color definition
\usepackage{fontspec,xunicode,xltxtra}
%\usepackage{fontspec,xunicode}
%\usepackage{polyglossia}
%\setdefaultlanguage[spelling=new, latesthyphen=true]{german}
%\setsansfont{DejaVu Sans}
%\setsansfont{Verdana}
%\setsansfont{Arial}
%\setromanfont[Mapping=tex-text]{Linux Libertine}
%\setsansfont[Mapping=tex-text]{Myriad Pro}
%\setmonofont[Mapping=tex-text]{Courier New}

%\setsansfont{Linux Biolinum}

\usepackage[ngerman]{babel}
\selectlanguage{ngerman}

%
% math/symbols
%
\usepackage{amssymb}
\usepackage{amsthm}
% \usepackage{latexsym}
\usepackage{amsmath}
%\usepackage{amsxtra} %Weitere Extrasymbole
%\usepackage{empheq} %Gleichungen hervorheben
%\usepackage{bm}
 %\bm{A} Boldface im Mathemodus

\usepackage{multimedia}
%\usepackage{tikz}

\usepackage{cellspace}
\setlength{\cellspacetoplimit}{2pt}
\setlength{\cellspacebottomlimit}{2pt}

%%%%%%%%%%%%%%%%%% Fuer Frames [fragile]-Option verwenden!
%Programm-Listing
%%%%%%%%%%%%%%%%%%
%Listingsumgebung fuer verbatim
%Grauhinterlegeter Text
%Automatischer Zeilenumbruch ist aktiviert
\usepackage{listings}
% This command allows you to typeset syntax highlighted Matlab
% code ``inline''.
\newcommand{\isage}[1]{\lstinline|#1|}

\definecolor{lgray}{gray}{0.80}
\definecolor{gray}{gray}{0.3}
\definecolor{darkgreen}{rgb}{0,0.4,0}
\definecolor{darkblue}{rgb}{0,0,0.8}
\definecolor{key}{rgb}{0,0.5,0} 
%\lstset{backgroundcolor=\color{lgray}, frame=single, basicstyle=\ttfamily, breaklines=true}
\lstnewenvironment{sage}[1][]{\lstset{xleftmargin=0.2cm,frame=none,backgroundcolor=\color{white},basicstyle=\color{darkblue}\ttfamily\small,#1}}{} 
\lstnewenvironment{sagein}[1][]{\lstset{#1}}{} 
%\lstnewenvironment{sage}{\lstset{,language=python, keywordstyle=color{blue},    commentstyle=color{green}, emphstyle=\color{red}, %frame=single, stringstyle=\color{red}, basicstyle=\ttfamily, ,mathescape =true,escapechar=§}}{}

\lstset{
language=python,
backgroundcolor=\color{lgray},
breaklines=true,
basicstyle=\ttfamily\small,
%otherkeywords={ =},
keywordstyle=\color{blue},
stringstyle=\color{darkgreen},
showstringspaces=false,
emph={class, pass, in, for, while, if, is, elif, else, not, and, or,
def, print, exec, break, continue, return},
emphstyle=\color{blue},
emph={[2]True, False, None, self},
emphstyle=[2]\color{key},
emph={[3]from, import, as},
emphstyle=[3]\color{blue},
upquote=true,
morecomment=[s]{"""}{"""},
commentstyle=\color{gray}\slshape,
%framexleftmargin=1mm, framextopmargin=1mm, 
frame=single,
mathescape =true,
escapechar=§
}


\usepackage{mydef}
%\usepackage{cmap} % you can search in the pdf for umlauts and ligatures
\usepackage{colonequals} %corrects the definition-symbols \colonequals (besides others)
\usepackage{ifthen}
%%%%%%%%%%%%%%%%%%%
%Neue Definitionen
%%%%%%%%%%%%%%%%%%%

%Newcommands
\newcommand{\Fun}[1]{\mathcal{#1}}      %Mathcal fuer Funktoren
\newcommand{\field}[1]{\mathbb{#1}}     %Grundkoerper ?? in mathds

\newcommand{\A}{\field{A}}              %Affines A
\newcommand{\C}{\field{C}}              %Complexes C
\newcommand{\Fp}{\field{F}_{\!p}}       %Endlicher Koerper mit p Elementen
\newcommand{\Fq}{\field{F}_{\!q}}       %Endlicher Koerper mit q Elementen
\newcommand{\Ga}{\field{G}_{a}}         %Add Gruppenschema
\newcommand{\K}{\field{K}}              %Generischer Koerper 
\newcommand{\N}{\field{N}}              %Nat Zahlen
\newcommand{\Pj}{\field{P}}             %Projektives P
\newcommand{\R}{\field{R}} 		%Reelle Zahlen
\newcommand{\Q}{\field{Q}}              %Rationale Zahlen  
\newcommand{\Qt}{\field{H}}             %Quaternionen 
\newcommand{\V}{\field{V}}              %Vektorbuendel V
\newcommand{\Z}{\field{Z}}              %Ganze Zahlen
\DeclareMathOperator{\Real}{Re}

\newcommand{\fdg}{\;|\;}                 %fuer die gilt

%Operatoren
\DeclareMathOperator{\Abb}{Abb}
%\usepackage{sagetex}

%
% Aufgaben
%
\parindent0cm % Abs�tze nicht einr�cken 
% Definieren einer neuen Farbe
\definecolor{light-gray}{gray}{.9}

\newcounter{zaehler}     % neuen Z�hler einf�hren
\stepcounter{zaehler}    % Z�hler einen hochz�hlen

\newenvironment{aufg}[1][0]
%---- Header
{\begin{samepage}%
%\colorbox{light-gray}{%                         % Box in gray
% \makebox[\textwidth]{%                           % Box in linewidth
%\textbf{Aufgabe \arabic{zaehler} } }\hspace{-\textwidth}\makebox[\textwidth]{\hfill #1 Punkte} }\\[0.05cm]       % Header
\dotfill\\
{\large\textbf{Aufgabe \arabic{zaehler} }\ifthenelse{0=#1}{}{\hfill #1 Punkte} }\\[0.4cm]
\begin{minipage}{\textwidth}
}
%-----  foot
{\end{minipage} \nopagebreak %\begin{minipage}{1cm} \end{minipage}
%\\ 
%\begin{minipage}{0.1cm} \end{minipage} 
%\hrulefill \begin{minipage}{1cm} \end{minipage}\\[1cm]  
\stepcounter{zaehler}                           % increase counter
\end{samepage}%
\\%
\bigskip%
}

\begin{document}
%--------------------------------------------------- Header
\begin{center}
    \textbf{\LARGE Einführung in Sage}\\
    {\large Zusammenfassung Einheit 02}
\end{center}
\textsl{Hinweis:} Textbausteine mit <name> weisen darauf hin, das anstatt des Ausdrucks eine passende Variable eingefügt werden muss.

\medskip
\textbf{Grundlagen}
\begin{itemize}
    \item Ausgabe- \href{http://docs.python.org/library/functions.html?highlight=print#print}{print} 
        \begin{sagein}
print ('Text {<format>} und {<format>} .. '.format(x,y,..))
        \end{sagein}
			\item Allgemeiner: <str>.format()
				\begin{sagein}
"x hat den Wert {} und y den Wert {}".format(x,y)
				\end{sagein}
\end{itemize}



\textbf{Dictionaries- } \href{http://docs.python.org/library/stdtypes.html?highlight=.update#mapping-types-dict}{\textbf{dictionaries}}

\begin{itemize}
 \item Deklarieren eines Dictionaries
\begin{sagein}
d = {<Index1>:<Wert1>,<Index2>:<Wert2>,...}
\end{sagein}
 \item Zugriff auf einen Index
\begin{sagein}
d[<Index>]
\end{sagein}
\item Liste aller Indizes
	\begin{sagein}
d.keys()
	\end{sagein}
\item Liste aller Werte
	\begin{sagein}
d.values()
	\end{sagein}
\end{itemize}

\textbf{Funktionen, Schleifen und Abfragen}

\begin{itemize}
 \item anonyme Funktion- \href{http://docs.python.org/howto/functional.html#small-functions-and-the-lambda-expression}{lambda}
\begin{sagein}
lambda <parameter_list>: <expression> 
\end{sagein}
% \item Objektfunktion: 
 \item Schleifen- \href{http://docs.python.org/tutorial/controlflow.html#for-statements}{for}
\begin{sagein}
for <var> in <list>:
    <codeBlock>
\end{sagein}
 \item Abfragen- \href{http://docs.python.org/reference/compound_stmts.html#the-if-statement}{if}
\begin{sagein}
if <boolean expr>:
    <codeBlock>
elif <boolean expr>:
		<codeBlock>
else:
    <codeBlock>
\end{sagein}

\end{itemize}
\textbf{Symbolisches Rechnen}
%\begin{itemize}
%\item Unbestimmtes Integral- \href{https://sage.math.uni-goettingen.de/doc/static/reference/sage/calculus/functional.html?highlight=diff#sage.calculus.functional.integral}{integrate()}
%\begin{sagein}
%integrate(<expr>,<variable>)
%\end{sagein}
% \item Bestimmtes Integral
%\begin{sagein}
%integrate(<expr>,<variable>,<lower>,<upper>)
%\end{sagein}
%\end{itemize}
  \begin{itemize}

      
      \item Sortieren nach \isage{<x>}- \href{https://sage.math.uni-goettingen.de/doc/static/reference/sage/symbolic/expression.html?highlight=simplify_full#sage.symbolic.expression.Expression.collect}{collect()}
       \begin{sagein}
<expr>.collect(<x>)  
       \end{sagein}
		 \item Zerlegen von Ausdr\"ucken
			 \begin{itemize}
				\item Anzahl der Operanden
					\begin{sagein}
<expr>.nops()
					\end{sagein}
				\item Operanden eines Ausdrucks
					\begin{sagein}
<expr>.operands()
					\end{sagein}
				\item Ist Operand in Ausdruck?
					\begin{sagein}
<expr>.has(<op>)
					\end{sagein}
				\end{itemize}
		 \item  Potenzgesetze- \href{https://sage.math.uni-goettingen.de/doc/static/reference/sage/symbolic/expression.html?highlight=simplify_full#sage.symbolic.expression.Expression.combine}{combine()}
       \begin{sagein}
<expr>.combine() 
       \end{sagein}
		 \item Ausmultiplizieren- \href{https://sage.math.uni-goettingen.de/doc/static/reference/sage/symbolic/expression.html?highlight=simplify_full#sage.symbolic.expression.Expression.expand}{expand()}, \href{https://sage.math.uni-goettingen.de/doc/static/reference/calculus/sage/symbolic/expression.html#sage.symbolic.expression.Expression.expand_trig}{expand\_trig()}
       \begin{sagein}
<expr>.expand() 
<expr>.expand_trig()
       \end{sagein}
   \item Faktorisieren- \href{https://sage.math.uni-goettingen.de/doc/static/reference/sage/symbolic/expression.html?highlight=simplify_full#sage.symbolic.expression.Expression.factor}{factor()}
       \begin{sagein}
<expr>.factor()  
       \end{sagein}
   \item Partialbruch-Zerlegung- \href{https://sage.math.uni-goettingen.de/doc/static/reference/sage/symbolic/expression.html?highlight=simplify_full#sage.symbolic.expression.Expression.partial_fraction}{partial\_fraction()}
       \begin{sagein}
<expr>.partial_fraction()
       \end{sagein}
   \item Vereinfachen
   \begin{itemize}
       \item  trigonometrisch- \href{https://sage.math.uni-goettingen.de/doc/static/reference/sage/symbolic/expression.html?highlight=simplify_full#sage.symbolic.expression.Expression.simplify_trig}{simplify\_trig()}
       \begin{sagein}
<expr>.simplify_trig() 
       \end{sagein}
        \item rational- \href{https://sage.math.uni-goettingen.de/doc/static/reference/sage/symbolic/expression.html?highlight=simplify_full#sage.symbolic.expression.Expression.simplify_rational}{simplify\_rational()}
       \begin{sagein}
<expr>.simplify_rational() 
       \end{sagein}
        \item log/ln/exp- \href{https://sage.math.uni-goettingen.de/doc/static/reference/sage/symbolic/expression.html?highlight=simplify_full#sage.symbolic.expression.Expression.simplify_radical}{simplify\_radical()}
       \begin{sagein}
<expr>.simplify_radical() 
       \end{sagein}
        \item Nutzung der Fakultät- \href{https://sage.math.uni-goettingen.de/doc/static/reference/sage/symbolic/expression.html?highlight=simplify_full#sage.symbolic.expression.Expression.simplify_factorial}{simplify\_factorial()}
       \begin{sagein}
<expr>.simplify_factorial()
       \end{sagein}
        \item alle Vereinfachungen- \href{https://sage.math.uni-goettingen.de/doc/static/reference/sage/symbolic/expression.html?highlight=simplify_full#sage.symbolic.expression.Expression.simplify_full}{simplify\_full()}
       \begin{sagein}
<expr>.simplify_full()
       \end{sagein}
       \end{itemize}

  \end{itemize}

\textbf{Gleichungen, Vergleiche, Logik}

\begin{itemize}
 %\item Wahrheitswert für die Gleichung- \href{http://docs.python.org/library/stdtypes.html#boolean-operations-and-or-not}{boolean}
  %   \begin{sagein}
%bool(<Ausdruck/Gleichung>)   
 %    \end{sagein}
	\item \"Uberpr\"ufung zweier Ausdr\"ucke auf Gleichheit
		\begin{sagein}
<expr1>==<expr2>
		\end{sagein}
	\item \textbf{ == != =}
 \item Wahrheitswerte: \isage{True}, \isage{False}
 \item Logische Verknüpfungen: \isage{and}, \isage{or}, \isage{not}
 \item Gleichungssysteme analytisch lösen mit \href{https://sage.math.uni-goettingen.de/doc/static/reference/sage/symbolic/relation.html?highlight=symbolic.relation#sage.symbolic.relation.solve}{solve()}:
\begin{sagein}
 solve(<equations>,<vars>,<options>)
\end{sagein}
\item Gleichungssysteme numerisch lösen mit \href{https://sage.math.uni-goettingen.de/doc/static/reference/sage/numerical/optimize.html?highlight=numerical.optimize#sage.numerical.optimize.find_root}{find\_root()}:
    \begin{sagein}
<expr>.find_root(<lowerBound>,<upperBound>)
    \end{sagein}


\end{itemize}

\end{document}

