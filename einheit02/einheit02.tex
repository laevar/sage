\documentclass[hyperref={xetex}]{beamer}
\title{Einführung in Sage - Einheit 2}
\subtitle{Grundlagen, Symbolisches Rechnen, Gleichungen}
\mode<article>
{
  \usepackage{fullpage}
  \usepackage{pgf}
  \usepackage[xetex]{hyperref}
  \setjobnamebeamerversion{beamer}
}

\mode<presentation>
{
  %\usetheme{Frankfurt}
 %\usetheme{My}
  \usetheme{Madrid}
  % or ...
%\usecolortheme{seagull}
  %\setbeamercovered{transparent}
  %\setbeamercovered{dynamic}
  % or whatever (possibly just delete it)
}
\usenavigationsymbolstemplate{}
\usefonttheme{structurebold}
\usepackage{multimedia}
%\usepackage{tikz}
\usepackage{fontspec,xunicode,xltxtra}

%\usepackage{polyglossia}
%\setdefaultlanguage[spelling=new, latesthyphen=true]{german}
%\setsansfont{DejaVu Sans}
%\setsansfont{Verdana}
%\setsansfont{Arial}
%\setromanfont{Linux Libertine O}
%\setsansfont{Linux Biolinum O}

\setbeamertemplate{footline}
{
\leavevmode
%\hbox{\begin{beamercolorbox}[wd=.5\paperwidth,ht=2.5ex,dp=1.125ex,
%leftskip=.3cm plus1fill,rightskip=.3cm]{author in head/foot}%
%    \usebeamerfont{author in head/foot}\insertshortauthor
%  \end{beamercolorbox}%
%  \begin{beamercolorbox}[wd=.5\paperwidth,ht=2.5ex,dp=1.125ex,leftskip=.3cm,
%rightskip=.3cm plus1fil]{title in head/foot}%
%    \usebeamerfont{title in head/foot}\insertshorttitle\hfill

\hfill\insertframenumber  \hspace{3pt}

%\inserttotalframenumber
%\hspace*{2ex}
%  \end{beamercolorbox}}%
  \vskip3pt%
}

\usepackage[ngerman]{babel}
\selectlanguage{ngerman}

%
% math/symbols
%
\usepackage{amssymb}
\usepackage{amsthm}
% \usepackage{latexsym}
\usepackage{amsmath}
%\usepackage{amsxtra} %Weitere Extrasymbole
%\usepackage{empheq} %Gleichungen hervorheben
%\usepackage{bm}
 %\bm{A} Boldface im Mathemodus

\usepackage{cellspace}
\setlength{\cellspacetoplimit}{2pt}
\setlength{\cellspacebottomlimit}{2pt}

%%%%%%%%%%%%%%%%%% Fuer Frames [fragile]-Option verwenden!
%Programm-Listing
%%%%%%%%%%%%%%%%%%
%Listingsumgebung fuer verbatim
%Grauhinterlegeter Text
%Automatischer Zeilenumbruch ist aktiviert
\usepackage{listings}
\definecolor{lgray}{gray}{0.80}
%\lstset{backgroundcolor=\color{lgray}, frame=single, basicstyle=\ttfamily, breaklines=true}
\lstnewenvironment{sage}{\lstset{backgroundcolor=\color{lgray},language=Python, emphstyle=\color{red}, frame=single, basicstyle=\ttfamily, breaklines=true,mathescape =true,escapechar=§}}{}


\usepackage{mydef}
%\usepackage{cmap} % you can search in the pdf for umlauts and ligatures
\usepackage{colonequals} %corrects the definition-symbols \colonequals (besides others)
\title{Einführung in Sage}
%
%\subtitle{Disputation} % (optional)

\author{Jochen Schulz}
% - Use the \inst{?} command only if the authors have different
%   affiliation.

\institute{Georg-August Universit\"at G\"ottingen \pgfimage[height=0.5cm]{../figures/unilogo3}}
% - Use the \inst command only if there are several affiliations.
% - Keep it simple, no one is interested in your street address.

\date{\today}

\subject{Sage}
% This is only inserted into the PDF information catalog. Can be left
% out. 

% If you have a file called "university-logo-filename.xxx", where xxx
% is a graphic format that can be processed by latex or pdflatex,
% resp., then you can add a logo as follows:

%\logo{\pgfimage[height=0.5cm]{figures/unilogo3}}


% Delete this, if you do not want the table of contents to pop up at
% the beginning of each subsection:

\AtBeginSection[]
{
  \begin{frame}<beamer>
    \frametitle{Aufbau}
    \tableofcontents[currentsection,currentsubsection]
  \end{frame}
}

\AtBeginSubsection[]
{
  \begin{frame}<beamer>
    \frametitle{Aufbau}
    \tableofcontents[currentsection,currentsubsection]
  \end{frame}
}



%%%%%%%%%%%%%%%%%%%
%Neue Definitionen
%%%%%%%%%%%%%%%%%%%

%Newcommands
\newcommand{\Fun}[1]{\mathcal{#1}}      %Mathcal fuer Funktoren
\newcommand{\field}[1]{\mathbb{#1}}     %Grundkoerper ?? in mathds

\newcommand{\A}{\field{A}}              %Affines A
\newcommand{\C}{\field{C}}              %Complexes C
\newcommand{\Fp}{\field{F}_{\!p}}       %Endlicher Koerper mit p Elementen
\newcommand{\Fq}{\field{F}_{\!q}}       %Endlicher Koerper mit q Elementen
\newcommand{\Ga}{\field{G}_{a}}         %Add Gruppenschema
\newcommand{\K}{\field{K}}              %Generischer Koerper 
\newcommand{\N}{\field{N}}              %Nat Zahlen
\newcommand{\Pj}{\field{P}}             %Projektives P
\newcommand{\R}{\field{R}} 		%Reelle Zahlen
\newcommand{\Q}{\field{Q}}              %Rationale Zahlen  
\newcommand{\Qt}{\field{H}}             %Quaternionen 
\newcommand{\V}{\field{V}}              %Vektorbuendel V
\newcommand{\Z}{\field{Z}}              %Ganze Zahlen

\newcommand{\fdg}{\;|\;}                 %fuer die gilt

%Operatoren
\DeclareMathOperator{\Abb}{Abb}
%\usepackage{sagetex}

\begin{document}
\lstset{basicstyle={\lstbasicfont\footnotesize}}


\begin{document}
\titlepage

\begin{frame}{Aufbau}
\tableofcontents
\end{frame}

%=================================================
\section{Grundlagen}
%=================================================

\subsection{Sage}

\begin{frame}[fragile]{Beispiel} 
Betrachte:
\begin{sagein}
f = x^2-3*x-18
\end{sagein}
\begin{itemize}
\item Wie geht Sage mit der Unbekannten $x$ um?
\item Welchen Datentyp hat $f$?
\item Was kann ich mit $f$ machen?
\end{itemize}
\end{frame}

\begin{frame}[fragile]{Bezeichner}
\begin{itemize}
\item \alert{Bezeichner} sind Namen, wie z.B. $x$ oder $f$. Sie können
im mathematischen Kontext sowohl Variablen als auch Unbestimmte repräsentieren.
\item Bezeichner sind aus Buchstaben, Ziffern und
Unterstrich \_ zusammengesetzt.
\item Sage unterscheidet zwischen Groß- und Kleinschreibung.
\item Bezeichner dürfen nicht mit einer Ziffer beginnen.
\end{itemize}
\textbf{Beispiele}
\begin{itemize}
\item zulässige Bezeichner:
\isage{x}, \isage{f}, \isage{x23}, \isage{_x_1}
\item unzulässige Bezeichner:
\isage{12x}, \isage{p~}, \isage{x>y}, \isage{Das System}
\end{itemize}
\end{frame}


\begin{frame}[fragile]{Wert eines Bezeichners}
\begin{itemize}
\item Der \alert{Wert} eines Bezeichners  ist ein \alert{Objekt} eines bestimmten
\alert{Datentyps}.
\item Ein \alert{Datentyp} ist durch seine Eigenschaften gegeben. \\
\textbf{Beispiel}: Natürliche Zahlen, rationale Zahlen, Bezeichner, Zeichenketten, \ldots  
\item Ein \alert{Objekt} ist eine Instanz (Einheit) eines Datentyps.
\end{itemize}
\end{frame}

\begin{frame}[fragile]{Zuweisungsoperator $=$}
    \begin{sagein}
<bezeichner> = <wert>
    \end{sagein}   
Zuweisung des Wertes \isage{wert} zu dem Bezeichner \isage{bez}.
\begin{itemize}
\item {\color{blue} \isage{func(arg)=expr(arg)}}: Definition der Funktion \isage{func} mit dem Argument \isage{arg} und Zuweisung des Ausdrucks \isage{expr} zu (abhängig von \isage{arg})
%\item Rückgabeparameter ist die rechte Seite (Eine Ausgabe erfolgt jedoch normalerweise nicht)
\item \textbf{Warnung:} Unterscheiden Sie  stets zwischen dem Zuweisungsoperator {\color{blue} $=$} und
dem logischen Operator {\color{blue} $==$}.   
\item {\color{blue} \isage{reset('<bezeichner>')}}: Löschen von Zuweisungen/Variablen.
\end{itemize}
\end{frame}


\begin{frame}{Operatoren}
\begin{itemize}
\item Typische \alert{Operatoren} sind \verb~+,-,*,/,..~
\item In Sage werden Objekte immer durch Funktionen miteinander
verbunden. Operatoren sind äquivalent zu Funktionen.
%\item Operatoren sind in Sage auch als Funktionen realisiert.
%\item Für wichtige Operatoren gibt es die gewohnte
%Kurzschreibweise.
\item Kombination verschiedener Operatoren: Die Regeln der \alert{Bindungsstärke} gelten (Punktrechnung vor Strichrechnung); Die Ordnung kann
durch Klammersetzung geändert werden.
\end{itemize}
\end{frame}

\begin{frame}[fragile]{Wichtige mathematische Operatoren}
\begin{center}
\begin{tabular}{|c|l|}
\hline
Operator/Funktion &  Erklärung\\
\hline
\hline
\verb!+! & Addition \\
\verb!-! & Subtraktion\\
\verb!*! & Multiplikation \\
\verb!/! & Division\\
\verb!^! & Potenz\\
\verb!%! &  Rest bei Division\\
\verb!factorial()! & Fakultät \\
\hline
\end{tabular}
\end{center}
\end{frame}

\begin{frame}[fragile]{Sage}
\begin{center}
\url{https://sage.math.uni-goettingen.de/home/pub/8/}
\end{center}
\end{frame}

\subsection{Python}
\begin{frame}[fragile]{Sage}
\begin{center}
\url{https://sage.math.uni-goettingen.de/home/pub/9/}
\end{center}
\end{frame}

\section{Symbolisches Rechnen I}
\begin{frame}[fragile]{Sage}
\begin{center}
\url{https://sage.math.uni-goettingen.de/home/pub/10/}
\end{center}
\end{frame}
\section{Gleichungen}
\begin{frame}[fragile]{Sage}
\begin{center}
\url{https://sage.math.uni-goettingen.de/home/pub/11/}
\end{center}
\end{frame}

\end{document}





















