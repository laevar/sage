\documentclass[notes=hide,hyperref={dvipdfmx,pdfpagelabels=false}]{beamer}
\title{Einführung in Sage - Einheit 2}
\subtitle{Grundlagen, Symbolisches Rechnen, Gleichungen}
\mode<article>
{
  \usepackage{fullpage}
  \usepackage{pgf}
  \usepackage[xetex]{hyperref}
  \setjobnamebeamerversion{beamer}
}

\mode<presentation>
{
  %\usetheme{Frankfurt}
 %\usetheme{My}
  \usetheme{Madrid}
  % or ...
%\usecolortheme{seagull}
  %\setbeamercovered{transparent}
  %\setbeamercovered{dynamic}
  % or whatever (possibly just delete it)
}
\usenavigationsymbolstemplate{}
\usefonttheme{structurebold}
\usepackage{multimedia}
%\usepackage{tikz}
\usepackage{fontspec,xunicode,xltxtra}

%\usepackage{polyglossia}
%\setdefaultlanguage[spelling=new, latesthyphen=true]{german}
%\setsansfont{DejaVu Sans}
%\setsansfont{Verdana}
%\setsansfont{Arial}
%\setromanfont{Linux Libertine O}
%\setsansfont{Linux Biolinum O}

\setbeamertemplate{footline}
{
\leavevmode
%\hbox{\begin{beamercolorbox}[wd=.5\paperwidth,ht=2.5ex,dp=1.125ex,
%leftskip=.3cm plus1fill,rightskip=.3cm]{author in head/foot}%
%    \usebeamerfont{author in head/foot}\insertshortauthor
%  \end{beamercolorbox}%
%  \begin{beamercolorbox}[wd=.5\paperwidth,ht=2.5ex,dp=1.125ex,leftskip=.3cm,
%rightskip=.3cm plus1fil]{title in head/foot}%
%    \usebeamerfont{title in head/foot}\insertshorttitle\hfill

\hfill\insertframenumber  \hspace{3pt}

%\inserttotalframenumber
%\hspace*{2ex}
%  \end{beamercolorbox}}%
  \vskip3pt%
}

\usepackage[ngerman]{babel}
\selectlanguage{ngerman}

%
% math/symbols
%
\usepackage{amssymb}
\usepackage{amsthm}
% \usepackage{latexsym}
\usepackage{amsmath}
%\usepackage{amsxtra} %Weitere Extrasymbole
%\usepackage{empheq} %Gleichungen hervorheben
%\usepackage{bm}
 %\bm{A} Boldface im Mathemodus

\usepackage{cellspace}
\setlength{\cellspacetoplimit}{2pt}
\setlength{\cellspacebottomlimit}{2pt}

%%%%%%%%%%%%%%%%%% Fuer Frames [fragile]-Option verwenden!
%Programm-Listing
%%%%%%%%%%%%%%%%%%
%Listingsumgebung fuer verbatim
%Grauhinterlegeter Text
%Automatischer Zeilenumbruch ist aktiviert
\usepackage{listings}
\definecolor{lgray}{gray}{0.80}
%\lstset{backgroundcolor=\color{lgray}, frame=single, basicstyle=\ttfamily, breaklines=true}
\lstnewenvironment{sage}{\lstset{backgroundcolor=\color{lgray},language=Python, emphstyle=\color{red}, frame=single, basicstyle=\ttfamily, breaklines=true,mathescape =true,escapechar=§}}{}


\usepackage{mydef}
%\usepackage{cmap} % you can search in the pdf for umlauts and ligatures
\usepackage{colonequals} %corrects the definition-symbols \colonequals (besides others)
\title{Einführung in Sage}
%
%\subtitle{Disputation} % (optional)

\author{Jochen Schulz}
% - Use the \inst{?} command only if the authors have different
%   affiliation.

\institute{Georg-August Universit\"at G\"ottingen \pgfimage[height=0.5cm]{../figures/unilogo3}}
% - Use the \inst command only if there are several affiliations.
% - Keep it simple, no one is interested in your street address.

\date{\today}

\subject{Sage}
% This is only inserted into the PDF information catalog. Can be left
% out. 

% If you have a file called "university-logo-filename.xxx", where xxx
% is a graphic format that can be processed by latex or pdflatex,
% resp., then you can add a logo as follows:

%\logo{\pgfimage[height=0.5cm]{figures/unilogo3}}


% Delete this, if you do not want the table of contents to pop up at
% the beginning of each subsection:

\AtBeginSection[]
{
  \begin{frame}<beamer>
    \frametitle{Aufbau}
    \tableofcontents[currentsection,currentsubsection]
  \end{frame}
}

\AtBeginSubsection[]
{
  \begin{frame}<beamer>
    \frametitle{Aufbau}
    \tableofcontents[currentsection,currentsubsection]
  \end{frame}
}



%%%%%%%%%%%%%%%%%%%
%Neue Definitionen
%%%%%%%%%%%%%%%%%%%

%Newcommands
\newcommand{\Fun}[1]{\mathcal{#1}}      %Mathcal fuer Funktoren
\newcommand{\field}[1]{\mathbb{#1}}     %Grundkoerper ?? in mathds

\newcommand{\A}{\field{A}}              %Affines A
\newcommand{\C}{\field{C}}              %Complexes C
\newcommand{\Fp}{\field{F}_{\!p}}       %Endlicher Koerper mit p Elementen
\newcommand{\Fq}{\field{F}_{\!q}}       %Endlicher Koerper mit q Elementen
\newcommand{\Ga}{\field{G}_{a}}         %Add Gruppenschema
\newcommand{\K}{\field{K}}              %Generischer Koerper 
\newcommand{\N}{\field{N}}              %Nat Zahlen
\newcommand{\Pj}{\field{P}}             %Projektives P
\newcommand{\R}{\field{R}} 		%Reelle Zahlen
\newcommand{\Q}{\field{Q}}              %Rationale Zahlen  
\newcommand{\Qt}{\field{H}}             %Quaternionen 
\newcommand{\V}{\field{V}}              %Vektorbuendel V
\newcommand{\Z}{\field{Z}}              %Ganze Zahlen

\newcommand{\fdg}{\;|\;}                 %fuer die gilt

%Operatoren
\DeclareMathOperator{\Abb}{Abb}
%\usepackage{sagetex}

\begin{document}
\lstset{basicstyle={\lstbasicfont\footnotesize}}


\maketitle

\begin{frame}{Aufbau}
\tableofcontents
\end{frame}

%=================================================
\section{Grundlagen von Sage}
%=================================================

% \begin{frame}[fragile]{Ein erstes Beispiel}
% \begin{sagein}
% f = x^2-3*x-18; solve(f==0,x)
% \end{sagein}
% \begin{sage}
% [x == 6, x == -3]
% \end{sage}
% \begin{sagein}
% solve(f<=0,x) ??
% \end{sagein}
% \begin{sage}
%  [-3, 6] union 
%  {I x + 3/2 | x in (-infinity, infinity)}
% \end{sage}
% \begin{sagein}
% assume(x,'real'): solve(f<=0,x) ??
% \end{sagein}
% \begin{sage}
%  [-3, 6]
%    \end{sage}
% \begin{sagein}   
% f(x=1)
% \end{sagein}
% \begin{sage}
%  -20
% \end{sage}
% \end{frame}

\begin{frame}[fragile]{Beispiel} 
Betrachte:
\begin{sagein}
f = x^2-3*x-18
\end{sagein}
\begin{itemize}
\item Wie geht Sage mit der Unbekannten $x$ um?
\item Welchen Datentyp hat $f$?
\item Was kann ich mit $f$ machen?
\end{itemize}
\end{frame}

\begin{frame}[fragile]{Bezeichner}
\begin{itemize}
\item \alert{Bezeichner} sind Namen, wie z.B. $x$ oder $f$. Sie können
im mathematischen Kontext sowohl Variablen als auch Unbestimmte repräsentieren.
\item Bezeichner sind aus Buchstaben, Ziffern und
Unterstrich \_ zusammengesetzt.
\item Sage unterscheidet zwischen Groß- und Kleinschreibung.
\item Bezeichner dürfen nicht mit einer Ziffer beginnen
\end{itemize}
\textbf{Beispiele}
\begin{itemize}
\item zulässige Bezeichner:
\isage{x}, \isage{f}, \isage{x23}, \isage{_x_1}
\item unzulässige Bezeichner:
\isage{12x}, \isage{p~}, \isage{x>y}, \isage{Das System}
\end{itemize}
\end{frame}


\begin{frame}[fragile]{Wert eines Bezeichners}
\begin{itemize}
\item Der \alert{Wert} eines Bezeichners  ist ein \alert{Objekt} eines bestimmten
\alert{Datentyps}.
\item Ein \alert{Datentyp} ist durch seine Eigenschaften gegeben. \\
\textbf{Beispiel}: Natürliche Zahlen, rationale Zahlen, Bezeichner, Zeichenketten, \ldots  
\item Ein \alert{Objekt} ist eine Instanz (Einheit) eines Datentyps.
\end{itemize}
\end{frame}

\begin{frame}[fragile]{Zuweisungsoperator $=$}
\begin{itemize}
\item Die Operation {\color{blue} \isage{bez=wert}} weist dem Bezeichner
\isage{bez} den Wert \isage{wert} zu.  
\item {\color{blue} \isage{func(arg)=expr(arg)}} definiert die Funktion \isage{func} mit dem Argument \isage{arg} und weist dieser den Ausdruck \isage{expr} zu, der von \isage{arg} abhängen sollte
%\item Rückgabeparameter ist die rechte Seite (Eine Ausgabe erfolgt jedoch normalerweise nicht)
\item Warnung: Unterscheiden Sie  stets zwischen dem Zuweisungsoperator {\color{blue} $=$} und
dem logischen Operator {\color{blue} $==$}.   
\item Löschen von Zuweisungen/Variablen: {\color{blue} \isage{reset('bezeichner')}}
\end{itemize}
\end{frame}

\begin{frame}[fragile]{Beispiele: Zuweisung}
\begin{sagein}
N=6; N
\end{sagein}
\begin{sage}
  6
\end{sage}
\begin{sagein}
x,y = var('x,y'); f = x+2*x*x-y; g(x) = x^2; f,g
\end{sagein}
\begin{sage}
(2*x^2 + x - y, x |--> x^2)
\end{sage}
\begin{sagein}
x=pi;y = cos(x); x,y
\end{sagein}
\begin{sage}
  (pi, -1)
\end{sage}
\end{frame}

\begin{frame}[fragile]{Beispiele: Auswertung}
\begin{sagein}
var('a') ; f(x) = x*x-3*x-a
\end{sagein}
\begin{sage}
x |--> x^2 - a - 3*x
\end{sage}
\begin{sagein}
f(a=2)
\end{sagein}
\begin{sage}
 x^2 - 3*x - 2  2
\end{sage}
\begin{sagein}
f(1)
\end{sagein}
\begin{sage}
  -a - 2
\end{sage}
\begin{sagein}
f(1,a=2)
\end{sagein}
\begin{sage}
  -4
\end{sage}
\end{frame}

\begin{frame}{Auswertung}
\begin{itemize}
\item Der {\color{red} \it Bezeichner} ist der Name einer Unbekannten.
\item Die  {\color{red} \it Auswertung} eines Bezeichners erfolgt ohne die Benutzung von bekannten Zuweisungen.
\item Der {\color{red} \it Wert} bezeichnet die Auswertung zum 
Zeitpunkt der Zuweisung. 
\end{itemize}
\end{frame}


\begin{frame}[fragile]{Beispiele für Datentypen}

\begin{sagein}
type(5)
\end{sagein}
\begin{sage}
  <type 'sage.rings.integer.Integer'>
\end{sage}
\begin{sagein}
f = x^2-3*x-18; type(f)
\end{sagein}
\begin{sage}
  <type 'sage.symbolic.expression.Expression'>
\end{sage}
\begin{sagein}
type(x)
\end{sagein}
\begin{sage}
  <type 'sage.symbolic.expression.Expression'>
\end{sage}
\begin{sagein}
f+f
\end{sagein}
\begin{sage}
2*x^2 - 6*x - 36
\end{sage}
\end{frame}

\begin{frame}[fragile]{Einige Datentypen}
\begin{center}
\begin{tabular}{|lll|}
\hline
Typ & Bedeutung & Beispiel\\
\hline
\isage{integer} & ganze Zahlen & \isage{-3,0,100}\\
\isage{rational} & rationale Zahlen & \isage{7/11}\\
\isage{float} & Gleitpunktzahl & \isage{0.123}\\
\isage{complex} & komplexe Zahlen & \isage{complex(1,3)}\\
\isage{expression} & symbolische Ausdrücke & \isage{x+y} \\
\isage{bool} & logische Werte: true/false& \isage{bool(1<2)} \\
\hline
\end{tabular}
\end{center}
\end{frame}


%\begin{frame}{Auswertung}
%\begin{itemize}
%\item Auf interaktiver Ebene wertet MuPAD in der Regel vollständig
%aus. 
%\item Maximale Auswertungstiefe wird durch die Konstanten {\color{blue} LEVEL}
%und {\color{blue} MAXLEVEL} gesteuert.\\
%(Default: {\color{blue} LEVEL}= 100, {\color{blue} MAXLEVEL}=100)
%\item Zuerst wird {\color{blue} MAXLEVEL} geprüft; erst dann {\color{blue} LEVEL}. 
%\item Ist {\color{blue} MAXLEVEL} erreicht, wird eine Fehlermeldung zurückgegeben.
%\item Die Auswertung wird bei {\color{blue} LEVEL} gestoppt, d.h. es wird keine Fehlermeldung zurückgegebnen.
%\end{itemize}
%\end{frame}

%\begin{frame}[fragile]{Auswertung}
%\begin{itemize}
%\item Bei Aufrufen {\color{blue} \%n} wird nicht ausgewertet. Ähnliches gilt
%bei Einträgen von Matrizen und Tabellen.
%\item Im Zusammenhang mit dem $\$$-Operator wird nicht ausgewertet. 
%\item Auswertungen können durch {\color{blue} eval} erzwungen werden.
%\item Auswertungen können durch {\color{blue} hold} unterbunden werden.
%\item Die Auswertungstiefe eines Bezeichners $a$ kann gezielt durch den Befehl
%  {\color{blue} \isage{level(a,n)}} gesteuert werden ($n$ Auswertungstiefe).
%\end{itemize}
%\end{frame}

% \begin{frame}[fragile]{Beispiele I}
% \begin{sagein}
% a:=sin(b)
% \end{sagein}
% \begin{sage}
%   sin(b)
% \end{sage}
% \begin{sagein}
% b:=0
% \end{sagein}
% \begin{sage}
%   0
% \end{sage}
% \begin{sagein}
% a
% \end{sagein}
% \begin{sage}
%   0
% \end{sage}
% \end{frame}
% 
% \begin{frame}[fragile]{Beispiele II}
% \begin{sagein}
% %3,eval(%3)
% \end{sagein}
% \begin{sage}
%   sin(b), 0
% \end{sage}
% \begin{sagein}
% a,hold(a)
% \end{sagein}
% \begin{sage}
%   0, a
% \end{sage} 
% \end{frame}

\begin{frame}{Operatoren}
\begin{itemize}
\item Typische Operatoren sind \verb~+,-,*,/,..~
\item In Sage werden Objekte immer durch Funktionen miteinander
verbunden. 
%\item Operatoren sind in Sage auch als Funktionen realisiert.
%\item Für wichtige Operatoren gibt es die gewohnte
%Kurzschreibweise.
\item Bei Kombination verschiedener Operatoren gelten die üblichen
Regeln der Bindungsstärke (Punktrechnung vor Strichrechnung); Die Ordnung kann
durch Klammersetzung geändert werden.
\end{itemize}
\end{frame}

\begin{frame}[fragile]{Wichtige mathematische Operatoren}
\begin{center}
\begin{tabular}{|c|l|}
\hline
Operator/Funktion &  Erklärung\\
\hline
\hline
\verb!+! & Addition \\
\verb!-! & Subtraktion\\
\verb!*! & Multiplikation \\
\verb!/! & Division\\
\verb!^! & Potenz\\
\verb!%! &  Rest bei Division\\
\verb!factorial()! & Fakultät \\
\hline
\end{tabular}
\end{center}
\end{frame}

\begin{frame}[fragile]{Zerlegen von Ausdrücken}
\begin{itemize}
\item Viele Ausdrücke sind zusammengesetzt. Ihre Bausteine heißen {\color{red}
Operanden.}
\item Durch {\color{blue} \isage{Ausdruck.nops()}} erhält man die Anzahl der
Operanden. 
\item Durch {\color{blue} \isage{Ausdruck.operands()}} erhält man alle 
Operanden
\item Mittels {\color{blue} \isage{Ausdruck.has(a)}} kann untersucht werden, ob
$a$ ein Operand vom \isage{Ausdruck} ist.
\item Die Befehle beziehen sich jeweils auf die automatisch
vereinfachten Objekte.
\end{itemize}
\end{frame}

\begin{frame}[fragile]{Beispiele I ???}
\begin{sagein}
f:=_plus(a,b,c)
\end{sagein}
\begin{sage}
  a + b + c
\end{sage}
\begin{sagein}
nops(f), op(f), op(f,2)
\end{sagein}
\begin{sage}
  3, a, b, c, b
\end{sage}
\begin{sagein}
op(f,0)
\end{sagein}
\begin{sage}
  _plus
\end{sage}
\begin{sagein}
has(f,a), has(f,a+b)
\end{sagein}
\begin{sage}
  TRUE, FALSE
\end{sage}
\end{frame}

\begin{frame}[fragile]{Beispiele II}
\begin{sagein}
_=var('z,y');f = x*z+3*x+sqrt(y)
f.operands(),(f.operands())[1], ((f.operands())[1]).nops()
\end{sagein}
\begin{sage}
([x*z, 3*x, sqrt(y)], 3*x, 2)
\end{sage}
\end{frame}

python grundlagen (lists, sets, for, while, if)

% \begin{frame}[fragile]{Darstellungsbaum}
% \begin{sagein}
% a = numerical_integral(exp(x^4),0,1)
% \end{sagein}
% \begin{center}
% \includegraphics[width=8cm, height=6cm]{figures/baum.png}
% \end{center}
% \end{frame}

% \begin{frame}[fragile]{Beispiel}
% so unsinnig..
% \begin{sagein}
% a:=int(exp(x^4),x=0..1):
% op(a)
% \end{sagein}
% \begin{sage}
%        4
%   exp(x ), x = 0..1
% \end{sage}
% \begin{sagein}
% op(a,0)
% \end{sagein}
% \begin{sage}
%   int
% \end{sage}
% \begin{sagein}
% op(op(a,1))
% \end{sagein}
% \begin{sage}
%    4
%   x
% \end{sage}
% \end{frame}

\begin{frame}[fragile]{Automatische Vereinfachung}
Sage führt oft automatische Vereinfachungen durch. Ansonsten muß
der Benutzer gezielt Vereinfachungen anfordern.

\begin{sagein}
sin(15*pi), exp(0)
\end{sagein}
\begin{sage}
  (0, 1)
\end{sage}
\begin{sagein}
2*Infinity-5
\end{sagein}
\begin{sage}
  +Infinity
\end{sage}
\begin{sagein}
y = (-4*x+x^2+4)*(7*x+x^2+12); y
\end{sagein}
\begin{sage}
 (x^2 - 4*x + 4)*(x^2 + 7*x + 12)
\end{sage}
\begin{sagein}
y.full_simplify()
\end{sagein}
\begin{sage}
x^4 + 3*x^3 - 12*x^2 - 20*x + 48
\end{sage}

\end{frame} 

%=================================================
\section{Symbolisches Rechnen I}
%=================================================

% \begin{frame}{Manipulation von Ausdrücken}
% \begin{itemize}
% \item Verbinden von Ausdrücken
% \item Vereinfachen
% \item Umformen
% \item Einsetzen der Unbekannten
% \end{itemize}
% \end{frame} 

\begin{frame}[fragile]{Verbinden von Ausdrücken}
Ausdrücke können beliebig addiert, subtrahiert, multipliziert und
dividiert werden. 
\begin{itemize}
\item Definition
\begin{sagein}
var('x,y'); f = x*x+3*x+y; g = x-y
\end{sagein}
\item Potenz
\begin{sagein}
f^g
\end{sagein}
\begin{sage}
   (x^2 + 3*x + y)^(x - y)
\end{sage}
\end{itemize}
\end{frame}

\begin{frame}[fragile]{Verbinden von Ausdrücken II}
\begin{itemize}
\item Addition / Subtraktion
\begin{sagein}
f+g, f-g
\end{sagein}
\begin{sage}
   (x^2 + 4*x, x^2 + 2*x + 2*y)
\end{sage}
\item Multiplikation/ Division
\begin{sagein}
f*g, f/g
\end{sagein}
\begin{sage}
    ((x - y)*(x^2 + 3*x + y), (x^2 + 3*x + y)/(x - y))
\end{sage}
\end{itemize}
\end{frame}

\begin{frame}[fragile]{collect()}
Durch {\color{blue} \isage{a.collect(Unbestimmte)}} wird der Ausdruck
\isage{a} bzgl. der \isage{Unbestimmten} sortiert.
\begin{sagein}
f = a*x^2+a*x+x^3+sin(x)+b*x+4*x+x*sin(x):
f.collect(x)
\end{sagein}
\begin{sage}
   a*x^2 + x^3 + (a + b + sin(x) + 4)*x + sin(x)
\end{sage}
\begin{sagein}
f.collect(x*sin(x))
\end{sagein}
\begin{sage}
  a*x^2 + x^3 + a*x + b*x + x*sin(x) + 4*x + sin(x)
\end{sage} 
Durch {\color{blue} \isage{a.combine()}} wird der Ausdruck
durch die Potenzgesetze zusammengefaßt.
\begin{sagein}
g = x^(a)*x^(b)
g.combine()
\end{sagein}
\begin{sage}
    x^(a + b)
\end{sage}

\end{frame}

\begin{frame}[fragile]{expand()}
Ausmultiplizieren von  Ausdrücken erfolgt durch {\color{blue}
\isage{a.expand()}} und {\color{blue} \isage{a.expand_trig()}}. \\
\begin{sagein}
expand((x+2)^4)
\end{sagein}
\begin{sage}
 x^4 + 8*x^3 + 24*x^2 + 32*x + 16
\end{sage}
\begin{sagein}
(sin(x+y)).expand_trig()
\end{sagein}
\begin{sage}
  sin(x)*cos(y) + sin(y)*cos(x)
\end{sage}
\end{frame}

\begin{frame}[fragile]{expand() bei Gleichungen}
\begin{sagein}
a = (16*x-13)^2 == (3*x+5)^2/2
a.expand()
\end{sagein}
\begin{sage}
  256*x^2 - 416*x + 169 == 9/2*x^2 + 15*x + 25/2
\end{sage}
\begin{sagein}
a.expand('left')
\end{sagein}
\begin{sage}
  256*x^2 - 416*x + 169 == 1/2*(3*x + 5)^2
\end{sage}
\begin{sagein}
a.expand('right')
\end{sagein}
\begin{sage}
  (16*x - 13)^2 == 9/2*x^2 + 15*x + 25/2
\end{sage}
\end{frame}

\begin{frame}[fragile]{factor()}
Der Befehl {\color{blue} \isage{factor(Ausdruck)}} faktorisiert Polynome und
Ausdrücke. 
\begin{itemize}
\item Sage faktorisiert nur, wenn die resultierenden Koeffizienten rationale 
Zahlen sind. 
\item Auch anwendbar auf rationale Funktionen. Es wird ein gemeinsamer
Hauptnenner gesucht.
\end{itemize}
\begin{sagein}
factor(x^2-2), factor(x^2-9/4)
\end{sagein}
\begin{sage}
(x^2 - 2, 1/4*(2*x - 3)*(2*x + 3))
\end{sage}
\begin{sagein}
factor(2 - 2/(x^2-1))
\end{sagein}
\begin{sage}
2*(x^2 - 2)/((x - 1)*(x + 1))
\end{sage}
\end{frame}

\begin{frame}[fragile]{partial\_fraction()}
Durch {\color{blue} \isage{a.partial_fraction()}} wird ein rationaler Ausdruck in eine
Summe rationaler Terme zerlegt, in denen jeweils der Zählergrad
kleiner als der Nennergrad ist. (Partialbruchzerlegung)
\begin{sagein}
f = x^2/(x^2-1); f.partial_fraction()
\end{sagein}
\begin{sage}
 1/2/(x - 1) - 1/2/(x + 1) + 1
\end{sage}
\begin{sage}
f = (x^2+2*x+3)/(x^3+4*x^2+5*x+2); f 
\end{sage}
\begin{sage}
 (x^2 + 2*x + 3)/(x^3 + 4*x^2 + 5*x + 2)
\end{sage}

\end{frame}


% \begin{frame}[fragile]{simplify}
% Durch {\color{blue} \isage{rewrite(Ausdruck, Option)}} wird versucht, den
% \isage{Ausdruck} so umzuformen, das gewisse Funktionen aus dem Ausdruck
% eliminiert werden. 
% \begin{itemize}
% \item Beispielsweise können \isage{sin} und \isage{cos} immer
% durch \isage{tan} ausgedrückt werden (Option: \isage{tan}).\\
% \item Optionen sind \isage{diff}, \isage{exp}, \isage{fact}, \isage{gamma}, 
% \isage{heavyside}, \isage{ln}, \isage{sign},
% \isage{sincos}, \isage{sinhcosh}, \isage{tan}.\\
% \item Man versucht die Ausdrücke mit Hilfe der in der Option genannten
% Funktion(en) auszudrücken.
% \end{itemize}
% \end{frame}

\begin{frame}[fragile]{Simplify}
\begin{itemize}
\item Durch {\color{blue} \isage{simplify_<target>(f)}} wird versucht den Ausdruck $f$
zu vereinfachen. \isage{target} entspricht verschiedenen Vereinfachungen.
\item Mögliche \isage{target} sind \isage{trig}, \isage{rational},\isage{radical}, \isage{factorial}, \isage{full}
\end{itemize}
\begin{sagein}
(2 - 2/(x^2-1)).simplify_rational()
\end{sagein}
\begin{sage}
 2*(x^2 - 2)/(x^2 - 1)
\end{sage}
\end{frame}

\begin{frame}[fragile]{Beispiele - Simplify I}
\begin{sagein}
f = x/(x+y)+y/(x+y)-sin(x)^2-cos(x)^2
f.simplify()
\end{sagein}
\begin{sage}
-sin(x)^2 - cos(x)^2 + x/(x + y) + y/(x + y)
\end{sage}
\begin{sagein}
g = sqrt(997)-(997^3)^(1/6)
g.simplify()
\end{sagein}
\begin{sage}
0
\end{sage}
\end{frame}


\begin{frame}[fragile]{Beispiele - Simplify II}
\begin{sagein}
(tan(x)).simplify_trig()
\end{sagein}
\begin{sage}
sin(x)/cos(x)
\end{sage}
\begin{sagein}
a = (2^(1/3)+4^(1/3))^3-6*(2^(1/3) + 4^(1/3))-6
a.simplify_full() 
\end{sagein}
\begin{sage}
 0
\end{sage}


\end{frame}


%=================================================
\section{Gleichungen}
%=================================================

\begin{frame}[fragile]{Gleichungen}
\begin{itemize}
\item lineares Beispiel
\begin{sagein}
var('x,y')
Gleichungen = [x+y == 1, x-y == 1]
solve(Gleichungen,x,y)
\end{sagein}
\begin{sage}
  [[x == 1, y == 0]]
\end{sage}
\item nichtlineares Beispiel
\begin{sagein}
Gleichungen1 = [x+y == 1,(x-y)^2 == 1]
solve(Gleichungen1,x,y)
\end{sagein}
\begin{sage}
  [[x == 0, y == 1], [x == 1, y == 0]]
\end{sage}
\end{itemize}
\end{frame}

\begin{frame}[fragile]{Vergleiche}
\begin{itemize}
\item Der Operator {\color{blue} \isage{==}} vergleicht zwei Objekte. 
\item {\color{blue}
\isage{a==b}} ist wahr (richtig), wenn $a$ und $b$ die gleichen Auswertungen
besitzen (und vom gleichen Typ sind). 
\item Zur Überprüfung von Aussagen gibt es die Funktion
{\color{blue} \isage{bool(Ausdruck)}}. Sie liefert als Ergebnis \isage{True} oder
\isage{False}.
\item Die inverse Operation zu {\color{blue} \isage{'=='}} ist {\color{blue}
\isage{'<>'}}, also \isage{a<>b} ist \isage{True}, falls $a$ nicht gleich $b$
ist. 
\end{itemize}
\end{frame}

\begin{frame}[fragile]{Beispiele - Vergleiche I}
\begin{sagein}
bool(4-3==1)
\end{sagein}
\begin{sage}
  True
\end{sage}
\begin{sagein}
bool(4*x==x); x=0; bool(4*x==x)
\end{sagein}
\begin{sage}
 False
 True
\end{sage}
\begin{sagein}
bool(x==0); bool(x<>0)
\end{sagein}
\begin{sage}
 True
 False
\end{sage}
\end{frame}

\begin{frame}[fragile]{Beispiele - Vergleiche II}
\begin{sagein}
bool(0.5==1/2)
\end{sagein}
\begin{sage}
  True
\end{sage}
\begin{sage}
??
\end{sage}
\begin{sage}
??
\end{sage}
\end{frame}

\begin{frame}[fragile]{Lösen von Gleichungssystemen}
\begin{itemize}
\item \isage{solve} ist der Befehl zum Lösen von Gleichungen und Gleichungssystemen.
\item Der Befehl ist von der Form {\color{blue} \isage{solve(Gleichungen,Variablen,solution_dict)}}. 
\item {\color{blue}\isage{Gleichungen}} kann ein System von Gleichungen sein.
\item  {\color{blue}\isage{Variablen}} gibt an, wonach aufgelöst wird.
\item Bei einzelnen Gleichungen wird der Lösungswert
zurückgegeben. Bei mehreren Gleichungen wird ein System äquivalenter
Gleichungen zurückgegeben.
\item Mit \isage{multiplicities=True} erhält man alle möglichen Lösungen.
\item {\color{blue}\isage{solution_dict=true}} gibt die Lösung als Dictonary zurück (Dazu später mehr)
\end{itemize}
\end{frame}

\begin{frame}[fragile]{Beispiele - Solve I} 
\begin{sagein}
solve(x^2+x == y/4,x)
\end{sagein}
\begin{sage}
[x == -1/2*sqrt(y + 1) - 1/2, x == 1/2*sqrt(y + 1) - 1/2]
\end{sage}
\begin{sagein}
solve(f == 0, x)
\end{sagein}
\begin{sage}
[x == -I, x == I, x == 1]
\end{sage}
\begin{sagein}
solve(f == 0, x, multiplicities=True)
\end{sagein}
\begin{sage}
([x == -I, x == I, x == 1], [1, 1, 5])
\end{sage}
\end{frame}

\begin{frame}[fragile]{Beispiele - Solve II} 
\begin{sagein}
assume(x>0); solve(x^2+x == y/4,y)
\end{sagein}
\begin{sage}
 [y == 4*x^2 + 4*x]
\end{sage}
\begin{sagein}
solve([x^2-y^2 == 0],[x,y])
\end{sagein}
\begin{sage}
([x == -y, x == y], [1, 1])
\end{sage}
\begin{sagein}
solve([x^2-y^2 == 0, x+y == 1],x,y)
\end{sagein}
\begin{sage}     
  {[x = 1/2, y = 1/2]}
\end{sage}
\end{frame}

\begin{frame}[fragile]{Numerisches Lösen von Gleichungssystemen}
\begin{sage}
 (x == sin(x)).find_root(-2,2)
\end{sage}
\begin{sage}
0.0
\end{sage}
\end{frame}

\end{document}





















