\documentclass[notes=hide,hyperref={dvipdfmx,pdfpagelabels=false}]{beamer}
\mode<article>
{
  \usepackage{fullpage}
  \usepackage{pgf}
  \usepackage[xetex]{hyperref}
  \setjobnamebeamerversion{beamer}
}

\mode<presentation>
{
  %\usetheme{Frankfurt}
 %\usetheme{My}
  \usetheme{Madrid}
  % or ...
%\usecolortheme{seagull}
  %\setbeamercovered{transparent}
  %\setbeamercovered{dynamic}
  % or whatever (possibly just delete it)
}
\usenavigationsymbolstemplate{}
\usefonttheme{structurebold}
\usepackage{multimedia}
%\usepackage{tikz}
\usepackage{fontspec,xunicode,xltxtra}

%\usepackage{polyglossia}
%\setdefaultlanguage[spelling=new, latesthyphen=true]{german}
%\setsansfont{DejaVu Sans}
%\setsansfont{Verdana}
%\setsansfont{Arial}
%\setromanfont{Linux Libertine O}
%\setsansfont{Linux Biolinum O}

\setbeamertemplate{footline}
{
\leavevmode
%\hbox{\begin{beamercolorbox}[wd=.5\paperwidth,ht=2.5ex,dp=1.125ex,
%leftskip=.3cm plus1fill,rightskip=.3cm]{author in head/foot}%
%    \usebeamerfont{author in head/foot}\insertshortauthor
%  \end{beamercolorbox}%
%  \begin{beamercolorbox}[wd=.5\paperwidth,ht=2.5ex,dp=1.125ex,leftskip=.3cm,
%rightskip=.3cm plus1fil]{title in head/foot}%
%    \usebeamerfont{title in head/foot}\insertshorttitle\hfill

\hfill\insertframenumber  \hspace{3pt}

%\inserttotalframenumber
%\hspace*{2ex}
%  \end{beamercolorbox}}%
  \vskip3pt%
}

\usepackage[ngerman]{babel}
\selectlanguage{ngerman}

%
% math/symbols
%
\usepackage{amssymb}
\usepackage{amsthm}
% \usepackage{latexsym}
\usepackage{amsmath}
%\usepackage{amsxtra} %Weitere Extrasymbole
%\usepackage{empheq} %Gleichungen hervorheben
%\usepackage{bm}
 %\bm{A} Boldface im Mathemodus

\usepackage{cellspace}
\setlength{\cellspacetoplimit}{2pt}
\setlength{\cellspacebottomlimit}{2pt}

%%%%%%%%%%%%%%%%%% Fuer Frames [fragile]-Option verwenden!
%Programm-Listing
%%%%%%%%%%%%%%%%%%
%Listingsumgebung fuer verbatim
%Grauhinterlegeter Text
%Automatischer Zeilenumbruch ist aktiviert
\usepackage{listings}
\definecolor{lgray}{gray}{0.80}
%\lstset{backgroundcolor=\color{lgray}, frame=single, basicstyle=\ttfamily, breaklines=true}
\lstnewenvironment{sage}{\lstset{backgroundcolor=\color{lgray},language=Python, emphstyle=\color{red}, frame=single, basicstyle=\ttfamily, breaklines=true,mathescape =true,escapechar=§}}{}


\usepackage{mydef}
%\usepackage{cmap} % you can search in the pdf for umlauts and ligatures
\usepackage{colonequals} %corrects the definition-symbols \colonequals (besides others)
\title{Einführung in Sage}
%
%\subtitle{Disputation} % (optional)

\author{Jochen Schulz}
% - Use the \inst{?} command only if the authors have different
%   affiliation.

\institute{Georg-August Universit\"at G\"ottingen \pgfimage[height=0.5cm]{../figures/unilogo3}}
% - Use the \inst command only if there are several affiliations.
% - Keep it simple, no one is interested in your street address.

\date{\today}

\subject{Sage}
% This is only inserted into the PDF information catalog. Can be left
% out. 

% If you have a file called "university-logo-filename.xxx", where xxx
% is a graphic format that can be processed by latex or pdflatex,
% resp., then you can add a logo as follows:

%\logo{\pgfimage[height=0.5cm]{figures/unilogo3}}


% Delete this, if you do not want the table of contents to pop up at
% the beginning of each subsection:

\AtBeginSection[]
{
  \begin{frame}<beamer>
    \frametitle{Aufbau}
    \tableofcontents[currentsection,currentsubsection]
  \end{frame}
}

\AtBeginSubsection[]
{
  \begin{frame}<beamer>
    \frametitle{Aufbau}
    \tableofcontents[currentsection,currentsubsection]
  \end{frame}
}



%%%%%%%%%%%%%%%%%%%
%Neue Definitionen
%%%%%%%%%%%%%%%%%%%

%Newcommands
\newcommand{\Fun}[1]{\mathcal{#1}}      %Mathcal fuer Funktoren
\newcommand{\field}[1]{\mathbb{#1}}     %Grundkoerper ?? in mathds

\newcommand{\A}{\field{A}}              %Affines A
\newcommand{\C}{\field{C}}              %Complexes C
\newcommand{\Fp}{\field{F}_{\!p}}       %Endlicher Koerper mit p Elementen
\newcommand{\Fq}{\field{F}_{\!q}}       %Endlicher Koerper mit q Elementen
\newcommand{\Ga}{\field{G}_{a}}         %Add Gruppenschema
\newcommand{\K}{\field{K}}              %Generischer Koerper 
\newcommand{\N}{\field{N}}              %Nat Zahlen
\newcommand{\Pj}{\field{P}}             %Projektives P
\newcommand{\R}{\field{R}} 		%Reelle Zahlen
\newcommand{\Q}{\field{Q}}              %Rationale Zahlen  
\newcommand{\Qt}{\field{H}}             %Quaternionen 
\newcommand{\V}{\field{V}}              %Vektorbuendel V
\newcommand{\Z}{\field{Z}}              %Ganze Zahlen

\newcommand{\fdg}{\;|\;}                 %fuer die gilt

%Operatoren
\DeclareMathOperator{\Abb}{Abb}
%\usepackage{sagetex}

\begin{document}
\lstset{basicstyle={\lstbasicfont\footnotesize}}


\subtitle{Einheit 2}
\maketitle

\begin{frame}{Aufbau}
\tableofcontents
\end{frame}

%\begin{frame}{Übersicht}
%\begin{enumerate}
%\item Bezeichner, Zuweisungen, Datentypen und Objekte
%\item Zerlegen von Objekten in seine Bestandteile
%\item Symbolisches Rechnen
%\item Gleichungen
%\end{enumerate}
%\end{frame}

%=================================================
\section{Grundlagen von Sage}
%=================================================

\begin{frame}[fragile]{Ein erstes Beispiel}
\begin{sage}
>> f = x^2-3*x-18; solve(f==0,x)
\end{sage}
\begin{sage}
[x == 6, x == -3]
\end{sage}
\begin{sage}
>> solve(f<=0,x) ??
\end{sage}
\begin{sage}
 [-3, 6] union 
 {I x + 3/2 | x in (-infinity, infinity)}
\end{sage}
\begin{sage}
>> assume(x,Type::Real): solve(f<=0,x) ??
\end{sage}
\begin{sage}
 [-3, 6]
   \end{sage}
\begin{sage}   
>> x:=1: f  ??
\end{sage}
\begin{sage}
 -20
\end{sage}
\end{frame}

\begin{frame}[fragile]{Beispiel} 
Betrachte:
\begin{sage}
>> f = x^2-3*x-18
\end{sage}
\begin{itemize}
\item Wie geht Sage mit der Unbekannten $x$ um?
\item Welchen Datentyp hat $f$?
\item Was kann ich mit $f$ machen?
\end{itemize}
\end{frame}

\begin{frame}[fragile]{Bezeichner}
\begin{itemize}
\item \alert{Bezeichner} sind Namen, wie z.B. $x$ oder $f$. Sie können
im mathematischen Kontext sowohl Variablen als auch Unbestimmte repräsentieren.
\item Bezeichner sind aus Buchstaben, Ziffern und
Unterstrich \_ zusammengesetzt.
\item Sage unterscheidet zwischen Groß- und Kleinschreibung.
\item Bezeichner dürfen nicht mit einer Ziffer beginnen
\end{itemize}
Beispiele für Bezeichner
\begin{itemize}
\item zulässige Bezeichner:
\verb+x+, \verb+f+, \verb+x23+, \verb+_x_1+
\item unzulässige Bezeichner:
\verb+12x+, \verb+p~+, \verb+x>y+, \verb+Das System+
\end{itemize}
\end{frame}


\begin{frame}[fragile]{Wert eines Bezeichners}
\begin{itemize}
\item Der \alert{Wert} eines Bezeichners  ist ein \alert{Objekt} eines bestimmten
\alert{Datentyps}.
\item Ein \alert{Datentyp} ist durch seine Eigenschaften gegeben. \\
Beispiel: Natürliche Zahlen, rationale Zahlen, Bezeichner, Zeichenketten, \ldots  
\item Ein \alert{Objekt} ist eine Instanz (Einheit) eines Datentyps.
\end{itemize}
\end{frame}

\begin{frame}[fragile]{Zuweisungsoperator $:=$}
\begin{itemize}
\item Die Operation {\color{blue} \verb+bez=wert+} weist dem Bezeichner
\verb+bez+ den Wert \verb+wert+ zu.  
\item Beispiele: {\color{blue} \verb+N=5+}, {\color{blue} \verb+f = x^2-3*x-18+}
\item Rückgabeparameter ist die rechte Seite (Eine Ausgabe erfolgt jedoch normalerweise nicht)
\item Warnung: Unterscheiden Sie  stets zwischen dem Zuweisungsoperator {\color{blue} $=$} und
dem logischen Operator {\color{blue} $==$}.   
\end{itemize}
\end{frame}

\begin{frame}[fragile]{Beispiele}
\begin{sage}
>> N=6; N
\end{sage}
\begin{sage}
  6
\end{sage}
\begin{sage}
>> x,y = var('x,y'); f = x+2*x*x-y; f
\end{sage}
\begin{sage}
    2*x^2 + x - y
\end{sage}
\begin{sage}
>> x=pi;y = cos(x); x,y
\end{sage}
\begin{sage}
  (pi, -1)
\end{sage}
\end{frame}

\begin{frame}[fragile]{Beispiele für Datentypen}

\begin{sage}
>> type(5)
\end{sage}
\begin{sage}
  <type 'sage.rings.integer.Integer'>
\end{sage}
\begin{sage}
>> f= x^2-3*x-18; type(f)
\end{sage}
\begin{sage}
  <type 'sage.symbolic.expression.Expression'>
\end{sage}
\begin{sage}
>> type(x)
\end{sage}
\begin{sage}
  <type 'sage.symbolic.expression.Expression'>
\end{sage}
\begin{sage}
>> f+f
\end{sage}
\begin{sage}
2*x^2 - 6*x - 36
\end{sage}
\end{frame}

\begin{frame}[fragile]{Einige Datentypen}
\begin{center}
\begin{tabular}{|lll|}
\hline
Domain-Typ & Bedeutung & Beispiel\\
\hline
\verb+rings.integer+ & ganze Zahlen & \verb+-3,0,100+\\
\verb+rings.rational+ & rationale Zahlen & \verb+7/11+\\
\verb+float+ & Gleitpunktzahl & \verb+0.123+\\
\verb+complex+ & komplexe Zahlen & \verb+complex(1,3)+\\
\verb+symbolic.expression+ & symbolische Ausdrücke & \verb!x+y!\\
\verb+bool+ & logische Werte: true/false&\verb+bool(1<2)+\\
\hline
\end{tabular}
\end{center}
\end{frame}

\begin{frame}[fragile]{Befehle im Umgang mit $=$}
\begin{itemize}
\item Löschen von Zuweisungen: {\color{blue} \verb+reset('bezeichner')+}
\end{itemize}
\end{frame}

\begin{frame}[fragile]{Beispiel: Auswertung}
\begin{sage}
>> var('a') ; f(x) = x*x-3*x-a
\end{sage}
\begin{sage}
x |--> x^2 - a - 3*x
\end{sage}
\begin{sage}
>> f(a=2)
\end{sage}
\begin{sage}
 x^2 - 3*x - 2  2
\end{sage}
\begin{sage}
>> f(1)
\end{sage}
\begin{sage}
  -a - 2
\end{sage}
\begin{sage}
>> f(1,a=2)
\end{sage}
\begin{sage}
  -4
\end{sage}
\end{frame}

\begin{frame}{Auswertung}
\begin{itemize}
\item Der {\color{red} \it Bezeichner} ist der Name einer Unbekannten.
\item Die  {\color{red} \it Auswertung} eines Bezeichners erfolgt ohne die Benutzung von bekannten Zuweisungen.
\item Der {\color{red} \it Wert} bezeichnet die Auswertung zum 
Zeitpunkt der Zuweisung. 
\end{itemize}
\end{frame}

%\begin{frame}{Auswertung}
%\begin{itemize}
%\item Auf interaktiver Ebene wertet MuPAD in der Regel vollständig
%aus. 
%\item Maximale Auswertungstiefe wird durch die Konstanten {\color{blue} LEVEL}
%und {\color{blue} MAXLEVEL} gesteuert.\\
%(Default: {\color{blue} LEVEL}= 100, {\color{blue} MAXLEVEL}=100)
%\item Zuerst wird {\color{blue} MAXLEVEL} geprüft; erst dann {\color{blue} LEVEL}. 
%\item Ist {\color{blue} MAXLEVEL} erreicht, wird eine Fehlermeldung zurückgegeben.
%\item Die Auswertung wird bei {\color{blue} LEVEL} gestoppt, d.h. es wird keine Fehlermeldung zurückgegebnen.
%\end{itemize}
%\end{frame}

%\begin{frame}[fragile]{Auswertung}
%\begin{itemize}
%\item Bei Aufrufen {\color{blue} \%n} wird nicht ausgewertet. Ähnliches gilt
%bei Einträgen von Matrizen und Tabellen.
%\item Im Zusammenhang mit dem $\$$-Operator wird nicht ausgewertet. 
%\item Auswertungen können durch {\color{blue} eval} erzwungen werden.
%\item Auswertungen können durch {\color{blue} hold} unterbunden werden.
%\item Die Auswertungstiefe eines Bezeichners $a$ kann gezielt durch den Befehl
%  {\color{blue} \verb+level(a,n)+} gesteuert werden ($n$ Auswertungstiefe).
%\end{itemize}
%\end{frame}

% \begin{frame}[fragile]{Beispiele I}
% \begin{sage}
% >> a:=sin(b)
% \end{sage}
% \begin{sage}
%   sin(b)
% \end{sage}
% \begin{sage}
% >> b:=0
% \end{sage}
% \begin{sage}
%   0
% \end{sage}
% \begin{sage}
% >> a
% \end{sage}
% \begin{sage}
%   0
% \end{sage}
% \end{frame}
% 
% \begin{frame}[fragile]{Beispiele II}
% \begin{sage}
% >> %3,eval(%3)
% \end{sage}
% \begin{sage}
%   sin(b), 0
% \end{sage}
% \begin{sage}
% >> a,hold(a)
% \end{sage}
% \begin{sage}
%   0, a
% \end{sage} 
% \end{frame}

\begin{frame}{Operatoren}
\begin{itemize}
\item Typische Operatoren sind \verb~+,-,*,/,...~
\item In Sage werden Objekte immer durch Funktionen miteinander
verbunden. 
%\item Operatoren sind in Sage auch als Funktionen realisiert.
%\item Für wichtige Operatoren gibt es die gewohnte
%Kurzschreibweise.
\item Bei Kombination verschiedener Operatoren gelten die üblichen
Regeln der Bindungsstärke (Punktrechnung vor Strichrechnung); Die Ordnung kann
durch Klammersetzung geändert werden.
\end{itemize}
\end{frame}

\begin{frame}[fragile]{Wichtige mathematische Operatoren}
\begin{center}
\begin{tabular}{|c|l|}
\hline
Operator/Funktion &  Erklärung\\
\hline
\hline
\verb!+! & Addition \\
\verb!-! & Subtraktion\\
\verb!*! & Multiplikation \\
\verb!/! & Division\\
\verb!^! & Potenz\\
\verb!factorial()! & Fakultät \\
\verb!mod()! &  Rest bei Division\\
\hline
\end{tabular}
\end{center}
\end{frame}

\begin{frame}[fragile]{Zerlegen von Objekten}
\begin{itemize}
\item Viele Objekte sind zusammengesetzt. Ihre Bausteine heißen {\color{red}
Operanden.}
%\item Durch {\color{blue} \verb+nops(Objekt)+} erhält man die Anzahl der
%Operanden. 
%\item Durch {\color{blue} \verb+op(Objekt)+} bzw. {\color{blue} \verb+op(Objekt,i)+}
%erhält man alle Operanden bzw. den $i$-ten Operand.
%\item Mittels {\color{blue} \verb+has(Objekt,a)+} kann untersucht werden, ob
%$a$ ein Operand von \verb+Objekt+ ist.
%\item Die Befehle beziehen sich jeweils auf die automatisch
%vereinfachten Objekte.
\end{itemize}
\end{frame}

% \begin{frame}[fragile]{Beispiele I}
% \begin{sage}
% >> f:=_plus(a,b,c)
% \end{sage}
% \begin{sage}
%   a + b + c
% \end{sage}
% \begin{sage}
% >> nops(f), op(f), op(f,2)
% \end{sage}
% \begin{sage}
%   3, a, b, c, b
% \end{sage}
% \begin{sage}
% >> op(f,0)
% \end{sage}
% \begin{sage}
%   _plus
% \end{sage}
% \begin{sage}
% >> has(f,a), has(f,a+b)
% \end{sage}
% \begin{sage}
%   TRUE, FALSE
% \end{sage}
% \end{frame}
% 
% \begin{frame}[fragile]{Beispiele II}
% \begin{sage}
% >> f:=x*z+3*x+sqrt(y):
% >> op(f), nops(op(f,2))
% \end{sage}
% \begin{sage}
%              1/2
%   3 x, x z, y   , 2
% \end{sage}
% \end{frame}

\begin{frame}[fragile]{Darstellungsbaum}
\begin{sage}
>> a = numerical_integral(exp(x^4),0,1)
\end{sage}
\begin{center}
\includegraphics[width=8cm, height=6cm]{figures/baum.png}
\end{center}
\end{frame}

\begin{frame}[fragile]{Beispiel}
so unsinnig..
\begin{sage}
>> a:=int(exp(x^4),x=0..1):
>> op(a)
\end{sage}
\begin{sage}
       4
  exp(x ), x = 0..1
\end{sage}
\begin{sage}
>> op(a,0)
\end{sage}
\begin{sage}
  int
\end{sage}
\begin{sage}
>> op(op(a,1))
\end{sage}
\begin{sage}
   4
  x
\end{sage}
\begin{sage}
>> prog::exprtree(a)
\end{sage}
\end{frame}

\begin{frame}[fragile]{Automatische Vereinfachung}
Sage führt oft automatische Vereinfachungen durch. Ansonsten muß
der Benutzer gezielt Vereinfachungen anfordern.

\begin{sage}
>> sin(15*pi), exp(0)
\end{sage}
\begin{sage}
  (0, 1)
\end{sage}
\begin{sage}
>> 2*Infinity-5
\end{sage}
\begin{sage}
  +Infinity
\end{sage}
\begin{sage}
>> y = (-4*x+x^2+4)*(7*x+x^2+12); y
\end{sage}
\begin{sage}
 (x^2 - 4*x + 4)*(x^2 + 7*x + 12)
\end{sage}
\begin{sage}
y.full_simplify()
\end{sage}
\begin{sage}
x^4 + 3*x^3 - 12*x^2 - 20*x + 48
\end{sage}

\end{frame} 

%=================================================
\section{Symbolisches Rechnen I}
%=================================================

\begin{frame}{Manipulation von Ausdrücken}
\begin{itemize}
\item Verbinden von Ausdrücken
\item Vereinfachen
\item Umformen
\item Einsetzen der Unbekannten
\end{itemize}
\end{frame} 

\begin{frame}[fragile]{Verbinden von Ausdrücken}
Ausdrücke können beliebig addiert, subtrahiert, multipliziert und
dividiert werden. 
\begin{itemize}
\item Definition
\begin{sage}
>> f:= x*x+3*x+y: g:=x-y:
\end{sage}
\item Potenz
\begin{sage}
>> f^g
\end{sage}
\begin{sage}
              2 x - y
  (3 x + y + x )
\end{sage}
\end{itemize}
\end{frame}

\begin{frame}[fragile]{Verbinden von Ausdrücken II}
\begin{itemize}
\item Addition / Subtraktion
\begin{sage}
>> f+g, f-g
\end{sage}
\begin{sage}
         2               2
  4 x + x , 2 x + 2 y + x
\end{sage}
\item Multiplikation/ Division
\begin{sage}
>> f*g, f/g
\end{sage}
\begin{sage}
                                     2
                      2   3 x + y + x
  (x - y) (3 x + y + x ), ------------
                             x - y
\end{sage}
\end{itemize}
\end{frame}

\begin{frame}[fragile]{collect}
Durch {\color{blue} \verb+collect(Ausdruck, Unbestimmte)+} wird der
\verb+Ausdruck+ bzgl. der \verb+Unbestimmten+ sortiert.
\begin{sage}
>> f:=a*x^2+a*x+x^3+sin(x)+b*x+4*x+x*sin(x):
>> collect(f,x)
\end{sage}
\begin{sage}
   3      2
  x  + a x  + (a + b + sin(x) + 4) x + sin(x)
\end{sage}
\begin{sage}
>> collect(f,[x,sin(x)])
\end{sage}
\begin{sage}
   3      2
  x  + a x  + x sin(x) + (a + b + 4) x + sin(x)
\end{sage} 
\end{frame}

\begin{frame}[fragile]{combine}
Durch {\color{blue} \verb+combine(Ausdruck,Option)+} wird der Ausdruck
zusammengefaßt. Dabei werden
mathematische Identitäten benutzt, die durch \verb+Option+ angegeben
werden. Optionen sind \verb+arctan+
,\verb+exp+,\verb+ln+,\verb+sincos+, 
\verb+sinhcosh+. Ohne
Angabe der Option werden nur die Potenzgesetze benutzt.  
\begin{sage}
>> g:= sin(a)*cos(b):
>> g=combine(g, sincos)
\end{sage}
\begin{sage}
                  sin(a + b)   sin(a - b)
  cos(b) sin(a) = ---------- + ----------
                      2            2
\end{sage}
\end{frame}

\begin{frame}[fragile]{expand}
Ausmultiplizieren von  Ausdrücken erfolgt durch {\color{blue}
\verb+expand(Ausdruck,f_1,f_2,..)+}. \\
\verb+f_1+, \verb+f_2+ sind
Ausdrücke, die \alert{nicht} expandiert werden sollen.
\begin{sage}
>> expand((x+2)^4)
\end{sage}
\begin{sage}
   4      3       2
  x  + 8 x  + 24 x  + 32 x + 16
\end{sage}
\begin{sage}
>> expand(sin(x+y))
\end{sage}
\begin{sage}
  cos(x) sin(y) + cos(y) sin(x)
\end{sage}
\end{frame}

\begin{frame}[fragile]{Beispiele zu expand}
\begin{sage}
>> f:=(x-y)^2+(x+y)^2: expand(f)
\end{sage}
\begin{sage}
     2      2
  2 x  + 2 y
\end{sage}
\begin{sage}
>> expand(f,x-y)
\end{sage}
\begin{sage}
   2    2          2
  x  + y  + (x - y)  + 2 x y
\end{sage}
\end{frame}

\begin{frame}[fragile]{factor}
Der Befehl {\color{blue} \verb+factor(Ausdruck)+} faktorisiert Polynome und
Ausdrücke. 
\begin{itemize}
\item MuPAD faktorisiert nur, wenn die resultierenden Koeffizienten rationale 
Zahlen sind. 
\item Auch anwendbar auf rationale Funktionen. Es wird ein gemeinsamer
Hauptnenner gesucht.
\end{itemize}
\end{frame}

\begin{frame}[fragile]{Beispiel: factor}
\begin{sage}
>> factor(x^2-2), factor(x^2-9/4)
\end{sage}
\begin{sage}
   2      (2 x - 3) (2 x + 3)
  x  - 2, -------------------
                 4
\end{sage}
\begin{sage}
>> factor(2 - 2/(x^2-1))
\end{sage}
\begin{sage}
      2
  2 (x  - 2)
  ---------------
  (x - 1) (x + 1)
\end{sage}
\end{frame}

\begin{frame}[fragile]{normal}
Durch {\color{blue} \verb+normal(f)+} wird eine 'Normalform' eines
rationalen Ausdrucks \verb+f+ erzeugt.
\begin{sage}
>> normal(2 - 2/(x^2-1))
\end{sage}
\begin{sage}
     2
  2 x  - 4
  --------
   2
  x  - 1
\end{sage}
\end{frame}

\begin{frame}[fragile]{partfrac}
Durch {\color{blue} \verb+partfrac(f)+} wird ein rationaler Ausdruck in eine
Summe rationaler Terme zerlegt, in denen jeweils der Zählergrad
kleiner als der Nennergrad ist. (Partialbruchzerlegung)
\begin{sage}
>> f:=x^2/(x^2-1): f=partfrac(f)
\end{sage}
\begin{sage}
     2
    x         1           1
  ------ = --------- - --------- + 1
   2      2 (x - 1)   2 (x + 1)
  x  - 1
\end{sage}
\end{frame}

\begin{frame}[fragile]{rewrite}
Durch {\color{blue} \verb+rewrite(Ausdruck, Option)+} wird versucht, den
\verb+Ausdruck+ so umzuformen, das gewisse Funktionen aus dem Ausdruck
eliminiert werden. 
\begin{itemize}
\item Beispielsweise können \verb+sin+ und \verb+cos+ immer
durch \verb+tan+ ausgedrückt werden (Option: \verb+tan+).\\
\item Optionen sind \verb+diff+, \verb+exp+, \verb+fact+, \verb+gamma+, 
\verb+heavyside+, \verb+ln+, \verb+sign+,
\verb+sincos+, \verb+sinhcosh+, \verb+tan+.\\
\item Man versucht die Ausdrücke mit Hilfe der in der Option genannten
Funktion(en) auszudrücken.
\end{itemize}
\end{frame}

\begin{frame}[fragile]{Beispiele - rewrite I}
\begin{sage}
>> rewrite(tan(x),sincos)
\end{sage}
\begin{sage}
  sin(x)
  ------
  cos(x)
\end{sage}
\begin{sage}
>> rewrite(tan(x),exp)
\end{sage}
\begin{sage}
              2
    I exp(I x)  - I
  - ---------------
             2
     exp(I x)  + 1
\end{sage}
\end{frame}

\begin{frame}[fragile]{Beispiele - rewrite II}
\begin{sage}
>> rewrite(tan(x),sinhcosh)
\end{sage}
\begin{sage}
  I sinh(-I x)
  ------------
  cosh(-I x)
\end{sage}
\end{frame}

\begin{frame}[fragile]{Simplify}
\begin{itemize}
\item Durch {\color{blue} \verb+simplify(f,target)+} wird versucht den Ausdruck $f$
zu vereinfachen. Optional können durch \verb+target+ spezielle
Vereinfachungen angefordert werden.
\item Mögliche \verb+targets+ sind \verb+exp+, \verb+ln+,\verb+cos+, \verb+sin+, \verb+sqrt+, \verb+logic+ und
\verb+relation+. 
\item Die Optionen \verb+logic+ und \verb+relation+ dienen
zur Vereinfachung von logischen Ausdrücken bzw. von Gleichungen und
Ungleichungen. 
\item Alternativ zu \verb+simplify(f,sqrt)+ kann auch die Funktion
\verb+radsimp+ verwendet werden.   
\end{itemize}
\end{frame}

\begin{frame}[fragile]{Beispiele: Simplify I}
\begin{sage}
>> f:=x/(x+y)+y/(x+y)-sin(x)^2-cos(x)^2:
>> f=simplify(f)
\end{sage}
\begin{sage}
     x       y           2         2
  ----- + ----- - cos(x)  - sin(x)  = 0
  x + y   x + y
\end{sage}
\end{frame}

\begin{frame}[fragile]{Beispiele: Simplify II}
\begin{sage}
>> g:= sqrt(4+2*sqrt(3)): 
\end{sage}
\begin{sage}
>> g=simplify(g,sin)
\end{sage}
\begin{sage}
 1/2   1/2     1/2    1/2   1/2     1/2
2    (3    + 2)    = 2    (3    + 2)
\end{sage}
\begin{sage}
>> g=simplify(g)
\end{sage}
\begin{sage}
    1/2   1/2     1/2    1/2         1/2
  (2    (3    + 2)    = 3    + 1) = 3    + 1
\end{sage}
\end{frame}

%=================================================
\section{Gleichungen}
%=================================================

\begin{frame}[fragile]{Gleichungen}
\begin{itemize}
\item lineares Beispiel
\begin{sage}
>> Gleichungen := {x+y = 1, x-y = 1}:
>> solve(Gleichungen)
\end{sage}
\begin{sage}
  {[x = 1, y = 0]}
\end{sage}
\item nichtlineares Beispiel
\begin{sage}
>> Gleichungen1:={x+y=1,(x-y)^2=1}:
>> solve(Gleichungen1)
\end{sage}
\begin{sage}
  {[x = 0, y = 1], [x = 1, y = 0]}
\end{sage}
\end{itemize}
\end{frame}

\begin{frame}[fragile]{Vergleiche}
\begin{itemize}
\item Der Operator {\color{blue} \verb+=+} vergleicht zwei Objekte. 
\item {\color{blue}
\verb+a=b+} ist wahr (richtig), wenn $a$ und $b$ die gleichen Auswertungen
besitzen (und vom gleichen Typ sind). 
\item Zur Überprüfung von Aussagen gibt es die Funktion
{\color{blue} \verb+bool(Ausdruck)+}. Sie liefert als Ergebnis \verb+TRUE+ oder
\verb+FALSE+.
\item Die inverse Operation zu {\color{blue} \verb+'='+} ist {\color{blue}
\verb+'<>'+}, also \verb+a<>b+ ist \verb+TRUE+, falls $a$ nicht gleich $b$
ist. 
\end{itemize}
\end{frame}

\begin{frame}[fragile]{Beispiele: Vergleiche I}
\begin{sage}
>> bool(4-3=1)
\end{sage}
\begin{sage}
  TRUE
\end{sage}
\begin{sage}
>> bool(4*x=x); x:=0: bool(4*x=x)
\end{sage}
\begin{sage}
  FALSE TRUE
\end{sage}
\begin{sage}
>> bool(x=0); bool(x<>0)
\end{sage}
\begin{sage}
  TRUE FALSE
\end{sage}
\end{frame}

\begin{frame}[fragile]{Beispiele: Vergleiche II}
\begin{sage}
>> bool(0.5=1/2)
\end{sage}
\begin{sage}
  FALSE
\end{sage}
\begin{sage}
>> domtype(1.0), domtype(1)
\end{sage}
\begin{sage}
  DOM_FLOAT, DOM_INT
\end{sage}
\end{frame}

\begin{frame}[fragile]{Solve}
\begin{itemize}
\item Solve ist der universelle Befehl zum Lösen von Gleichungen und
Ungleichungen und auch Differentialgleichungen.
\item Der Befehl ist von der Form {\color{blue} \verb+solve(Gleichungen,Variablen)+}. 
\item \verb+Gleichungen+ kann ein System von Gleichungen sein.
\item \verb+Variablen+ gibt an, wonach aufgelöst wird.
\item Bei einzelnen Gleichungen wird der Lösungswert
zurückgegeben. Bei mehreren Gleichungen wird ein System äquivalenter
Gleichungen zurückgegeben.
\item Weitere Optionen werden später erklärt.
\end{itemize}
\end{frame}

\begin{frame}[fragile]{Beispiele - Solve I} 
\begin{sage}
>> solve(x^2+x=y/4,x)
\end{sage}
\begin{sage}
  {          1/2               1/2       }
  {   (y + 1)           (y + 1)          }
  { - ---------- - 1/2, ---------- - 1/2 }
  {       2                 2            }
\end{sage}
\end{frame}

\begin{frame}[fragile]{Beispiele - Solve II} 
\begin{sage}
>> assume(x>0): solve(x^2+x=y/4,y)
\end{sage}
\begin{sage}
            2
  {4 x + 4 x }
\end{sage}
\begin{sage}
>> solve({x^2-y^2=0},{x,y})
\end{sage}
\begin{sage}
  {[x = z, y = z], [x = -z, y = z]}
\end{sage}
\begin{sage}
>> solve({x^2-y^2=0,x+y=1},{x,y})
\end{sage}
\begin{sage}     
  {[x = 1/2, y = 1/2]}
\end{sage}
\end{frame}

\end{document}





















