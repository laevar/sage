\documentclass[a4paper,10pt,DIV15]{scrartcl}
\usepackage[psamsfonts]{amssymb}
\usepackage{amsmath}
\usepackage[svgnames]{xcolor} %color definitions

\usepackage{fontspec,xunicode,xltxtra}
%\usepackage{fontspec,xunicode}
%\usepackage{polyglossia}
%\setdefaultlanguage[spelling=new, latesthyphen=true]{german}
%\setsansfont{DejaVu Sans}
%\setsansfont{Verdana}
%\setsansfont{Arial}
%\setromanfont[Mapping=tex-text]{Linux Libertine}
%\setsansfont[Mapping=tex-text]{Myriad Pro}
%\setmonofont[Mapping=tex-text]{Courier New}

%\setsansfont{Linux Biolinum}

\usepackage[ngerman]{babel}
\selectlanguage{ngerman}

%
% math/symbols
%
\usepackage{amssymb}
\usepackage{amsthm}
% \usepackage{latexsym}
\usepackage{amsmath}
%\usepackage{amsxtra} %Weitere Extrasymbole
%\usepackage{empheq} %Gleichungen hervorheben
%\usepackage{bm}
 %\bm{A} Boldface im Mathemodus

\usepackage{multimedia}
%\usepackage{tikz}

\usepackage{cellspace}
\setlength{\cellspacetoplimit}{2pt}
\setlength{\cellspacebottomlimit}{2pt}

%%%%%%%%%%%%%%%%%% Fuer Frames [fragile]-Option verwenden!
%Programm-Listing
%%%%%%%%%%%%%%%%%%
%Listingsumgebung fuer verbatim
%Grauhinterlegeter Text
%Automatischer Zeilenumbruch ist aktiviert
\usepackage{listings}
% This command allows you to typeset syntax highlighted Matlab
% code ``inline''.
\newcommand{\isage}[1]{\lstinline|#1|}

\definecolor{lgray}{gray}{0.80}
\definecolor{gray}{gray}{0.3}
\definecolor{darkgreen}{rgb}{0,0.4,0}
\definecolor{darkblue}{rgb}{0,0,0.8}
\definecolor{key}{rgb}{0,0.5,0} 
%\lstset{backgroundcolor=\color{lgray}, frame=single, basicstyle=\ttfamily, breaklines=true}
\lstnewenvironment{sage}[1][]{\lstset{xleftmargin=0.2cm,frame=none,backgroundcolor=\color{white},basicstyle=\color{darkblue}\ttfamily\small,#1}}{} 
\lstnewenvironment{sagein}[1][]{\lstset{#1}}{} 
%\lstnewenvironment{sage}{\lstset{,language=python, keywordstyle=color{blue},    commentstyle=color{green}, emphstyle=\color{red}, %frame=single, stringstyle=\color{red}, basicstyle=\ttfamily, ,mathescape =true,escapechar=§}}{}

\lstset{
language=python,
backgroundcolor=\color{lgray},
breaklines=true,
basicstyle=\ttfamily\small,
%otherkeywords={ =},
keywordstyle=\color{blue},
stringstyle=\color{darkgreen},
showstringspaces=false,
emph={class, pass, in, for, while, if, is, elif, else, not, and, or,
def, print, exec, break, continue, return},
emphstyle=\color{blue},
emph={[2]True, False, None, self},
emphstyle=[2]\color{key},
emph={[3]from, import, as},
emphstyle=[3]\color{blue},
upquote=true,
morecomment=[s]{"""}{"""},
commentstyle=\color{gray}\slshape,
%framexleftmargin=1mm, framextopmargin=1mm, 
frame=single,
mathescape =true,
escapechar=§
}


\usepackage{mydef}
%\usepackage{cmap} % you can search in the pdf for umlauts and ligatures
\usepackage{colonequals} %corrects the definition-symbols \colonequals (besides others)
\usepackage{ifthen}
%%%%%%%%%%%%%%%%%%%
%Neue Definitionen
%%%%%%%%%%%%%%%%%%%

%Newcommands
\newcommand{\Fun}[1]{\mathcal{#1}}      %Mathcal fuer Funktoren
\newcommand{\field}[1]{\mathbb{#1}}     %Grundkoerper ?? in mathds

\newcommand{\A}{\field{A}}              %Affines A
\newcommand{\C}{\field{C}}              %Complexes C
\newcommand{\Fp}{\field{F}_{\!p}}       %Endlicher Koerper mit p Elementen
\newcommand{\Fq}{\field{F}_{\!q}}       %Endlicher Koerper mit q Elementen
\newcommand{\Ga}{\field{G}_{a}}         %Add Gruppenschema
\newcommand{\K}{\field{K}}              %Generischer Koerper 
\newcommand{\N}{\field{N}}              %Nat Zahlen
\newcommand{\Pj}{\field{P}}             %Projektives P
\newcommand{\R}{\field{R}} 		%Reelle Zahlen
\newcommand{\Q}{\field{Q}}              %Rationale Zahlen  
\newcommand{\Qt}{\field{H}}             %Quaternionen 
\newcommand{\V}{\field{V}}              %Vektorbuendel V
\newcommand{\Z}{\field{Z}}              %Ganze Zahlen
\DeclareMathOperator{\Real}{Re}

\newcommand{\fdg}{\;|\;}                 %fuer die gilt

%Operatoren
\DeclareMathOperator{\Abb}{Abb}
%\usepackage{sagetex}

%
% Aufgaben
%
\parindent0cm % Abs�tze nicht einr�cken 
% Definieren einer neuen Farbe
\definecolor{light-gray}{gray}{.9}

\newcounter{zaehler}     % neuen Z�hler einf�hren
\stepcounter{zaehler}    % Z�hler einen hochz�hlen

\newenvironment{aufg}[1][0]
%---- Header
{\begin{samepage}%
%\colorbox{light-gray}{%                         % Box in gray
% \makebox[\textwidth]{%                           % Box in linewidth
%\textbf{Aufgabe \arabic{zaehler} } }\hspace{-\textwidth}\makebox[\textwidth]{\hfill #1 Punkte} }\\[0.05cm]       % Header
\dotfill\\
{\large\textbf{Aufgabe \arabic{zaehler} }\ifthenelse{0=#1}{}{\hfill #1 Punkte} }\\[0.4cm]
\begin{minipage}{\textwidth}
}
%-----  foot
{\end{minipage} \nopagebreak %\begin{minipage}{1cm} \end{minipage}
%\\ 
%\begin{minipage}{0.1cm} \end{minipage} 
%\hrulefill \begin{minipage}{1cm} \end{minipage}\\[1cm]  
\stepcounter{zaehler}                           % increase counter
\end{samepage}%
\\%
\bigskip%
}


%\usepackage{tikz}
%\usetikzlibrary{shadows}
%\usetikzlibrary{fit}
%\usetikzlibrary{shapes}
%\usetikzlibrary{backgrounds}

\parindent0cm % Abs�tze nicht einr�cken 

% Definieren einer neuen Farbe
\definecolor{light-gray}{gray}{.9}

\newcounter{zaehler}     % neuen Z�hler einf�hren
\stepcounter{zaehler}    % Z�hler einen hochz�hlen

\newenvironment{aufg}%
%---- Header
{\begin{samepage}
\colorbox{light-gray}{                         % Box in gray
 \makebox[\textwidth]{                           % Box in linewidth
\textbf{Aufgabe} \arabic{zaehler} :}}\\[0.1cm]       % Header
%\begin{minipage}{0.5cm} \end{minipage}    % Insert 0.5cm
\begin{minipage}{\textwidth}}
%-----  foot
{\end{minipage} \nopagebreak %\begin{minipage}{1cm} \end{minipage}
\\[0.1cm] 
%\begin{minipage}{0.1cm} \end{minipage} 
%\hrulefill \begin{minipage}{1cm} \end{minipage}\\[1cm]  
\stepcounter{zaehler}                           % increase counter
 \end{samepage}%
}

%-------------------------------------------------------------------------------
\begin{document}
%-------------------------------------------------------------------------------

%--------------------------------------------------- Header
\begin{center}
\textbf{\LARGE Einf\"uhrung in Sage }\\
\end{center}
\begin{minipage}{6cm}
Dr. J. Schulz\\
C. Rügge
\end{minipage}\hfill
\begin{minipage}{2.5cm}
\begin{flushright}
\textbf{Einheit 6}\\
WS 2009/2010
\end{flushright}
\end{minipage}\\[1cm]

\begin{aufg}
Berechnen Sie die folgenden Grenzwerte
\[ \lim_{n \rightarrow \infty} \left( \frac{(n+1)^2-n^2}{n} \right),
\quad 
\lim_{n \rightarrow \infty} \left( \sqrt{n+1} - \sqrt{n} \right),  \quad 
\lim_{n \rightarrow \infty} \left( \frac{\sum_{i=1}^n i^{99}}{n^{100}} \right).
\] 
\end{aufg}

\begin{aufg}
Bestimmen Sie die Werte der folgenden Summen
\[ \sum_{i=1}^\infty \frac{1}{i^3}, \quad \sum_{i=0}^\infty x^i,
0<x<1, \quad \sum_{i=1}^\infty \frac{1}{i+2i^2}.
\]
\end{aufg}

\begin{aufg}
Bestimmen Sie den Konvergenzradius der folgenden Potenzreihen
\[  \sum_{n=1}^\infty (n^4 - 4n^3)x^n, \quad \sum_{n=1}^\infty
n^{\log(n)/n}x^n, \quad \sum_{n=1}^\infty \left( \sum_{k=1}^n
\frac{1}{k} \right) x^n.\]
\end{aufg}


\begin{aufg}
\begin{itemize}
\item Berechnen Sie die ersten $100$ Glieder der Fibonacci-Folge
\[ a_1=a_2:=1, \quad a_{n+2}:=a_{n+1}+a_n, \ n \in \mathbb{N}. \]
\item Berechnen Sie für $n=1, \dots, 100$
\[ b_n:=\frac{a_{n}}{a_{n+1}}. \]
Geben Sie für die Glieder $b_n$ eine obere Schranke an, und raten Sie
den Grenzwert!  
\item Berechnen Sie die ersten $100$ Glieder der Folge 
\[ c_n = \frac{\left( \frac{1+\sqrt{5}}{2}\right)^n - 
\left( \frac{1-\sqrt{5}}{2}\right)^n}{\sqrt{5}}. \]
\item Bestimmen Sie $c_n-a_n$ und $\frac{c_n}{a_n}$ für $n=1,\dots
  ,100$.
\item Berechnen Sie den Grenzwert von $c_{n+1}/c_n$ approximativ.
\end{itemize}
\end{aufg}

\begin{aufg}
Erzeugen Sie die Tabelle \verb+table(1=ln(1),2=ln(2),...,10=ln(10))+.
Erweitern Sie die Tabelle um den Eintrag \verb+11=ln(11)+. 
Erstellen Sie aus
der Tabelle eine Liste aller Indizes und eine Liste aller Werte!
Sortieren Sie beide Listen!
\end{aufg}
\begin{aufg}
  Zeigen Sie 
  \[\lim_{x \rightarrow \infty} x^a = \left\{ \begin{array}{c}\infty, \mbox{ für } a > 0
        \\1, \mbox{ für } a = 0
        \\0, \mbox{ für } a < 0 \end{array}\right. .\]

\end{aufg}


\begin{aufg}
  Bestimmen Sie von der Potenzreihe
  $f(x)=\sum_{n=0}^\infty(-1)^n\frac{x^{2n+1}}{2n+1}$ den Konvergenzradius.
  Berechnen Sie $4(4f(\frac{1}{5})-f(\frac{1}{239}))$.
\end{aufg}

%-----------------------------------------------------------------------------------

\begin{aufg}
Bestimmen Sie die Funktion dritten Grades, die folgende Eigenschaften erf\"ullt:
\begin{enumerate}
\item
  Die Funktion besitzt ein Maximum im Punkt $(-2,3)$.
\item
  Der Graph der Funktion berührt die Parabel $ f(x) = -x^2+2x+4 $ an der
  Stelle $ x = -1$.
\end{enumerate}
Zeichnen Sie alle Eigenschaften sowie die zugeh\"orige Funktion in einem Plot.
\end{aufg}
%-----------------------------------------------------------------------------------

\begin{aufg}
Schreiben Sie eine Prozedur, die als Eingabeparameter eine Liste mit
beliebigen Einträgen hat. Die Prozedur soll daraus eine Tabelle berechnen und
zurückgeben, die die verschiedenen Elemente der Liste (als Indizes) versehen mit ihrer
Häufigkeit (als Werte) enthält.
\end{aufg}

%-----------------------------------------------------------------------------------

\begin{aufg}
Schreiben Sie eine Prozedur, die zu einem gegebenen Startwert $x_0$ und einer
gegebenen Funktion $f$ die ersten 20 Iterationen
\[ x_{k+1} = x_k - \frac{f(x_k)}{f'(x_{k})}, \quad k=0,1,\dots ,19  \]
berechnet. \\

Wenden Sie die Prozedur auf $f(x)=\exp(-x)-x$ und $g(x)=x^3-2x+3$
an. Verwenden Sie jeweils $x_0=0$ und pr\"ufen Sie, ob $x_{20}$ eine gute
Approximation einer Nullstelle von $f$ ist.
\end{aufg}
%-----------------------------------------------------------------------------------
\end{document}

