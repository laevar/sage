\documentclass[a4paper,12pt,DIV15]{scrartcl}
\usepackage[xetex,bookmarks=true,pdfstartview=FitH,bookmarksopen=true,
    colorlinks,citecolor=Blue,linkcolor=DarkBlue,urlcolor=Green,
    pagebackref=true,plainpages=false,pdfpagelabels=true,unicode=true,
    breaklinks=true,naturalnames=false,setpagesize=true,a4paper=true,hyperindex]{hyperref}
\usepackage[svgnames,hyperref]{xcolor} %color definition
\usepackage{tikz}

%\usepackage{fontspec,xunicode}
% %\usepackage{polyglossia}
%\setdefaultlanguage[spelling=new, latesthyphen=true]{german}
%\setsansfont{DejaVu Sans}
%\setsansfont{Verdana}
%\setsansfont{Arial}
%\setromanfont[Mapping=tex-text]{Linux Libertine}
%\setsansfont[Mapping=tex-text]{Myriad Pro}
%\setmonofont[Mapping=tex-text]{Courier New}

%\setsansfont{Linux Biolinum}

\usepackage[ngerman]{babel}
\selectlanguage{ngerman}

%
% math/symbols
%
\usepackage{amssymb}
\usepackage{amsthm}
% \usepackage{latexsym}
\usepackage{amsmath}
%\usepackage{amsxtra} %Weitere Extrasymbole
%\usepackage{empheq} %Gleichungen hervorheben
%\usepackage{bm}
 %\bm{A} Boldface im Mathemodus
\usepackage{fontspec,xunicode,xltxtra}

\usepackage{multimedia}
%\usepackage{tikz}

\usepackage{cellspace}
\setlength{\cellspacetoplimit}{2pt}
\setlength{\cellspacebottomlimit}{2pt}

%%%%%%%%%%%%%%%%%% Fuer Frames [fragile]-Option verwenden!
%Programm-Listing
%%%%%%%%%%%%%%%%%%
%Listingsumgebung fuer verbatim
%Grauhinterlegeter Text
%Automatischer Zeilenumbruch ist aktiviert
%\usepackage{listings}
\usepackage[framed]{mcode}
%\usepackage{mcode}
% This command allows you to typeset syntax highlighted Matlab
% code ``inline''.
% mcode fuer matlab

\definecolor{lgray}{gray}{0.80}
\definecolor{gray}{gray}{0.3}
\definecolor{darkgreen}{rgb}{0,0.4,0}
\definecolor{darkblue}{rgb}{0,0,0.8}
\definecolor{key}{rgb}{0,0.5,0} 
\definecolor{NU0}{RGB}{68,85,136} % #458
\definecolor{KW3}{RGB}{85,68,136}
\definecolor{KW4}{RGB}{153,0,0}
\definecolor{dred}{RGB}{221,17,68} % #d14
\definecolor{BG}{RGB}{240,240,240}
%\lstset{backgroundcolor=\color{lgray}, frame=single, basicstyle=\ttfamily, breaklines=true}
%\lstnewenvironment{sage}{\lstset{,language=python, keywordstyle=color{blue},    commentstyle=color{green}, emphstyle=\color{red}, %frame=single, stringstyle=\color{red}, basicstyle=\ttfamily, ,mathescape =true,escapechar=§}}{}

\lstdefinelanguage{fooHaskell} {%
  basicstyle=\footnotesize\ttfamily,%
  commentstyle=\slshape\color{gray},%
  keywordstyle=\bfseries,%\color{KW4},
  breaklines=true,
  sensitive=true,
  xleftmargin=1pc,
  emph={[1]
    FilePath,IOError,abs,acos,acosh,all,and,any,appendFile,approxRational,asTypeOf,asin,
    asinh,atan,atan2,atanh,basicIORun,break,catch,ceiling,chr,compare,concat,concatMap,
    const,cos,cosh,curry,cycle,decodeFloat,denominator,digitToInt,div,divMod,drop,
    dropWhile,either,elem,encodeFloat,enumFrom,enumFromThen,enumFromThenTo,enumFromTo,
    error,even,exp,exponent,fail,filter,flip,floatDigits,floatRadix,floatRange,floor,
    fmap,foldl,foldl1,foldr,foldr1,fromDouble,fromEnum,fromInt,fromInteger,fromIntegral,
    fromRational,fst,gcd,getChar,getContents,getLine,head,id,inRange,index,init,intToDigit,
    interact,ioError,isAlpha,isAlphaNum,isAscii,isControl,isDenormalized,isDigit,isHexDigit,
    isIEEE,isInfinite,isLower,isNaN,isNegativeZero,isOctDigit,isPrint,isSpace,isUpper,iterate,
    last,lcm,length,lex,lexDigits,lexLitChar,lines,log,logBase,lookup,map,mapM,mapM_,max,
    maxBound,maximum,maybe,min,minBound,minimum,mod,negate,not,notElem,null,numerator,odd,
    or,ord,otherwise,pi,pred,primExitWith,print,product,properFraction,putChar,putStr,putStrLn,quot,
    quotRem,range,rangeSize,read,readDec,readFile,readFloat,readHex,readIO,readInt,readList,readLitChar,
    readLn,readOct,readParen,readSigned,reads,readsPrec,realToFrac,recip,rem,repeat,replicate,return,
    reverse,round,scaleFloat,scanl,scanl1,scanr,scanr1,seq,sequence,sequence_,show,showChar,showInt,
    showList,showLitChar,showParen,showSigned,showString,shows,showsPrec,significand,signum,sin,
    sinh,snd,span,splitAt,sqrt,subtract,succ,sum,tail,take,takeWhile,tan,tanh,threadToIOResult,toEnum,
    toInt,toInteger,toLower,toRational,toUpper,truncate,uncurry,undefined,unlines,until,unwords,unzip,
    unzip3,userError,words,writeFile,zip,zip3,zipWith,zipWith3,listArray,doParse
  },%
  emphstyle={[1]\color{NU0}},%
  emph={[2]
    Bool,Char,Double,Either,Float,IO,Integer,Int,Maybe,Ordering,Rational,Ratio,ReadS,Show,ShowS,String,
    Word8,InPacket
  },%
  emphstyle={[2]\bfseries\color{KW4}},%
  emph={[3]
    case,class,data,deriving,do,else,if,import,in,infixl,infixr,instance,let,
    module,of,primitive,then,type,where
  },
  emphstyle={[3]\color{darkblue}},
  emph={[4]
    quot,rem,div,mod,elem,notElem,seq
  },
  emphstyle={[4]\color{NU0}\bfseries},
  emph={[5]
    EQ,False,GT,Just,LT,Left,Nothing,Right,True,Show,Eq,Ord,Num
  },
  emphstyle={[5]\color{KW4}\bfseries},
  morestring=[b]",%
  morestring=[b]',%
  stringstyle=\color{darkgreen},%
  showstringspaces=false
}
\lstnewenvironment{hs}
{\lstset{language=fooHaskell,backgroundcolor=\color{BG}}}
{\smallskip}
\newcommand{\ihs}[1]{\lstset{language=fooHaskell,basicstyle=\color[gray]{0.6}}\lstinline|#1|}


\lstdefinelanguage{fooMatlab} {%
backgroundcolor=\color[gray]{0.9},
breaklines=true,
basicstyle=\ttfamily\small,
%otherkeywords={ =},
%keywordstyle=\color{blue},
%stringstyle=\color{darkgreen},
showstringspaces=false,
%emph={for, while, if, elif, else, not, and, or, printf, break, continue, return, end, function},
%emphstyle=\color{blue},
%emph={[2]True, False, None, self, NaN, NULL},
%emphstyle=[2]\color{key},
%emph={[3]from, import, as},
%emphstyle=[3]\color{blue},
%upquote=true,
%morecomment=[s]{"""}{"""},
%commentstyle=\color{gray}\slshape,
%framexleftmargin=1mm, framextopmargin=1mm, 
%title=\tiny matlab,
frame=single,
%mathescape =true,
%escapechar=§
}
\newcommand{\imatlab}[1]{\lstset{language=fooMatlab,basicstyle=\color[gray]{0.6}}\lstinline|#1|}
\lstnewenvironment{matlab}[1][]{\lstset{language=fooMatlab,xleftmargin=0.2cm,frame=none,backgroundcolor=\color{white},basicstyle=\color{darkblue}\ttfamily\small,#1}}{} 
\lstnewenvironment{matlabin}[1][]{\lstset{language=fooMatlab,#1}}{} 
\newcommand{\matinput}[1]{\lstset{language=fooMatlab}\lstinputlisting{#1}}

\lstdefinelanguage{fooPython} {%
language=python,
backgroundcolor=\color[gray]{0.7},
breaklines=true,
basicstyle=\ttfamily\small,
%otherkeywords={ =},
keywordstyle=\color{blue},
stringstyle=\color{darkgreen},
morestring=[b]",%
morestring=[b]',%
showstringspaces=false,
emph={class, pass, in, for, while, if, is, elif, else, not, and, or,
def, print, exec, break, continue, return, import, from, lambda, null},
emphstyle=\color{blue},
emph={[2]True, False, None, self},
emphstyle=[2]\color{key},
emph={[3]from, import, as},
emphstyle=[3]\color{blue},
upquote=true,
morecomment=[s]{"""}{"""},
comment=[l]{\#},
commentstyle=\color{gray},
%framexleftmargin=1mm, framextopmargin=1mm, 
%title=\tiny python,
%caption=python,
frame=single
%frameround=tttt,
%mathescape =true,
%escapechar=§
}

\newcommand{\pyinput}[1]{\lstset{language=fooPython}\lstinputlisting{#1}}
\newcommand{\isage}[1]{{\lstset{language=fooPython,basicstyle=\color[gray]{0.3}}\lstinline|#1|}}

\lstnewenvironment{pyout}[1][]{\lstset{language=fooPython,xleftmargin=0.2cm,frame=none,backgroundcolor=\color{white},basicstyle=\color{darkblue}\ttfamily\small,#1}}{}
\lstnewenvironment{pyin}[1][]{\lstset{language=fooPython,#1}}{}
\lstnewenvironment{sageout}[1][]{\lstset{language=fooPython,xleftmargin=0.2cm,frame=none,backgroundcolor=\color{white},basicstyle=\color{darkblue}\ttfamily\small,#1}}{}
\lstnewenvironment{sagein}[1][]{\lstset{language=fooPython,#1}}{}

%\usepackage{caption}
%\DeclareCaptionFont{white}{ \color{white} }
%\DeclareCaptionFormat{listing}{
%  \colorbox[cmyk]{0.43, 0.35, 0.35,0.01 }{
%      \parbox{\textwidth}{\hspace{15pt}#1#2#3}
%        }
%        }
%        \captionsetup[lstlisting]{ format=listing, labelfont=white, textfont=white, singlelinecheck=false, margin=0pt, font={bf,footnotesize} }


\usepackage{mydef}
%\usepackage{cmap} % you can search in the pdf for umlauts and ligatures
\usepackage{colonequals} %corrects the definition-symbols \colonequals (besides others)

\usepackage{ifthen}

%%%%%%%%%%%%%%%%%%%
%Neue Definitionen
%%%%%%%%%%%%%%%%%%%

%Newcommands
\newcommand{\Fun}[1]{\mathcal{#1}}      %Mathcal fuer Funktoren
\newcommand{\field}[1]{\mathbb{#1}}     %Grundkoerper ?? in mathds

\newcommand{\A}{\field{A}}              %Affines A
\newcommand{\Fp}{\field{F}_{\!p}}       %Endlicher Koerper mit p Elementen
\newcommand{\Fq}{\field{F}_{\!q}}       %Endlicher Koerper mit q Elementen
\newcommand{\Ga}{\field{G}_{a}}         %Add Gruppenschema
\newcommand{\K}{\field{K}}              %Generischer Koerper 
\newcommand{\N}{\field{N}}              %Nat Zahlen
\newcommand{\Pj}{\field{P}}             %Projektives P
\newcommand{\R}{\field{R}} 		%Reelle Zahlen
\newcommand{\Q}{\field{Q}}              %Rationale Zahlen  
\newcommand{\Qt}{\field{H}}             %Quaternionen 
\newcommand{\V}{\field{V}}              %Vektorbuendel V
\newcommand{\Z}{\field{Z}}              %Ganze Zahlen
\DeclareMathOperator{\Real}{Re}

\newcommand{\fdg}{\;|\;}                 %fuer die gilt

%Operatoren
\DeclareMathOperator{\Abb}{Abb}
%\usepackage{sagetex}


%
% Aufgaben
%
\parindent0cm % Abs�tze nicht einr�cken 
% Definieren einer neuen Farbe
\definecolor{light-gray}{gray}{.9}

\newcounter{zaehler}     % neuen Z�hler einf�hren
\newenvironment{aufgn}[2][0]
%---- Header
{\begin{samepage}%
%\colorbox{light-gray}{%                         % Box in gray
% \makebox[\textwidth]{%                           % Box in linewidth
%\textbf{Aufgabe \arabic{zaehler} } }\hspace{-\textwidth}\makebox[\textwidth]{\hfill #1 Punkte} }\\[0.05cm]       % Header
\dotfill\\
{\large\textbf{Aufgabe \arabic{zaehler} \ifthenelse{ \equal{#2}{} }{}{: \emph{ #2 } }}\ifthenelse{-1=#1}{(testierbar)}{}\ifthenelse{0=#1 \or -1=#1}{}{\hfill #1 Punkte} }\\[0.4cm]
%{\large\textbf{Exercise \arabic{zaehler}  #2 }\ifthenelse{-1=#1}{(testierbar)}{}\ifthenelse{0=#1 \or -1=#1}{}{\hfill #1 Punkte} }\\[0.4cm]
\begin{minipage}{\textwidth}%
}%
%-----  foot
{\end{minipage}\nopagebreak%\begin{minipage}{1cm} \end{minipage}
%\\ 
%\begin{minipage}{0.1cm} \end{minipage} 
%\hrulefill \begin{minipage}{1cm} \end{minipage}\\[1cm]  
\stepcounter{zaehler}                           % increase counter
\end{samepage}%
\\%
\bigskip%
}


\newenvironment{aufg}[1][0]
%---- Header
{\begin{samepage}%
\refstepcounter{zaehler}% increase counter
%\colorbox{light-gray}{%                         % Box in gray
% \makebox[\textwidth]{%                           % Box in linewidth
%\textbf{Aufgabe \arabic{zaehler} } }\hspace{-\textwidth}\makebox[\textwidth]{\hfill #1 Punkte} }\\[0.05cm]       % Header
\dotfill\\
{\large\textbf{Aufgabe \arabic{zaehler} }\ifthenelse{-1=#1}{(testierbar)}{}\ifthenelse{0=#1 \or -1=#1}{}{\hfill #1 Punkte} }\\[0.4cm]
\begin{minipage}{\textwidth}%
}%
%-----  foot
{\end{minipage}\nopagebreak%\begin{minipage}{1cm} \end{minipage}
%\\ 
%\begin{minipage}{0.1cm} \end{minipage} 
%\hrulefill \begin{minipage}{1cm} \end{minipage}\\[1cm]  
\end{samepage}%
\\%
\bigskip%
}

\begin{document}
\begin{center}
    \textbf{\LARGE Einführung in Sage}\\
    {\large Einheit 06}\\
    {\large Zusammenfassung: Folgen und Reihen}
\end{center}

\section{Folgen}
\begin{itemize}
\item {\color{red} reelle Zahlenfolge}: Abbildung $a:\mathbb{N} \ \rightarrow \ \mathbb{R}$.
\item Alternative Notation: $(a_n)_{n \in \mathbb{N}}$ oder $(a_n)_{n}$.
\item {\color{red} Glieder} der Folge: Die Zahlen $a_n$.
\item {\color{red} Teilfolge}: $(a_{n_i})_{n_i}$ ist eine Abbildung $a:N \
\rightarrow \ \mathbb{R}$, wobei $N \subset \mathbb{N}$ eine Menge mit
unendlich vielen Elementen ist.
\item Bemerkung: Wir beschränken uns auf den Fall reeller Zahlenfolgen. 
\end{itemize}

\begin{defn}[Konvergenz von Folgen]
Eine Zahlenfolge $(a_n)_n$ ist {\color{red} konvergent} gegen den {\color{red}
Grenzwert} oder {\color{red} Limes} $a\in \mathbb{R}$, wenn es zu jedem
$\varepsilon >0$ ein $n_0 \in \mathbb{N}$ gibt, so dass für alle $n \geq
n_0$ die Abschätzung 
\[ |a_n - a|< \varepsilon \]
 gilt. Man schreibt
\[ a=\lim_{n \rightarrow \infty} a_n. \]

{\color{red} divergent}: nicht konvergente Folge. 
\end{defn}


\paragraph{Bemerkungen:}
\begin{itemize}
    \item {\color{red} Nullfolge}: Folge konvergiert gegen $0$.
\item {\color{red} Häufungspunkt}: Grenzwert einer konvergenten Teilfolge $(a_{n_i})_{n_i}$.
\begin{itemize}
\item Ein Folge kann keinen aber auch mehrere Häufungspunkte besitzten
\item konvergente Folgen haben genau einen Häufungspunkt.
\end{itemize}
\item  {\color{red} Cauchy-Folge}: eine Folge $(a_n)_n$ bei der für
alle $\varepsilon>0$ ein $n_0 \in \mathbb{N}$ existiert, so dass für alle
$n,m \geq n_0$ gilt:
$|a_n - a_m| < \varepsilon$. \\
\item In $\mathbb{R}$ ist eine Folge konvergent,
genau dann wenn sie eine Cauchy-Folge ist (Vollständigkeit).
\item {\color{red} $\varepsilon$-Umgebung}: $U_\varepsilon(a)$ von $a$ ist
definiert durch
\[
U_\varepsilon(a) := (a-\varepsilon, a+\varepsilon) := \{ x \in \mathbb{R} \ | \ |x - a| < \varepsilon \}.
\] 
\end{itemize}




\begin{thm}[Konvergenzkriterien]
\begin{itemize}
\item Jede monotone, beschränkte Folge konvergiert.
\item Konvergenz bei Addition: Sind $(a_n)_n$ und $(b_n)_n$ konvergente Folgen, $\alpha, \beta \in \mathbb{R}$, so ist auch die
                   Folge $( \alpha a_n+\beta b_n)_n$ konvergent mit
                   dem Grenzwert
 \[ \lim_{n \rightarrow \infty} ( \alpha a_n + \beta b_n)= \alpha
                   \lim_{n \rightarrow \infty} a_n + \beta \lim_{n
                   \rightarrow \infty} b_n .\]
\item Konvergenz bei Multiplikation: Sind $(a_n)_n$ und $(b_n)_n$ konvergente Folgen, so ist auch die
                   Folge $(a_n b_n)_n$ konvergent mit
                   dem Grenzwert
 \[
  \lim_{n \rightarrow \infty} ( a_n b_n)= 
                   (\lim_{n \rightarrow \infty} a_n) \cdot  (\lim_{n
                   \rightarrow \infty} b_n).
 \]
\item Bemerkung: Weglassen oder Hinzufügen endlich vieler Glieder verändert das
                   Konvergenzverhalten nicht.
\end{itemize}
\end{thm}

\begin{thm}[Wichtige Sätze]
\begin{itemize}
    \item \textsl{\color{red}(Bolzano-Weierstrass)}: Jede beschränkte Folge besitzt (mindestens) eine
konvergente Teilfolge.
\item Jede Teilfolge einer konvergenten Folge konvergiert gegen den
Grenzwert der ursprünglichen Folge.
\item Jede konvergente Folge ist beschränkt, d.h. es gibt ein $K>0$,
so dass $|a_n|\leq K$ gilt für alle $n \in \mathbb{N}$.
\item \textsl{\color{red}Zwischenfolge}: Seien $(a_n)_n$ und $(b_n)_n$ konvergente Folgen mit $\lim_{n
\rightarrow \infty} a_n = \lim_{n \rightarrow \infty} b_n$. Dann gilt
für eine Folge $(c_n)_n$ mit $a_n \leq c_n \leq b_n$, $n \in
\mathbb{N}$, dass sie konvergiert mit  $\lim_{n
\rightarrow \infty} c_n = \lim_{n \rightarrow \infty} b_n$.
\end{itemize}
\end{thm}


\section{Reihen}
\begin{defn}[unendliche Reihe]
Sei $(a_n)_n$ eine Folge reeller Zahlen. Eine {\color{red} (unendliche)
Reihe} mit den {\color{red} Gliedern} $a_n$, in Zeichen
\[ \sum_{n=1}^\infty a_n =a_1 + a_2 + a_3 + \dots, \]
ist definiert durch die Folge $(s_n)_n$ der {\color{red} Partialsummen}\\
\[
s_n=\sum_{k=1}^n a_k = a_1+a_2+ \dots +a_n 
\]
Der Grenzwert $s$ der Folge $(s_n)_n$ wird als {\color{red} Wert} oder 
{\color{red} Summe} der Reihe bezeichnet. Man schreibt
\[s= \sum_{n=1}^\infty a_n.\] 
\end{defn}

\paragraph{Bemerkungen:}
\begin{itemize}
\item  Reihen sind eine spezielle Art von Folgen.
\item Indizierung mit $m$: $\sum_{n=m}^\infty a_n$.
\item Bei Abänderung, Weglassen oder Hinzufügen endlich vieler Glieder
bleiben Konvergenz und Divergenz unberührt. I.A. wird sich aber der
Grenzwert ändern.
\end{itemize}


% \begin{frame}[fragile]{Beispiele}
% \begin{itemize}
% \item Die Reihe  $\sum_{n=1}^\infty \frac{1}{n^s}$ konvergiert für $s>1$.
% \begin{sagein}
% var('s');assume(s>1);sum(1/k^s,k,1,oo)
% \end{sagein}
% \begin{sage}
% 
% \end{sage}
% 
% \item Die Reihe  $\sum_{n=2}^\infty \frac{1}{n(\log n)^s}$ konvergiert
% für $s>1$ und divergiert für $s=1$.
% \begin{sagein}
% sum(1/(n*log(n)^s),k,2,oo)
% \end{sagein}
% \begin{sage}
% 
% \end{sage}
% 
% \end{itemize}
% 
% \end{frame}


\begin{thm}[Konvergenzkriterien]
\begin{itemize}
\item {\color{red}Cauchykriterium:} Eine Reihe $\sum_{n=1}^\infty a_n$ konvergiert
                  genau dann, wenn es zu jedem $\varepsilon>0$ ein $n_0
                  \in \mathbb{N}$ gibt, so dass für alle $m,n \geq n_0$
                  gilt $| \sum_{k=m}^n a_k|<\varepsilon$.
\item {\color{red}Notwendiges Kriterium:} Konvergiert eine Reihe, so bilden ihre Glieder eine Nullfolge. Dieses Kriterium ist {\color{red}nicht} hinreichend!
\item {\color{red}Verdichtungskriterium:} Eine Reihe $\sum_{n=1}^\infty a_n$ mit
                  einer Folge nichtnegativer, monoton fallender
                  Glieder konvergiert genau dann, wenn die Reihe
                  $\sum_{n=1}^\infty 2^n a_{2^n}$ konvergiert.
\end{itemize}

Gilt $0 \leq c_n \leq a_n \leq b_n$ für alle $n \in \mathbb{N}$
\begin{itemize}
 \item {\color{red} Minorante}: $\sum_{n=1}^\infty c_n$
 \item {\color{red} Majorante}: $\sum_{n=1}^\infty b_n$ 
\end{itemize}
Die Reihe $\sum_{n=1}^\infty a_n$ {\color{red}konvergiert}, wenn...
\begin{description}
\item[Majorantenkriterium:] eine konvergente Majorante besitzt (nichtnegative Glieder).

\item[Quotientenkriterium:] Die Glieder positiv sind und ein $q<1$
existiert, so dass für $n \in \mathbb{N}$ gilt $\frac{a_{n+1}}{a_n}
\leq q$. 
\item[Wurzelkriterium:] Die Glieder positiv sind und ein $q<1$
existiert, so dass für $n \in \mathbb{N}$ gilt $\sqrt[n]{a_n} \leq
q$. 
\item[Leibnizsches Kriterium:] wenn die Folge $(a_n)_n$ bei $\sum_{n=1}^\infty (-1)^n a_n$
eine monoton fallende  Nullfolge ist.
\end{description}
Die Reihe $\sum_{n=1}^\infty a_n$ \textsl{divergiert}, wenn...
\begin{description}
\item[Majorantenkriterium:] sie eine divergente Minorante besitzt.
\end{description}
\end{thm}


\begin{defn}[Absolute und bedingte Konvergenz]

{\color{red} absolut konvergent}: Ist eine Reihe $\sum_{n=0}^\infty a_n$ 
genau dann wenn $\sum_{n=0}^\infty |a_n|$ konvergiert. 

{\color{red} bedingt konvergent}: konvergent, aber nicht absolut konvergent.
\begin{itemize}
\item Absolut konvergente Reihen können beliebig umgeordnet werden.
\item Dies ist i.d.R. bei nicht absolut konvergenten Reihen falsch!
\end{itemize}
\end{defn}

\section{Potenzreihen}

{\color{red} Potenzreihe}:
\[ \sum_{n=0}^\infty a_n (x-x_0)^n \]
mit $x_0 \in \mathbb{R}$. 

{\color{red} Konvergenzradius}: 
\[  \rho := \frac{1}{ \limsup_{n \rightarrow \infty} \sqrt[n]{|a_n|}}
\]
Ist $a_n \neq 0$ für alle $n > n_0$:
\[
 \rho = \limsup_{n \rightarrow \infty} \frac{|a_{n}|}{|a_{n+1}|}.
\]

Konvergenzverhalten:
\begin{itemize}
 \item konvergiert absolut für $|x -x_0|< \rho$.
 \item divergiert für $|x-x_0|>\rho$.
\item Die Konvergenz an den Stellen $x_0-\rho$ und $x_0+\rho$ muss bei
jeder Reihe individuell geprüft werden.   
\end{itemize}


\end{document}
