\documentclass[a4paper,12pt,DIV15]{scrartcl}
\usepackage[xetex,bookmarks=true,pdfstartview=FitH,bookmarksopen=true,
    colorlinks,citecolor=Blue,linkcolor=DarkBlue,urlcolor=Green,
    pagebackref=true,plainpages=false,pdfpagelabels=true,unicode=true,
    breaklinks=true,naturalnames=false,setpagesize=true,a4paper=true,hyperindex]{hyperref}
\usepackage[svgnames,hyperref]{xcolor} %color definition
\usepackage{tikz}
\usepackage{fontspec,xunicode,xltxtra}
%\usepackage{fontspec,xunicode}
%\usepackage{polyglossia}
%\setdefaultlanguage[spelling=new, latesthyphen=true]{german}
%\setsansfont{DejaVu Sans}
%\setsansfont{Verdana}
%\setsansfont{Arial}
%\setromanfont[Mapping=tex-text]{Linux Libertine}
%\setsansfont[Mapping=tex-text]{Myriad Pro}
%\setmonofont[Mapping=tex-text]{Courier New}

%\setsansfont{Linux Biolinum}

\usepackage[ngerman]{babel}
\selectlanguage{ngerman}

%
% math/symbols
%
\usepackage{amssymb}
\usepackage{amsthm}
% \usepackage{latexsym}
\usepackage{amsmath}
%\usepackage{amsxtra} %Weitere Extrasymbole
%\usepackage{empheq} %Gleichungen hervorheben
%\usepackage{bm}
 %\bm{A} Boldface im Mathemodus

\usepackage{multimedia}
%\usepackage{tikz}

\usepackage{cellspace}
\setlength{\cellspacetoplimit}{2pt}
\setlength{\cellspacebottomlimit}{2pt}

%%%%%%%%%%%%%%%%%% Fuer Frames [fragile]-Option verwenden!
%Programm-Listing
%%%%%%%%%%%%%%%%%%
%Listingsumgebung fuer verbatim
%Grauhinterlegeter Text
%Automatischer Zeilenumbruch ist aktiviert
\usepackage{listings}
% This command allows you to typeset syntax highlighted Matlab
% code ``inline''.
\newcommand{\isage}[1]{\lstinline|#1|}

\definecolor{lgray}{gray}{0.80}
\definecolor{gray}{gray}{0.3}
\definecolor{darkgreen}{rgb}{0,0.4,0}
\definecolor{darkblue}{rgb}{0,0,0.8}
\definecolor{key}{rgb}{0,0.5,0} 
%\lstset{backgroundcolor=\color{lgray}, frame=single, basicstyle=\ttfamily, breaklines=true}
\lstnewenvironment{sage}[1][]{\lstset{xleftmargin=0.2cm,frame=none,backgroundcolor=\color{white},basicstyle=\color{darkblue}\ttfamily\small,#1}}{} 
\lstnewenvironment{sagein}[1][]{\lstset{#1}}{} 
%\lstnewenvironment{sage}{\lstset{,language=python, keywordstyle=color{blue},    commentstyle=color{green}, emphstyle=\color{red}, %frame=single, stringstyle=\color{red}, basicstyle=\ttfamily, ,mathescape =true,escapechar=§}}{}

\lstset{
language=python,
backgroundcolor=\color{lgray},
breaklines=true,
basicstyle=\ttfamily\small,
%otherkeywords={ =},
keywordstyle=\color{blue},
stringstyle=\color{darkgreen},
showstringspaces=false,
emph={class, pass, in, for, while, if, is, elif, else, not, and, or,
def, print, exec, break, continue, return},
emphstyle=\color{blue},
emph={[2]True, False, None, self},
emphstyle=[2]\color{key},
emph={[3]from, import, as},
emphstyle=[3]\color{blue},
upquote=true,
morecomment=[s]{"""}{"""},
commentstyle=\color{gray}\slshape,
%framexleftmargin=1mm, framextopmargin=1mm, 
frame=single,
mathescape =true,
escapechar=§
}


\usepackage{mydef}
%\usepackage{cmap} % you can search in the pdf for umlauts and ligatures
\usepackage{colonequals} %corrects the definition-symbols \colonequals (besides others)
\usepackage{ifthen}
%%%%%%%%%%%%%%%%%%%
%Neue Definitionen
%%%%%%%%%%%%%%%%%%%

%Newcommands
\newcommand{\Fun}[1]{\mathcal{#1}}      %Mathcal fuer Funktoren
\newcommand{\field}[1]{\mathbb{#1}}     %Grundkoerper ?? in mathds

\newcommand{\A}{\field{A}}              %Affines A
\newcommand{\C}{\field{C}}              %Complexes C
\newcommand{\Fp}{\field{F}_{\!p}}       %Endlicher Koerper mit p Elementen
\newcommand{\Fq}{\field{F}_{\!q}}       %Endlicher Koerper mit q Elementen
\newcommand{\Ga}{\field{G}_{a}}         %Add Gruppenschema
\newcommand{\K}{\field{K}}              %Generischer Koerper 
\newcommand{\N}{\field{N}}              %Nat Zahlen
\newcommand{\Pj}{\field{P}}             %Projektives P
\newcommand{\R}{\field{R}} 		%Reelle Zahlen
\newcommand{\Q}{\field{Q}}              %Rationale Zahlen  
\newcommand{\Qt}{\field{H}}             %Quaternionen 
\newcommand{\V}{\field{V}}              %Vektorbuendel V
\newcommand{\Z}{\field{Z}}              %Ganze Zahlen
\DeclareMathOperator{\Real}{Re}

\newcommand{\fdg}{\;|\;}                 %fuer die gilt

%Operatoren
\DeclareMathOperator{\Abb}{Abb}
%\usepackage{sagetex}

%
% Aufgaben
%
\parindent0cm % Abs�tze nicht einr�cken 
% Definieren einer neuen Farbe
\definecolor{light-gray}{gray}{.9}

\newcounter{zaehler}     % neuen Z�hler einf�hren
\stepcounter{zaehler}    % Z�hler einen hochz�hlen

\newenvironment{aufg}[1][0]
%---- Header
{\begin{samepage}%
%\colorbox{light-gray}{%                         % Box in gray
% \makebox[\textwidth]{%                           % Box in linewidth
%\textbf{Aufgabe \arabic{zaehler} } }\hspace{-\textwidth}\makebox[\textwidth]{\hfill #1 Punkte} }\\[0.05cm]       % Header
\dotfill\\
{\large\textbf{Aufgabe \arabic{zaehler} }\ifthenelse{0=#1}{}{\hfill #1 Punkte} }\\[0.4cm]
\begin{minipage}{\textwidth}
}
%-----  foot
{\end{minipage} \nopagebreak %\begin{minipage}{1cm} \end{minipage}
%\\ 
%\begin{minipage}{0.1cm} \end{minipage} 
%\hrulefill \begin{minipage}{1cm} \end{minipage}\\[1cm]  
\stepcounter{zaehler}                           % increase counter
\end{samepage}%
\\%
\bigskip%
}

\begin{document}
\begin{center}
    \textbf{\LARGE Einführung in Sage}\\
    {\large Einheit 08}\\
    {\large Zusammenfassung: Differentation und Integration}
\end{center}

\section{Differentation}
\begin{defn}[Ableitung]
Eine Funktion $f: D \ \rightarrow \mathbb{R}$ heißt {\color{red}  differenzierbar} an
der Stelle $x_0 \in D$, wenn 
\[ \lim_{x \rightarrow x_0} \frac{f(x)-f(x_0)}{x -x_0} \]
existiert. \\
Der Grenzwert wird {\color{red} Ableitung} oder {\color{red} Differentialquotient}
von $f$ an $x_0$ genannt und mit $f\,'(x_0)=f^{\,(1)} (x_0)$ bezeichnet.
\end{defn}


\paragraph{Bemerkungen:}
\begin{itemize}
\item $f$ {\color{red} differenzierbar} (auf $D$): Wenn $f$ an jeder
Stelle von $D$ differenzierbar ist. 
\item {\color{red} Ableitung} von $f$: $f\,'$ für die gilt $f$ differenzierbar auf $D$.
\item Eine Funktion $f:D \rightarrow \mathbb{R}$ ist an $x_0 \in D$
genau dann differenzierbar, wenn es eine lineare Abbildung
$T:\mathbb{R} \rightarrow \mathbb{R}$ gibt, so dass 
\[ f(x_0+h)=f(x_0) +Th + R(h)h, \quad \lim_{h \rightarrow 0} R(h)=0 \]
gilt ($T$ und $R$ hängen i.A. von $x_0$ ab). 
\item $f$ an $x_0 \in D$ differenzierbar => stetig.
\end{itemize}

\paragraph{Beispiele:}
\begin{itemize}
\item $(x^n)'=n x^{n-1}$
\item $(e^x)'=e^x$
\item $(\log(x))'= \frac{1}{x}$, $x>0$
\item $(\sin(x))'=\cos(x)$
\item $(\cos(x))'=-\sin(x)$
\end{itemize}

\begin{defn}[Landau-Symbol]
Sei $f$ eine Funktion, die auf einem Intervall definiert ist, das $0$
enthält. 
\begin{itemize}
 \item {\color{red} $f(x)=O(x^n)$}: Es gibt eine Kontante $C >0$, so dass
\[  \lim_{x \rightarrow 0} \frac{|f(x)|}{|x|^n} \leq C\]
($f(x)$ geht gegen $0$ mindestens so schnell wie $x^n$)
\item {\color{red} $f(x)=o(x^n)$}: 
\[  
 \lim_{x \rightarrow 0} \frac{|f(x)|}{|x|^n}=0
\]
($f(x)$ geht schneller gegen $0$ als $x^n$)
\end{itemize}
\end{defn}


\begin{defn}[Höhere Ableitungen]
\begin{itemize}
\item {\color{red}$f$ $n$-mal differenzierbar, $f^{\,(n)}$}: Ist $f:D
\rightarrow \mathbb{R}$ $(n-1)$-mal differenzierbar mit der
$(n-1)$-ten Ableitung $f^{\,(n-1)}$ und ist $f^{\,(n-1)}$ wiederum
differenzierbar.
\item Ist $f$ $n$-mal differenzierbar und ist die $n$-te Ableitung $f^{\,(n)}$
stetig, so heißt $f$ $n$-mal {\color{red} stetig differenzierbar}.
\item $f$ heißt {\color{red} unendlich oft differenzierbar}, wenn $f$ $n$-mal
differenzierbar ist für alle $n\in \mathbb{N}$. 
\end{itemize}
\end{defn}




\begin{thm}[Differentationsregeln]
Seien $f,g:D \rightarrow \mathbb{R}$ differenzierbare Funktionen und
$x_0 \in D$. Dann gilt
\begin{itemize}
\item {\color{red}Summe}: $(f+g)'(x_0)=f\,'(x_0)+ g\,'(x_0)$
\item {\color{red}Produktregel}: \[(f \cdot g)'(x_0) = f\,'(x_0) \cdot g(x_0) + f(x_0)g\,'(x_0)\] 
\item {\color{red}Quotientenregel}: \[\left(\frac{f}{g}\right)'(x_0) = \frac{f\,'(x_0) g(x_0) - f(x_0)
g'(x_0)}{(g(x_0))^2}\]
falls $g(x_0) \neq 0$ .
\item {\color{red}Kettenregel}: Seien $f:D_f \rightarrow \mathbb{R}$ und $g:D_g
\rightarrow \mathbb{R}$ mit $f(D_f) \subset D_g$. Ferner seien $f$ an
$x_0 \in D_f$ und $g$ an $y_0=f(x_0)$ differenzierbar. Dann gilt
\[ (g \circ f)(x_0) = g\,'(y_0) \cdot f\,'(x_0).  \]
\item {\color{red}Leibnizsche Regel}:
\[
(f \cdot g)^{(n)} = \sum_{k=0}^n \binom{n}{k} f^{\,(k)} g^{\,(n-k)}.
\]
\item {\color{red}Umkehrfunktion $f^{\,-1}$}:
\[(f^{\,-1})' \circ f = \frac{1}{f\,'}.\]
\end{itemize}
\end{thm}

\begin{thm}
Ist eine Funktion $f$ an $x_0$ differenzierbar und hat sie ein
lokales Extremum, so gilt $f\,'(x_0)=0$.
\end{thm}
\begin{thm}[Mittelwertsatz] 
Sei $f$ eine in dem Intervall $[a,b]$ stetige
und in $(a,b)$ differenzierbare Funktion. Dann gibt es ein $\xi \in
(a,b)$, so dass gilt
\[ \frac{f(b)-f(a)}{b-a}= f\,'(\xi). \]  
\end{thm}
\begin{thm}
Eine differenzierbare Funktion auf dem Intervall $I$ ist genau
dann konstant, wenn $f\,'(x)$ auf $I$ identisch verschwindet.
 \end{thm}

\begin{defn}[Regel von L'Hospital]
Seien $f,g: D \rightarrow \mathbb{R}$
differenzierbar, und $g\,'(x) \neq 0$, $x \in D$ und 
{\color{red} $\lim_{x \rightarrow a} f(x) = \lim_{x \rightarrow a} g(x)= 0$}. Dann gilt
\[ \lim_{x \rightarrow a} \frac{f(x)}{g(x)} =  \lim_{x \rightarrow a}
\frac{f\,'(x)}{g\,'(x)}, \]
falls der Limes auf der rechten Seite existiert.

\end{defn}
\begin{bemerk}
Der Fall $a=\pm\infty$ ist auch erlaubt!

Die Aussage gilt auch für $\lim_{x \rightarrow a} f(x)=\lim_{x \rightarrow a} g(x)= \infty$ 
\end{bemerk}

 \begin{center}
     Sage-Notebook zur Differentation: 
     \url{http://sage.math.uni-goettingen.de/home/pub/47}
 \end{center}


\section{Taylorsche Formel}

\begin{thm}[Taylorsche Formel]
Sei $f:I \rightarrow \mathbb{R}$ $(n+1)$-mal differenzierbar und seien
$x,x_0 \in I$, $x \neq x_0$. Dann gibt es $\xi \in \mathbb{R}$, so
dass
\begin{eqnarray*}
f(x)& = & f(x_0) + \frac{f\,'(x_0)}{1!}(x-x_0)+
\frac{f\,''(x_0)}{2!}(x-x_0)^2\\[0.5cm]
 & & + \cdots + \frac{f^{\,n}(x_0)}{n!}(x-x_0)^n +R_n(x,x_0) 
\end{eqnarray*}
gilt mit dem  {\color{red} Lagrangschen Restglied}
\[ R_n(x,x_0) := \frac{f^{\,(n+1)}(\xi)}{(n+1)!} (x-x_0)^{n+1}. \]
\end{thm}

\paragraph{Restglied:}
Für das Restglied gibt es noch andere Darstellungen:
\begin{itemize}
\item Darstellung von {\color{red}Cauchy}
\[ R_n(x,x_0)= \frac{f^{\,(n+1)}(\xi)}{n!} (x- \xi)^{n}(x-x_0).\]
\item {\color{red}Integraldarstellung}
\[ R_n(x,x_0)= \int_{x_0}^x \frac{f^{\,(n+1)}(t)}{n!}  (x-t)^n dt.\]
\end{itemize}

\begin{defn}[Taylorreihe]
Sei $f:I \rightarrow \mathbb{R}$ unendlich oft  differenzierbar und seien
$x,x_0 \in I$, $x \neq x_0$. Dann nennt man die Reihe
\[ \sum_{n=0}^\infty \frac{f^{\,(n)}(x_0)}{n!}(x-x_0)^n \]
die {\color{red} Taylorreihe} von $f(x)$ um den Entwicklungspunkt
{\color{red} $x_0$}.

\bigskip
Die Taylorreihe stellt die Funktion $f$, wenn das Restglied $R_n(x,x_0)$ für $n
\rightarrow \infty$ gegen $0$ geht.
\end{defn}

\paragraph{Hinreichende Bedingungen:}
\begin{itemize}
\item Die Funktion $f$ läßt sich auf $I\cap (x_0-\delta, x_0+\delta)$ durch die Taylorreihe darstellen, wenn
\[
\delta:= \frac{1}{\limsup_{n \rightarrow \infty} \sqrt[n]{A_n}}>0
 \]
mit $A_n=\sup_{x \in I}  \frac{|f^{\,(n)}(x)|}{n!}$ gilt.
\item Gibt es ein $M>0$, so dass $|f^{\,(n)}(x)|\leq M^n$ ist für $x \in
I$, $n \in \mathbb{N}$, so läßt sich $f$ auf $I$ durch die Taylorreihe
darstellen.
\item Es gebe ein $M$ mit  $f^{\,(n)}(x)\geq -M^n$, $x \in [a,b]$. Dann
gilt die Taylorreihendarstellung für alle $x \in [a,b]$ mit
$|x-x_0|<b-x$. 
\end{itemize}

\paragraph{Beispiele:}
\begin{itemize}
\item Ist $f(x) = \sum_{n=0}^k a_n x^n$ ein Polynom, so gilt für jeden Punkt $x_0 \in \R$
\[
 f(x) = \sum_{n=0}^\infty \frac{f^{\,(n)}(x_0)}{n!}(x-x_0)^n 
 = \sum_{n=0}^k  \frac{f^{\,(n)}(x_0)}{n!}(x-x_0)^n
\]
\item Für $f(x)=\exp(x)$ und $x_0 \in \mathbb{R}$ gilt 
\[ \exp(x)= \sum_{n=0}^\infty \frac{\exp(x_0)}{n!} (x-x_0)^n, \quad x
\in \mathbb{R}.\]
\item Für $f(x)=\log(x)$ und $x_0=1$ gilt
\[ \log (x) = \sum_{n=1}^\infty \frac{(-1)^{n-1}}{n}(x-1)^n, \quad 0 <
x \leq 2.\]
\item Für $f(x)= (1+x)^a$, $a\in \mathbb{R}$  und $x_0=0$ gilt
\[ (1+x)^a= \sum_{n =0}^\infty \binom{a}{n}{x^n}, \quad -1 <
x < 1. \]
\end{itemize}

\section{Integration}
\begin{itemize}
\item  Sei $[a,b]$ ein Intervall. Gilt $a=a_0 < a_1 <a_2 < \dots <
                                 a_n=b$, so nennt man $Z=(a_0, \dots
                                 ,a_n)$ eine {\color{red} Zerlegung} von
                                 $[a,b]$.
\item Eine Funktion $\phi:[a,b] \rightarrow \mathbb{R}$ heißt {\color{red}
                                 Treppenfunktion}, wenn es eine
                                 Zerlegung $Z$ von $[a,b]$ gibt, so dass
                                 $\phi$ konstant ist auf jedem
                                 Teilintervall $(a_i,a_{i+1})$ von $Z$.
\item Wir erklären das Integral einer Treppenfunktion $\phi$ durch
\[\int_Z \phi := \sum_{k=1}^n c_k (a_k -a_{k-1}).\]
Dabei ist $Z$ die zugehörige Zerlegung und $c_k=\phi(x)$, $x\in
(a_{k-1},a_k)$.  
\item Die Menge aller Treppenfunktionen auf $[a,b]$ sei $T[a,b]$.
\end{itemize}

%eventuell zu sage umschalten?

\begin{defn}
Für eine beschränkte Funktion $f[a,b]\rightarrow \mathbb{R}$
heißt
\[ \int^*f:= \inf  \{ \int \psi \ | \ \psi \in T[a,b], f \leq \psi \}. \]
das {\color{red} Oberintegral} und 
\[ \int_*f:= \sup  \{ \int \psi \ | \ \psi \in T[a,b], f \geq \psi \}.\]
das {\color{red} Unterintegral}. Es gilt für $\phi, \psi \in T[a,b]$ mit
$\phi \leq f \leq \psi$ die Ungleichung
\[ \int \phi \leq \int_* f \leq \int^* f \leq \int \psi.\]
\end{defn}

\begin{defn}[Das Riemannsche Integral]
Eine beschränkte Funktion $f:[a,b] \rightarrow \mathbb{R}$ heißt {\color{red}
integrierbar}, wenn Ober- und Unterintegral von $f$ auf $[a,b]$ übereinstimmen.\\
Der gemeinsame Wert heißt das {\color{red} Integral} von $f$ und wird mit 
\[ \int_a^b f(x)dx \]
bezeichnet.\\ 
($f$ der Integrand, $x$ Integrationsvariable, $a$, $b$ Integrationsgrenzen).
\end{defn}

\paragraph{Eigenschaften:}
\begin{itemize}
\item Linearität
\[ \int_a^b (\alpha f + \beta g) dx = \alpha \int_a^b fdx + \beta
\int_a^b g dx \]
\item Monotonie
\[  f \leq g \quad \Rightarrow \quad \int_a^b f dx \leq \int_a^b g
dx.\]
\item Ist $f,g$ integrierbar, so auch $f+g$, $f*g$, $f-g$,
$\max\{f,g\}$, $\min \{f,g \}$. 
\end{itemize}

\begin{thm}
{\color{red} Stammfunktion} $F$: Ist $f$ stetig auf einem Intervall $I$ und $a \in I$, so ist 
\[F(x)= \int_a^x f(t) dt, x \in I \] eine differenzierbare Funktion mit
$F\,'(x)=f(x)$. 
\end{thm}
\begin{thm}
{\color{red}Unbestimmtes Integral}: alternative Notation für eine Stammfunktion $F$:  $\int f(x) dx$ 
\end{thm}
\begin{thm}
Ist $f$ stetig auf $[a,b]$, so gibt es ein $\xi \in [a,b]$ mit
\[\int_a^b f(x)dx = f(\xi) (b-a)\]
\end{thm}



\begin{defn}[Uneigentliche Integrale]
Sei $f$ auf $[a,b)$ erklärt (eventuell $b=\infty$) und sei $f$ auf jedem
abgeschlossenen Teilintervall integrierbar. Man definiert
\[  \int_a^b f(x)dx := \lim_{z \rightarrow b (-0)} \int_a^z f(x)dx,\]
falls der Limes existiert. Man spricht von einem {\color{red}
uneigentlichen Integral}.
\end{defn}

\subsection{Funktionenfolgen}
\begin{itemize}
\item Sei $(f_n)_n$ eine Folge integrierbarer Funktionen auf
$[a,b]$. Konvergiert $(f_n)_n$ gleichmäßig, so ist die Grenzfunktion
integrierbar mit
\[ \int_a^b \lim_{n \rightarrow \infty} f_n(x) dx = \lim_{n
\rightarrow \infty} \int_a^b f_n(x) dx .\]
\item Sei $(f_n)_n$ eine Folge differenzierbarer Funktionen auf
$[a,b]$. $(f\,'_n)_n$ konvergiere gleichmäßig und es existiere ein $x_0
\in [a,b]$ für den $f_n(x_0)$ konvergiert. Dann konvergiert $(f_n)_n$
gleichmäßig mit $( \lim_{n \rightarrow \infty} f_n(x))' = \lim_{n
\rightarrow \infty} f\,'_n(x)$.  
\end{itemize}
 
\begin{center}
    Sage-Notebook zur Integration:
     \url{http://sage.math.uni-goettingen.de/home/pub/48}
 \end{center}
\end{document}
