\documentclass[a4paper,12pt,DIV15]{scrartcl}
\usepackage[xetex,bookmarks=true,pdfstartview=FitH,bookmarksopen=true,
    colorlinks,citecolor=Blue,linkcolor=DarkBlue,urlcolor=Green,
    pagebackref=true,plainpages=false,pdfpagelabels=true,unicode=true,
    breaklinks=true,naturalnames=false,setpagesize=true,a4paper=true,hyperindex]{hyperref}
\usepackage[svgnames,hyperref]{xcolor} %color definition

%\usepackage{fontspec,xunicode}
% %\usepackage{polyglossia}
%\setdefaultlanguage[spelling=new, latesthyphen=true]{german}
%\setsansfont{DejaVu Sans}
%\setsansfont{Verdana}
%\setsansfont{Arial}
%\setromanfont[Mapping=tex-text]{Linux Libertine}
%\setsansfont[Mapping=tex-text]{Myriad Pro}
%\setmonofont[Mapping=tex-text]{Courier New}

%\setsansfont{Linux Biolinum}

\usepackage[ngerman]{babel}
\selectlanguage{ngerman}

%
% math/symbols
%
\usepackage{amssymb}
\usepackage{amsthm}
% \usepackage{latexsym}
\usepackage{amsmath}
%\usepackage{amsxtra} %Weitere Extrasymbole
%\usepackage{empheq} %Gleichungen hervorheben
%\usepackage{bm}
 %\bm{A} Boldface im Mathemodus
\usepackage{fontspec,xunicode,xltxtra}

\usepackage{multimedia}
%\usepackage{tikz}

\usepackage{cellspace}
\setlength{\cellspacetoplimit}{2pt}
\setlength{\cellspacebottomlimit}{2pt}

%%%%%%%%%%%%%%%%%% Fuer Frames [fragile]-Option verwenden!
%Programm-Listing
%%%%%%%%%%%%%%%%%%
%Listingsumgebung fuer verbatim
%Grauhinterlegeter Text
%Automatischer Zeilenumbruch ist aktiviert
%\usepackage{listings}
\usepackage[framed]{mcode}
%\usepackage{mcode}
% This command allows you to typeset syntax highlighted Matlab
% code ``inline''.
% mcode fuer matlab

\definecolor{lgray}{gray}{0.80}
\definecolor{gray}{gray}{0.3}
\definecolor{darkgreen}{rgb}{0,0.4,0}
\definecolor{darkblue}{rgb}{0,0,0.8}
\definecolor{key}{rgb}{0,0.5,0} 
\definecolor{NU0}{RGB}{68,85,136} % #458
\definecolor{KW3}{RGB}{85,68,136}
\definecolor{KW4}{RGB}{153,0,0}
\definecolor{dred}{RGB}{221,17,68} % #d14
\definecolor{BG}{RGB}{240,240,240}
%\lstset{backgroundcolor=\color{lgray}, frame=single, basicstyle=\ttfamily, breaklines=true}
%\lstnewenvironment{sage}{\lstset{,language=python, keywordstyle=color{blue},    commentstyle=color{green}, emphstyle=\color{red}, %frame=single, stringstyle=\color{red}, basicstyle=\ttfamily, ,mathescape =true,escapechar=§}}{}

\lstdefinelanguage{fooHaskell} {%
  basicstyle=\footnotesize\ttfamily,%
  commentstyle=\slshape\color{gray},%
  keywordstyle=\bfseries,%\color{KW4},
  breaklines=true,
  sensitive=true,
  xleftmargin=1pc,
  emph={[1]
    FilePath,IOError,abs,acos,acosh,all,and,any,appendFile,approxRational,asTypeOf,asin,
    asinh,atan,atan2,atanh,basicIORun,break,catch,ceiling,chr,compare,concat,concatMap,
    const,cos,cosh,curry,cycle,decodeFloat,denominator,digitToInt,div,divMod,drop,
    dropWhile,either,elem,encodeFloat,enumFrom,enumFromThen,enumFromThenTo,enumFromTo,
    error,even,exp,exponent,fail,filter,flip,floatDigits,floatRadix,floatRange,floor,
    fmap,foldl,foldl1,foldr,foldr1,fromDouble,fromEnum,fromInt,fromInteger,fromIntegral,
    fromRational,fst,gcd,getChar,getContents,getLine,head,id,inRange,index,init,intToDigit,
    interact,ioError,isAlpha,isAlphaNum,isAscii,isControl,isDenormalized,isDigit,isHexDigit,
    isIEEE,isInfinite,isLower,isNaN,isNegativeZero,isOctDigit,isPrint,isSpace,isUpper,iterate,
    last,lcm,length,lex,lexDigits,lexLitChar,lines,log,logBase,lookup,map,mapM,mapM_,max,
    maxBound,maximum,maybe,min,minBound,minimum,mod,negate,not,notElem,null,numerator,odd,
    or,ord,otherwise,pi,pred,primExitWith,print,product,properFraction,putChar,putStr,putStrLn,quot,
    quotRem,range,rangeSize,read,readDec,readFile,readFloat,readHex,readIO,readInt,readList,readLitChar,
    readLn,readOct,readParen,readSigned,reads,readsPrec,realToFrac,recip,rem,repeat,replicate,return,
    reverse,round,scaleFloat,scanl,scanl1,scanr,scanr1,seq,sequence,sequence_,show,showChar,showInt,
    showList,showLitChar,showParen,showSigned,showString,shows,showsPrec,significand,signum,sin,
    sinh,snd,span,splitAt,sqrt,subtract,succ,sum,tail,take,takeWhile,tan,tanh,threadToIOResult,toEnum,
    toInt,toInteger,toLower,toRational,toUpper,truncate,uncurry,undefined,unlines,until,unwords,unzip,
    unzip3,userError,words,writeFile,zip,zip3,zipWith,zipWith3,listArray,doParse
  },%
  emphstyle={[1]\color{NU0}},%
  emph={[2]
    Bool,Char,Double,Either,Float,IO,Integer,Int,Maybe,Ordering,Rational,Ratio,ReadS,Show,ShowS,String,
    Word8,InPacket
  },%
  emphstyle={[2]\bfseries\color{KW4}},%
  emph={[3]
    case,class,data,deriving,do,else,if,import,in,infixl,infixr,instance,let,
    module,of,primitive,then,type,where
  },
  emphstyle={[3]\color{darkblue}},
  emph={[4]
    quot,rem,div,mod,elem,notElem,seq
  },
  emphstyle={[4]\color{NU0}\bfseries},
  emph={[5]
    EQ,False,GT,Just,LT,Left,Nothing,Right,True,Show,Eq,Ord,Num
  },
  emphstyle={[5]\color{KW4}\bfseries},
  morestring=[b]",%
  morestring=[b]',%
  stringstyle=\color{darkgreen},%
  showstringspaces=false
}
\lstnewenvironment{hs}
{\lstset{language=fooHaskell,backgroundcolor=\color{BG}}}
{\smallskip}
\newcommand{\ihs}[1]{\lstset{language=fooHaskell,basicstyle=\color[gray]{0.6}}\lstinline|#1|}


\lstdefinelanguage{fooMatlab} {%
backgroundcolor=\color[gray]{0.9},
breaklines=true,
basicstyle=\ttfamily\small,
%otherkeywords={ =},
%keywordstyle=\color{blue},
%stringstyle=\color{darkgreen},
showstringspaces=false,
%emph={for, while, if, elif, else, not, and, or, printf, break, continue, return, end, function},
%emphstyle=\color{blue},
%emph={[2]True, False, None, self, NaN, NULL},
%emphstyle=[2]\color{key},
%emph={[3]from, import, as},
%emphstyle=[3]\color{blue},
%upquote=true,
%morecomment=[s]{"""}{"""},
%commentstyle=\color{gray}\slshape,
%framexleftmargin=1mm, framextopmargin=1mm, 
%title=\tiny matlab,
frame=single,
%mathescape =true,
%escapechar=§
}
\newcommand{\imatlab}[1]{\lstset{language=fooMatlab,basicstyle=\color[gray]{0.6}}\lstinline|#1|}
\lstnewenvironment{matlab}[1][]{\lstset{language=fooMatlab,xleftmargin=0.2cm,frame=none,backgroundcolor=\color{white},basicstyle=\color{darkblue}\ttfamily\small,#1}}{} 
\lstnewenvironment{matlabin}[1][]{\lstset{language=fooMatlab,#1}}{} 
\newcommand{\matinput}[1]{\lstset{language=fooMatlab}\lstinputlisting{#1}}

\lstdefinelanguage{fooPython} {%
language=python,
backgroundcolor=\color[gray]{0.7},
breaklines=true,
basicstyle=\ttfamily\small,
%otherkeywords={ =},
keywordstyle=\color{blue},
stringstyle=\color{darkgreen},
morestring=[b]",%
morestring=[b]',%
showstringspaces=false,
emph={class, pass, in, for, while, if, is, elif, else, not, and, or,
def, print, exec, break, continue, return, import, from, lambda, null},
emphstyle=\color{blue},
emph={[2]True, False, None, self},
emphstyle=[2]\color{key},
emph={[3]from, import, as},
emphstyle=[3]\color{blue},
upquote=true,
morecomment=[s]{"""}{"""},
comment=[l]{\#},
commentstyle=\color{gray},
%framexleftmargin=1mm, framextopmargin=1mm, 
%title=\tiny python,
%caption=python,
frame=single
%frameround=tttt,
%mathescape =true,
%escapechar=§
}

\newcommand{\pyinput}[1]{\lstset{language=fooPython}\lstinputlisting{#1}}
\newcommand{\isage}[1]{{\lstset{language=fooPython,basicstyle=\color[gray]{0.3}}\lstinline|#1|}}

\lstnewenvironment{pyout}[1][]{\lstset{language=fooPython,xleftmargin=0.2cm,frame=none,backgroundcolor=\color{white},basicstyle=\color{darkblue}\ttfamily\small,#1}}{}
\lstnewenvironment{pyin}[1][]{\lstset{language=fooPython,#1}}{}
\lstnewenvironment{sageout}[1][]{\lstset{language=fooPython,xleftmargin=0.2cm,frame=none,backgroundcolor=\color{white},basicstyle=\color{darkblue}\ttfamily\small,#1}}{}
\lstnewenvironment{sagein}[1][]{\lstset{language=fooPython,#1}}{}

%\usepackage{caption}
%\DeclareCaptionFont{white}{ \color{white} }
%\DeclareCaptionFormat{listing}{
%  \colorbox[cmyk]{0.43, 0.35, 0.35,0.01 }{
%      \parbox{\textwidth}{\hspace{15pt}#1#2#3}
%        }
%        }
%        \captionsetup[lstlisting]{ format=listing, labelfont=white, textfont=white, singlelinecheck=false, margin=0pt, font={bf,footnotesize} }


\usepackage{mydef}
%\usepackage{cmap} % you can search in the pdf for umlauts and ligatures
\usepackage{colonequals} %corrects the definition-symbols \colonequals (besides others)

\usepackage{ifthen}

%%%%%%%%%%%%%%%%%%%
%Neue Definitionen
%%%%%%%%%%%%%%%%%%%

%Newcommands
\newcommand{\Fun}[1]{\mathcal{#1}}      %Mathcal fuer Funktoren
\newcommand{\field}[1]{\mathbb{#1}}     %Grundkoerper ?? in mathds

\newcommand{\A}{\field{A}}              %Affines A
\newcommand{\Fp}{\field{F}_{\!p}}       %Endlicher Koerper mit p Elementen
\newcommand{\Fq}{\field{F}_{\!q}}       %Endlicher Koerper mit q Elementen
\newcommand{\Ga}{\field{G}_{a}}         %Add Gruppenschema
\newcommand{\K}{\field{K}}              %Generischer Koerper 
\newcommand{\N}{\field{N}}              %Nat Zahlen
\newcommand{\Pj}{\field{P}}             %Projektives P
\newcommand{\R}{\field{R}} 		%Reelle Zahlen
\newcommand{\Q}{\field{Q}}              %Rationale Zahlen  
\newcommand{\Qt}{\field{H}}             %Quaternionen 
\newcommand{\V}{\field{V}}              %Vektorbuendel V
\newcommand{\Z}{\field{Z}}              %Ganze Zahlen
\DeclareMathOperator{\Real}{Re}

\newcommand{\fdg}{\;|\;}                 %fuer die gilt

%Operatoren
\DeclareMathOperator{\Abb}{Abb}
%\usepackage{sagetex}


%
% Aufgaben
%
\parindent0cm % Abs�tze nicht einr�cken 
% Definieren einer neuen Farbe
\definecolor{light-gray}{gray}{.9}

\newcounter{zaehler}     % neuen Z�hler einf�hren
\newenvironment{aufgn}[2][0]
%---- Header
{\begin{samepage}%
%\colorbox{light-gray}{%                         % Box in gray
% \makebox[\textwidth]{%                           % Box in linewidth
%\textbf{Aufgabe \arabic{zaehler} } }\hspace{-\textwidth}\makebox[\textwidth]{\hfill #1 Punkte} }\\[0.05cm]       % Header
\dotfill\\
{\large\textbf{Aufgabe \arabic{zaehler} \ifthenelse{ \equal{#2}{} }{}{: \emph{ #2 } }}\ifthenelse{-1=#1}{(testierbar)}{}\ifthenelse{0=#1 \or -1=#1}{}{\hfill #1 Punkte} }\\[0.4cm]
%{\large\textbf{Exercise \arabic{zaehler}  #2 }\ifthenelse{-1=#1}{(testierbar)}{}\ifthenelse{0=#1 \or -1=#1}{}{\hfill #1 Punkte} }\\[0.4cm]
\begin{minipage}{\textwidth}%
}%
%-----  foot
{\end{minipage}\nopagebreak%\begin{minipage}{1cm} \end{minipage}
%\\ 
%\begin{minipage}{0.1cm} \end{minipage} 
%\hrulefill \begin{minipage}{1cm} \end{minipage}\\[1cm]  
\stepcounter{zaehler}                           % increase counter
\end{samepage}%
\\%
\bigskip%
}


\newenvironment{aufg}[1][0]
%---- Header
{\begin{samepage}%
\refstepcounter{zaehler}% increase counter
%\colorbox{light-gray}{%                         % Box in gray
% \makebox[\textwidth]{%                           % Box in linewidth
%\textbf{Aufgabe \arabic{zaehler} } }\hspace{-\textwidth}\makebox[\textwidth]{\hfill #1 Punkte} }\\[0.05cm]       % Header
\dotfill\\
{\large\textbf{Aufgabe \arabic{zaehler} }\ifthenelse{-1=#1}{(testierbar)}{}\ifthenelse{0=#1 \or -1=#1}{}{\hfill #1 Punkte} }\\[0.4cm]
\begin{minipage}{\textwidth}%
}%
%-----  foot
{\end{minipage}\nopagebreak%\begin{minipage}{1cm} \end{minipage}
%\\ 
%\begin{minipage}{0.1cm} \end{minipage} 
%\hrulefill \begin{minipage}{1cm} \end{minipage}\\[1cm]  
\end{samepage}%
\\%
\bigskip%
}

\begin{document}
\begin{center}
    \textbf{\LARGE Einführung in Sage}\\
    {\large Einheit 03}\\
    {\large Zusammenfassung: Mengen und Zahlen}
\end{center}

\section{Mengen}

\begin{defn}[Menge]
\begin{quote}
Unter einer \emph{Menge} verstehen wir jede Zusammenfassung $M$ von bestimmten wohlunterschiedenen Objekten $m$ unserer Anschauung oder unseres Denkens zu einem Ganzen.\\
{\tiny (G. Cantor; Beiträge zur Begründung der transfiniten Mengenlehre; Mathematische Annalen; Bd. 46; 1895; S. 481-512) }
\end{quote}
 \begin{itemize}
\item \emph{Element}: Ein Objekt $x$ in der Menge $M$ ($x \in  M$). 
\item \emph{Enthalten}: Es gilt für alle $x \in M$ auch $x \in N$ ($M \subset N$).
\item \emph{Gleichheit}: Es gilt $M\subset N$ und $N \subset M$ ($M=N$).
\end{itemize}
\end{defn}

\section{Natürliche Zahlen}

\begin{defn}[Natürliche Zahlen $\mathbb{N}$ nach Peano]
Sei $\mathbb{N} := \{0,1,2,3,\dots\}$ die Menge der natürlichen Zahlen mit folgenden Eigenschaften:
\begin{enumerate}
\item $0 \in \mathbb{N}$
\item Es gibt eine Nachfolgerabbildung $nf : \mathbb{N} \
  \rightarrow \mathbb{N} \smallsetminus \{ 0 \}$
\item $nf$ ist injektiv.
\item Ist $M \subset \mathbb{N}$ mit $0 \in M$ und folgt für alle  $m \in M$ das
  $nf(m) \in M$ gilt, so ist $M=\mathbb{N}$.
\end{enumerate}
  
\end{defn}
% andere einfuehrung!
Bemerkungen:
\begin{itemize}
\item Nachfolgefunktion: $nf(m)=m+1$
\item Es besteht ein enger Zusammenhang zwischen den natürlichen
  Zahlen und vollständiger  Induktion.
%TODO:erklaerung  .
\end{itemize}

\begin{defn}[Äquivalenzrelation]
Sei $M$ eine Menge. Eine {\color{red} Äquivalenzrelation} $R$ auf $M$ ist eine Teilmenge
\[R\subseteq M \times M\]
 mit den folgenden Eigenschaften (Schreibweise: $(x,y) \in R$, $x \sim_R y$, $x \sim y$):
\begin{enumerate}
\item \textbf{Reflexivität:} für alle $x \in M$ gilt $x \sim x$. 
\item \textbf{Symmetrie:} für alle $x,y \in M$ folgt aus $x \sim y$ das $y \sim x$.
\item \textbf{Transitivität:} für alle $x,y,z \in M$ und $x \sim y$, $y \sim z$ folgt
  $x \sim z$.  
\end{enumerate}
\end{defn}

\begin{defn}[Äquivalenzklasse] Sei $\sim_R$ eine Äquivalenzrelation auf einer Menge $M$.
\begin{itemize}
\item Eine Teilmenge $A \subset M$ heißt {\color{red} Äquivalenzklasse}, falls gilt:
\begin{itemize}
\item [(a)] $A \neq \emptyset$.
\item [(b)] $x,y \in A \ \Rightarrow \ x \sim y$.
\item [(c)] $x \in A$, $y \in M$, $x \sim y$ $\Rightarrow$ $y \in A$.
\end{itemize}
\item Eine Äquivalenzrelation zerlegt eine Menge in disjunkte
Äquivalenzklassen.  
\item Andersrum definiert eine disjunkte Zerlegung einer Menge eine Äquivalenzrelation.
\item  Ein $a \in A$ ist ein {\color{red} Repräsentant} der Äquivalenzklasse
$A$. Schreibweise: $\overline{a}$, $a \bmod R$ für ein Äquivalenzklasse $A$. 
\end{itemize}
\end{defn}

\section{Ganze Zahlen}
\begin{defn}[Ganze Zahlen $\mathbb{Z}$]
Sei $\mathbb{Z}:=\{ 0,1,-1,2,-2,\dots \}$ die Menge der ganzen Zahlen mit folgenden Eigenschaften:
\begin{itemize}
\item Äquivalenzrelation auf $\mathbb{N} \times \mathbb{N}$:\\
$(m,n) \sim (p,q) \mbox{ genau dann, wenn } m+q=n+p \mbox{ gilt.} $
\item Nichtnegative Zahlen: $(m,0)$. Sie sind paarweise nicht äquivalent
zueinander.
\item Negative Zahlen: $(0,m)$. 
\item Die ganzen Zahlen $\mathbb{Z}$ sind  gegeben durch die Menge
der Äquivalenzklassen.
\item Addition:
\[
\overline{(m,n)}+\overline{(u,v)}:=\overline{(m+u,n+v)}\]
\item Multiplikation:
\[
\overline{(m,n)}\cdot\overline{(u,v)}:=\overline{(m u+nv,mv+nu)}
\]
\end{itemize}
  
\end{defn}



\section{Rationale Zahlen}
\begin{defn}[Rationale Zahlen $\mathbb{Q}$]
  Die rationale Zahlen können eingeführt werden über die
Äquivalenzrelation auf $\mathbb{Z}\times(\mathbb{Z}\smallsetminus \{ 0\})$:
\[ (m,n) \sim (p,q) \mbox{ genau dann, wenn } mq=np \mbox{ gilt.} \]
Statt $(m,n)$ schreibt man $\frac{m}{n}$.
\begin{itemize}
\item Die Äquivalenzklasse $\overline{(0,n)}$, $n \in \mathbb{Z}$ ist
die $0$ in $\mathbb{Q}$.
\item Mit $(n,m)$ gehören auch alle Erweiterungen $(kn,km)$ zu einer
Ä.-klasse.  
\item Addition:  
\[ 
\overline{\genfrac(){}{}{m}{n}}+\overline{\genfrac(){}{}{p}{q}}=\overline{\genfrac(){}{}{mq+pn}{nq}},
\]
Multiplikation:
\[
\overline{\genfrac(){}{}{m}{n}} \ \cdot \ \overline{\genfrac(){}{}{p}{q}}=\overline{\genfrac(){}{}{mp}{nq}}.
\]
%\item Die rationalen Zahlen bilden einen \alert{Körper}.
\end{itemize} 
  
\end{defn}


\begin{defn}[Gruppe]
Eine {\color{red} Gruppe} ist ein Paar $(G,\cdot)$ bestehend aus einer Menge
$G$ und einer Verknüpfung $\cdot$ auf $G$, d.h. einer Abbildung
\[
 \cdot: G \times G \ \rightarrow \ G, \quad (a,b) \mapsto a \cdot b
\]
mit folgenden Eigenschaften
\begin{itemize}
\item [(G1)] $(a \cdot b) \cdot c =a \cdot (b \cdot c)$ für alle
$a,b,c \in G$.
\item [(G2)] Es existiert ein $e \in G$ ({\it neutrales Element}) mit $e \cdot a =a$ für alle $a
\in G$ und zu jedem $a \in G$ existiert ein $a' \in G$ ({\it inverses
Element}) mit $a' \cdot
a=e$. 
\end{itemize}  
{\it abelsche} Gruppe: $a \cdot b = b \cdot a$ \textsl{für alle} $a,b \in G$.
\end{defn}

\paragraph{Eigenschaften einer Gruppe}
\begin{itemize}
\item Für ein neutrales Element gilt auch $a \cdot e= a$ für alle
$a\in G$.
\item Es gibt genau ein neutrales Element $e \in G$.
\item Zu jedem $a \in G$ ist das inverse Element $a' \in G$ eindeutig
und wird durch $a^{-1}$ bezeichnet. 
\item Es gilt auch $a \cdot a'=e$. 
\item Für abelsche Gruppen schreibt man oft $+$ statt $\cdot$. Das Inverse zu $a$ wird dann mit $-a$, das Neutrale mit $0$ bezeichnet.
\end{itemize}

\paragraph{Beispiele von Gruppen:}

\begin{itemize}
\item $(\mathbb{Z},+)$, die ganzen Zahlen mit Addition.
\item $(\mathbb{Z}/n\mathbb{Z},+)$, die Restklassen modulo $n$ mit Addition.
\item $(\mathbb{Q},+)$, $(\mathbb{Q} \smallsetminus \{ 0 \} ,\cdot)$
\item $(\mathop{Add}(M,\mathbb{R}),+)$, die reellwertigen Funktionen auf einer Menge $M$ mit punktweiser Addition.
\end{itemize} 

\begin{defn}[Körper]
Ein {\color{red} Körper} ist ein Tripel $(K,+,\cdot)$ bestehend aus einer
Menge $K$ und zwei Verknüpfungen $+$ und $\cdot$ mit folgenden
Eigenschaften:
\begin{itemize}
\item [(K1)] $(K,+)$ ist eine abelsche Gruppe. (Das neutrale Element
heiße $0$. Das inverse Element zu $a \in K$ sei $-a$.) 
\item [(K2)] $(K \smallsetminus \{ 0 \}, \cdot)$ sei eine abelsche
Gruppe. (Das neutrale Element dazu sei $1$.)
\item [(K3)] Distributivgesetze
\begin{eqnarray*}
a \cdot (b + c) & = & (a \cdot b) + (a \cdot c)\\
(a+b) \cdot c & = &   (a \cdot c) + (b \cdot c) \mbox{ für alle }
a,b,c \in K.
\end{eqnarray*}
\end{itemize}
%(Ein Körper ist ein kommutativer unitärer Ring)
\end{defn}

\paragraph{Beispiele von Körpern:}
\begin{itemize}
\item Die rationalen Zahlen $\mathbb{Q}$ mit den Verknüpfungen $+$ und $\cdot$.
\item Die reellen Zahlen $\mathbb{R}$ mit den Verknüpfungen $+$ und $\cdot$.
\item Die komplexen Zahlen $\mathbb{C}$ mit den Verknüpfungen $+$ und $\cdot$.
\item Für $p$ Primzahl $\mathbb{Z}/p\mathbb{Z}$, die Restklassen modulo $p$ mit $+$ und $\cdot$.
\end{itemize}

\begin{defn}[Anordnung]
Sei $K$ ein Körper. Er heißt {\color{red} angeordnet}, wenn es einen {\color{red}
Positivbereich} $P \subset K$ gibt mit
\begin{itemize}
\item Die Mengen $P$, $\{ 0 \}$, und $-P:=\{-x\;|\;x \in P \}$ sind
disjunkt. 
\item $K = P \cup \{ 0 \} \cup -P$.
\item Aus $x,y \in P$ folgt $x+y \in P$ und $x \cdot y \in P$. 
\end{itemize}
Man definiert:
\begin{eqnarray*}
 x >y& \text{genau dann, wenn}& x-y\in P,\\
 x \geq y&  \text{genau dann, wenn} &x-y \in P \cup \{ 0 \}.
\end{eqnarray*}
Analog definiert man $<$ und $\leq$. 
\end{defn}

\begin{defn}[Schranken]
Sei $K$ ein angeordneter Körper.
\begin{itemize}
\item {\color{red} obere Schranke} $y\in K$: Für $M \subset K$, wenn für alle $x\in M$ die Relation $x \leq y$ gilt.
\item nach oben {\color{red} beschränkt}: Wenn eine Teilmenge $M$ von $K$ eine obere Schranke besitzt (analog {\color{red} untere Schranke}).
\item {\color{red} Maximum} von $M$: Eine obere Schranke $y$ einer Teilmenge $M \subset K$, wenn $y\in M$ (analog {\color{red} Minimum}).
\item {\color{red} Supremum}: Die kleinstmögliche obere Schranke $y$ einer Teilmenge $M \subset K$ (analog {\color{red} Infimum}) 
(Nicht notwendigerweise in $M$ oder $K$).  
\end{itemize}
\end{defn}

\section{Reelle Zahlen}
\begin{defn}[Reelle Zahlen $\mathbb{R}$]
Die reellen Zahlen können definiert werden über die Einführung folgender Äquivalenzklasse:
  \begin{itemize}
\item Sei $M$ die Menge aller Teilmengen von $\mathbb{Q}$ mit oberer
Schranke.
\item Äquivalenzrelation: Zwei Elemente aus $M$ seien äquivalent, wenn sie dieselben
Mengen von oberen Schranken haben. 
\end{itemize}
  
\end{defn}

Bemerkungen:
\begin{itemize}
\item Es lassen sich die üblichen Verknüpfungen auf $\mathbb{R}$
definieren. 
\item Die reellen Zahlen können auch als Vervollständigung von
$\mathbb{Q}$ definiert werden oder durch den Dedekindschen Schnitt.
\item Die rationalen Zahlen sind als Äquivalenzklassen der
einelementigen Mengen $\{ x \}$, $x \in \mathbb{Q}$ enthalten.

 \end{itemize}





\section{Komplexe Zahlen}
\begin{defn}[Komplexe Zahlen $\mathbb{C}$]
Der Körper $\mathbb{C}$ der {\color{red} komplexen Zahlen} ist
definiert über die Menge $\mathbb{R}^2=\mathbb{R} \times \mathbb{R}$ mit 
\begin{itemize}
 \item   Addition: $(k,l)+(n,m)=(k+n,l+m)$
\item Multiplikation:  $ (k,l) \cdot (n,m) = (kn-lm, km+ln) $
\end{itemize}
\begin{itemize}
 \item {\color{red} $i:=(0,1)$}  mit 
\begin{itemize}
\item $ i^2=(0,1) \cdot (0,1) = (-1,0) $
\item $\forall (x,y) \in \mathbb{C}:  (x,y)=x \cdot (1,0) + y \cdot (0,1)= x +i y$
\end{itemize}
\item \textsl{Betrag}: $|z|=|(x,y)|:=\sqrt{x^2+y^2}$ 
\end{itemize}
  
\end{defn}


\begin{thm}[Fundamentalsatz der Algebra]
    Jedes nicht konstante Polynom (mit komplexen Koeffizienten) hat mindestens eine Nullstelle in $\mathbb{C}$. 
\end{thm}
\paragraph{Eigenschaften von $\mathbb{C}$:}
\begin{itemize}
\item Polarkoordinaten $(r, \varphi)$ zu $(x,y) \in \mathbb{C}$
\[  r:= \sqrt{x^2+y^2}, \quad \tan(\varphi)=\frac{y}{x} \] 

\item Es gilt: $z = (x,y)_{\text{Rechtwinklig}} = (r,\varphi)_{\text{Polar}} = re^{i\varphi}$
\end{itemize}

\end{document}
