\documentclass[hyperref={xetex}]{beamer}
\title{Einführung in Sage - Einheit 3}
\subtitle{Mengen, Zahlen}
\mode<article>
{
  \usepackage{fullpage}
  \usepackage{pgf}
  \usepackage[xetex]{hyperref}
  \setjobnamebeamerversion{beamer}
}

\mode<presentation>
{
  %\usetheme{Frankfurt}
 %\usetheme{My}
  \usetheme{Madrid}
  % or ...
%\usecolortheme{seagull}
  %\setbeamercovered{transparent}
  %\setbeamercovered{dynamic}
  % or whatever (possibly just delete it)
}
\usenavigationsymbolstemplate{}
\usefonttheme{structurebold}
\usepackage{multimedia}
%\usepackage{tikz}
\usepackage{fontspec,xunicode,xltxtra}

%\usepackage{polyglossia}
%\setdefaultlanguage[spelling=new, latesthyphen=true]{german}
%\setsansfont{DejaVu Sans}
%\setsansfont{Verdana}
%\setsansfont{Arial}
%\setromanfont{Linux Libertine O}
%\setsansfont{Linux Biolinum O}

\setbeamertemplate{footline}
{
\leavevmode
%\hbox{\begin{beamercolorbox}[wd=.5\paperwidth,ht=2.5ex,dp=1.125ex,
%leftskip=.3cm plus1fill,rightskip=.3cm]{author in head/foot}%
%    \usebeamerfont{author in head/foot}\insertshortauthor
%  \end{beamercolorbox}%
%  \begin{beamercolorbox}[wd=.5\paperwidth,ht=2.5ex,dp=1.125ex,leftskip=.3cm,
%rightskip=.3cm plus1fil]{title in head/foot}%
%    \usebeamerfont{title in head/foot}\insertshorttitle\hfill

\hfill\insertframenumber  \hspace{3pt}

%\inserttotalframenumber
%\hspace*{2ex}
%  \end{beamercolorbox}}%
  \vskip3pt%
}

\usepackage[ngerman]{babel}
\selectlanguage{ngerman}

%
% math/symbols
%
\usepackage{amssymb}
\usepackage{amsthm}
% \usepackage{latexsym}
\usepackage{amsmath}
%\usepackage{amsxtra} %Weitere Extrasymbole
%\usepackage{empheq} %Gleichungen hervorheben
%\usepackage{bm}
 %\bm{A} Boldface im Mathemodus

\usepackage{cellspace}
\setlength{\cellspacetoplimit}{2pt}
\setlength{\cellspacebottomlimit}{2pt}

%%%%%%%%%%%%%%%%%% Fuer Frames [fragile]-Option verwenden!
%Programm-Listing
%%%%%%%%%%%%%%%%%%
%Listingsumgebung fuer verbatim
%Grauhinterlegeter Text
%Automatischer Zeilenumbruch ist aktiviert
\usepackage{listings}
\definecolor{lgray}{gray}{0.80}
%\lstset{backgroundcolor=\color{lgray}, frame=single, basicstyle=\ttfamily, breaklines=true}
\lstnewenvironment{sage}{\lstset{backgroundcolor=\color{lgray},language=Python, emphstyle=\color{red}, frame=single, basicstyle=\ttfamily, breaklines=true,mathescape =true,escapechar=§}}{}


\usepackage{mydef}
%\usepackage{cmap} % you can search in the pdf for umlauts and ligatures
\usepackage{colonequals} %corrects the definition-symbols \colonequals (besides others)
\title{Einführung in Sage}
%
%\subtitle{Disputation} % (optional)

\author{Jochen Schulz}
% - Use the \inst{?} command only if the authors have different
%   affiliation.

\institute{Georg-August Universit\"at G\"ottingen \pgfimage[height=0.5cm]{../figures/unilogo3}}
% - Use the \inst command only if there are several affiliations.
% - Keep it simple, no one is interested in your street address.

\date{\today}

\subject{Sage}
% This is only inserted into the PDF information catalog. Can be left
% out. 

% If you have a file called "university-logo-filename.xxx", where xxx
% is a graphic format that can be processed by latex or pdflatex,
% resp., then you can add a logo as follows:

%\logo{\pgfimage[height=0.5cm]{figures/unilogo3}}


% Delete this, if you do not want the table of contents to pop up at
% the beginning of each subsection:

\AtBeginSection[]
{
  \begin{frame}<beamer>
    \frametitle{Aufbau}
    \tableofcontents[currentsection,currentsubsection]
  \end{frame}
}

\AtBeginSubsection[]
{
  \begin{frame}<beamer>
    \frametitle{Aufbau}
    \tableofcontents[currentsection,currentsubsection]
  \end{frame}
}



%%%%%%%%%%%%%%%%%%%
%Neue Definitionen
%%%%%%%%%%%%%%%%%%%

%Newcommands
\newcommand{\Fun}[1]{\mathcal{#1}}      %Mathcal fuer Funktoren
\newcommand{\field}[1]{\mathbb{#1}}     %Grundkoerper ?? in mathds

\newcommand{\A}{\field{A}}              %Affines A
\newcommand{\C}{\field{C}}              %Complexes C
\newcommand{\Fp}{\field{F}_{\!p}}       %Endlicher Koerper mit p Elementen
\newcommand{\Fq}{\field{F}_{\!q}}       %Endlicher Koerper mit q Elementen
\newcommand{\Ga}{\field{G}_{a}}         %Add Gruppenschema
\newcommand{\K}{\field{K}}              %Generischer Koerper 
\newcommand{\N}{\field{N}}              %Nat Zahlen
\newcommand{\Pj}{\field{P}}             %Projektives P
\newcommand{\R}{\field{R}} 		%Reelle Zahlen
\newcommand{\Q}{\field{Q}}              %Rationale Zahlen  
\newcommand{\Qt}{\field{H}}             %Quaternionen 
\newcommand{\V}{\field{V}}              %Vektorbuendel V
\newcommand{\Z}{\field{Z}}              %Ganze Zahlen

\newcommand{\fdg}{\;|\;}                 %fuer die gilt

%Operatoren
\DeclareMathOperator{\Abb}{Abb}
%\usepackage{sagetex}

\begin{document}
\lstset{basicstyle={\lstbasicfont\footnotesize}}


\begin{document}
\titlepage

\begin{frame}{Aufbau}
\tableofcontents
\end{frame}

%===================================================
\section{Mengen}
%==================================================


\begin{frame}{Sage}
\begin{center}
\url{https://sage.math.uni-goettingen.de/home/pub/13/}
\end{center}
\end{frame}



% \begin{frame}[fragile]{Multimengen}
%  Oft ist es nützlich Elemente von Mengen mit Vielfachheit zu zählen. In MuPAD 
%  gibt es deshalb den Datentyp  \isage{Dom::Multiset}. Die Elemente der Menge werden mit Vielfachheit aufgeführt.
% \begin{sagein}
% Multimenge := Dom::Multiset(a,a,b,c,a,c,d)
% \end{sagein}
% \begin{sage}
%   {[a, 3], [b, 1], [c, 2], [d, 1]}
% \end{sage}
% \begin{sagein}
% Multimenge union {a}; Multimenge minus {c}
% \end{sagein}
% \begin{sage}
%   {[a, 4], [b, 1], [c, 2], [d, 1]}
%   {[a, 3], [b, 1], [c, 1], [d, 1]}
% \end{sage}
% \end{frame}


%-------------------------------------
\section{Zahlen}
%--------------------------------------

\begin{frame}[fragile]{Sage}
\begin{center}
\url{https://sage.math.uni-goettingen.de/home/pub/14/}
\end{center}
\end{frame}


%=======================================
%\section{Ungleichungen}
%=======================================

% \begin{frame}[fragile]{Ungleichungen}
% MuPAD kann mittels des Befehls {\color{blue} \isage{solve}} auch
% Ungleichungen lösen.
% \begin{sagein}
% domtype(%)
% \end{sagein}
% \begin{sage}
%   Dom::Interval
% \end{sage}
% \begin{sagein}
% >>assume(x>0): solve({exp(x)<=4, exp(x)>=1},x)
% \end{sagein}
% \begin{sage}
%   (0, infinity) intersect (0, 2 ln(2)]               
% \end{sage}
% \end{frame}

% \begin{frame}[fragile]{Intervalle}
% \begin{itemize}
% \item Intervalle in Sage werden durch {\color{blue} \isage{:}} definiert. 
% \item Zu einer Zahl $a$ wird durch {\color{blue} \isage{hull(a)}} ein
%   Intervall bestimmt, in dem $a$ liegt.
% \item Funktionen wie \isage{sin} oder \isage{exp} akzeptieren als Eingaben
%   Intervalle. 
% \end{itemize}
% \end{frame}
% 
% \begin{frame}[fragile]{Beispiele I}
% \begin{sagein}
% a = range(1,8)
% a:=1/3...1; domtype(a) 
% \end{sagein}
% \begin{sage}
%   0.3333333333 ... 1.0
%   DOM_INTERVAL 
% \end{sage}
% \begin{sagein}
% exp(a)+1, sin(a)/a
% \end{sagein}
% \begin{sage}
%   2.395612425 ... 3.718281829, 
%   0.3271946967 ... 2.524412955
% \end{sage}
% \end{frame}
% 

%TODO: Platz fuer mehr!

\end{document}
