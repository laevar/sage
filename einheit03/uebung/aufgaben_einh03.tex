\documentclass[a4paper,10pt,DIV15]{scrartcl}
\usepackage[psamsfonts]{amssymb}
\usepackage{amsmath}
\usepackage[svgnames]{xcolor} %color definitions

\usepackage{fontspec,xunicode,xltxtra}
%\usepackage{fontspec,xunicode}
%\usepackage{polyglossia}
%\setdefaultlanguage[spelling=new, latesthyphen=true]{german}
%\setsansfont{DejaVu Sans}
%\setsansfont{Verdana}
%\setsansfont{Arial}
%\setromanfont[Mapping=tex-text]{Linux Libertine}
%\setsansfont[Mapping=tex-text]{Myriad Pro}
%\setmonofont[Mapping=tex-text]{Courier New}

%\setsansfont{Linux Biolinum}

\usepackage[ngerman]{babel}
\selectlanguage{ngerman}

%
% math/symbols
%
\usepackage{amssymb}
\usepackage{amsthm}
% \usepackage{latexsym}
\usepackage{amsmath}
%\usepackage{amsxtra} %Weitere Extrasymbole
%\usepackage{empheq} %Gleichungen hervorheben
%\usepackage{bm}
 %\bm{A} Boldface im Mathemodus

\usepackage{multimedia}
%\usepackage{tikz}

\usepackage{cellspace}
\setlength{\cellspacetoplimit}{2pt}
\setlength{\cellspacebottomlimit}{2pt}

%%%%%%%%%%%%%%%%%% Fuer Frames [fragile]-Option verwenden!
%Programm-Listing
%%%%%%%%%%%%%%%%%%
%Listingsumgebung fuer verbatim
%Grauhinterlegeter Text
%Automatischer Zeilenumbruch ist aktiviert
\usepackage{listings}
% This command allows you to typeset syntax highlighted Matlab
% code ``inline''.
\newcommand{\isage}[1]{\lstinline|#1|}

\definecolor{lgray}{gray}{0.80}
\definecolor{gray}{gray}{0.3}
\definecolor{darkgreen}{rgb}{0,0.4,0}
\definecolor{darkblue}{rgb}{0,0,0.8}
\definecolor{key}{rgb}{0,0.5,0} 
%\lstset{backgroundcolor=\color{lgray}, frame=single, basicstyle=\ttfamily, breaklines=true}
\lstnewenvironment{sage}[1][]{\lstset{xleftmargin=0.2cm,frame=none,backgroundcolor=\color{white},basicstyle=\color{darkblue}\ttfamily\small,#1}}{} 
\lstnewenvironment{sagein}[1][]{\lstset{#1}}{} 
%\lstnewenvironment{sage}{\lstset{,language=python, keywordstyle=color{blue},    commentstyle=color{green}, emphstyle=\color{red}, %frame=single, stringstyle=\color{red}, basicstyle=\ttfamily, ,mathescape =true,escapechar=§}}{}

\lstset{
language=python,
backgroundcolor=\color{lgray},
breaklines=true,
basicstyle=\ttfamily\small,
%otherkeywords={ =},
keywordstyle=\color{blue},
stringstyle=\color{darkgreen},
showstringspaces=false,
emph={class, pass, in, for, while, if, is, elif, else, not, and, or,
def, print, exec, break, continue, return},
emphstyle=\color{blue},
emph={[2]True, False, None, self},
emphstyle=[2]\color{key},
emph={[3]from, import, as},
emphstyle=[3]\color{blue},
upquote=true,
morecomment=[s]{"""}{"""},
commentstyle=\color{gray}\slshape,
%framexleftmargin=1mm, framextopmargin=1mm, 
frame=single,
mathescape =true,
escapechar=§
}


\usepackage{mydef}
%\usepackage{cmap} % you can search in the pdf for umlauts and ligatures
\usepackage{colonequals} %corrects the definition-symbols \colonequals (besides others)
\usepackage{ifthen}
%%%%%%%%%%%%%%%%%%%
%Neue Definitionen
%%%%%%%%%%%%%%%%%%%

%Newcommands
\newcommand{\Fun}[1]{\mathcal{#1}}      %Mathcal fuer Funktoren
\newcommand{\field}[1]{\mathbb{#1}}     %Grundkoerper ?? in mathds

\newcommand{\A}{\field{A}}              %Affines A
\newcommand{\C}{\field{C}}              %Complexes C
\newcommand{\Fp}{\field{F}_{\!p}}       %Endlicher Koerper mit p Elementen
\newcommand{\Fq}{\field{F}_{\!q}}       %Endlicher Koerper mit q Elementen
\newcommand{\Ga}{\field{G}_{a}}         %Add Gruppenschema
\newcommand{\K}{\field{K}}              %Generischer Koerper 
\newcommand{\N}{\field{N}}              %Nat Zahlen
\newcommand{\Pj}{\field{P}}             %Projektives P
\newcommand{\R}{\field{R}} 		%Reelle Zahlen
\newcommand{\Q}{\field{Q}}              %Rationale Zahlen  
\newcommand{\Qt}{\field{H}}             %Quaternionen 
\newcommand{\V}{\field{V}}              %Vektorbuendel V
\newcommand{\Z}{\field{Z}}              %Ganze Zahlen
\DeclareMathOperator{\Real}{Re}

\newcommand{\fdg}{\;|\;}                 %fuer die gilt

%Operatoren
\DeclareMathOperator{\Abb}{Abb}
%\usepackage{sagetex}

%
% Aufgaben
%
\parindent0cm % Abs�tze nicht einr�cken 
% Definieren einer neuen Farbe
\definecolor{light-gray}{gray}{.9}

\newcounter{zaehler}     % neuen Z�hler einf�hren
\stepcounter{zaehler}    % Z�hler einen hochz�hlen

\newenvironment{aufg}[1][0]
%---- Header
{\begin{samepage}%
%\colorbox{light-gray}{%                         % Box in gray
% \makebox[\textwidth]{%                           % Box in linewidth
%\textbf{Aufgabe \arabic{zaehler} } }\hspace{-\textwidth}\makebox[\textwidth]{\hfill #1 Punkte} }\\[0.05cm]       % Header
\dotfill\\
{\large\textbf{Aufgabe \arabic{zaehler} }\ifthenelse{0=#1}{}{\hfill #1 Punkte} }\\[0.4cm]
\begin{minipage}{\textwidth}
}
%-----  foot
{\end{minipage} \nopagebreak %\begin{minipage}{1cm} \end{minipage}
%\\ 
%\begin{minipage}{0.1cm} \end{minipage} 
%\hrulefill \begin{minipage}{1cm} \end{minipage}\\[1cm]  
\stepcounter{zaehler}                           % increase counter
\end{samepage}%
\\%
\bigskip%
}


%\usepackage{tikz}
%\usetikzlibrary{shadows}
%\usetikzlibrary{fit}
%\usetikzlibrary{shapes}
%\usetikzlibrary{backgrounds}

\parindent0cm % Abs�tze nicht einr�cken 

% Definieren einer neuen Farbe
\definecolor{light-gray}{gray}{.9}

\newcounter{zaehler}     % neuen Z�hler einf�hren
\stepcounter{zaehler}    % Z�hler einen hochz�hlen

\newenvironment{aufg}%
%---- Header
{\begin{samepage}
\colorbox{light-gray}{                         % Box in gray
 \makebox[\textwidth]{                           % Box in linewidth
\textbf{Aufgabe} \arabic{zaehler} :}}\\[0.1cm]       % Header
%\begin{minipage}{0.5cm} \end{minipage}    % Insert 0.5cm
\begin{minipage}{\textwidth}}
%-----  foot
{\end{minipage} \nopagebreak %\begin{minipage}{1cm} \end{minipage}
\\[0.1cm] 
%\begin{minipage}{0.1cm} \end{minipage} 
%\hrulefill \begin{minipage}{1cm} \end{minipage}\\[1cm]  
\stepcounter{zaehler}                           % increase counter
 \end{samepage}%
}

%-------------------------------------------------------------------------------
\begin{document}
%-------------------------------------------------------------------------------

%--------------------------------------------------- Header
\begin{center}
\textbf{\LARGE Einf\"uhrung in Sage }\\
\end{center}
\begin{minipage}{6cm}
Dr. J. Schulz\\
C. Rügge
\end{minipage}\hfill
\begin{minipage}{2.5cm}
\begin{flushright}
\textbf{Einheit 3}\\
WS 2009/2010
\end{flushright}
\end{minipage}\\[1cm]

\begin{aufg}
Bestimmen Sie die Teilmenge der Potenzmenge von $\{ 1,2,3,4,5,6 \}$,
  die $1$ und $4$ enthält.
\end{aufg}

\begin{aufg}
Bestimmen Sie eine Näherung der Ausdrücke $\pi, e^2, \log_2 3$ und
$\sqrt[3]{5}$ mit $100$ signifikanten Stellen.
\end{aufg}

\begin{aufg}
Gegeben seien die folgenden Mengen $ M1 =  \lbrace1,2,3,\lbrace
c,4\rbrace,a,b\rbrace$, $M2= \lbrace6,7,8,a,c,d\rbrace$ und
$M3=\lbrace1,2,a,b,f,g\rbrace$. 
  \begin{enumerate}
  \item
    F\"ugen Sie das Element \{2,3\} der Menge $M2$ hinzu.
  \item
    Bestimmen Sie den Schnitt und die Vereinigung von $M1$, $M2$ und $M3$.
  \item
    Bestimmen Sie die Differenz zwischen $M1$ und $M2$.
  \item
    Bestimmen Sie die Teilmenge der Potenzmenge von $M1$, die sowohl $2$ als auch $b$
    enth\"alt. 
  \end{enumerate}
\end{aufg}

\begin{aufg}
  \begin{enumerate}
  \item
    Sei $a=0.304627*10^6$ eine Darstellung von $a$ zur Basis
    $10$. Berechnen Sie die $5$-adische Darstellung von $a$.
  \item
    Sei $b=0.152005*6^6$ eine Darstellung zur Basis $6$. Berechnen Sie
    die Dezimaldarstellung von $b$.
 \item
    Sei $c=0.123456*7^6$ eine Darstellung zur Basis $7$. Erstellen Sie
    die $11$-adische Darstellung von $c$.
  \end{enumerate}

%Vorlesung: Zahlen! Zp, QQ, RR, etc. 

\end{aufg}

\begin{aufg}
 Berechnen Sie von $f(x)=\frac{2x^2-3x}{x-4}$ die lokalen Extrema samt
 Funktionswerte und überprüfen Sie, ob Minima oder Maxima
 vorliegen. 

Die berechneten Extremwertstellen sollen für die weiteren Berechnungen
direkt verwendet werden und sollen nicht 'per Hand' eingegeben werden. 
\end{aufg}

\begin{aufg}
  Gegeben sei \(a=1-2i , b=\frac{3+4i}{3i}\) und \( c=7-6i\) mit \(  a,b,c\in
  \mathbb{C}\). 
  \begin{enumerate}
  \item
    Berechnen Sie den Realteil und den Imaginärteil von \(d=ab,
    e=\frac{ac}{b}\) und \(f=b^{ac} \). 
  \item
    Bestimmen Sie die Polarkoordinaten der Zahlen $a,b$ und $c$.
  \end{enumerate}
\end{aufg}

\begin{aufg}
Berechnen die Summe
\[ \sum_{i=1}^{100} \sum_{j=1}^i \frac{1}{j+i} . \]
\end{aufg}

\begin{aufg}
Nach \verb+x = (10^50/3.0).n(digits=100)+ sind nur die ersten Dezimalstellen in
$x$ garantiert richtig. Was passiert wenn man die Anzahl
der signifikanten Stellen erhöht?
\end{aufg}

\begin{aufg} 
Informieren Sie sich auf der Seite
{\small
\begin{verbatim}
http://cs.furman.edu/digitaldomain/themes/risks/risks_numeric.htm
\end{verbatim}
}
über die Risiken bei numerischen Berechnungen und rekonstruieren
Sie, was jeweils passierte.
\end{aufg}

\begin{aufg}
Geben Sie folgende Zeilen ein:
\begin{sage}
>> def f(x):
>> 	return f(x+1)
>> f(0)
\end{sage}
Erklären Sie den Fehler. Wie können Sie die Funktion \verb+f(x)+ anpassen, damit der
Fehler nicht mehr auftritt?  
\end{aufg}
\end{document}

