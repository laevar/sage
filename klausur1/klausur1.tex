\documentclass[a4paper,12pt,DIV15]{scrartcl}
\usepackage[psamsfonts]{amssymb}
\usepackage{amsmath}
\usepackage[svgnames]{xcolor} %color definitions

\usepackage{fontspec,xunicode,xltxtra}
%\usepackage{fontspec,xunicode}
%\usepackage{polyglossia}
%\setdefaultlanguage[spelling=new, latesthyphen=true]{german}
%\setsansfont{DejaVu Sans}
%\setsansfont{Verdana}
%\setsansfont{Arial}
%\setromanfont[Mapping=tex-text]{Linux Libertine}
%\setsansfont[Mapping=tex-text]{Myriad Pro}
%\setmonofont[Mapping=tex-text]{Courier New}

%\setsansfont{Linux Biolinum}

\usepackage[ngerman]{babel}
\selectlanguage{ngerman}

%
% math/symbols
%
\usepackage{amssymb}
\usepackage{amsthm}
% \usepackage{latexsym}
\usepackage{amsmath}
%\usepackage{amsxtra} %Weitere Extrasymbole
%\usepackage{empheq} %Gleichungen hervorheben
%\usepackage{bm}
 %\bm{A} Boldface im Mathemodus

\usepackage{multimedia}
%\usepackage{tikz}

\usepackage{cellspace}
\setlength{\cellspacetoplimit}{2pt}
\setlength{\cellspacebottomlimit}{2pt}

%%%%%%%%%%%%%%%%%% Fuer Frames [fragile]-Option verwenden!
%Programm-Listing
%%%%%%%%%%%%%%%%%%
%Listingsumgebung fuer verbatim
%Grauhinterlegeter Text
%Automatischer Zeilenumbruch ist aktiviert
\usepackage{listings}
% This command allows you to typeset syntax highlighted Matlab
% code ``inline''.
\newcommand{\isage}[1]{\lstinline|#1|}

\definecolor{lgray}{gray}{0.80}
\definecolor{gray}{gray}{0.3}
\definecolor{darkgreen}{rgb}{0,0.4,0}
\definecolor{darkblue}{rgb}{0,0,0.8}
\definecolor{key}{rgb}{0,0.5,0} 
%\lstset{backgroundcolor=\color{lgray}, frame=single, basicstyle=\ttfamily, breaklines=true}
\lstnewenvironment{sage}[1][]{\lstset{xleftmargin=0.2cm,frame=none,backgroundcolor=\color{white},basicstyle=\color{darkblue}\ttfamily\small,#1}}{} 
\lstnewenvironment{sagein}[1][]{\lstset{#1}}{} 
%\lstnewenvironment{sage}{\lstset{,language=python, keywordstyle=color{blue},    commentstyle=color{green}, emphstyle=\color{red}, %frame=single, stringstyle=\color{red}, basicstyle=\ttfamily, ,mathescape =true,escapechar=§}}{}

\lstset{
language=python,
backgroundcolor=\color{lgray},
breaklines=true,
basicstyle=\ttfamily\small,
%otherkeywords={ =},
keywordstyle=\color{blue},
stringstyle=\color{darkgreen},
showstringspaces=false,
emph={class, pass, in, for, while, if, is, elif, else, not, and, or,
def, print, exec, break, continue, return},
emphstyle=\color{blue},
emph={[2]True, False, None, self},
emphstyle=[2]\color{key},
emph={[3]from, import, as},
emphstyle=[3]\color{blue},
upquote=true,
morecomment=[s]{"""}{"""},
commentstyle=\color{gray}\slshape,
%framexleftmargin=1mm, framextopmargin=1mm, 
frame=single,
mathescape =true,
escapechar=§
}


\usepackage{mydef}
%\usepackage{cmap} % you can search in the pdf for umlauts and ligatures
\usepackage{colonequals} %corrects the definition-symbols \colonequals (besides others)
\usepackage{ifthen}
%%%%%%%%%%%%%%%%%%%
%Neue Definitionen
%%%%%%%%%%%%%%%%%%%

%Newcommands
\newcommand{\Fun}[1]{\mathcal{#1}}      %Mathcal fuer Funktoren
\newcommand{\field}[1]{\mathbb{#1}}     %Grundkoerper ?? in mathds

\newcommand{\A}{\field{A}}              %Affines A
\newcommand{\C}{\field{C}}              %Complexes C
\newcommand{\Fp}{\field{F}_{\!p}}       %Endlicher Koerper mit p Elementen
\newcommand{\Fq}{\field{F}_{\!q}}       %Endlicher Koerper mit q Elementen
\newcommand{\Ga}{\field{G}_{a}}         %Add Gruppenschema
\newcommand{\K}{\field{K}}              %Generischer Koerper 
\newcommand{\N}{\field{N}}              %Nat Zahlen
\newcommand{\Pj}{\field{P}}             %Projektives P
\newcommand{\R}{\field{R}} 		%Reelle Zahlen
\newcommand{\Q}{\field{Q}}              %Rationale Zahlen  
\newcommand{\Qt}{\field{H}}             %Quaternionen 
\newcommand{\V}{\field{V}}              %Vektorbuendel V
\newcommand{\Z}{\field{Z}}              %Ganze Zahlen
\DeclareMathOperator{\Real}{Re}

\newcommand{\fdg}{\;|\;}                 %fuer die gilt

%Operatoren
\DeclareMathOperator{\Abb}{Abb}
%\usepackage{sagetex}

%
% Aufgaben
%
\parindent0cm % Abs�tze nicht einr�cken 
% Definieren einer neuen Farbe
\definecolor{light-gray}{gray}{.9}

\newcounter{zaehler}     % neuen Z�hler einf�hren
\stepcounter{zaehler}    % Z�hler einen hochz�hlen

\newenvironment{aufg}[1][0]
%---- Header
{\begin{samepage}%
%\colorbox{light-gray}{%                         % Box in gray
% \makebox[\textwidth]{%                           % Box in linewidth
%\textbf{Aufgabe \arabic{zaehler} } }\hspace{-\textwidth}\makebox[\textwidth]{\hfill #1 Punkte} }\\[0.05cm]       % Header
\dotfill\\
{\large\textbf{Aufgabe \arabic{zaehler} }\ifthenelse{0=#1}{}{\hfill #1 Punkte} }\\[0.4cm]
\begin{minipage}{\textwidth}
}
%-----  foot
{\end{minipage} \nopagebreak %\begin{minipage}{1cm} \end{minipage}
%\\ 
%\begin{minipage}{0.1cm} \end{minipage} 
%\hrulefill \begin{minipage}{1cm} \end{minipage}\\[1cm]  
\stepcounter{zaehler}                           % increase counter
\end{samepage}%
\\%
\bigskip%
}


%\usepackage{tikz}
%\usetikzlibrary{shadows}
%\usetikzlibrary{fit}
%\usetikzlibrary{shapes}
%\usetikzlibrary{backgrounds}

\parindent0cm % Abs�tze nicht einr�cken 

% Definieren einer neuen Farbe
\definecolor{light-gray}{gray}{.9}

%-------------------------------------------------------------------------------
\begin{document}
%-------------------------------------------------------------------------------

%--------------------------------------------------- Header
\begin{center}
\textbf{\LARGE Mathematische Anwendersysteme }\\
\textbf{\LARGE Einführung in Sage}\\\medskip
\end{center}
%\begin{minipage}{6cm}
%Jochen Schulz,\\
%Christoph Rügge
%\end{minipage}
\hfill
\begin{minipage}{4cm}
%\textbf{Klausur}\\
10.03.2011
\end{minipage}\\[1cm]
%------------------------------

\begin{center}
\Huge \textbf{Klausur}
\end{center}
\bigskip\bigskip\bigskip
\Large
\begin{center}
\begin{tabular}{|Sl||p{0.5cm}|p{0.5cm}|p{0.5cm}|p{0.5cm}|p{0.5cm}|p{0.5cm}|p{0.5cm}||Sc|}
\hline
Aufgabe & \textbf{1} & \textbf{2} & \textbf{3} & \textbf{4} & \textbf{5} & \textbf{6} & \textbf{7} & \textbf{Summe}\\
\hline
Mögl. Pkt. &  4  & 2  & 2  & 3  & 5  & 8  & 5  &  29  \\
\hline
Erreichte Pkt. &    &   &   &   &   &   &     &    \\
\hline
\end{tabular}
\end{center}

\bigskip\bigskip\bigskip
Bitte eintragen:\\
\begin{center}
\begin{tabular}{|Sl|p{8cm}|}
\hline
Nachname: & \\
\hline
Vorname: & \\
\hline
Studiengang: & \\
\hline 
Semester: & \\
\hline 
Immatrikulationsnummer: & \\
\hline
\end{tabular}\\[1cm]
\textbf{Hinweise:}
\begin{itemize}
\item Die Klausur beginnt um 10.30 Uhr und endet um 12.00 Uhr.
\item Ben\"otigte Hilfsmittel sind Stift und Papier.
\item Erlaubte Hilfsmittel sind gedruckte sowie handgeschriebene Notizen oder Skripte. 
\item Benutzen Sie zum Aufschreiben der Aufgaben möglichst exakten Sage-Syntax.
\end{itemize}
\end{center}

\newpage
\normalsize



%-----------------------------------------------------------------------------------
% \begin{aufg}{4}
% \begin{enumerate}
% \item Sei $a=1234$ im Dezimalsystem. 
%     Berechnen Sie für $a$ die Darstellung zur Basis $5$.
% \item  Sei $1101$ die Darstellung einer Zahl zur Basis $2$. Berechnen Sie die Darstellung dieser Zahl zur Basis 3.
% \end{enumerate}
% \end{aufg}
%-------------------------------------------------------------------------------------
% \begin{aufg}{6}
% Zu welchen Auswertungen führt der Aufruf von \isage{x1}, \ldots, \isage{x6} nach
% den folgenden Eingaben.
% \begin{enumerate}
% \item 
% \verb~a:=2: b:=2: c:=2: x1:=a+b+c: x1~
% \item 
% \isage{L:=[1,2,3]: M:=[2,3,4]: x2:= L.M: x2}
% \item 
% \verb~k:=r+s: m:=r-s: x3:=k+m: x3~ 
% \item 
% \verb~U:=brat: V:=wurst: x4:=bratwurst: x4~ 
% \item 
% \verb~e:=PI: f:=sin: x5:=f(e): x5~ 
% \item 
% \verb~R:={4,2,1,3,4,2}: x6:=nops(R): x6~
% \end{enumerate}
% \end{aufg}

\begin{aufg}{3}
Schreiben sie eine Funktion mit \isage{def}, die 
\[ f(s) = \int_0^1 e^{-s t} \sin( t^2) dt \]
numerisch approximiert. 
\end{aufg}

\begin{aufg}{3}
 Was sind die Unterschiede zwischen Listen, Tuple und Dictionaries ? Geben sie auch
jeweils Beispiele an, welches die Unterschiede hervorhebt.
\end{aufg}


\begin{aufg}{2}
Führen Sie eine Kurvendiskussion durch für die Funktion (andere Funktion!)
\[  f: x \quad \mapsto \quad \exp(1/x)+\frac{1}{4}x\exp(1/x)\]
\begin{itemize}
\item Untersuchen Sie das Verhalten von $f$ bei der Polstelle $x=0$.
\item Untersuchen Sei das Verhalten von $f$ für $x\r \Rightarrow \pm \infty$.
\item Berechnen Sie Nullstellen, Extremstellen und
  Wendepunkte. Überlegen Sie sich, ob lokale/globale Maxima oder Minima an den
  Extremstellen vorliegen und geben Sie sie an. 
\item Plotten Sie den Graphen  auf dem Intervall $[-5,10]$.
\end{itemize}

\end{aufg}

\begin{aufg}{3}
Mit welcher Befehlsfolge kann man die Vandermonde-Matrix
 
\[ 
V:= \left(\begin{array}{ccccc} 
1 & x_1 & x_1^2 & \hdots & x_1^{n-1}\\
1 & x_2 & x_2^2 & \hdots & x_2^{n-1}\\
\vdots & \vdots & \vdots & \vdots & \vdots\\
1 & x_n & x_n^2 & \hdots & x_n^{n-1}\\
\end{array} \right) 
\]
erstellen?
\end{aufg}


\begin{aufg}{5}
Schreiben sie eine Funktion mit Input-Variablen $x_0$ und $TOL$, die die Folge
\[ x_{n+1} = x_n - \frac{x_n^2-5}{2 x_n}, \quad n \in \mathbb{N} \]
berechnet und abbricht, wenn $|x_{n}-x_{n-1}| \leq TOL$ ist. Die Funktion soll
$x_{n}$ und das zugeh\"orige $n$ zur\"uckgeben. 
\end{aufg}


%-------------------------------------------------------------------------
\begin{aufg}{2}
Welche Möglichkeiten stehen zur Verfügung um die Gleichheit bei einer Gleichung des Typs 
\isage{<Ausdruck1> == <Ausdruck2>} zu überprüfen ?
\end{aufg}
%-------------------------------------------------------------------------
% \begin{aufg}{2}z
% Schreiben Sie eine MuPAD-Prozedur \isage{programm1} die für eine Zahl $n>10$ die Zeichenkette \isage{MuPAD ist super} und für $n\leq10$ die Zeichenkette   \isage{MuPAD ist Mist} zurückgibt (Typenüberprüfung braucht nicht durchgeführt zu werden).
% \end{aufg}
%-----------------------------------------------------------------------------------
\begin{aufg}{4}
Schreiben Sie ohne Verwendung der Funktion \isage{max} eine Sage-Prozedur\\ \isage{programm2}, die aus einer Liste von Zahlen die größte Zahl zurückliefert (Typenüberprüfung braucht nicht durchgeführt zu werden).
\end{aufg}

\begin{aufg}{2}
Erklären sie, wie man in Sage die ganzen, rationalen und reellen Zahlen darstellt und welche Unterschiede in der Behandlung seitens des Computers vorhanden sind.
\end{aufg}


%------------------------------------------------------------------------
\end{document}

